\documentclass{article}
\usepackage{amsmath, amsthm, amssymb, amsfonts, mathtools,enumitem, stmaryrd,physics, cancel, tikz-cd, graphicx, float, booktabs}
\usetikzlibrary{arrows}
\usepackage{geometry}
    \geometry{
        a4paper,
        left = 40mm,
        top = 20mm,
        right = 40mm,
        bottom = 30mm
    }
\setlength{\parindent}{0pt}

\theoremstyle{definition}
\newtheorem{problem}{Problem}
\newtheorem{solution}{Solution}
\newtheorem*{example}{Example}
\newtheorem*{exercise}{Exercise}
\newtheorem*{definition}{Definition}
\newtheorem{theorem}{Theorem}
\newtheorem*{theorem*}{Theorem}
\newtheorem{proposition}[theorem]{Proposition}
\newtheorem*{proposition*}{Proposition}
\newtheorem{lemma}[theorem]{Lemma}
\newtheorem*{lemma*}{Lemma}
\newtheorem{corollary}[theorem]{Corollary}
\newtheorem*{corollary*}{Corollary}
\newtheorem*{remark}{Remark}

\title{M522 Topology II}
\author{Thanic Nur Samin}
\date{\vspace{-5ex}}

\begin{document}
    \maketitle

    \section*{Monday, 1/13/2025}
    
    \section*{Singular Homology and CW Complexes}

    We want to talk about the \underline{Homology} of a space \(X\).

    \begin{definition}
        [Homology] Let \(X\) be a topological space. Consider the sequence of abelian groups:

        \[
            H_0 X, H_1 X, H_2 X, \cdots 
        \]

        These are the homemorphism invariants.

    \end{definition}

    For example, consider the \(2\)-torus \(T^2\) and the \(2\)-sphere \(S^2\). They are not homeomorphic, we can see that from their fundamental groups.

    % insert

    \(H_1 T^2 = \mathbb{Z} \oplus \mathbb{Z}\).

    \(H_1 S^2 = 0\).

    \(\therefore S^2 \not\cong T^2\) 

    Do note that, even if all elements from the sequence are isomorphic the spaces might not be isomorphic!

    Some application: see Davis and Kirk ``Homology Greatest Hits''.

    Knot theory seems very intuitive but proving statements is very troublesome. For example, how do you prove that the trefoil and the unknot are not the same?

    \begin{theorem}
        [Brouwer's Fixed Point Theorem] Every \(f:D^n \to D^n\) has a fixed point.
    \end{theorem}

    \begin{theorem}
        [Euler's Formula] For every `triangulation' of \(S^2\) we have:

        \[
            v - e + f = 2 = \chi(S^2)
        \]

        \(\chi\) denotes the \underline{Euler Characteristic}.

        eg pyramid \(4-6+4=2\), triangulated bipyramid \(5-9+6 = 2\), cube \(8-12+6=2\).

        % insert
    \end{theorem}

    \begin{theorem}
        [Hairy Ball Theorem]

        % insert

        \(\not\exists f\colon S^2 \to S^2\) s.t. \(\forall x\in S^2, x \cdot f(x) = 0\).
        
        So you can't comb the hairy ball.
    \end{theorem}
    
    \begin{theorem}
        [Jordan Curve Theorem]

        % insert

        The complement of a closed curve in plane has two components.


    \end{theorem}

    \begin{theorem}
        [Brouwer's Theorem on Invariance of Domain]

        \(m \neq n \implies \mathbb{R}^m \not\cong \mathbb{R}^n\).

        Consider open \(U \subset \mathbb{R}^n\) [a domain] and let \(f\) be a continuous injection \(f\colon U \to \mathbb{R}^n\).

        Then \(f(U)\) is open in \(\mathbb{R}^n\).
    \end{theorem}

    \subsection*{Variants of Homology}

    \begin{table}[H]
        \centering
        \begin{tabular}{c|c|c}
            \toprule
                 & defined for &  \\
            \midrule
                Singular Homology &Top Spaces & Easy to define  but hard to compute \\
                simplicial homology & simplicial complexes and \(\Delta\)-complexes & Easy to define and compute, but difficult to show homeo inv. \\
                Cellular homology & CW-complexes & hard to define, easy to compute, flexible. \\
            \bottomrule
        \end{tabular}
        \caption{Variants of Homology}
    \end{table}

    \section*{Definition of Singular Homology}

    \begin{definition}
        [Standard \(n\)-simplex]

        \[
            \Delta^n = \left\{ (t_0, \cdots , t_n) \mid \sum_{i} t_i = 1, 1 \geq t_i \geq 0 \right\} \subset \mathbb{R}^{n+1}
        \]

        \[
            = \left\{ \sum_{i} t_i \underline{e}_i \mid 1 \geq t_i \geq 0, \sum_{i} t_i = 1 \right\} 
        \]

        \[
            = \text{convex hull of } \{ \underline{e_0}, \cdots , \underline{e_n} \} 
        \]

        % insert
    \end{definition}

    Recall that convex hull is the intersection of all convex sets containing the original set.

    \section*{Singular \(n\)-simplex in \(X\)}

    \(n\)-simplices are images of standard simplices under continuous maps.

    % insert

    We define singular \(n\)-chains \(S_n X\). These are free abelian groups with \(\mathbb{Z}\)-basis the singular \(n\)-simplicies in \(X\).

    A typical element will be a finite sum:

    \[
        n_1 \sigma_1 + \cdots + n_k \sigma_k \in S_n X
    \]

    Note: Davis and Kirk uses \(S_n X\), Hatcher uses \(C_n X\).

    For example, let \(X\) be the punctured plane \(X = \mathbb{R}^2 - \{ 0 \}\).

    \(\sigma_1 + \sigma_2 - \sigma_3 \in S_1 X\).

    % insert

    This is an example of a special \(1\)-chain callled the \(1\)-cycle.

    \underline{Goal}: Define a boundary map \(\partial_n : S_n X \to S_{n-1} X\) [Read Davis Kirk].
    
    Then, \(H_n\) is given by the quotient map:

    \[
        H_n = \frac{\ker \partial_n}{\operatorname{im} \partial_{n+1}} = \frac{n \text{-cycles}}{n \text{-boundaries}}
    \]

\end{document}