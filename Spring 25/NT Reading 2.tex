\documentclass{article}
\usepackage{amsmath, amsthm, amssymb, amsfonts, mathtools, mathrsfs, enumitem, stmaryrd,physics, cancel, tikz-cd, graphicx, float, booktabs}
\usetikzlibrary{arrows}
\usepackage{geometry}
    \geometry{
        a4paper,
        left = 40mm,
        top = 20mm,
        right = 40mm,
        bottom = 30mm
    }
\setlength{\parindent}{0pt}

\theoremstyle{definition}
\newtheorem{problem}{Problem}
\newtheorem{solution}{Solution}
\newtheorem*{example}{Example}
\newtheorem*{exercise}{Exercise}
\newtheorem*{definition}{Definition}
\newtheorem{theorem}{Theorem}
\newtheorem*{theorem*}{Theorem}
\newtheorem{proposition}[theorem]{Proposition}
\newtheorem*{proposition*}{Proposition}
\newtheorem{lemma}[theorem]{Lemma}
\newtheorem*{lemma*}{Lemma}
\newtheorem{corollary}[theorem]{Corollary}
\newtheorem*{corollary*}{Corollary}
\newtheorem*{remark}{Remark}

\title{NT Reading 2}
\author{Thanic Nur Samin}
\date{\vspace{-5ex}}

\begin{document}
    \maketitle

    \tableofcontents

    \setcounter{section}{8}

    \section{Thursday, 1/16/2025, Cartan Subalgebras (CSA) by Rostyslav}

    \begin{corollary}
        [15.3] Let \(L\) be semisimple. CSA's of \(L\) are precisely the maximal toral subalgebras of \(L\).
    \end{corollary}

    \begin{proof}
        \(\implies\): Let \(H\) be a maximal toral subalgebra.

        \(\implies\) \(H\)-abelian \(\implies H\)-nilpotent, \(N_L(H) = H\) since \(L = H + \bigsqcup_{\alpha\in \Phi} L_\alpha\) with \([H,L_\alpha] = L_\alpha\) for \(\alpha \in \Phi \implies H\)-CSA.
        
        \(\impliedby\): Let \(H\)-CSA. \(x = x_s + x_h\) by Jordan decomposition.

        \(\implies L_0(\operatorname{ad} x_s) \subset L_0(\operatorname{ad} x)\) for \(x\in L\) semisimple.

        \(L_0(\operatorname{ad} x_s) = C_L(x_s)\) since \(\operatorname{ad} x\) is diagonal.
        
        \(H\)- minimal Engel.

        By defintion, \(\implies L_0(\operatorname{ad} x_s) = C_L(x_s) = H\).
        
        But \(C_L(x_g)\) contains maximal toral subalgebra which is CSA. Thus it itself is minimal Engel. \(H\)-maximal toral.

        Detaills:

        \(L \subset H\). \(H\) is CSA \(\iff H\) is nilpotent, \(N_L(H)=H\).


        \(L_0(\operatorname{ad}(x_s)) \subset L_0(\operatorname{ad}(x)).\)
        
        \(L_0(\operatorname{ad}(x_s)) = \{ y\in L \mid \operatorname{ad}(x_s)(y) = 0\} = \{ y\in L \mid [x_s,y] = 0 \} \).
        
        \([x,y] = \underbrace{[x_s,y]}_{=0} + [x_n,y]\).

        \(\operatorname{ad}(x)^m(y) = \operatorname{ad}(x_n)^m(y) = 0\) for \(m \gg 0\).
        
    \end{proof}

    \begin{lemma}
        [15.4.B] Let \(\phi: L \to L^{\prime}\) [epimorphism]. Let \(H^{\prime}\) be CSA of \(L^{\prime}\). Then, any CSA of \(\phi ^{-1} (H^{\prime})\) is also a CSA of \(L\).
    \end{lemma}

    \begin{definition}
        \(x\in L\) is callled \underline{strongly ad-nilpotent} if \(\exists y\in L\) and \(\exists a\neq 0\) eigenvalue of \(\operatorname{ad} y\) such that \(x\in L_a(\operatorname{ad} y)\).

        \[
            [L_a(\operatorname{ad} y), L_b(\operatorname{ad} y)] \subset L_{a,b}(\operatorname{ad} y)
        \]
    \end{definition}

    \begin{remark}
        By \underline{Lemma 15.1} if \(x\) is strongly ad-nilpotent then \(x\) is \(\operatorname{ad}\)-nilpotent.  

        Recall: \(x\in L_a(\operatorname{ad} y)\) means \(\exists m > 0 : (\operatorname{ad} y - a \cdot \operatorname{id}_{})^m (x) = 0\).
    
        This implies that \(\operatorname{ad} x\) is nilpotent after some calculation.
    \end{remark}

    \begin{definition}
        \(\mathcal{N} (L)\) is the set of strongly ad-nilpotent elements. 
    \end{definition}

    \begin{definition}
        \(\mathscr{E} (L) < \operatorname{Int} L \coloneqq \langle \exp (\operatorname{ad} x) \mid x \text{ is ad-nilpotent}  \rangle \) generated by \(\forall \text{exp} \operatorname{ad} x\) where \(x\in \mathcal{N} (L)\).
    \end{definition}

    \begin{remark}
        \(\mathcal{N}(L)\) is stable under \(\forall x \in \operatorname{Aut}(L)\).
        
        Thus, \(\mathscr{E}(L) \trianglelefteq \operatorname{Aut}(L)\).

        \(K \subset L \implies \mathcal{N}(K) \subset \mathcal{N}(L)\).

        Then,
    \end{remark}

    \begin{definition}
        \(\mathscr{E} (L,K)\) is generated by \(\exp \operatorname{ad} x \forall x\in \mathscr{E}(K)\) 
    \end{definition}

    Then, \(\mathscr{E}(K) = \mathscr{E}(L,K)\).

    If \(\phi : L \to L^{\prime}\) is an epimorphism then \(\phi(L_a(\operatorname{ad} y)) = L^{\prime}_a(\operatorname{ad} \phi(y))\).
    
    \(\implies  \phi (\mathcal{N}(L)) = \mathcal{N}(L^{\prime})\).

    \begin{lemma}
        [16.1] Let \(\phi: L \to L^{\prime}\) be an epimorphism. If \( \sigma'^{\prime} \in \mathscr{E}(L^{\prime}) \implies \exists \sigma\in \mathscr{E}(L)\) such that:

        \begin{center}
            \begin{tikzcd}
                L \ar[r,"\phi"] \ar[d,"\sigma"] & L^\prime \ar[d,"\sigma^\prime"] \\ L \ar[r,"\phi"] & L'
            \end{tikzcd}
        \end{center}

    \end{lemma}

    \begin{proof}
        If \(\sigma ^{\prime} = \exp \operatorname{ad} x^{\prime}\) \(x^{\prime} \in \mathcal{N}(L^{\prime})\) \(\exists x\) s.t. \(\phi(x)=x^{\prime}\).
        
        \(\forall z\in L, (\phi \circ \exp \operatorname{ad}_L x)(z) = \phi(z + [x,z] + [x,[x,z]]) = \phi(z) + [x^{\prime}, \phi(z)] + [x^{\prime}, [x^{\prime},\phi(z)]]\)
        
        \(= (\exp \operatorname{ad}_L x^{\prime})(\phi(z)) \implies\) QED. 
        
    \end{proof}

    \begin{theorem}
        [16.2] Let \(L\) be solvable. Let \(H_1, H_2\) be CSA's of \(L\). 
        
        Then, \(H_1\) is conjugate with \(H_2\) by an element of \(\mathscr{E}(L)\).
    \end{theorem}

    \section{Thursday, 1/23/2025, Cartan Subalgebra (CSA) by Rostyslav}

    \begin{proof}
        Induction on \(\dim L\).

        Base case: suppose \(\dim L=1\). Since \(L\) is nilpotent \(L\) must be trivial.

        Assume that \(L\) is not nilpotent. \(L\)--solvable \(\implies\) \(L\) has non-zero abelian ideals. eg least non-zero term of the derived series. Choose such \(A\) of least possible dimension.

        Set \(L^{\prime} = L / A\). We have \(\phi: L \to L / A = L^{\prime}\) given by \(x \mapsto x^{\prime}\).

        Lemma 15.4(image of CSA is CSA) implies \(H_1^{\prime} , H_2^{\prime}\) are CSAs of the solvable algebra \(L^{\prime}\). By induction \(\exists \sigma \in \mathscr{E} (L^{\prime})\) such that \(\sigma(H_1^{\prime})=H_2^{\prime}\).

        Lemma 16.1(the commutative diagram) implies \(\exists \sigma \in \mathscr{E} (L)\) such that the diagram commutes. So, \(\sigma\) maps \(K_1 = \phi ^{-1} (H_1^{\prime})\) to \(\phi ^{-1} (H_2^{\prime}) = K_2\)
        
        But now \(H_2\) and \(\sigma(H_2)\) are both CSA''s of \(K_2\).

        If \(K_2\) is smaller than \(L\) induction allows us to find \(\tau ^{\prime} \in \mathscr{E} (K_2)\) such that \(\tau ^{\prime} \sigma (H_1)=H_2\).

        But \(\mathscr{E}(K_2)\) consists of restrictions of \(\mathscr{E}(L,K_2)\) to \(K_2\).

        \(\exists \tau\) such that \(\tau \sigma (H_1) = H_2\) for \(\tau \in \mathscr{E} (L) \implies\) done.

        Otherwise \(L = K_2 = \sigma(L_1)\).

        \(K_2 = K_1\) and \(L = H_2 + A = H_1 + A\).

        Theorem 15.3 \(\implies\) CSA \(H_2 = L_0(\operatorname{ad} x)\) for suitable \(x\in L\).

        \(A\) being \(\operatorname{ad} x\) stable, so by lemma 15.1,

        \[
            A = A_0 (\operatorname{ad} x) \oplus A_{\ast} (\operatorname{ad} x)
        \]

        and each summand is stable under \(H_2 + A\).

        Since \(A\)  is minimal, \(A = A_0 (\operatorname{ad} x)\) or \(A = A_{\ast}(\operatorname{ad} x)\)  

        \(A\) cannot be equal to \(A_0(\operatorname{ad} x)\) since in that case \(A \subset H_2, L = H_2\).

        But since \(L\) is \underline{not} nilpotennt we have a contradiction.

        Thus, \(A = A_{\ast} (\operatorname{ad} x) \implies A = L_{\ast} (\operatorname{ad} x)\). Since \(L = H_1 + A\) we can write \(x = y+z\) with \(y\in H_1, z \in A_{\ast} (\operatorname{ad} x)\).
        
        Since \(\operatorname{ad} x\) is invertible on \(L_{\ast} (\operatorname{ad} x)\) we can write \(z = [x,z^{\prime}]\) where \(z^{\prime} \in L_{\ast} (\operatorname{ad} x)\).
        
        \(A\)-abelian \(\implies (\operatorname{ad} z^{\prime})^2 = 0\).
        
        Thus, \(\exp \operatorname{ad} z^{\prime} = 1_L + \operatorname{ad} z^{\prime}\).

        Applying to \(x\) we have, \(x - z = y\).

        \(\implies H_0 = L_0(\operatorname{ad} y)\) must also be a CSA of \(L\). Since \(y\in H_1, H \supset H_1\) and both minimal Engel, \(H = H_1\).

        \(H_1\) is conjuate to \(H_2\) using \(\exp \operatorname{ad} z^{\prime}\).

        We only need to show that, \(\exp \operatorname{ad} z^{\prime} \in \mathscr{E} (L)\).

        \(z^{\prime} \) can be written as sum of strongly \(\operatorname{ad}\)-nilpotent elements of \(A = L_{\ast} (\operatorname{ad} x)\)
        
        \(\implies A\)-abelian so \(\exp \operatorname{ad} z^{\prime} = \prod \exp \operatorname{ad} z_i \in \mathscr{E} (L)\).

        So we're done.

    \end{proof}

    Consider \(B = \) upper triangular matrices. It is a lie algebra. What is a CSA of this?

    Attempt: we have \(H =\) upper triangular matrices with \(0\) diagonal. \(N_B (H) = B\). But it is not nilpotent so it doesn't work.

    However, attempt 2: we can take \(H = \) diagonal matrices.

    In fact, if \(\mathfrak{g} \subset \mathfrak{gl}_n\) is a subalgebra and \(\mathfrak{g}\) contains a diagonal matrix with all entries different, then the subalgebra \(\mathfrak{h}\) of \(\mathfrak{g}\) containing all diagonal matrices on \(\mathfrak{g}\) is a CSA.

    \begin{definition}
        A maximal solvable subalgebra of a lie algebra \(L\) is called a \underline{Borel} \underline{subalgebra}.
    \end{definition}

    \begin{lemma}
        [16.3.A] If \(B\) is a borel subalgebra of \(L\) then \(B = N_L(B)\). Aka, Borel subalgebras are self normalizing.
    \end{lemma}

    \begin{proof}
        Let \(x\in N_L(B)\). Then, \(B + Fx\) is a subalgebra of \(L\). It is solvable since \([B+Fx,B+Fx] \subset B\). Since \(B\) is maximal, we must have \(x\in B\).
    \end{proof}

    \begin{lemma}
        [16.3.B] If \(\operatorname{Rad} L \neq L\) then there is a bijection between the sets of Borel subalgebras of \(L\) and Borel subalgebras of \(L / \operatorname{Rad} L\).
    \end{lemma}

    \begin{proof}
        \(\operatorname{Rad} L\) is a solvable ideal of \(L\).

        Therefore, \(B + \operatorname{Rad} L\) is a solvable subalgebra of \(L\).

        \(\implies\) by maximality, we're done.
    \end{proof}

    \begin{definition}
        Let \(H\) be a CSA in a semisimple lie algebra \(L\), \(\Phi\) a root system of \(L\) relative to \(H\). Fix a base \(\Delta\) and a set of positive roots.

        Set \(B(\Delta) = H \sqcup_{\alpha > 0} L_\alpha\)

        And \(N(\Delta) = \sqcup_{\alpha \not> 0} L_\alpha\).

        Then \(B(\Delta)\) is the standard Borel subalgebra relative to \(H\).

        \(N(\Delta)\) is the derived algebra of \(B(\Delta)\).
    \end{definition}

    \begin{lemma}
        [16.3.C.1]

        \(N(\Delta)\) is nilpotent.

        If \(x\in L_\alpha (\alpha > 0)\) then,

        Application of \(\operatorname{ad} x\) to root vector increases the \underline{height} by at least \(1\).

        \(\implies\) decreasing central series goes to zero.

        Thus, \(B(\Delta)\) is solvable.

        Let \(K \supset B(\Delta)\). Then, \(K\) is stable under \(\operatorname{ad} H\).

        Then \(K\) must include some \(L_\alpha\) with \(\alpha < 0\).

        Thus, simple \(S_\alpha \subset K \implies K\) is not solvable.

    \end{lemma}

    Note: \(S_\alpha = \langle L_\alpha , L_{-\alpha}, H \rangle \) 

    \begin{lemma}
        [16.3.C2] All standard Borel subalgebras of \(L\) relative to \(H\) are conjugate under \(\mathscr{E} (L)\).
    \end{lemma}

    \begin{proof}
        By 14.3 the reflection \(\sigma_\alpha\) acting on \(H\) may be extended to an inner automorphism \(\tau_\alpha\) of \(L\) which is, by construction, in \(\mathscr{E}(L)\).

        \(\tau_\alpha\) would send \(B(\Delta)\) to \(B(\sigma\Delta)\).

        The \underline{Weyl Group} is generated by those reflections, so we see that \(\mathscr{E}(L)\) will act transitively on standard Borel subalgebras relative to \(H\).

    \end{proof}

    \begin{theorem}
        [16.4] The Borel subalgebras of an arbitrary Lie algebra \(L\) are all conjugate under \(\mathscr{E}(L)\).
    \end{theorem}

    We omit the proof for now.

    \begin{corollary}
        All the CSAs of lie algebra \(L\) are all conjugate under \(\mathscr{E}(L)\). 
    \end{corollary}

    \begin{proof}
        Let \(H,H^{\prime}\) be CSAs. They're nilpotent by definition. Therefore, they're solvable. 

        Therefore, \(H \subset B, H^{\prime} \subset B^{\prime}\) where \(B,B^{\prime}\) are some borel subalgebra.

        By the previous theorem, \(\exists \sigma \in \mathscr{E}(L)\) such that \(\sigma(B) = B^{\prime}\).

        Thus, \(\sigma(H)\) and \(H^{\prime}\) are CSAs of \(B^{\prime}\).

        By theorem 16.2, \(\exists \tau^{\prime} \in \mathscr{E}(B^{\prime})\) such that \(\tau ^{\prime} \sigma (H) = H^{\prime}\).

        But \(\tau ^{\prime} \) is a restriction of some \(\tau \in \mathscr{E}(L,B^{\prime}) \subset \mathscr{E}(L)\). Therefore,

        \[
            \tau \sigma (H) = H^{\prime}, \, \tau \sigma \in \mathscr{E}(L)
        \]

    \end{proof}

    \section{Thursday, 1/30/2025, Cartan Subalgebra (CSA) by Rostyslav}

    \begin{center}
        \begin{tikzcd}
            & L \ar[d,no head] & \\
            & K \ar[dl, no head] \ar[dr, no head] & \\
            C \ar[d,no head] & & C' \ar[d,no head] \\
            B\cap K \ar[rd, no head] & & B'\cap K \ar[ld, no head] \\
            & B\cap B' \ar[d, no head] \\ & N'
        \end{tikzcd}
    \end{center}

    We prove theorem 16.4.

    \begin{proof}
        \underline{First induction hypothesis} \(\dim L\) upwards.

        base: \(\dim L = 1\) is trivial.

        WLOG by lemma 16.1 and 16.3B, we can assume that \(L\) is semisimple.

        Fix a standard borel subalgebra relative to some CSA.

        Suffices to shwo that \(\forall B^{\prime}\) -other borel subalgebra is conjugate to \(B\) under \(\mathscr{E} (L)\).

        If \(B^{\prime} \cap B = B\) then \(B^{\prime} = B\) by maximality (both are borel).

        \underline{Second induction hypothesis} 

        \(\dim (B\cap B^{\prime})\) downwards for all larger dimension are conjugate.

        (1) Suppose that \(B\cap B^{\prime} \neq 0\).

        Case i: set \(N^{\prime}\) of nilpotent elements of \(B \cap B^{\prime}\) is nonzero.
        
        \(B\)-standard \(\implies N^{\prime}\) subspace derived alg of \(B \cap B^{\prime}\) consists of nilpotent elements.

        \(\implies N^{\prime}\)-ideal of \(B\cap B^{\prime}\).

        \(N^{\prime}\)-not an ideal of \(L \implies K = N_L(N)\) is proper.

        Consider the action of \(N^{\prime}\) on \(B / (B \cap B^{\prime})\) induced by \(\operatorname{ad}\) for all \(x\in N^{\prime}\) acts nilpotently on this vector space.

        Theorem 33 \(\implies \exists y\) such that \(y + (B\cap B^{\prime})\) killed by \(\forall x\in N\). ie st \([xy]\in B\cap B^{\prime} , y \notin B\cap B^{\prime}\) but \([xy]\) is also \([B,B] \implies [xy]\) is nilpotent \(\implies [xy] \in N^{\prime}\) or \(y\in N_R(N^{\prime})\). \(y\notin B\cap B^{\prime} = B\cap K\)
        
        Same way \(B\cap B^{\prime} \subsetneq B^{\prime} \cap K\). 
        
        \(B\cap K, B^{\prime} \cap K\) solvable subalgebra.

        \(C,C^{\prime}\) borel subalgebra containing them \(K\neq L\) by induction \(\exists \sigma \in \mathscr{E} (L,K) \subset \mathscr{E} (L)\) such that \(\sigma (C^{\prime})=C\).
        
        Since \(B\cap B \subsetneq C\) and \(B\cap B^{\prime} \subseteq C^{\prime}\) second induction hypothesis implies \(\exists \tau \in \mathscr{E} (L)\) such that \(\tau \sigma (C^{\prime}) \subset B\)
        
        \(B \cap \tau \sigma (B^{\prime}) \supset \tau \sigma (C^{\prime}) \cap \tau \sigma(B^{\prime}) \supset \tau \sigma(B^{\prime} \cap K) \supsetneq \tau \sigma (B\cap B^{\prime})\) 

        \(\implies\) Second induction hypothesis

        \(B\) is conjugate under \(\mathscr{E} (L)\) to \(\tau \sigma (B^{\prime})\) so we have proved case i.

        Case ii: There are no non-zero nilpotent elements in \(B\cap B^{\prime}\). 

        Then, 4.2.c and 16.3.a implies that \(B\cap B^{\prime} = T =\) toral (semisimple).
        
        \(B\) is standard: \(B(\Delta) = H + N\), \(N(\Delta) = N\).

        \([B,B] = N, T\cap N = 0\).

        Thus, \(N_B(T) = C_B(T)\).

        Let \(C\) be a a CSA of \(C_B(T)\). By (one of the) definitions of CSA we know it is self normalizing and \(N\)-nilpotent.

        Thus, \(T \subset N_{C_B(T)}(C) = C\).
        
        In \(n\in N_B(C^{\prime}) (\operatorname{ad} t)^k n = 0\). \(t \subset T \subset C\).
        
        \(\operatorname{ad} t\)-semisimple \(\implies k = 1, n \in C_B(T)\) 

        \(\operatorname{ad} t \cdot n = [t,n]\).

        \(\implies N_B(C) = N_{C_B(T)}(C) = C\).

        \(\implies C\) is self-normalizing not only in the centralizer, but also in \(B\). It is also nilpotent.

        Therefore, \(C\)  is a CSA of \(B\).

        \(C\)-maximal toral of \(L\) is conjugate under \(\mathscr{E} (B) \implies\) under \(\mathscr{E} (L)\).

        Thus, WLOG we can assume that \(T \subset H\).

        Suppose \(T = H\). Then, \(B^{\prime} \supsetneq H\).

        \(\implies B^{\prime}\) includes at least on e \(L_\alpha\) with \(\alpha < 0\) relative to \(\Delta\).
        
        \(\tau_\alpha (B^{\prime}) = B^{\prime\prime}, B^{\prime\prime} \cap B \supset H + L_\alpha \) 

        \(\implies\) second induction hypothesis

        \(B^{\prime\prime} \) is conjugate to \(B\) under \(\mathscr{E} (L)\)

        Let \(T \subsetneq H\).

        \(B^{\prime} \subset C_L(T)\)

        By first induction hypothesis we know it will have less degree.
        
        \(\dim C_L(T) < \dim L\).
        
        \(H \subset C_L(T)\) we can find a boorel subalgebra \(B^{\prime\prime}\) of \(C_L (T)\) that will contain \(H\).

        \(\implies B^{\prime}\) and \(B^{\prime\prime}\) are conjugate under \(\mathscr{E} (L, C_L(T)) \subset \mathscr{E} (L)\).

        \(B^{\prime} \subsetneq C_L(T), T = B\cap B^{\prime}\).

        We can find an eigenvector \(x\in B^{\prime}\) for \(\operatorname{ad} T\) and \(t\in T\) such that \([t,x] = ax\). \(a\in\mathbb{Q}_+\).

        \(S \coloneqq H + \sqcup_{\alpha \in \Phi} L_\alpha\) 

        \(\alpha(t) \in \mathbb{Q}_+\).

        \(S\) is subalgebra of \(L\).
        
        Similarly to llemma 16.3.c.2, \(S\) is solvable.

        \(B^{\prime\prime} \supset C\) -  Borel subalgebra.

        \(B^{\prime\prime} \cap B^{\prime}  \supset T + Fx \supsetneq T = B \cap B^{\prime}\) here \(x\) is eigenvector

        \(\dim B^{\prime\prime} \cap B^{\prime} > \dim B\cap B^{\prime}\)

        Second induction hypothesis \(\implies B^{\prime\prime}\) is conjugate to \(B\).

        Similarly, we can prove that \(B^{\prime\prime}\) is conjugate to \(B^{\prime}\)

        Thus, \(B\) is conjugate to \(B^{\prime}\).

        (2) \(B\cap B^{\prime} = 0\)

        Then \(\dim L \geq \dim B + \dim B^{\prime}\).

        \(B\) is standard so \(\dim B \geq \frac{1}{2}\dim L\).

        Let \(T\)-maximal toral subalgebra of \(B^{\prime}\).

        Assume \(T=0 \implies B\) only has nilpotent elements \(\implies\) by Engel's theorem, \(B\) has to be nilpotent.

        By lemma 16.3.A, since \(B\) is borel, it is self normalizing \(B = N_L (B)\).

        Then \(B\) is a CSA of \(L\).

        15.3 \(\implies\) all of CSAs of \(L\) are toral.

        Being toral and nilpotent is a contradiction.

        Thus, \(T\neq 0\).

        \(T \subset H_0\)-maximal toral of \(L\)

        \(\implies B \cap B^{\prime\prime} \neq 0 \implies B^{\prime} \) is conjugate to \(B\).

        \(\dim B^{\prime} = \dim B^{\prime\prime} > \frac{1}{2} \dim L \implies\) we have a contradiction.
    \end{proof}

    There's a relationship between \(\mathscr{E}(L)\) and the inner automorphisms.

    Suppose \(L\) is semisimple lie algebra, \(H\)-CSA of \(L\), \(\Delta\)-base, \(\Phi\)-root system

    Let \(\tau \in \operatorname{Aut} L\). We can see that \(\tau (B)\) is conjugate to \(B\) by \(\sigma_1 \in \mathscr{E} (L)\).

    We can find \(\sigma_2 \in \mathscr{E} (L,B) \subset \mathscr{E} (L)\) that sends \(\sigma_1 \tau (H)\) to \(H\) by 16.2.
    
    \(\sigma_2 \sigma_1 \tau\) preserves \(H\) and \(B\) \(\implies\) it induces an automorphism on \(\Phi\). Leave \(\Delta\) invariant.

    Let \(\rho\) be such automorphism. It is not unique, \underline{but} \(\rho \sigma_2 \sigma_1 \tau (x_\alpha) = c_\alpha x_\alpha\) [\(\alpha > 0\)], \(\rho \sigma_2 \sigma_1 \tau (y_\alpha) = c_\alpha ^{-1} y_\alpha\)

    \(\rho \sigma_2 \sigma_1 \tau (h_\alpha) = h_\alpha\)

    \(\tau\) differes from \(\mathscr{E} (L) \cdot \Gamma(L)\) by a diagonal automorphism.
    
    And diagonal automorphisms are inner automorphisms.

    Therefore, \(\operatorname{Aut} (L) = \operatorname{Inn} (L) . \Gamma(L)\). 

    Note: in the semidirect case, \(\mathscr{E} (L) = \operatorname{Inn} (L)\).

    \section{Thursday, 2/6/2025, Universal Enveloping Algebra by Hechi}

    Notation:

    \(\mathbb{F} =\) field, \(\mathcal{L} / \mathbb{F}\) lie algebra, \(V / \mathbb{F}\) vector space.

    \begin{definition}
        [Tensor Algebra] \(T^m V = V^{\otimes m}\).

        \[
            T^{0} V = \mathbb{F} ,T^1 V = V,
        \]

        \[
            T^m V = \underbrace{V \otimes \cdots \otimes V}_{m \text{ copies}} 
        \]

        \(T(V) = \coprod_{i=0}^{\infty} T^i V\) 

        Multiplication by Tensor Product.

    \end{definition}

    Universal Property:

    \begin{center}
        \begin{tikzcd}
            V \ar[r,"i"] \ar[rd, "\rho"] & T(V) \ar[d,dotted] \\ & A
        \end{tikzcd}
    \end{center}

    \(A\) is an \(\mathbb{F}\)-algebra with \(1\).

    Symmetric Algebra:

    Let \(I \subset T(V)\) be the two-sided ideal generated by all elements of the form:

    \[
        x \otimes y - y \otimes x, x,y\in V
    \]

    We define \(S(V) = T(V) / I\).

    We can write:

    \[
        S(V) = \coprod_{i=0}^{\infty} S^i V
    \]

    By writing it as a direct sum of terms of a degree.

    \(S^0 V = \mathbb{F}, S^1 V = V\) [since \(I\) contains \(\deg \geq 2\) terms, they remain unchanged].

    We can wrie:

    \[
        I = \coprod_{i=2}^{\infty} I^i, I^i = I \cap T^i
    \]

    This also enjoys a Universal Property: when \(A\) is commutative:

    \begin{center}
        \begin{tikzcd}
            T \ar[r,"i"] \ar[rd,"\rho"] & T(V) \ar[d,dotted] \\ & A
        \end{tikzcd}
    \end{center}
    
    If \(V\) is a finite dimensional vector space then \(S(V)\) are \underline{polynomials}!

    \begin{definition}
        [Universal Enveloping Algebra]
        An universal enveloping algebra is the pair \((U,i)\) where \(U\) is an algebra and \(i: \mathcal{L} \to U\) such that:

        \[
            i([xy]) = i(x) i(y) - i(y) i(x) \tag*{\(\ast\)}
        \]

        We also must have a universal property: suppose \(A\) is an \(\mathbb{F}\)-algebra with \(1\). Then, if \(j: \mathcal{L} \to A\) satisfies \((\ast)\) then,

        \begin{center}
            \begin{tikzcd}
                \mathcal{L} \ar[r,"i"] \ar[rd,"j"] & U \ar[d,"\phi"] \\ & A
            \end{tikzcd}
        \end{center}

        From this we also know that \((U,i)\) is unique upto isomorphism.

    \end{definition}

    \subsection*{Construction of the Universal Enveloping Algebra}

    Let \(J \subset T(\mathcal{L})\) be the \(2\)-sided ideal generated by elements \(x \otimes y - y \otimes x - [xy]\).
    
    Define \(U(\mathcal{L}) \coloneqq T(\mathcal{L}) / J\). Then \(U(L)\) satisfies \((\ast)\).

    But this is not very explicit. We can explicitly construct it using the PBW theorem.

    From now on let \(T\coloneqq T(\mathcal{L}), S \coloneqq S(\mathcal{L}), U = U(\mathcal{L})\). We have a canonical projection \(\pi: T \to U\).
    
    Let \(T^m = T^m \mathcal{L}, S^m = S^m \mathcal{L}\).

    We also define the following filtrations:

    \[
        T_m = T^0 \oplus \cdots \oplus T^m
    \]

    \[
        U_m = \pi(T_m) \subset U, U_{-1} = 0
    \]

    Facts: \(U_m U_p \subset U_{m+p}\). \(U_m \subset U_{m+1}\).
    
    Thus it makes sense to define \(G^m = U_m / U_{m-1}\). This is a \(\mathbb{F}\)-vector space.

    The multiplication on \(U\) induces a well defined map:

    \[
        G^m \times G^p \to G^{m+p}
    \]

    Since lower degree terms just become \(0\).

    We can extend this to \(G = \coprod_{i=0}^{\infty}G^i\)

    Then, we have multiplication on \(G\):

    \[
        G \otimes G \to G
    \]

    This gives \(G\) an \(\mathbb{F}\)-algebra structure. The algebra is abelian.

    We can define the map:

    \[
        \phi_m : T^m \to U_m \to G^m
    \]

    Combining all the \(m\), we have a surjective homomorphism:

    \[
        \phi: T \twoheadrightarrow G
    \]

    \begin{lemma}
        [17.3]

        \(\phi(I) = 0\) and therefore \(\phi\) induces a surjective map \(\omega: S \twoheadrightarrow G\).
    \end{lemma}

    \begin{proof}
        By definition of \(J\),

        \[
            \pi (\underbrace{x \otimes y - y \otimes x}_{\deg 2}) = \pi(\underbrace{[xy]}_{\deg 1})
        \]

        Thus, \(\phi(x \otimes y - y \otimes x) \in U_1 / U_1 \subseteq U_2 / U_1\) thus \(\phi(x \otimes y - y \otimes x) = 0\).

        Hence the result.
    \end{proof}

    \begin{theorem}
        [Poincar\'e-Birkhoff-Witt, 17.3] \(\omega: S \xrightarrow{\cong} G\).
    \end{theorem}

    \begin{corollary}
        [17.3A] We want to give a basis of \(U\). If \(W \subset T^m\) is a subspace such that \(T^m \to S^m\) sends \(W \xrightarrow{\cong} S^m\) then \(\pi(W)\) is a complement of \(U_{m-1}\) in \(U_m\)
    \end{corollary}

    \begin{proof}

        The following diagram commutes by construction

        \begin{center}
            \begin{tikzcd}
                & U_m \ar[rd, two heads] & \\
                T^m \ar[ru, "\pi_m"] \ar[rd] \ar[d,"\cup"] \ar[rr, bend left=120, "\phi_m"] & & G^m \\ W \ar[rru, bend right=60, "\cong"] \ar[r,"\cong"] & S^m \ar[ru,"\cong"] &
            \end{tikzcd}
        \end{center}

        Bottom map gives \(W\xrightarrow{\cong} G^m\). Top map, then \(\phi_m(W)\) must be complement of \(U_{m-1}\)
    \end{proof}

    \begin{corollary}
        [17.3B] \(i: \mathcal{L} \to U\) is injective.
    \end{corollary}

    \begin{proof}
        Take \(W = T^1\).
    \end{proof}

    \begin{corollary}
        [17.3C] This is traditionally known as the PBW theorem. Assume \(\mathcal{L}\) has a countable basis \((x_1, \cdots)\). Then,

        \(\{x_{\sigma(1)}\cdots x_{\sigma(m)}\}, m\in \mathbb{Z}_{\geq 0}, \sigma\) permutation so that \(\sigma(1) \leq  \sigma(2) \leq \cdots \leq \sigma(m)\) is a basis for \(U\).
    \end{corollary}

    \begin{proof}
        Let \(W = \operatorname{span}\left\{ x_{\xi(1)} \otimes \cdots \otimes x_{\sigma(m)}: \sigma(1) \leq  \cdots \leq \sigma(m) \right\} \subseteq T^m\).
        
        Clearly \(W \xrightarrow{\cong} S^m\). Then we use corollary 17.3A. 
    \end{proof}

    \begin{proof}
        [Proof of PBW]

        By well-ordering-principle we can define \((\chi_\lambda, \lambda \in \Omega)\) be an ordered basis of \(\mathcal{L}\). This gives isomorphism \(S \cong\mathbb{F}[z_{\lambda}]_{\lambda \in \Omega}\) where \(z_\lambda\) are just variables indexed by \(\Omega\).

        Let \(\Sigma = (\lambda_1, \cdots , \lambda_m)\) index with length \(m\). Then \(z_\Sigma = z_{\lambda_1} \cdots z_{\lambda_m} \in S^m\).

        \(x_\sigma = x_{\lambda_1} \otimes \cdots \otimes x_{\lambda_m} \in T^m\).

        We say \(\Sigma\) is increasing if \(\lambda_1 \leq \cdots \leq \lambda_m\) or \(\varnothing\).
        
        We define \(z_{\varnothing} = 1\).

        \(\left\{ z_\Sigma : \Sigma \text{ is increasing}  \right\}\) is a basis of \(S\).

        We say \(\lambda \leq \Sigma\) if \(\lambda \leq \mu \forall \mu \in \Sigma\).

        The idea is to give \(S\) a structure of \(L\)-module.

        \begin{lemma}
            [17.4A] Fix \(m\in \mathbb{Z}_{>0}\). Then there exists a unique linear map \(f_m: \mathcal{L} \otimes S_m \to S\) with the following properties:

            \(A_m\): \(f_m(\chi_\lambda \otimes z_\Sigma) = z_\lambda z_\Sigma \forall \lambda \leq \Sigma , z_\sigma \in S^m\).

            \(B_m\): \(f_m(\chi_\lambda \otimes z_\Sigma)- z_\lambda z_\Sigma \in S_k \forall k \leq m, z_\Sigma \in S_k\)

            \(C_m\): \(f_m(\chi_\lambda \otimes f_m(x_\mu \otimes z_{\tau} )) = f_m(x_\mu \otimes f_m(\chi_\lambda \otimes z_{\tau})) + f_m([\chi_\lambda x_\mu] \otimes z_{\tau}) \forall z_{\tau} \in S_{m-1}\).
        \end{lemma}

        \underline{Proof of Lemma}: induction.

        \begin{lemma}
            [17.4B] \(\exists\) representation \(\rho : \mathcal{L} \to \mathfrak{gl}(S)\) with:

            \begin{enumerate}[label=\alph*)]
                \item \(\rho(\chi_\lambda) z_\Sigma = z_\lambda z_\sigma, \forall k\lambda \leq \Sigma\)
                \item similar
                \item similar 
            \end{enumerate} 
        \end{lemma}

        \underline{Proof of Lemma}: Comining \(A_m, B_m, C_m\) for all \(m\).

        \begin{lemma}
            [17.4C]

            Let \(t \in T_m \cap J\), then \(t_m\) (homogeneous degree \(m\) part of \(t\)) is in \(I\).
        \end{lemma}

        \underline{Proof of Lemma}: Can be seen from the commutative diagram:
        
        \begin{center}
            \begin{tikzcd}
                \mathcal{L} \ar[r,"\rho"] \ar[rd] \ar[rdd] & \mathfrak{gl}(S) \\ & U\ar[u,dotted] \\ & T\ar[u]
            \end{tikzcd}
        \end{center}

        \(J \subset \ker (T \to \mathfrak{gl}(S))\) so \(\rho(t) = 0\). Then we can write \(t_m\) as linear combination of \(x_\Sigma, \Sigma\) has length \(m\).
        
        Then look at the highest deggree term of \(\rho(t)\). It is a linear combination of \(z_\Sigma\). But this term is \(0\). So \(t_m \in I\).

        We're done with all the lemma. We finally prove PBW.

        Apply 17.4C to \(t - t^{\prime}\). 
        
        Let \(t \in T^m\) such that \(\pi (t) \in U_{m-1}\).

        Then \(\exists t^{\prime}\) soo that \(\pi(t) = \pi(t^{\prime})\). Note that \(t^{\prime}\) has degree strictly smaller than \(t\).
        
        Then, \(t-t^{\prime} \in J\).

        Thus it satisfies the condition of 17.4C.

        Therefore the \(m\)-degree part of \(t - t^{\prime}\) must be \(t\) [since \(t\in T^m\)].

        Thus, by C, \(t\in I\).

    \end{proof}

    \section{Thursday, 2/13/2025, Universal Enveloping Algebra by Hechi}

    \begin{definition}
        Given a set \(X\) a lie algebra \(J\) is \underline{free on \(X\)} if:

        \begin{center}
            \begin{tikzcd}
                X \ar[r] \ar[d, hook] & J \\ L \ar[ur,"\exists!",dotted]
            \end{tikzcd}
        \end{center}
    \end{definition}
    
    Construction: Let \(V\) be the vector space with a basis labeled by \(X\), so \(V =\mathbb{F}^{(X)}\).

    Let \(T(V)\) be the tensor algebra. Then \(\mathcal{L}\) is the lie subalgebra generated by \(X\).

    \subsection*{Serre's Theorem}

    Let \(\mathcal{L}\) be a lie algebra over an algebraically closed characteristic \(0\) field. \(\mathcal{L} , H, \Delta = \{ \alpha_1, \cdots , \alpha_l \}\). \(x_i\in \mathcal{L}_{\alpha_i}, y_i \in \mathcal{L}_{\alpha_i}, h_i \in H\) where,
      
    \(S_1\): \([h_i h_j] = 0\) 

    \(S_2\): \([x_i y_i] = h_i, [x_i y_j] = 0, i \neq j\) 
    
    \(S_3\): \([h_i x_j] = \langle ;\alpha_j. \alpha_i \rangle x_j, [h_i y_j] = - \langle \alpha_j, \alpha_i \rangle u_j\) 

    \(S_{ij}^+, S_{ij}^-\) 

    \begin{theorem}
        [Serre's Theorem] Given root system \(\Phi\) with basis \(\Delta\), there exists \(\mathcal{L}\) generated by \(\{ x_i, y_i, h_i \}\) satisfying \(S_1, S_2, S_3, S_{ij}^\pm\) and \(\mathcal{L}\) has root system \(\Phi\). (\(\mathcal{L}\) has CSA \(H\) etc).
    \end{theorem}

    Idea: use the free Lie algebra and quotient by relations.

    By the classification of root systems, there are only:
    
    \(A_l (l \geq 1)\)
    
    \(B_l (l\geq 2)\)
    
    \(C_l (l\geq 3)\)
    
    \(D_l(l \geq 4)\)
    
    \(E_6\)
    
    \(E_7\)
    
    \(E_8\)

    \(F_4\)
    
    \(G_2\) 

    The first four are given in the first chapter.

    Only thing needed to show is that they are semisimple!

    From now we assume \(\dim L < \infty\).

    \begin{theorem}
        \(L\) is called \underline{reductive} if \(\operatorname{rad}(L) = Z(L)\). If \(L\) is reductive then,

        \[
            L = [LL] \oplus Z(L)
        \]

    \end{theorem}

    \begin{proof}
        Note that \([L / Z(L), L / Z(L)] = L / Z(L)\) since \(Z(L) = \operatorname{rad}(L)\). Thus, \([LL]\) maps surjectively to \(L / Z(L)\) by the canonical projection.

        We also know that \(L^{\prime} = L / Z(L)\) acts on \(L\) by the adjoint action. Thus, \(L = M \oplus Z(L)\) for some \(M\). We want to show that \(M = [LL]\).

        \([LL] = [M \oplus Z(L), M \oplus Z(L)] = [MM] \subseteq M\). But \([LL]\) maps surjectively to \(L / Z(L)\), which isn't possible if it is properly contained.

        Therefore, \(M = [LL]\). Thus we have the isomorphism.
    \end{proof}

    \begin{theorem}
        If \(L \subseteq \mathfrak{gl}(V)\) and \(L\) acts irreducibly on \(V\), then \(L\) is reductive and \(\dim Z(L) \leq 1\).        
    \end{theorem}

    \begin{corollary}
        
    \end{corollary}If moreover \(L \subseteq \mathfrak{sl}(V)\) then \(L\) is semisimple.

    \begin{definition}
        Suppose \(L\) is semisimple. \(L\) is called \underline{simply-laced} if it's Dynkin diagram has only simple edges.

        This is equivalent to, if \(\alpha, \beta \in \Delta : \langle \alpha. \beta^\wedge \rangle = 2 \frac{(\alpha, \beta)}{(\beta, \beta)} \in \{ 0,-1,-2 \}\).

        Then, the simply laced are: \(A_l, D_l, E_6, E_7, E_8\).

        The letters here are \(A, D, E\). This gives us the `A-D-E' classification which is very common in algebra.

        
        
    \end{definition}

    \section{Thursday, 2/27/2025, Representation Theory by Zoia}
    
    Notation:

    \(L\)-Semisimple Lie Algebra over \(F\)-algebraically closed field of char \(0\).

    \(H\)-fixed CSA of \(L\)
    
    \(H^{\ast}\)-dual space of the CSA \(H\).

    \(\Phi\): the root system, \(\Delta = \{ \alpha_1, \cdots , \alpha_l \}\), base of \(\Phi\).
    
    \(\mathcal{W}\)- the Weyl Group.

    Weight Spaces:

    \(V\)- finite dim \(L\)-module.

    \(H\) acts diagonally on \(\).

    \(V = \bigoplus_\lambda V_\lambda\) where \(\lambda\) runs over \(H^{\ast}\).

    \(V_\lambda = \left\{ v\in V \mid h \cdot v = \lambda(h)v \forall h \in H \right\}\).

    If \(V_\lambda \neq 0\) we call \(V_\lambda\) a weight space. \(\lambda\) is the weight of \(H\) on \(V\).

    \(V^{\prime}\)- the sum of all weight spaces \(V_\lambda\) [always direct].

    Examples:

    \begin{enumerate}[label=\arabic*)]
        \item Consider \(L\) as an \(L\)-submodule via adjoint representation.
        
        Weights are the roots of \(\alpha \in \Phi \) with weight spaces \(L_\alpha\) of dim \(1\).

        If \(L = \mathfrak{sl}(2,F)\) then a linear functional \(\lambda\) on \(H\) is completely defined by \(\lambda(h)\) at the basis vector \(h\).


    \end{enumerate} 

    Exercise 1: If \(V\) is an arbitrary \(L\)-module, then the sum of its weight space is direct.

    \begin{lemma}
        Let \(V\) be an arbitrary \(L\)-module. Then,

        \begin{enumerate}[label=\alph*)]
            \item \(L_\alpha\) maps \(V_\lambda\) into \(V_{\lambda + \alpha}\) with \(\lambda \in H^{\ast} , \alpha \in \varphi\).
            
            \item The sum \(V^{\prime} = \sum_{\lambda \in H^{\ast}} V_\lambda\) is diret, and \(V^{\prime}\) is an \(L\)-submodule of \(V\).
            
            \item If \(\dim V < \infty\) then \(V = V^{\prime}\) [ex1]
            
            \item If \(x\in L_\alpha, v \in V_\lambda, h \in H\),
            \[
                h\cdot \times \cdot v = x \cdot h \cdot v + [hx] \cdot v = (\lambda(h) + \alpha(h)) xv
            \]

            Thus, \(L_\alpha\) sends \(V_\lambda\) to \(V_{\lambda+\alpha}\).

            \item \(L_\alpha, \alpha \in \Phi\) permutes the weight spaes.
        \end{enumerate} 
    \end{lemma}
    
    \subsection*{Standard Cyclic Modules}

    \begin{definition}
        A maximal vector (of weight \(\lambda\)) in an \(L\)-module \(V\) is a nonzero \(v^+\in V_\lambda\) is killed by all \(L_\alpha (\alpha \in \Phi^+), \alpha \in \Delta\).

        Meaning, \(x_\alpha v^+ = 0\, \forall x_\alpha \in L_\alpha\) 
    \end{definition}
    
    Example: If \(L\) is simple and \(\beta\) is the maximal root in \(\Phi\) relative to \(\Delta\) then any nonzero element from \(L_\beta\) is a \underline{maximal vector} (for the adjoint representation of \(L\)).

    Note that the following fact about the Borel Subalgebra

    \[
        B(\Delta) = H + \bigoplus_{\alpha \succ 0} L_\alpha 
    \]

    has a common eigenvector: which is a maximal vector.

    \begin{definition}
        If \(V = \mathcal{U}(L) \cdot v^+\) for \(v^+\) maximal vector of weight \(\lambda\) then we say \(V\) is standard cyclic.

        Then \(\lambda\) is the highest weight on \(V\).
    \end{definition}

    Example: If \(V\) is an arbitrary \(A\)-module, then the sum of its weight spaces is direct.
    
    Structure of such a submodule:

    Fix nonzero \(x_\alpha \in L_\alpha, \alpha \succ 0\) and choose \(y_\alpha \in L_\alpha\) uniquely so that \([x_\alpha y_\alpha] = h_\alpha\).
    
    Recall we have a partial order:

    \(\lambda \succ \mu \iff \lambda - \mu\) is a sum of positive roots.

    \begin{theorem}
        Consider \(V, v^+\in V_\lambda, \Phi^+ = \{ \beta_1, \cdots , \beta_m \}\). Then,

        \begin{enumerate}[label=\alph*)]
            \item \(V\) is spanned by vectors \(y_{\beta_1}^{i_1} \cdots y_{\beta_m}^{i_m} \cdot v^+\) where \(i_j \in \mathbb{Z}_{\geq 0}\).
            \item The weights of of \(V\) are of the form:
            
            \[
                \mu = \lambda - \sum_{i=1}^l k_i \alpha_i \, \lambda - \mu \succ 0 \forall \lambda
            \]

            \item \(\forall \mu \in H^{\ast}, \dim V_\mu < \infty\) and \(\dim V_\lambda = 1\).
            
            \item Each submodule of \(V\) is the direct sum of its weight spaces.
            \item \(V\) is indecomposable with a unique maximal submodule and unique irreducible quotient.
            \item Every nonzero homo-c image of \(V\) is also standard cyclic of weight \(\lambda\).
        \end{enumerate} 
    \end{theorem}

    \begin{proof}
        \begin{enumerate}[label=\alph*)]
            \item \(L = B(\Delta) + \bigoplus_{\alpha \prec 0} L_\alpha\). Then by the PBW theorem, it holds that \(\mathcal{U}(L)\cdot v^+ = \mathcal{U} \left( \bigoplus_{\alpha \prec 0} L_{\alpha} \right) \mathcal{U} \left( B(\Delta) \right) \cdot v^+ = \mathcal{U} \left( \bigoplus_{\alpha \prec 0} L_\alpha \right) \cdot F v^+\) [recall \(v^+\) is a common eigenvector for \(B\)].
            \item Consider the vector \(y_{\beta_1}^{i_1} \cdots y_{\beta_m}^{i_m} \cdot v^+\) has weight \(\lambda - \sum i_j \beta_j\) Rewrite \(\beta\) as  a non-negative \(\mathbb{Z}\)-linear combination of simple roots.
            \item \(\exists\) only finite number of polynomials \(\sum i_j \beta_j\) from part b. These span the weight space \(V_\mu\), if \(\mu = \lambda - \sum_{i} k_i \alpha_i\). The only vector of the form \(\sum_{j} i_j \beta_j\) which has weight \(\mu = \lambda\) is \(v^+\).
            \item Let \(W\) be a submodule of \(V\) and \(w\in W\) such that \(w\) is a sum of \(v_i\) such that \(v_i \in V_{\mu_i}\) and all the weights \(\mu_i\) are distinct. We want to show that all \(v_i \in W\).
            
            Let \(w = v_1 + \cdots + v_n\). We apply contradiction. WLOG suppose there exists \(v_2 \notin W\).
            
            Suppose \(h\in H\) such that \(\mu_1(h)\neq \mu_2(h)\).

            \(h \cdot w=\sum_{i} \mu_i(h)v_i \in W \implies (h - \mu_1(h)\cdot L)\cdot w=(\mu_2(h)-\mu_1(h))v_2 + \cdots  + (\mu_n(h) - \mu_1(h))\cdot v_n \neq 0\)
            
            Then \(v_2 \in W\). Contradiction.

            \item By c and d each proper submodule of \(V\) lies in the sum of lies in the sum of weight spaces other than \(V_\lambda\). Then the sum \(W\) of all submodule is still proper.
            
            Thus \(V\) has a unique maximal submodule and unique irreducible quotient.

            Thus, \(V\) cannot be the direct sum of two proper submodules since each of them are contained in \(W\).
        \end{enumerate} 
    \end{proof}

    \section{Thursday, 3/13/2025, Representation Theory by Zoia}

    \subsection*{Existence and Uniqueness of Standard Cyclic Modules}

    WTS: \(\forall \lambda \in H^{\ast}, \exists !\) irreducible cylic \(L\)-module of highest weight \(\lambda\) [which can be infinite dimensional].
    
    \begin{theorem}
        Let \(V,W\) be standard cyclic modules of highest weight \(\lambda\). If \(V,W\) are irreducible, then they are isomorphic.
    \end{theorem}
    
    \begin{proof}
        Consider their sum: the \(L\)-module \(X\) such that \(X = V \oplus W\) such that \(v^+\) and \(w^+\) represent the maximal vectors if weight \(\lambda\) in \(V,W\) respectively.

        Let \(x^+=(v^+,w^+)\in X\). Then, \(x^+\) is the maximal vector of weight \(\lambda\).

        Let's consider \(Y\) the \(L\)-submodule of \(X\) generated by \(x^+\). Then \(Y\) is standard cyclic as well.

        Let's consider the projection maps \(P: Y \to V, P^{\prime}: Y \to W\).

        In this case, \(P, P^{\prime}\) are \(L\)-module homomorphisms.

        Since \(P(x^+)=v^+, P^{\prime}(x^+)=w^+\) we can conclude that \(\operatorname{im} P = V, \operatorname{im} P^{\prime} = W\).

        Since \(V,W\) are irreducible quotients of a standard cyclic module \(Y\), \(V\) and \(W\) must be isomorphic by the previous theorem.
    \end{proof}

    \subsection*{Construction of \(Z(\lambda)\) by generators and relations}

    Also sometimes called Verma Modules.

    Note that construction directly proves existence!

    We define:

    \[
        Z(\lambda) = U(L) \otimes_{U(B)} D_\lambda
    \]

    where \(D_\lambda\) is a one dimensional vector space having \(v^+\) as basis, and define action of \(B\) on \(D_\lambda\) by \((h+\sum_{\alpha \succ 0} X_\alpha)\cdot v^+ = h \cdot v^+ = \lambda(h)v^+\) for fixed \(\lambda \in H^{\ast}\).  

    Choose nonzero element \(x_\alpha \in L_\alpha\) where \(\alpha \prec 0\).

    Let \(I(\lambda)\) be the left ideal in \(U(L)\) generated by all such \(x_\alpha\) along with \(h_{\alpha} - \lambda(h_\alpha)\cdot 1\) \((\alpha \in \Phi^+)\).
    
    These generators annihilate \(v^+\) of \(Z(\lambda) \implies I(\lambda)\) also annihilates \(Z(\lambda)\).

    \(\exists\) a canonical homomorphism of left \(U(L)\)-modules \(U(L) / I(\lambda) \to Z(\lambda)\) sending the coset of \(1\) onto \(v^+\).

    Meaning, \(\overline{W} = W + I(\lambda) \mapsto W \otimes 1\).

    PBW basis of \(U(L)\implies\) we can see that this map sends the cosets of \(U*B()\) onto \(Fv^+ \implies \) this is one-to-one.

    Thus, \(Z(\lambda) \cong U(L) / I(\lambda)\).

    \begin{theorem}
        [Existence] Let \(\lambda \in H^{\ast}\) then \(\exists\) an irreducible standard cyclic module \(V(\lambda)\) of weight \(\lambda\).
    \end{theorem}

    \begin{proof}
        \(Z(\lambda)\) is standard cyclic of weight \(\lambda\) and has a unique maximal submodule \(Y(\lambda)\), and by the theorem in previous recitation. Then \(V(\lambda) \equiv  Z(\lambda) / Y(\lambda)\) is an irreducible and standard cyclic module of weight \(\lambda\).
    \end{proof}

    \subsection*{Finite Dimensional Moddules}

    Necessary Conditions for finite dimensiona:

    Let \(V\) be a finite dimensional irreducible \(L\)-module. Thus, \(V\cong V(\lambda)\).

    \(\forall \alpha_i\) set \(s_i (s_{\alpha_i})\) be the corresponding \(\mathfrak{sl}(2,F)\) in \(L\). Then, \(s_i = L_{\alpha_i} \oplus L_{-\alpha_i} \oplus [L_{\alpha_i}, L_{-\alpha_i}]\). We have \(x_i \in L_{\alpha_i}, y_i \in L_{-\alpha_i}\) and set \(h_i = [x_i, y_i]\). Then \([h_i, x_i] = 2x_i, [h_i, y_i]=-2y_i\). This is how we get the copy: this subalgebra is isomorphic to \(\mathfrak{sl}(2,F)\).
    
    Then, \(\lambda(h_i)\) determines completely \(H_i \subset s_i\). By the theorem from 7.2, \(\lambda(h_i)\) is a non-negative integer.
    
    \begin{theorem}
        If \(V\) is a finite dimensional irreducible \(L\)-module of highest weight \(\lambda\) then \(\lambda(h_i)\) is a non-negative integer \(1 \leq i \leq l\).
    \end{theorem}

    \section{Thursday, 3/27/2025, Representation Theory by Zoia}
    
    Sufficient Condition for Finite Dimension

    \begin{lemma}
        Fix standard generators \(\{ x_i, y_i \} \) of \(L\). Then the identities hold in \(\mathcal{U}(L)\) for \(k \geq 0; i \leq i,j \leq l\).

        \begin{enumerate}[label=\alph*)]
            \item \([x_j, y_i^{k+1}] = 0\) if \(i \neq j\)
            \item \([h_j, y_i^{k+1}] = -(k+1) \alpha_i (h_j) y_i^{k+1}\)
            \item \([x_i, y_j^{k+1}] = -(k+1)y_i^k(kj-h_{i})\)  
        \end{enumerate} 
    \end{lemma}

    \begin{proof}
        a is by 10.1 \(\alpha_j - \alpha_i\) not a root.

        b, c induction.
    \end{proof}

    \begin{theorem}
        If \(\lambda \in H^{\ast}\) is dominant integral then the irreducible \(L\)-module \(V = V(\lambda)\) is finite dimensional, and its set of weights \(\Pi(\lambda)\) is permuted by \(W\), the weyl group with \(\dim V_\mu = \dim V_{\sigma \mu}\) for \(\sigma \in W\).
    \end{theorem}

    \begin{proof}
        Denote by \(\varphi : L \to gl(V)\). Fix a maximal vector \(v^+\) of \(V\) of weight \(\lambda\) and \(m_i = \lambda(h_i), 1 \leq i \leq l\)

        1. WTS: \(y_i^{m_i + 1} . v^+ = 0\).

        Denote \(\omega = y_i^{m_i + 1} \cdot v^+\).

        \(x_j \cdot \omega = 0\) 

        \(x_i y_i^{m+1} \cdot v^+ = y_i^{m_i+1} x_i v^+\)

        \((m_i + D y_i^{m_i}) \cdot (m_i v^+ - m_i v^+) = 0\).
        
        \(\implies x_i \omega = 0\). If \(\omega \neq 0\) then \(\omega\) is a max vector in \(V\) with weight of \(\lambda - (m_i + 1) \alpha_i \neq \lambda\)

        \(\implies\) contradition in 20.2.

        \(\implies \omega = 0\) 

        2. For \(1 \leftarrow i \leq l\) \(V\) contains a nonzero fin. dimensional \(S_i \) module. The subspace spanned by \(v^+, y_i v^+, y_i^{e_i} v^+, y_i^{m_i}v^+\) is stable under \(y_i\) according to 1. It is also stable under \(h_i\) sine each generator belongs to a weighr space of \(V\). Thus, it is stable under \(h_i\) since ech belongs to a weight space of \(V\). By c it is stable under \(x_i\).
        
        3. \(V\) is the sum of finite dimensional \(S_i\)-submmodules. Denote by \(V^{\prime}\) the sum of all other submodules of \(V\).

        by 2 \(V^{\prime}\) is nonzero. On the other hand, let \(W\) be any finite dimensional \(S_i\)-module of \(V\). The span of all subspaces \(x_\alpha (\alpha \in \Phi)\) is finite dimensional and \(S_i\) is stable under \(L\).
        
        Since \(V\) is stable under \(L\), \(V^{\prime} = V\) by irreducibility.

        4. For \(1 \leq i \leq l\), \(\varphi (x_i)\) and \(\varphi(y_i)\) are locally nilpotent endomorphim of \(V\).
        
        If \(v\in V \implies \delta\) is in finite sum of fin dim \(S_i\) submodules, \(\varphi(x_i)\) and \(\varphi(y_i)\) are nilpotent.

        5. \(S_i = \exp \varphi (x_i), \exp \varphi (-y_i) \exp \varphi(x_i)\) is a well defined automorphism of \(V\) by 4.
        
        6. If \(\mu\) is any weight of \(V\) then \(S_i(V_\mu) = V_{\sigma \mu}\) where \(\sigma_i\) is reflect by \(d_i\). \(V_\mu\) lies in a fin. dim \(S_i\)-submodule \(V^{\prime}\) and \(\eval{S_i}_{V^{\prime}}\) is the same as automorphism \(\tau\) from 7.2.

        7. The set of weights \(\Pi(\lambda)\) is stable under \(W\) and \(\dim V_\mu = \dim V_{\sigma \mu} [\mu \in \Pi (\lambda), \sigma \in W]\).
        
        Since \(W\) is generated by \(\sigma_1 , \cdots , \sigma_L\) this follows from 6.

        8. \(\Pi(\lambda)\) is finite \(\implies\) the set of \(W\)-conjugates of all dominant integral linear functions \(\mu \prec \lambda\) is finite and \(\Pi(\lambda)\) is in this set.

        9. \(\dim V\) is finite. by 20.2c, \(\dim V_\mu\) is finite \(\forall \mu \in \Pi(\lambda)\). 

    \end{proof}

    \section{Thursday, 4/3/2025}
    
    Skipped

    \section{Thursday, 4/10/2025 by Hyeonmin}
    
    \underline{Convention} 

    \(\mathfrak{G}\) a semisimple lie algebra over an algebraically closed field \(F\) of \(\operatorname{char} 0\), \(F = \mathbb{C}\).

    \(\mathfrak{h}\) a CSA, \(\dim \mathfrak{h} = \ell, \vert \Phi ^+ \vert = m, \Phi \not\ni 0\).
    
    \(\mathfrak{n} = \bigoplus_{\alpha > 0} \mathfrak{G}_\alpha , \mathfrak{b} = \mathfrak{h} \oplus \mathfrak{n}, \mathfrak{n}^- = \bigoplus_{\alpha < 0} \mathfrak{G}_\alpha\).
    
    \(\Lambda_r = \mathbb{Z} \Phi\).

    \(Z(\mathfrak{G}) =\) the center of \(U(\mathfrak{G})\). 

    \section*{1.1: Axioms and Consequences}

    \begin{definition}
        The BGG Category \(\mathcal{O}\) is the full subcategory of \(\operatorname{Mod}_{U(\mathfrak{G})}\) satisfying:

        \begin{enumerate}[label=$\mathcal{O}$\arabic*)]
            \item \(M\in \mathcal{O}\) is a f.g. \(U(\mathfrak{G})\)-module.
            \item \(M \in \mathcal{O}\) is \(\mathfrak{h}\)-s.s., i.e. \(M = \bigoplus_{\lambda \in \mathfrak{h}^{\ast}} M_\lambda\) where \(M_\lambda = \{ x\in M \mid \forall h\in \mathfrak{h}, h . x = \lambda(h)x \}\).
            \item \(\forall v\in M, U(\mathfrak{n})v\) is finite dimensional.  
        \end{enumerate} 
    \end{definition}

    \begin{proposition}
        \(M\in \mathcal{O}\) satiisfies:

        \(\mathcal{O}4\): All weight space \(M_\lambda\) are finite dimensional.

        \(\mathcal{O}5\): \(\Pi(M) \coloneqq \{ \lambda \in \mathfrak{h}^{\ast} \mid M_\lambda \neq 0 \}\) is contained in the union of finitely many sets of the form \(\lambda-\Gamma\) where \(\lambda \in \mathfrak{h}^{\ast}, \Gamma\) is the semigroup in \(\Lambda_r\) generated by \(\Phi^+\).
    \end{proposition}

    \begin{proof}
        \(\mathcal{O} 1, \mathcal{O} 2 \implies M\)  has a finite generating set consisting weight vectors [from axiom 2, we have a direct sum. So, a vector in \(M\) can be represented as a finite sum of weight vectors by the definition of direct sum].

        Thus, we may assume that \(M = U(\mathfrak{G}) v\) [\(M\) is generated by exactly one weight vector] with a weight vector \(v\) of weight \(\lambda\).

        \(\mathcal{O} 3 \implies V = U(\mathfrak{n})v\) is finite dimensional.
        
        \(U(\mathfrak{h})\) is stable on this so \(U(\mathfrak{b})v\) is finite dimensional.
        
        Since the action of \(U(\mathfrak{n}^-)\) produces weights lower, then \(\exists \mu \in \mathfrak{h}^{\ast}: \Pi (M) \subseteq \mu - \Gamma\) [in fact \(\mu - \lambda\)]. This gives us \(\mathcal{O}5\).
        
        Write \(\forall\) weight \(v, M_v = \left\{ u \cdot w \mid u\in U(\mathfrak{n}^-), w\in V, \operatorname{wt}(u \cdot w) = v \right\} \)

        Since \(\forall w\in V, \exists\) only finitely many monomials \(u = y_1^{i_1} \cdots y_m^{i_m}\) such that \(\operatorname{wt}(u \cdot w) = v\) and \(V =\) finite dimensional, then \(M_v\) is finite dimensional [\(\mathcal{O} 4\)].
    \end{proof}

    \begin{theorem}
        [1.1] Category \(\mathcal{O}\) satisies:

        \begin{enumerate}[label=\alph*)]
            \item \(\mathcal{O}\) is a noetherian category.
            \item \(\mathcal{O}\) is closed under submodules, quotients, finite direct sums.
            \item \(\mathcal{O}\) is an abelian cateogry.
            \item If \(L\) is finite dimensional \(\mathfrak{G}\)-module, then \(\mathcal{O} \xrightarrow{L \otimes_\mathbb{C} -}\mathcal{O}\) is an exact functor.
            \item \(M\in \mathcal{O}\) is a \(Z(\mathfrak{G})\)-finite. i.e., \(\forall v\in M, \operatorname{span}(Z(\mathfrak{G})v)\) is finite dimensional.
            \item \(M\in \mathcal{O}\) is finitely generated \(U(\mathfrak{n}^-)\)-module.  
        \end{enumerate} 
    \end{theorem}

    \begin{proof}
        \begin{enumerate}[label=\alph*)]
            \item \(U(\mathfrak{G})\) is a noetherian ring. [It has a filtration which we can pass to a graded ring. Which is isomorphic to a polynomial ring.] We can invoke \(\mathcal{O}1 \implies\) \(\mathcal{O}\) is a noetherian category.
            \item Quotient and finite direct sum ok. \(U(\mathfrak{G})\) nnoetherian so submodule f.g.
            \item Since \(\operatorname{Mod}_{U(\mathfrak{G})}\) is an abelian cattegory, we only need to check \(\exists\) kernels, cokernels, finite diret sums. (b) gives us this.
            \item \(\mathcal{O}2: (L \otimes M)_\nu = \bigoplus_{\mu +\lambda =\nu} (L_\mu \otimes M_\lambda) v\). \(\mathcal{O}3\) ok. \(\mathcal{O}1: \{ v_1, \cdots , v_n \}\) basis of \(L, \{ w_1, \cdots , w_p \}\) generating set of \(M\). Let \(N \coloneqq\) the submodule of \(L \otimes M\) generated by \(\{ v_i \otimes w_j \}\). WTS: \(L \otimes M = N\). We already know \(N \subseteq L \otimes M\). Now, for any \(v\in L\), since \(L\) is just a finite dimensional vector space, \(v \otimes w_j \in N\) for any \(j\). \(\forall x\in \mathfrak{G}, x \cdot (v \otimes w_j) = (x \cdot v) \otimes w_j + v \otimes (x \cdot w_j)\). Since \(x \cdot (v \otimes w_j) \in N\) and \((x \cdot v) \otimes w_j \in N \implies v \otimes (x \cdot w_j) \in N\). Iterating, \(\forall u\in U(\mathfrak{G}), v \otimes (u \cdot w_j) \in N\).
            \item Since \(v\in M\) is the direct sum of weight vectors, then may assume that \(v\in M_\lambda\) for some \(\lambda \in \mathfrak{h}^{\ast}\). Then \(Z(\mathfrak{G}) \cdot v \in M_\lambda\). [\(\forall z\in Z(\mathfrak{G}): h \cdot (zv) = z(hv) = z(\lambda(h)v) = \lambda(h)(zv)\)]. \(\mathcal{O}4\) implies the result.
            \item \(\{ m_1, \cdots , m_p \}\) a generating set of \(M, N_0 = \operatorname{span}\{ m_1, \cdots , m_p \}, N = U(\mathfrak{b}) N_0.\) \(\mathcal{O}3\) implies \(U(\mathfrak{n})N_0\) f.g. Thus A basis of \(N\) generates \(M\) as a \(U(\mathfrak{n}^-)\)-module.
        \end{enumerate} 
    \end{proof}

    \subsection*{1.2: Highest Weight Modules}

    \begin{definition}
        \(M = U(\mathfrak{G}) . v^+\) is a highest weight module where \(v^+\) is maximal vector of \(\lambda\). 
    \end{definition}

    \begin{remark}
        \(M \in \mathcal{O}, \exists\) a maximal vector \(v^+\) in \(M\).
    \end{remark}

    \begin{theorem}
        [1.2] \(M = U(\mathfrak{G}) \cdot v^+\) a highest weight module of weight \(\lambda\).. Chose \(y_i \neq 0\) in \(G_{-\alpha_i}\). Then,

        \begin{enumerate}[label=\alph*)]
            \item \(M\) is spanned by \(y_1^{i_1} \cdots y_m^{i_m} v^+ (i_j \geq 0)\). Thus \(M\) is \(\mathfrak{h}\)-s.s.
            \item All wt \(\mu\) satisfies \(\mu \leq \lambda\).
            \item \(\forall\) wt \(\mu, \dim M_\mu < \infty, \dim M_\lambda = 1 \implies M \in \mathcal{O}\).
            \item Each nonzero quotient is again a highest weight module of weight \(\lambda\).
            \item A submodule generated by a maximal vector \(\mu < \lambda\) is proper.
            \item \(M\) has a unique maximal submodule and unique simple quotient. \(M\) Mis indecomposable.
            \item All simple highest weight module \(M\) of weight \(\lambda\) are isomorphic. Moreover, \(\dim \operatorname{End}_{\mathcal{O}} M = 1\)  
        \end{enumerate} 
    \end{theorem}

    \begin{corollary}
        [1.2] \(M \neq \mathcal{O}\), nonzero. Then \(M\) has a filtration \(0 \subseteq M_1 \subseteq \cdots \subseteq M_n = M\) with nonzero quotients. Each of which is a highest weight module.
    \end{corollary}

    \begin{proof}
        [Sketch] \(\forall V \coloneqq \mathfrak{n}\)-submodule generated by a finite generating set of \(M\) of weight vectors. \(\mathcal{O}3\) implies finite dimensional. Now induct on \(\dim V\).
    \end{proof}

    \subsection*{1.3 Verma modules and Simple modules}

    \begin{definition}
        \(\mathbb{C}_\lambda = \mathbb{C}\) a \(\mathfrak{b}\)-module:

        \begin{itemize}
            \item \(\forall h\in \mathfrak{h, v \in \mathbb{C}_\lambda}: h \cdot v = \lambda(h)v\).
            \item \(\forall n\in \mathfrak{n}, v \in C_\lambda: n v = 0\). 
        \end{itemize} 

        \(M(\lambda): = U(\mathfrak{G}) \otimes_{U(\mathfrak{b})} \mathbb{C}_\lambda\): a \underline{Verma Module}.
        
        \begin{itemize}
            \item Finite \(U(\mathfrak{n}^-)\)-module of rank \(1\).
            \item \(M(\lambda) = U(\mathfrak{G}).v^+\) where \(v^+ = 1 \otimes 1\).  
        \end{itemize} 
    \end{definition}

    \begin{remark}
        Let \(N\) be a finite dimensional \(U(\mathfrak{b})\)-module on which \(\mathfrak{h}\) acts semisimply. Then \(U(\mathfrak{G}) \otimes_{U(\mathfrak{b})} N \in \mathcal{O}\). This defines an exact functor: action of \(\mathfrak{h} \to \mathcal{O}\). 
    \end{remark}

    \begin{definition}
        \(L(\lambda)\) [resp. \(N(\lambda)\)] is the unique simple quotient (resp. unique maximal submodule) of \(M(\lambda)\) [from theorem 1.2(f)].
    \end{definition}

    \begin{theorem}
        [1.3] Every simple module in \(\mathcal{O}\) of maximal weight \(\lambda\) is isomorphic to \(L(\lambda)\). Moreover, \(\dim \operatorname{Hom}(L(\mu), L(\lambda)) = \delta_{\mu \lambda}\). 
    \end{theorem}

    \begin{proof}
        Let \(M\in \mathcal{O}\), simple of maximal weight \(\lambda\). Let \(v^+\in M_\lambda \implies U(\mathfrak{G})v^+\in M\) highest weight module.
        
        Theorem 1.2g implies the result.

        When \(\mu = \lambda\) we know that \(\dim \operatorname{End}_{\mathcal{O}}M = 1\). by 1.2g.

        When \(\mu \neq \lambda\) we claim that \(\operatorname{Hom}(L(\mu),L(\lambda)) = 0\).
        
        Let \(0\neq f \in \operatorname{Hom}(L(\mu), L(\lambda))\). These are simple modules, thus \(\ker f = 0, \operatorname{im} f = L(\lambda)\). So this is in fact an isomorphism.

        \(\forall m \in L(\mu)_\nu, h \cdot f(m) = f(h \cdot m) = f(\nu(h)m) = \nu(h) f(m) \implies f(m) \in L(\lambda)_\nu\).

        \(f: L(\mu)_\nu \xrightarrow{\sim} L(\lambda)_\nu\).

        So, \(\mu\) is a weight of \(L(\lambda)\). Therefore \(\mu \leq \lambda\). Isomorphism implies \(\lambda \leq \mu\). Thus \(\mu = \lambda\). Contradiction.
    \end{proof}

    \section{Thursday, 4/17/2025}
    
    \subsection*{1.4 Maximal vectors in Verma modules}

    \begin{proposition}
        [1.4] Given \(\lambda \in \mathfrak{h}^{\ast}\) and fixed \(\alpha \in \Delta\), suppose \(n = :\langle \lambda, \alpha^\vee \rangle \in \mathbb{Z}_{\geq 1}\).

        If \(v^+\) is a maximal vector of weight \(\lambda\) in \(M(\lambda)\) then \(y_\alpha^{n+1} \cdot v^+\) is the maximal vector of weight \(\mu = \lambda - (n+1)\alpha < \lambda\).

        Thus, \(\exists\) a nonzero hom \(M(\mu) \to M(\lambda)\) whose image is in \(N(\lambda)\).
    \end{proposition}

    \begin{lemma}
        [1.4] 

        \begin{enumerate}[label=\alph*)]
            \item \(\forall i\neq j, [x_j, y_i^{k+1}] = 0\).
            \item \([h_j, y_i^{k+1}] = -(k+1)\alpha)i(h_j) y_i^{k+1}\)
            \item \([x_i, y_i^{k+1}]=-(k+1)y_i^k (k \cdot 1 - h_i)\).    
        \end{enumerate} 
    \end{lemma}

    \begin{proof}
        maximality:

        \begin{itemize}
            \item \(\alpha_i = \alpha : x_\alpha(y_\alpha^{n+1} v^+) = [x_\alpha, y_\alpha^{n+1}]v^+ + y_\alpha^{n+1} \cancel{x_\alpha v^+} \underset{c}{=} -(n+1)y_\alpha^n \cancel{(n-\lambda(h_\alpha))}v_+ = 0\).
            \item \(\alpha_i \neq \alpha: x_i(y_\alpha^{n+1} v^+) = [x_i, y_\alpha^{n+1}]v^+ \underset{a}{=} 0\) 
            \item Weight of \(y_\alpha^{n+1} v^+ = \mu\): \(\forall 1 \leq j \leq l, h_j(y_\alpha^{n+1} v^+)=[h_j y_\alpha^{n+1}]v^+ + y_\alpha^{n+1} (\lambda(h_j)v^+) \overset{b}{=} \underbrace{(-(n+1)\alpha(h_j) + \lambda(h_j))}_{\mu(h_j)} y_\alpha^{n+1} v^+\) 
        \end{itemize} 

        Consider \(f: M(\mu) \to M(\lambda)\) by \(v_\mu \mapsto y_\alpha^{n+1} v^+\).
        
        Then \(f(M(\mu))=U(\mathfrak{g}) \cdot y_\alpha^{n+1} \cdot v^+=\) a proper submodule \(\subseteq N(\lambda)\)  
    \end{proof}

    \begin{corollary}
        [1.4] Let \(v^+\) be instead a maximal vector of weight \(\lambda\) in \(L(\lambda)\).

        Then \(y_\alpha^{n+1} v^+ = 0\).
    \end{corollary}

    \begin{proof}
        \(L(\lambda)\) is simple so there doesn't exist maximal vector of \(\mu < \lambda\).
    \end{proof}

    \subsection*{1.5 \(\mathfrak{sl}_2(\mathbb{C})\)}

    Fix the standard basis \(\{ h,x,y \}\).

    \(\dim \mathfrak{h}^{\ast} = 1 \implies h^{\ast} \xrightarrow{\cong} \mathbb{C} , \lambda \mapsto \lambda(h)\).

    Identically, \(\Lambda = \left\{ \lambda \in \mathfrak{h}^{\ast} \mid \langle \lambda, \alpha^\vee \rangle \in \mathbb{Z}, \forall \alpha \in \Phi \right\}\) with \(\mathbb{Z}\) and identically \(\Lambda_r = \mathbb{Z} \Phi\) with \(2\mathbb{Z}\).
    
    eg \(\Phi = \{ \alpha , -\alpha \}; \alpha(h) = 2 \implies \rho = \frac{\alpha }{2} \implies \Lambda = \mathbb{Z} \rho, \Lambda_r = \mathbb{Z}\alpha = 2 \mathbb{Z} \rho\).

    \(M(\lambda)\) has weights \(\lambda, \lambda-2, \lambda-4\) each with mul \(1\).

    Basis vector \((i \geq 0)\) for \(M(\lambda)\) can be chosen so that \((v_{-1} = 0)\):

    \begin{itemize}
        \item \(h \cdot v_i = (\lambda - 2i) v_i\) 
        \item \(x \cdot v_i = (\lambda - i + 1) v_{i-1}\)
        \item \(y \cdot v_i = (i+1) v_{i+1}\)  
    \end{itemize} 

    Claim 1: \(\dim L(\lambda) < \infty\) iff \(\lambda \in \mathbb{Z}_{\geq 0}\).

    Note: weight of \(L(\lambda): \lambda , \lambda -2, \cdots ,-\lambda\)

    Therefore, \(N(\lambda) \cong L(-\lambda - 2)\)

    Claim 2: \(M(\lambda)\) simple iff \(\lambda \notin \mathbb{Z}_{\geq 0}\).

    \(\implies \) is done. \(\impliedby:\) suppose \(M(\lambda)\) is not simple. Then \(\exists N \subsetneq M(\lambda)\) having a maximal vector \(w\) which is not in \(\mathbb{C} v^+\).

    Then, \(\exists k \in \mathbb{Z}_{\geq 0}: w = y^{k+1} v^+\) (up to scalar). Then \(0 = xw = [x,y^{k+1}] v^+ = -(k+1)y^k(k-\lambda)v^+\).
    
    Thus \(k = \lambda\).

    \subsection*{1.6 Finite Dimensional Modules}

    \begin{theorem}
        [1.6] TFAE:

        \begin{enumerate}[label=\alph*)]
            \item \(L(\lambda)\) fin dim
            \item \(\lambda \in \Lambda^+ = \left\{ \lambda \in \mathfrak{h}^{\ast} \mid \forall \alpha \in \Phi , \langle \lambda , \alpha^\vee \rangle \in \mathbb{Z}_{\geq 0} \right\} \) 
            \item \(\dim L(\lambda)_\mu = \dim L(\lambda)_{w \mu} \forall w\in W, \mu \in \mathfrak{h}^{\ast}\). 
        \end{enumerate} 
    \end{theorem}

    \subsection*{1.7 Action of the Center}

    \begin{definition}
        Let \(M = M(\lambda)\) be gen by \(v^+\). For \(z\in Z(\mathfrak{g})\) define \(\chi_\lambda(z) \in \mathbb{C}: z v^+ = \chi_\lambda(z)v^+\) [since \(zv^+\in M_\lambda, \dim M_\lambda = 1\)].
        
        Then \(\chi_\lambda: Z(\mathfrak{g}) \to \mathbb{C}\) the \underline{central character} associated with \(\lambda\)  
    \end{definition}

    Note: \(\forall v\in M, z v = \chi_\lambda (z) v\) since \(v = u \cdot v^+, u\in U(\mathfrak{n}^-)\) and \(zu = uz\).

    \(\chi_\lambda\): alg hom and \(\ker \chi_\lambda\) is a maximal ideal in \(Z(\mathfrak{g})\).
    
    More generally, any alg  hom \(\chi: Z(\mathfrak{g}) \to \mathbb{C}\) is called a central character.
    
    \begin{definition}
        Let \(pr: U(\mathfrak{g}) \to U(\mathfrak{h})\) be the projection by sending other monomials to \(0\).
        
        Then \(\xi = \eval{pr}_{Z(\mathfrak{g})}\) is called the \underline{Harish-Chandra homomorphism}.  
    \end{definition}

    Note: \(\forall z\in Z(\mathfrak{g}), \chi_\lambda(z) = \lambda(\xi(z))\).
    
    Therefore \(\chi_\lambda(z) v^+ = z v^+ = pr(z) v^+ = \lambda(pr(z))v^+\)

\end{document}