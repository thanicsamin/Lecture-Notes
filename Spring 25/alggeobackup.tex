\documentclass{article}
\usepackage{amsmath, amsthm, amssymb, amsfonts, mathtools,enumitem, stmaryrd,physics, cancel, tikz-cd, graphicx, float, booktabs}
\usetikzlibrary{arrows}
\usepackage{geometry}
    \geometry{
        a4paper,
        left = 40mm,
        top = 20mm,
        right = 40mm,
        bottom = 30mm
    }
\setlength{\parindent}{0pt}

\theoremstyle{definition}
\newtheorem{problem}{Problem}
\newtheorem{solution}{Solution}
\newtheorem*{example}{Example}
\newtheorem*{exercise}{Exercise}
\newtheorem*{definition}{Definition}
\newtheorem{theorem}{Theorem}
\newtheorem*{theorem*}{Theorem}
\newtheorem{proposition}[theorem]{Proposition}
\newtheorem*{proposition*}{Proposition}
\newtheorem{lemma}[theorem]{Lemma}
\newtheorem*{lemma*}{Lemma}
\newtheorem{corollary}[theorem]{Corollary}
\newtheorem*{corollary*}{Corollary}
\newtheorem*{remark}{Remark}

\title{alggeobackup}
\author{me}

\begin{document}
    %\maketitle

    \section*{Friday, 4/4/2025}

    Let \(f\in k[X], X \subset \mathbb{A}^N\) an affine variety. 
    
    For \(p\in X\), we had \(d_p f \in \mathfrak{m}_p / \mathfrak{m}_p^2 [\cong (T_p X)\ast]\) where,

    \[
        d_p f = (f - f(p)) \pmod{\mathfrak{m}_p^2}
    \]

    \(d_p: k[x] \to (\mathfrak{m}_p / \mathfrak{m}_p^2)\) is a derivation!

    Reason: \(fg - f(p)g(p) = (f-f(p))g + f(p)(g-g(p)) = \underbrace{(f-f(p))(g-g(p))}_{\in \mathfrak{m}_p^2} + g(p)(f-f(p))+f(p)(g-g(p))\) 

    Thus, we can say:

    \[
        d_p f = \sum_{j=1}^n \frac{\partial f}{\partial x_j} (p) \cdot d_p x_j
    \]

    \begin{remark}
        Viewing \(f: X \to \mathbb{A}^1\), we see that \(d_p f = d_p f: T_p X \to T_{f(p)}\mathbb{A}^1 \cong k\).

        Since \(d_p f\) is the linear dual of map.

        Note that,

        \[
            \frac{\mathfrak{m}_{f(p)}}{\mathfrak{m}_{f(p)}^2} = k[(x-f(p))] \xrightarrow{f^{\ast}} \frac{\mathfrak{m}_p}{\mathfrak{m}_p^2}
        \]

        \[
            x - f(p) \mapsto (y \mapsto f(y) - f(p))
        \]

        \[
            [x-f(p)] \mapsto [f-f(p)] = d_p f
        \]


    \end{remark}

    Hence, \(\forall p\in X, d_p f \in \mathfrak{m}_p / \mathfrak{m}_p^2 \forall f\in k[x]\).
    
    Hence, fixing \(f\in k[x]\) and varying \(p\), 

    \[
        p \mapsto d_p f \in \mathfrak{m}_p / \mathfrak{m}_p^2
    \]

    \hrule

    Now let \(X\) be any variety.

    We obtain a function \(df: X \to \bigsqcup_{p\in X} \mathfrak{m}_p / \mathfrak{m}_p^2 = \bigsqcup_{p\in X} (T_p X)^{\ast}\).

    Consider \(\Phi(X) = \left\{ \varphi : X \to \bigsqcup_{p\in X} T_p^{\ast} X \mid \varphi (p) \in T_p^{\ast} X \forall p\in X \right\}\)

    \begin{lemma}
        \(\Phi(X)\) is a \(k[X]\)-module.
    \end{lemma}

    \begin{proof}
        Given \(g\in k[X], \varphi \in \Phi(X)\), we define:

        \[
            (g \cdot \varphi)(p) \coloneqq g(p) \varphi(p)
        \]

        Check that this defines a \(k[X]\)-module structure.
    \end{proof}

    \begin{definition}
        A regular \(1\)-form \(\omega\in \Phi (X)\) is an element such that \(\forall p\in X, \exists p\in U \underset{\text{open, affine}}{\subset} X\) such that \(\eval{\omega}_U\) is the \(k[U]\)-submodule of \(\Phi(U)\) generated by \(df, f\in k[U]\).   

        \[
            \Omega^1[X] = \left\{ \text{reg. \(1\)-forms on \(X\)} \right\} \underset{\text{\(k[X]\)-submodule}}{\subset} \Phi(X) 
        \]
    \end{definition}

    Examples: \(\Omega^1[\mathbb{A}^n] = \bigoplus_{i=1}^n k[\mathbb{A}^n] \cdot dx_i\) 

    \begin{proof}
        
        \(\forall p\in \mathbb{A}^n\), recall that \(\{ d_p x_1, \cdots , d_p x_n \}\) is a basis of \(\mathfrak{m}_p / \mathfrak{m}_p^2\).

        Hence, any \(\varphi \in \Phi(\mathbb{A}^n)\) can be written as \(\sum_{i=1}^n \varphi_i \cdot dx_i\) where \(\varphi_i \in \operatorname{Fun}(\mathbb{A}^n, k)\).

        Then, if \(\omega = \sum_{i=1}^n \omega_i d x_i \in \Omega^1[\mathbb{A}^n]\), then \(\forall p\in \mathbb{A}^n, \exists U \ni p \underset{\text{open}}{\subset} \mathbb{A}^n\).

        \(\omega \in k[U] . \{ df \mid f \in k[U] \} \).

        Since \(df = \sum_{i=1}^n \frac{\partial f}{\partial x_i} \cdot d x_i\),

        \(f \in k[U] \implies df \in df \in \bigoplus_{i=1}^n k[U] . d x_i\) 

        \(\implies \omega \in \bigoplus_{i=1}^n k[U] . \, \mathrm{d}x _i \iff \omega_i \in k[U]\).

        Since this holds \(\forall p\in \mathbb{A}^n, \omega_i \in k[\mathbb{A}^n]\).

    \end{proof}

    \(X\in \mathbb{P}^1, \Omega^1[X] = 0\).
    
    \begin{proof}
        Note that \(\mathbb{P}^1 = \underbrace{\mathbb{A}^1_0}_{(x_0 = 1)} \cup \underbrace{\mathbb{A}_1^1}_{(x_1 = 1)}\).

        \(\mathbb{P}^1 = \{ (x_0: x_1) \}\).
        
        \(t = \frac{x_1}{x_0}, u = \frac{x_0}{x_1}\).
        
        By example 1, \(\omega \in \Omega^1[X] \implies \eval{\omega}_{\mathbb{A}^1_0} \in \Omega^1 [\mathbb{A}_0^1]\) so \(\eval{\omega}_{\mathbb{A}^1_0} = p(t) \cdot d t\).

        Similarly, \(\eval{\omega}_{\mathbb{A} ^1_1} = q(u) \cdot du \).
        
        For these to define \(\omega \in \Omega^1[X]\) we require \(p(t) dt = q(u) du=  q(1 / t) - dt / t^2\) on \(\Omega^1[\mathbb{A}^1_0 \cap \mathbb{A}^1_1]\) where \(ut = 1\).

        LHS is regular at \(0\).

        RHS has a pole of order \(\geq 2\) at \(0\).
        
        Thus this is only possible when \(p=q=0\).
        
    \end{proof}

    Let \(X \subset \mathbb{P}^2\) defined by \(\xi_0^2 + \xi_1^2 + \xi_2^2 = 0\).

    Since \(X\) is projective, \(k[X] = k\). Nevertheless, \(\Omega^1[X] \neq 0\)!!

    \begin{proof}
        Let \(\mathbb{A}^2_0 = \{ \xi_0\neq 0\}, \mathbb{A}^2_1 = \{ \xi_1 \neq 0 \}, \mathbb{A}^2_2 = \{ \xi_2 \neq 0 \}  \) 

        Let \(X = U_{01} \cup U_{12} \cup U_{13}\) where \(U_{ij} = \mathbb{A}^2_i \cap \mathbb{A}^2_j\). Main idea: two of \(\{ \xi_0, \xi_1, \xi_2 \}\) cannot be zero!

        We define:

        \(U_{01}: x = \frac{\xi_1}{\xi_0}, y = \frac{\xi_2}{\xi_0}, \varphi = \frac{dy}{x^2}\).
        
        \(U_{12}: u = \frac{\xi_2}{\xi_1}, v=\frac{\xi_0}{\xi_1}, \psi = \frac{dv}{u^2}\).
        
        \(U_{02}: s = \frac{\xi_0}{\xi_2}, t = \frac{\xi_1}{\xi_2}, \chi = \frac{dt}{s^2}\)

        On \(\mathbb{A}^2_0 \cap \mathbb{A}^2_1 \cap \mathbb{A}^2_2, = U_{01} \cap U_{12} = U_{01} \cap U_{02} = U_{12}\cap U_{02}, \varphi = \psi = \chi\).
        
        Hence \(\exists \omega \in \Omega^1[x]\) such that \(\eval{\omega}_{U_{01}} = \varphi, \eval{\omega}_{U_{12}} = \psi, \eval{\omega}_{U_{02}}=\chi\).
    \end{proof}

    \hrule

    \begin{theorem}
        Let \(p\in X\) be nonsigunlar. Then \(\exists U\ni p \underset{\text{affine open}}{\subset} X\) such that \(\Omega^1[U]\) is a free \(k[U]\)-module of rank \(n = \dim_p X\).  
    \end{theorem}

    \begin{proof}
        WLOG \(X \underset{\text{affine}}{\subset} \mathbb{A}^n, X\) irreducible. Let \(\mathfrak{a}_X = \{ F_1, \cdots , F_m\} \).
        
        \(\dim T_p X = n = \dim X\).

        Since \(T_p X = \left\{ (a_1, \cdots , a_N) \in \mathbb{A}^N \mid \sum_{j=1}^N \frac{\partial F_i}{\partial x_j} (p) a_j = 0 \right\} \),

        \[
            T_p X = \ker \left( \begin{bmatrix}
                \frac{\partial F_1}{\partial x_1} (p) & \cdots & \frac{\partial F_1}{\partial x_N}(p) \\
                \vdots &  & \vdots \\
                \frac{\partial F_m}{\partial x_1}(p) & \cdots & \frac{\partial F_m}{\partial x_N}(p) \\
            \end{bmatrix}  \right) 
        \]

        Hence, \(\operatorname{rank} (\cdots) = N - n\). 
    \end{proof}

\end{document}