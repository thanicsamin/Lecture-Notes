\documentclass{article}
\usepackage{amsmath, amsthm, amssymb, amsfonts, mathtools,enumitem, stmaryrd,physics, cancel, tikz-cd, graphicx, float, booktabs}
\usetikzlibrary{arrows}
\usepackage{geometry}
    \geometry{
        a4paper,
        left = 40mm,
        top = 20mm,
        right = 40mm,
        bottom = 30mm
    }
\setlength{\parindent}{0pt}

\theoremstyle{definition}
\newtheorem{problem}{Problem}
\newtheorem{solution}{Solution}
\newtheorem*{example}{Example}
\newtheorem*{exercise}{Exercise}
\newtheorem*{definition}{Definition}
\newtheorem{theorem}{Theorem}
\newtheorem*{theorem*}{Theorem}
\newtheorem{proposition}[theorem]{Proposition}
\newtheorem*{proposition*}{Proposition}
\newtheorem{lemma}[theorem]{Lemma}
\newtheorem*{lemma*}{Lemma}
\newtheorem{corollary}[theorem]{Corollary}
\newtheorem*{corollary*}{Corollary}
\newtheorem*{remark}{Remark}

\title{M630 Algebraic Geometry}
\author{Thanic Nur Samin}
\date{\vspace{-5ex}}

\begin{document}
    \maketitle

    \section*{Monday, 1/13/2025}
    
    Textbook: Basic Alg Geom Vol 1 (I R Shafarevich)

    Let \(k\) be a field.

    An \underline{affine \(n\)-space over \(k\)} is \(k^n = \left\{ (x_1, \cdots , x_n) \mid x_j \in k \right\}\).

    We denote this by \(\mathbb{A}_k^n\).

    In particular, \(\mathbb{A}_k^2\) is the affine plane.

    \begin{definition}
        An \underline{affine plane curve} is the set of zeros of a polynomial \(f(x,y) \in k[x,y]\).
        
        \[
            \left\{ (x,y) \in \mathbb{A}_k^2 \mid f(x,y) = 0 \right\} 
        \]
    \end{definition}

    Let \(\Gamma = \left\{ (x,y) \in \mathbb{A}_k^2 \mid f(x,y)=0 \right\} \) 

    \begin{itemize}
        \item The degree of \(\Gamma\) is the degree of \(f\).
        \item \(k[x,y]\) is a UFD.
        \item \(\Gamma\) is irreducible if \(f\) is irreducible.
        \item If \(f = f_1^{k_1} \cdots f_r^{k_r}\) where \(f_i\) are irreducible and \(f_i \neq f_j\) when \(i\neq j\), then,
        \[
            \Gamma = \Gamma_1 \cup \cdots \cup \Gamma_r
        \]

        where \(\Gamma_i = \{ f_i = 0 \}\).

        The \(\Gamma_i\) are called irreducible components of \(\Gamma\).
    \end{itemize} 

    \begin{remark}
        If \(k\) is not algebraically closed, these notions are may not make sense.

        For example, let \(k = \mathbb{R}\) and consider the curve \(\Gamma = \{ x^2 + y^2 = 0 \}\). Then \(\Gamma = \{ (0,0) \}\).

        Over the reals, \(x^2 + y^2\) is irreducible with degree \(2\).

        But in \(\mathbb{R}^2\), this is not the only equation that describes this.

        \(\Gamma = \{ x^4 + y^4 = 0 \}\). This is also not irreducible, and the degree is \(4\). Which one do we take?

        This illustrates that the degree of \(\Gamma\) is ambigious. Also, \(x^2 + y^2 \in \mathbb{R}[x,y]\) is irreducible but \(x^4 + y^4 = (x^2 + y^2 + \sqrt{2}xy)(x^2 + y^2 - \sqrt{2}xy)\) is not irreducible.
        
        Thus, whether \(\Gamma\) is irreducible or not is not defined either.

        Such issues do not arise in algebraically closed fields.
    \end{remark}

    \begin{lemma}
        Let \(k\) be a field. Let \(f\in k[x,y]\) be an irreducible polynomial.

        If \(f\nmid g\) then the set of common zeros of \(f\) and \(g\) is finite.
    \end{lemma}

    \begin{proof}
        \(k[x,y] \subset k(x)[y]\) where \(k(x)\) is the field of rational functions in \(x\). So we can consider \(f\in k(x)[y]\).
        
        
        We claim that \(f\) is irreducible as a member of \(k(x)[y]\).

        (If \(f\) were to be factorizable in \(f(x)[y]\), clearing denominators yields a non-unique factorization of \(qf\) for some polynomial \(q\in k[x,y]\).)

        Similarly, \(f\nmid g\) in \(k(x)[y]\). Hence, since \(k(x)[y]\) is a Euclidean domain, \(\gcd(f,g) = 1\).

        Therefore, there exists \(\widetilde{u}, \widetilde{v} \in k(x)[y]\) so that \(\widetilde{u} f + \widetilde{v} g = 1\). Clearing denominators, \(\exists u,v\in k[x,y]\) such that:

        \[
            uf + vg = a(x)
        \]

        Since the denominators are polynomials in \(x\).

        Now, \(f(\alpha,\beta) = g(\alpha, \beta) = 0 \implies a(\alpha)=0\). Thus \(\alpha\) is one of the finitely many roots of \(a(x)\).

        For each such \(\alpha\) a root of \(a(x)\), the polynomials \(f(\alpha, y)\) and \(g(\alpha,y)\) have finitely many roots. Hence there are finitely many common zeroes \((\alpha, \beta)\) of \(f\) and \(g\).

    \end{proof}

    \begin{corollary}
        For curves over an algebriacally closed field, the notions of degree, irreducibility, etc. are well defined.
    \end{corollary}

    \begin{proof}
        Consider an irreducible curve \(\Gamma = \{ f = 0 \}\), \(f\) irreducible. If \(\Gamma = \{ g = 0 \}\), then \(f\mid g\). (Indeed, if \(k\) is algebraically closed, for all \(\alpha \in k, f(\alpha,y)\in k[y]\) has roots. Since \(\#k\) is infinite, \(\Gamma\) has infinitely many points.)
        
        Hence \(f\) appears as an irreducible factor of \(g\) (and can be recovered from \(g\) upto a unit).

        Hence \(\deg \Gamma\) is defined.

        If \(\Gamma = \{ g=0 \}\) where \(g = f_1^{k_1}\cdots f_r^{k_r}\) where \(f_i\)'s are irreducible.

        Note that \(\Gamma = \{ f_1 \cdots f_r = 0 \} \). So, \(\deg \Gamma = \deg f_1 \cdots f_r = \sum_{i} \deg f_i\) 
        
    \end{proof}

    Unless explicitly stated otherwise, we assume that fields are algebraically closed.

    \underline{Another Justification}

    \begin{theorem}
        [Fundamental Theorem of Algebra] Let \(f(x)\in k[x]\) be a polynomial of degree \(n\). Then,

        \(\# \left\{ y - f(x) = 0 \right\} \cap \{ y=0 \} = n\).

        We have to be count properly, taking multiplicity into account.
    \end{theorem}

    More generally, we have:

    \begin{theorem}
        [Bezout's Theorem] (Approximate Statement)
        
        Let \(f,g\in k[x,y], \deg f = m, \deg g = n\). Assume \(f,g\) have no common irreducible factors.

        Then, \(\#\{ f=0 \} \cap \{ g=0 \} =0\) (counted properly under certain assumptions).
        
        For example, in \(k=\mathbb{R}\) we can draw a pair of ellipses with \(0,2,4\) intersections. We have \(4\) intersections via Bezout's theorem.

        % insert
    \end{theorem}

    Question: Does algebraic geometry not comment at all about fields that aren't algebraically closed?

    Answer: One still obtains information about these by passing to the algebraic closure.

    Assume \(k_0 = \mathbb{R}\). Consider the circle \(x^2 + y^2 = r^2\). From a point \(P\) outside the circle, we can consider the tangents \(PT_1\) and \(PT_2\). Let \(\ell\) be the line between \(T_1\) and \(T_2\). This is the polar line of \(P\) w.r.t.\ \(\Gamma\).
    
    What happens when \(P\) is inside \(\Gamma\)? We can still find \(T_1\) and \(T_2\) provided we can pass to the complex numbers.

    We pass to \(\mathbb{C}^2\). Then there are still two lines tangent to \(x^2 + y^2 = r^2\) passing through \(P\) in \(\mathbb{C}^2\). We thus have \(T_1, T_2 \in \mathbb{C}^2\).

    Since the original data is real it must be invariant under complex conjugation. Thus, we must have \(T_1 = \overline{T_2}\).

    Therefore, the line \(\ell\) joining \(T_1\) and \(T_2\) must be real.

    \section*{Wednesday, 1/15/2025}
    
    \begin{remark}
        Algebraic Geometry over fields that are not algebraically closed has its applications.

        For example,

        \begin{enumerate}[label=\arabic*)]
            \item \(K = \mathbb{Q}\): for example, finding rational points on  \(x^n + y^n = 1 \rightsquigarrow\) FLT.
            \item \(k = \mathbb{F}_p\). Studying points on \(f(x,y)=0 \iff\) studying solutions of \(\widetilde{f}(x,y)\equiv 0\mod p\).
            \item \(k = \mathbb{C}(z)\). A surface \(F(x,y,z)=0\) where \(F \in \mathbb{C} [x,y,z]\) is often studied by vieweing it as a curve in \(k(x,y)\). 
        \end{enumerate} 
    \end{remark}

    \subsection*{Rational Curves}

    Consider the curve \(\overbrace{y^2 = x^2 + x^3}^{X}\) over \(k=\mathbb{C}\). This is an irreducible (why?) curve.

    Note that \((0,0)\in X\).

    If \(P\in X, P \neq (0,0)\), consider the line joining \(O\) and \(P\), \(y = tx\). It meets \(X\) at points whose \(x\)-coordinates satisfy:

    \[
        (tx)^{2} = x^{2} +x^3 \iff x^{2} (t^2 - x - 1) = 0
    \]

    It has \(x=0\) as a double root, we have \(x=t^2 - 1, y = t(t^2 - 1)\).
    
    It turns out that every point on \(X\) is of the form \((t^2 - 1, t(t^2 - 1))\).

    Since \(t^2 - 1, t(t^2 - 1) \in \mathbb{Q}[t] \subset \mathbb{R} [t]\), the \(\mathbb{Q}\) (resp. \(\mathbb{R}\)) points of \(X\) can be recovered by letting \(t\) roam over \(\mathbb{Q}\) (resp. \(\mathbb{R}\)).

    \begin{definition}
        Let \(X = \left\{ f(x,y) = 0 \right\}\) be an irreducible affine plane curve over an algebraically closed field \(k\). \(X\) is called rational if there exists rational functions \(\phi(t), \psi(t) \in k(t)\) such that \(f(\phi(t),\psi(t)) \equiv 0\) [in \(k(t)\)]
    \end{definition}

    For example, any irreducible conic is rational.

    \begin{proof}
        WLOG assume our conic \(X\) has equation \(y^2 = ax^2 + bx + c\), if necessary by shifting the \(y\)-coordinate.
        
        Let \((x_0, y_0)\in X\). Consider the line \(y - y_0 = t(x - x_0)\).
        
        It intersects \(X\) at points whose \(x\)-coordinate satisfy:

        \[
            (y_0 + t(x-x_0))^2 = ax^2 + bx + c
        \]

        \[
            \iff t^2(x^2 - 2xx_0 + x_0^2) + 2t(x-x_0)y_0 + y_0^2 = ax^2 + bx + c \, (\star)
        \]

        \(\star\) is a quadratic, \(x_0\) is one root. The other root, by Vieta Jumping, is:
        
        \[
            \frac{2t^2 x_0 - 2ty_0 + b}{t^2 - a} - x_0 = \phi(t)
        \]

        Thus, \(\psi(t) \equiv  y_0 + t(\phi(t) - x_0)\)

        Then clearly \((\phi(t), \psi(t))\in X \forall t, \phi, \psi \in k(t)\).

        Sub example: \(c > 0, x_0 = 0, y_0 = \sqrt{c} \).
        
        Then, \(\phi(t) = \frac{b - 2t \sqrt{c}}{t^2 - a}, \psi(t) = \sqrt{c} + t\left(\frac{b - 2t \sqrt{c}}{t^2 - a}\right) \) 
    \end{proof}

    Application: If \(X\) is a rational curve over \(\mathbb{R}\) then the equation \(f(x,y)=0\) defines \(y\) implicitly as a function of \(x\).

    If \(g(x,y)\in k(x,y)\) we have:

    \[
        \int g(x,y(x))\,\mathrm{d}x = \int g(\phi(t),\psi(t))\phi^{\prime}(t)\, \mathrm{d}t \, (\star\star)
    \]

    where \((\phi(t),\psi(t))\) is a parametrization of \(X\). This turns LHS of \((\star\star)\) to an integral of a rational function.

    For example if \(y^2 = ax^2 + bx + c, c > 0\) then putting \(x = \phi(t) = \frac{b-2t\sqrt{c}}{t^2 - a}, \psi(t) = \sqrt{c} + t \left( \frac{b - 2t \sqrt{c}}{t^2 - a} \right)\) turns \(\int g(x, \sqrt{ax^2 + bx + c})\) into \(\int (\text{rational function})\, \mathrm{d}t\).

    These are called Euler substitutions of the second type.

    Question: Characterize all rational curves.

    \subsection*{Field of rational function on a curve}.

    Let \(X = \left\{ f(x,y) = 0 \right\} \) be an irreducible affine algebraic curve.

    \(S = \left\{ g \in k[x,y] \mid f \nmid g \right\}\). Then \(S\) is a multiplicative set.

    Consider \(k(X) \coloneqq \frac{k[x,y][S ^{-1}]}{(f)}\).

    Equivalently (check!): \(k(x)\) is the set of all fractions of the form \(\frac{p(x,y)}{q(x,y)}\) where \(o,q\in k[x,y]\) with \(f\nmid q\).
    
    \(\frac{p}{q} \sim \frac{p^{\prime}}{q^{\prime} =} \iff f \mid pq^{\prime} -p^{\prime} q\). 

    Claim: \(k(X)\) is a field.

    \begin{proof}
        If \(\frac{p(x,y)}{q(x,y)}\neq 0\) in \(k(X)\),then \(f \nmid p(x,y) \implies \frac{q(x,y)}{p(x,y)}\in k(X) \).
    \end{proof}

    \(k(X)\) is called the field of rational functions on \(X\). Note that since \(f\nmid q\) and \(q(x,y)=0\) on at most finitely many points on \(X\). Thus \(\frac{p}{q}\) can be viewed as a function defined on all but finitely many points on \(X\).

    \(\operatorname{tr} \deg_k k(X) = 1\) since \(x\) and \(y\) generate \(k(X)\) but are algebraically dependent since \(f(x,y)=0\).  
    
    \begin{definition}
        A rational function \(u\in K(X)\)  is \underline{regular} at a point \(P\in X\) if \(u\) can be written as \(\frac{p(x,y)}{q(x,y)}\) such that \(q(P) \neq 0\).
    \end{definition}

    For example, let \(X = \{ x^2 + y^2 = 1 \}\) and let \(u(x,y) = \frac{1-y}{x}\). We want to f0ind all points on \(X\) where \(u\) is regular.

    \(u\) is regular whenever \(x\neq 0\). That leaves \((0,\pm 1)\).

    Note that \(u(x,y) = \frac{1-y}{x} = \frac{x}{1+y}\). Thus it is also regular at \((0,1)\).

    Exercise: find out if \(u\) is regular at \((0,-1)\).

    \begin{proposition}
        If an irreducible curve \(X\) is rational, then \(k(X) \cong k(t)\).
    \end{proposition}

    \section*{Friday, 1/17/2025}
    
        Recall: if \(X\) is an irreducible algebraic plane curve, one has \(k(X)\) the field of rational functions on \(X\).

        \(\operatorname{tr} \deg_k k(X) = 1\).
        
        \begin{lemma}
            If \(K\) is a field extension of \(k\) of \(\operatorname{tr} \deg 1\) generated by \(x,y\in K\) then \(K \cong k(x)\) for some irreducible curve \(X\).
        \end{lemma}

        \begin{proof}
            Since \(\operatorname{tr} \deg_k K = 1, x,y\) satisfy \(f(x,y) = 0\) for some irreducible polynomial \(f(x,y)\in k[x,y]\). Then \(K \cong k(X)\),

            \[
                X = \{ (x,y)\in \mathbb{A}_k^2 \mid f(x,y) = 0 \}
            \]

        \end{proof}

        \begin{proposition}
            Let \(X\) be irreducible algebraic plane curve over \(k\). Then, \(X\) is rational \(\iff k(X) \cong k(t)\).
        \end{proposition}

        \begin{proof}
            \(X\) rational \(\implies k(X) \cong k(t)\)

            Let \(X = \{ (x,y)\in \mathbb{A}^2_k \mid f(x,y)=0 \}\) where \(f\in k[x,y]\) irreducible since \(X\) is rational.

            \(\exists \phi (t), \psi (t)\in k(t)\) such that \(f(\phi (t),\psi (t))\equiv 0\).

            We construct a map:

            \(i: k(x) \to k(t)\) 

            \(x \mapsto \phi(t)\) 

            \(y \mapsto \psi(t)\) 

            \[
                \frac{p(x,y)}{q(x,y)} \mapsto \frac{p(\phi(t),\psi(t))}{q(\phi(t),\psi(t))}
            \]

            We claim that \(i\) is well defined if \(q(\phi(t),\psi(t))\equiv 0\)

            Then, note that as \(t\) runs over \(k\) the point \((\phi(t),\psi(t))\in X\) [whenever defined] takes infinitely many values.
            
            Since \(f\) does not divide \(q\), \(q(\phi(t),\psi(t))\not\equiv 0\).
            
            Hence, \(i \left( \frac{p(x,y)}{q(x,y)} \right)\) makes sense for \(\frac{p}{q}\in k(X)\).
            
            Checking that \(i\) is a ring homomorphism is easy [left as exercise].

            \(\implies i \left( \frac{p}{q} \right) = i \left( \frac{p^{\prime}}{q^{\prime}} \right) \)

            Since \(k(X),k(t)\) are fields, \(i\) is injective and \(\operatorname{im} (i) \hookrightarrow k(t)\) as a subfield.

            Clearly, \(k \subsetneq \operatorname{im} (i) \subseteq k(t)\) 

            By Luroth's theorem, \(\operatorname{im} (i) = k(g(t))\) for some rational function \(g(t)\).

            Thus, \(i: k(X) \xrightarrow{\cong} \underbrace{k(g(t))}_{=k(v)} \hookrightarrow k(t)\)

            Hence \(k(X) \cong k(t)\) for \(X\) rational since \(k(v) \cong k(t)\).

            Conversely, if \(k(X) \cong k(t), X = \{ f(x,y)=0 \}\) let \(i: k(X)\) be a given isomorphism.

            Let \(\phi (t) \coloneqq i(x), \psi (t) \coloneqq i(y)\) then, \(\phi , \psi \in k(t)\). Since \(f(x,y)\equiv 0\) in \(k(X)\) we have \(f(\phi(t),\psi(t))\equiv 0\).

        \end{proof}

        \begin{proposition}
            Let \(X\) be a rational curve. Let \((\phi, \psi): k(X) \xrightarrow{\cong} k(t)\) be an isomorphism (here \((\phi ,\psi )\) stands for the ring homomorphism \(k(X) \to k(t)\) sending \(x \rightsquigarrow \phi(t), y \rightsquigarrow \psi(t)\)).

            Then, if \(\chi(x,y) = (\phi,\psi)^{-1} (t)\),

            The maps \(X \xrightarrow{\Psi} \mathbb{A}_k^1\) given by \((x,y) \mapsto \chi(x,y)\) 

            \(\mathbb{A}_k^1 \xrightarrow{\Phi} X\) givn by \(t \mapsto (\phi(t),\psi(t))\) 

            satisfying \(\Phi \circ \Psi = \operatorname{id}, \Psi \circ \Phi = \operatorname{id}\) wherever both compositions are defined.
        \end{proposition}

        \begin{proof}
            Indeed, \(\chi(\phi(t),\psi(t)) \equiv  (\phi, \psi)(\phi,\psi)^{-1}(t) = t\) in \(k(t)\).
            
            The inverse of the given homomorphism sends \(\phi(t)\) to \(x\) and \(\psi(t)\) to \(y\). 

            \(x = (\phi, \psi) ^{-1} \phi(t) = \phi(\chi(x,y))\), similar for \(y\).
            
            Suppose \((x_0,y_0)\in X, \Psi(x_0, y_0)\) is well defined. Suppose moreover that \(\Phi \circ \Psi (x_0,y_0)\) is also defined. Then, \(\Phi \circ \Psi(x_0,y_0)=\Phi(\chi(x_0,y_0)) = (\phi(\chi(x_0,y_0)),\psi(\chi(x_0,y_0))) = (x_0,y_0)\).
            
            Similarly, if \(\Phi(t) = (\phi (t), \psi (t))\) defined for some \(t_0 \in \mathbb{A}_k^1\), and \(\Psi(\Phi(t_0))\) is defined,
            
            \(\Psi(\Phi(t_0)) = (\chi(\phi(t_0),\psi(t_0)))=t_0\) 
        \end{proof}

        \begin{corollary}
            Let \(U \subset X\) be the set of all points where \(\Psi\) is defined. Let \(V \subset \mathbb{A}^1_k\) be the set of all points where \(\Phi\) is defined.

            Then the pair: \(\Psi : X \to \mathbb{A}_k^1, \Phi: \mathbb{A}_k^1 \to X\) induces a bijection of sets. 

            \[
                \Psi : U \cap \Psi ^{-1} (V) \to  V \cap \Phi ^{-1} (U)
            \]

            Note that \(U \cap \Psi ^{-1} (V)\) consists of all but finitely many points of \(X\), and \(V \cap \Phi ^{-1} (U)\) is all but finitely many points of \(\mathbb{A}^1_k\).

        \end{corollary}



\end{document}