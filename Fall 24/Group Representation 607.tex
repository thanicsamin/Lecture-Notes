\documentclass{article}

\usepackage{amsmath, amsthm, amssymb, amsfonts, mathtools,enumitem, stmaryrd,physics, cancel}
\usepackage{tikz-cd}
\usetikzlibrary{arrows}
\usepackage{graphicx}
\usepackage{float}
\usepackage{booktabs}
\usepackage{geometry}
    \geometry{
        a4paper,
        left = 40mm,
        top = 20mm,
        right = 40mm,
        bottom = 30mm
    }
\setlength{\parindent}{0pt}

\theoremstyle{definition}
\newtheorem{problem}{Problem}
\newtheorem{solution}{Solution}
\newtheorem*{example}{Example}
\newtheorem*{exercise}{Exercise}
\newtheorem*{definition}{Definition}
\newtheorem{theorem}{Theorem}
\newtheorem*{theorem*}{Theorem}
\newtheorem{proposition}[theorem]{Proposition}
\newtheorem*{proposition*}{Proposition}
\newtheorem{lemma}[theorem]{Lemma}
\newtheorem*{lemma*}{Lemma}
\newtheorem{corollary}[theorem]{Corollary}
\newtheorem*{corollary*}{Corollary}
\newtheorem*{remark}{Remark}

\newcommand{\Frac}{\operatorname{Frac}}
\newcommand{\im}{\operatorname{im}}
\newcommand{\acts}{\curvearrowright}
%\renewcommand{\rank}{\operatorname{rank}}
\newcommand{\End}{\operatorname{End}}
\newcommand{\Hom}{\operatorname{Hom}}
\newcommand{\Char}{\operatorname{char}}
%\renewcommand{\Tr}{\operatorname{Tr}}

\title{Group Representations MATH 607}
\author{Thanic Nur Samin}
\date{\vspace{-5ex}}

\begin{document}

\maketitle

Texts: Lang, Algebra, Revised Third Edition, Chapter 17 (sections 1-5) and 18 (sections 1-8)

Serre, Linear Representations of Finite Groups, Parts II and III

\section*{Monday, 8/26/2024}

Today:

History

Modular

Quotients

Matrices

Lang XVII, Section 1

\subsection*{(Fake) History}

History of Groups 

Most notions (let's say what is a vector spce, what is a group) were vague.

Originally, groups were seen as:

\begin{itemize}
    \item Symmetry Groups \(S_n\) 
    \item \(GL_n(\mathbb{R})\) aka \(n \times n\) invertible matrices
    \item Subgroups of the above
    \item \underline{Representations} of the above
\end{itemize}

For representation, consider \(G\) and a homomorphism \(G \to S_n\) [which is a group action \(G \curvearrowright \{ 1,2,\dots ,n \} \) ] or a homomorphism \(G \to GL_n\) [which is a group action on vector space].

Part I of this course will be Ring Theory.

\subsection*{Part I: Ring Theory}

\subsection*{Module}

Convention: \(R = \) Ring with unity

\begin{definition}
    [Left Module]

    Left Module is an abelian group \(M\) with a function \(R \times M \to M\) so that \((r,m) \mapsto rm\) such that \(R \times M \to M\) is \(\mathbb{Z}\)-billinear.

    Meaning, we have:

    \((r+r^{\prime} )m = rm + r^{\prime} m\) 

    \(r(m + m^{\prime} ) = r m + r m^{\prime} \) 

    Also \((r r^{\prime} )m = r(r^{\prime} m)\) 

    And finally \(1m = m\) 
\end{definition}

By default, module = left module (since Jim doesn't want Trump to get reelected, he prefers left module)

module / field [module over field] = vector space

We can have submodules \(M^{\prime} \triangleleft M\)

We have quotients \(M / M^{\prime} \) 

We have the short exact sequence:

\[
    0 \to M^{\prime} \to M \to M / M^{\prime} \to 0
\]

which means in each homomorphism, im = ker

So, \(M^{\prime} \to M\) is injective and \(M \to M / M^{\prime} \) is surjective.

Also, kernel of \(M \to M / M^{\prime} \) is \(M^{\prime} \)


\begin{remark}
    Note that \(R\) is itself an \(R\)-module. 

    Convention: Submodule \(M\) of \(R\) = left ideal of \(R\). 

    Left ideals are not enough to take quotients (like how we need normal subgroup for group quotients).
\end{remark}

So we need two sided ideals.

\begin{definition}
    [Two Sided Ideals]

    \(I \subset R\) is \underline{2-sided ideal} if \(I\) is abelian subgroup and \(ri \in I, ir \in I\) aka ``closed''.
\end{definition}

\begin{example}

    Consider a homomoprhism \(f: R \to R^{\prime} \). Then \(\ker f\) is a 2-sided ideal of \(R\).

\end{example}

For ring homomorphism we need:

\(f(r+r^{\prime} )=f(r)+f(r^{\prime} )\) 

\(f(r r^{\prime} )=f(r)f(r^{\prime} )\) 

\(f(1)=1\) 

If \(I \subset R\) is 2-sided then \(R / I\) is a quotient ring.

For example, \(M_2(\mathbb{R})\) has no proper \(2\)-sided ideal. But there exists left ideals!

\(\begin{pmatrix}
    \ast &  0 \\
    \ast &  0 \\
\end{pmatrix}\) is a left ideal 

Matrices are a good `source' of non-commutative rings.

Given any ring \(R\) we can consider ring \(M_n(R)\) of \(n \times n\) matrices.

Given \(R\)-module \(M\) we can get \(\text{End}_R(M) = \{ f: M \to M, f \text{ is } R\text{-module map} \}\)

We have \((f+g)m=f(m)+g(m), (fg)m = f(g(m))\).

This is a `coordinate free approach' to matrices.

\begin{remark}

    \(M_n(R)\) and \(\text{End}_R(R^n)\) often looks the same, but in general \(M_n(R) \not\cong \text{End}_R(R^n)\).

\end{remark}

Let's first take \(n = 1\). Let \(r_0 \in R\).

Consider \(R \to R\) map \(r \mapsto r_0 r\) 

We don't like this because \underline{this is not a left module map!!!} 

So this is not even in \(\text{End}_R(R)\) 

What if we consider \(r \mapsto r r_0\)?

This is a left module map, aka \(\in \text{End}_R(R)\) 

But \(R \to \text{End}_R(R)\) is not a ring homomorphism.

So we are going to take the opposite ring.

\underline{Fix 1}:

Given ring \(R\), we can look into the mirror and find opposite ring \(R^{op}\)

Elements of \(R^{op} =\) elements of \(R\).

\(0, 1, +\) remain the same 

But multiplication is reversed: define \(r \cdot_{op} r^{\prime} = r^{\prime} r\) 

Alternate notation, we write \(op\) on elements.

Then \(r^{op}(r^{\prime} )^{op} = (r^{\prime} r)^{op}\) 

Then we have isomorphism \(R^{op} \cong \text{End}_R(R)\) which is a ring homomoprhism!

\begin{exercise}


    \begin{enumerate}[label=\arabic*)]

        \item \(R \cong R^{op} \iff \exists\) antiautomorphism \(\alpha : R \to R\) 

        Antiautomorphism means \(\alpha\) preserves \(0, 1, +\) but reverses mutliplication

        \item \(R\) commutative, then \((M_n R) \cong (M_n R)^{op}\) 

        \item Real quaternions \(\mathbb{H} \cong \mathbb{H} ^{op}\) 

    \end{enumerate}

\end{exercise}


\begin{remark}
    If you take right modules, you don't need op.
\end{remark}

There is a \underline{contravariant endofunctor} in the category of rings which takes objects of rings to their opposite.

\(\text{Ring}^{op} \to \text{Ring}\) [opposite category, not the same thing]

\(R \mapsto R^{op}\) 

\underline{Fix 2}: [From Lang]

Suppose we have module homomorphism \(\phi : E = E_1 \oplus \cdots \oplus  E_n \to F_1 \oplus \cdots \oplus F_m = F\) 

Then we have \(E_j \to E \overset{\phi }{\to } F \to F_i\) which we define to be \(E_j \overset{\phi_{ij}}{\to } F_i\) 

Then we have a matrix \(M(\phi)\) so that \(M(\phi)=(\phi)_{ij}\) 

Then for \(\begin{pmatrix}
     x_1 \\
     \vdots \\
     x_n \\
\end{pmatrix}\in E_1 \oplus \cdots \oplus E_n\) 

Then \(\phi (x) = (\phi_{ij}) \begin{pmatrix}
     x_1 \\
     \vdots \\
     x_n \\
\end{pmatrix}\) 

So, if we have \(E^n = E \oplus \dots \oplus E\) [n times]

Lang says, there is a ring isomorphism

\[
    \text{End}_R(E^n) \overset{\cong}{\to } M_n(\text{End}_R E)
\]

\[
    \phi \mapsto (\phi_{ij})
\]

If \(E = R\) as left module, then \(\text{End}_R R \cong R^{op}\) 

By combining these, \(\text{End}_R(R^n)\cong M_n(R^{op})\) 


\section*{Wednesday, 8/28/2024}

Today:

Group ring

Category

Simple modules

Question: The course is about `group representations'. So why study rings?

Answer: A group representation [homomorphism \(G \to GL_n(\mathbb{R})\)] is exactly the same as a module over the ring \(\mathbb{R} G\).

So knowing everything about modules would tell us everything about representation.

Abelian Category!

Suppose we have a ring \(R\) and a group \(G\). We can get a ring out of \(G\)

\begin{definition}
    [Group Ring \(RG\)]

    As an abelian group, this is the free \(R\)-module with basis the elements of \(G\).

    Elements are symbols of the form \(r_1 g_1 + \dots + r_n g_n\) [finite linear combination].

    \(0\) is the trivial linear combination. So \(0 = 0\)

    \(1 = 1e = 1_R e_G\)
    
    Multiplication is defined in the obvious way.

    \((\sum_{i} r_i g_i)(\sum_{j} r_j^{\prime} g_j^{\prime} ) = \sum_{i, j} r_i r_j^{\prime} g_i g_j^{\prime} \) 

\end{definition}

Suppose \(V\) is a \(R\)-module.

Then a homomorphism \(\rho : G \to \text{Aut}_R(V) \leftrightarrow V\) is \(RG\)-module.

\(\rho \mapsto (\sum_{i} r_i g_i)v \coloneqq \sum_{i} r_i \rho(g_i)v\)

\(g \mapsto (v \to gv) \leftarrow V\) \(RG\) module.

\begin{example}
    \(C_2 = \{ 1,t \}\) 

    Then we have \(\mathbb{Z} C_2 = \{ a + bt \mid a, b\in\mathbb{Z}, t^2 = 0\} = \mathbb{Z} [t] / (t^2) \) 

    Note that \((1+t)(1-t) = 1-t^2 = 0\) so we have zero divisors.

    Take \(C_\infty = \langle t \rangle \) 

    Then \(\mathbb{Z} C_\infty =\mathbb{Z} [t, t^{-1}]\) the laurent polynomial ring.

    \(\mathbb{Q} C_\infty = \mathbb{Q}[t, t^{-1}]\) is a PID [since it is a euclidean ring]

\end{example}

Now we see categories.

If we fix \(R\) then we have a functor \(\text{Group} \to \text{Ring}\) given by \(G \mapsto RG\)

Or we could say we have a functor \(\text{Ring} \times \text{Group} \to \text{Ring}\) given by \((R, G) \to RG\) 

\begin{definition}
    A category \(\mathcal{C}\) consists of:

    \begin{itemize}
        \item objects \(\text{Ob } \mathcal{C}\) 

        \item morphism \(C(X,Y)\) for \(X, Y \in \text{Ob } \mathcal{C}\)
    
        \item compositions \(C(X,Y) \times C(Y,Z) \to C(X,Z)\) given by \((g,f) \mapsto f \circ g\) 

        \item identity \(\text{Id}_X \in C(X,X) \forall X\in \text{Ob} \mathcal{C}\) 

    \end{itemize}
    
    Such that we have:

    \begin{itemize}
  
    \item associativity: \((f \circ g)\circ h = f\circ (g\circ h)\) 

    \item composition with identity: \(\text{Id}_Y \circ f = f = f\circ \text{Id}_X\) for \(f\in C(X,Y)\)

    \end{itemize}

    For example in the cateogry of groups, we have objects groups and morphisms homomorphism. 

    Morphism notations: \(f: X \to Y\) or \(X \overset{f}{\to } Y\) for \(f\in C(X,Y)\) 

\end{definition}

\begin{definition}
    \(f: X \to Y\) is isomorphism if \(\exists g:Y \to X\) such that \(f\circ g = \text{Id}, g\circ f = \text{Id}\). Thehen we say \(X\) and \(Y\) are isomorphic and write \(X \cong Y\).
\end{definition}

\begin{example}

    Example of Categories:

    \begin{itemize}
        \item Set 
        \item Ring
        \item Group
        \item Ab (Abelian Groups)
        \item \(R\)-modules (objects are modules, morphisms are homomorphisms \(h(rm) = rh(m)\) )
        \item Given a group \(G\) we can get a category \(BG\) such that:

        \(\text{Ob } BG = \{ \ast \} \)  and \(BG(\ast,\ast) = G\) 
        
        In this category, there is only one object \(\ast\). The elements of the group are morphisms.
        
    \end{itemize} 



\end{example}

\begin{definition}
    Functor \(F: \mathcal{C} \to \mathcal{D}\) is \(F: \text{Ob } \mathcal{C} \to \text{Ob } \mathcal{D}\) given by \(X \mapsto F(X)\)

    And \(F: \mathcal{C} (X,Y) \to \mathcal{D}(F(X),F(Y))\) such that

    \(X \overset{f}{\to} Y\) gives us \(F(X) \overset{F(f)}{\to} F(Y)\) 

    such that \(F(f\circ g) = F(f)\circ F(g)\) and \(F(\text{Id}_X) = \text{Id}_{F(X)}\) 
\end{definition}

\begin{example}

    Unit Functor Ring \(\to\) Group given by \(R \mapsto R^\times = \{ r\in R \mid \exists s\in R, rs = 1 = sr \} \)

    For example, \(\mathbb{Q} ^\times \cong C_2 \oplus \mathbb{Z} ^{\infty} [=\pm p_1^{e_1}p_2^{e_2}\cdots]\) 

    \(\mathbb{Z} ^\times \cong \{ \pm 1 \} = C_2\) 

    \((\mathbb{Z} C_2)^\times \cong \{ \pm 1, \pm t \} \cong C_2 \times C_2\)

\end{example}

\begin{definition}
    \(R\) is a division ring (= skew field) if \(1 \neq 0\) and \(R^\times = R - 0\).
\end{definition}

\begin{definition}
    Quaternions

    \(\mathbb{H} = \{ a + bi + cj + dh \mid a,b,c,d,\in \mathbb{R} \} \) 

    Where \(i^2 = j^2 = k^2 = -1\) 

    \(ij = k, jk = i, ki = j, ji = -k, kj = -i, ik = -j\) 

    This is a division ring since we can write down inverses.

    \(\alpha = a + bi + cj + dk\) gives us \(\overline{\alpha} = a - bi - cj - dk\) 

    So, \(\text{norm}(\alpha) = \alpha \overline{\alpha } = a^2 + b^2 + c^2 + d^2\) 

    So, \(\alpha ^{-1} = \frac{\overline{\alpha }}{\text{norm}(\alpha)}\) 
\end{definition}

\begin{remark}

    Note that the quaternion group \(Q_8 = \{ \pm 1, \pm i, \pm j, \pm k\} \) is a subgroup of \(\mathbb{H} ^\times = GL_1(\mathbb{H})\).

    So, \(\mathbb{H}\) is a \(\mathbb{R} Q_8\) module.

\end{remark}

\begin{theorem}
    [Weddenburn's Little Theorem]

    a. A finite commutative domain is a field [easy]

    b. A finite skew field is a field [aka commutative]

\end{theorem}

a is easy: suppose \(F\) is finite commutative domain. For \(0 \neq f\in F,\) consider multiplication by \(f\) as a map \(F \to F\). It is injective, and finiteness implies surjective. So, it is bijective, and there exsits inverse.

eg \(\mathbb{Z} / p\) is a field.

\subsection*{Simple Modules}

These are like primes. We also have some analogue of prime factorization.

\begin{definition}
    \(R\)-module \(E\) is \underline{simple} if:

    \(E \neq 0\)

    No proper submodules, aka \(M \triangleleft E \implies M = 0\) or \(E\)

    In other words, \(E\) is a simple module if it only has two submodules: \(0\) and \(E\).

\end{definition}

eg simple \(\mathbb{R}\)-modules are 1 dim vector spaces, aka \(\mathbb{R}\) 

\begin{exercise}

    \begin{enumerate}[label=\alph*)]

        \item \(\mathbb{R}^2\) is a simple \(M_2(\mathbb{R})\)-module 

        \item Express \(M_2(\mathbb{R})\) as direct sum of simple modules. 
    \end{enumerate}

\end{exercise}

\section*{Friday, 8/30/2024}

\begin{exercise}

    Suppose finite \(G \neq 1\) and \(R \neq 0\) Prove that \(RG\) has zero divisors.

\end{exercise}

\begin{definition}
    Direct product of rings \(R \times S\), addition and multiplication is done componentwise.

    It is a product in the category of rings. aka:

    \begin{center}  
        \begin{tikzcd}
            & T \ar[ld, "f_1", swap] \ar[d, dotted, "f"] \ar[rd, "f_2"] & \\
            R & R \times S \ar[l, "\pi_1"] \ar[r, "\pi_2",swap] & S
        \end{tikzcd}
    \end{center}

    for any pair of ring homomorphisms \(T \overset{f_1}{\to} R\) and \(T \overset{f_2}{\to} S\) we have a unique ring homomorphism \(f: T \overset{f}{\to } R \times S\) so that the diagram commutes.

\end{definition}

\begin{definition}
    \(e\in R\) is an \underline{idempotent} if \(e^2 = e\). 

\end{definition}

\(0,1\) are trivial idempotents.

\(\begin{pmatrix}
    0 &  0 \\
    0 &  1 \\
\end{pmatrix}\) is an idempotent in \(M_2(\mathbb{R})\) 

\((0,1)\) is an idempotent in \(\mathbb{R} \times \mathbb{R} \) 

If \(e\) is an idempotent so is \(1 - e\) 

\begin{definition}
    Idempotent \(e\in R\) is central if \(\forall r\) we have \(er = re\) 
\end{definition}

\(\begin{pmatrix}
    0 &  0 \\
    0 &  1 \\
\end{pmatrix}\) is not central, but \((0,1)\) is.

\begin{exercise}

    A ring can be written as a product ring, aka \(R \cong R_1 \times R_2\) with \(R_i \neq 0\) if and only if there exists a nontrivial central idempotent.

\end{exercise}

\subsection*{Semisimiple Modules}

\begin{definition}
    \(E\) is a \underline{simple} \(R\)-module if it doesn't have any nontrivial submodules.

    If \(E \neq 0\) and \(M \triangleleft E\) then \(M \neq 0\) or \(M = E\) 
\end{definition}

\begin{example}

    \(R^2\) is a simple \(M_2\mathbb{R}\)-module.


    \(\mathbb{R} \times 0\) is a simple \(\mathbb{R} \times \mathbb{R}\) module.

    \(\mathbb{Z} / p\mathbb{Z}\) is a simple \(\mathbb{Z}\)-module

\end{example}

\begin{lemma}

    [Schur's Lemma]: Let \(E, F\) be simple \(R\)-modules. Then any nonzero homomoprhism \(f: E\to F\) is an isomorphism.

\end{lemma}

\begin{proof}
    \(f \neq 0\) means \(\ker f \neq E\) and \(\text{im } f \neq 0\).
    
    Since they are submodules, \(\ker f = 0\) and \(\text{im } f = F\) 

    So \(f\) is bijective.
\end{proof}

\begin{corollary}

    If \(E\) is simple, then \(\text{End}_R E\) is a skew field [any non-zero element is invertible]

\end{corollary}

\begin{example}

    Commutative example: \(\text{End}_{M_2\mathbb{R}}(\mathbb{R}^2)\) is a skew field.

    In fact, \(\text{End}_{M_2\mathbb{R}}(\mathbb{R}^2) \cong \mathbb{R}\) 

\end{example}



\begin{definition}[Direct Sum]
    Suppose \(M_i \triangleleft M\) for \(i \in I\) 

    Then, \(M = \bigoplus_{i\in I} M_i\) means, \(\forall m\in M_i\) we have \(m = \sum_{i\in I} m_i\) with \(m_i \in M_i\) \underline{uniquely}.

    There are notions of internal and external direct sums. The above is an internal direct sum.

    External direct sum: given \(\{ M_i \}_{i\in I}\) we can construct \(\bigoplus_{i\in I} M_i\) 

\end{definition}

\begin{proposition}
    [Universal Property] Given a collection of homomorphisms \(\{ t_i : M_i \to N \}_{i\in I}\), it extends directly to a homomorphism \(\bigoplus M_i \to N\). We denote this by \(\bigoplus f_i\)
\end{proposition}

 
\begin{remark}
    Note: Maps to product are easy, maps from direct sum are easy.    
\end{remark}

\begin{proposition}
    [1.2, Lang XVII]

    Suppose we have isomorphism \(E_1^{n_1} \oplus \dots \oplus E_r^{n_r} \overset{\cong }{\to } F_1^{m_1} \oplus \dots \oplus F_s^{m_s}\) with \(E_i\) and \(F_j\) simple and non-isomorphic [ie for all \(k \neq i, E_k \not \cong E_i\) and \(k \neq j, F_k \not\cong F_j\)  ]
    
    Then \(r = s\) and there exists a permutatation \(\sigma \in S_r\) so that \(E_j \cong F_{\sigma(j)}\) and \(n_j = m_{\sigma(j)}\) 
\end{proposition}

\underline{Corollary}: If \(E\) is a finite direct sum of simple modules, then the isomorphism class of simple components of \(E\) and multiplicities are well-defined.

\begin{proof}
    We use Schur's Lemma.

    We write \(\phi\) as a matrix \((\phi_{ji}: E_i^{n_i} \to F_j^{m_j})\)

    Since \(\phi\) is injective, for all \(i\) there exists a \(j\) such that \(\phi_{ji} \neq 0\)

    Then, \(E_i \cong F_j\) by Schur's Lemma

    Note that \(F_j\) are isomorphic. So, for all \(i\), the \(j\) such that \(\phi_{ji} \neq 0\) is unique!

    We also get \(\sigma:\{ 1, \dots , r \} \to \{ 1,\dots ,s \} \) so that \(\sigma(i)=j\) 

    Since \(\sigma ^{-1} \) exists\, \(\sigma ^{-1}\) exists, and thus \(r = s\) 

    Since \(\phi\) is an isomorphism, individual \(\phi_{ji} : E_i^{n_i} \to F_{\sigma(i)}^{m_{\sigma(i)}}\) are isomorphisms.

    To complete the proof, we need a lemma

    \underline{Lemma}: Let \(E\) be simple. If \(E^n \cong E^m\) then \(n = m\)
    
    Proof of lemma; Let \(D = \text{End}_R E\). By Schur's Lemma, \(D\) is a division ring.

    Since \(E^n \cong E^m\), we have \(\text{End}_R(E^n) \cong \text{End}_R(E^m)\)

    So, \(M_n(D) \cong M_m(D)\) 

    Also, isomorphism not just as rings, but also as \(D\)-modules.

    Every module over a skew field is free, and the number of dimensions is the same.

    So, \(n^2 = m^2 \implies n = m\) 

    This finishes the proof.

\end{proof}

\subsection*{Lang XVII section 2}

\begin{theorem}
    Let \(E\) be an \(R\)-module. Then TFAE:

    SS1: \(E\) is a sum of simple modules [so, we can write \(m\in E\) as sum of \(m_i\) but it is not unique]

    SS2: \(E\) is a direct sum of simple modules [we can write as a sum, and it's unique]

    SS3: Every submodule of \(E\) is a summand.

    \(F \triangleleft E \impliedby \) we can find \(F^{\prime}\) so that \(E = F \oplus F^{\prime}\) 

    SS3\(^{\prime} \)  : any monomorphism \(F \to E\) `splits'

    SS3\(^{\prime\prime} \) Short exact sequence

    \[
        0 \to F \to E \to H \to 0
    \]

    splits.

\end{theorem}

This leads us to:

\begin{definition}
    \(E\) is semisimple if it satisfies one of the above.
\end{definition}

Davies: SS2 is best

eg: \(R = \mathbb{R} \times \mathbb{R} \) 

\(E = \mathbb{R} \times \mathbb{R}\) is semisimple but not simple.

Because: \(E = \mathbb{R} \times 0 \oplus 0 \times \mathbb{R}\) 


\section*{Wednesday, 9/4/2024}

Recap: Semisimple modules.

\begin{lemma}
    If \(E = \sum_{i\in I} E_i\) with \(E_i\) simple. Then, \(\exists J \subset I\) such that \(E = \bigoplus_{j\in J} E_j\)
\end{lemma}

\begin{corollary}
    SS1 \(\implies\) SS2
\end{corollary}

\begin{proof}
    Let \(J \subset I\) be maximal such that \(\sum_{j\in J} E_j = \bigoplus_{j\in J} E_j\)

    This exists by Zorn's lemma.

    \(\forall i\in I - J\), we have \(E_i \cap \bigoplus_{j\in J}E_j \neq \varnothing\) by maximality.

    Since \(E_i\) is simple, \(E_i \subset \bigoplus_{j\in J}E_j\). Therefore, \(E = \bigoplus_{j\in J} E_j\).

\end{proof}

\underline{True of False}? Every module has a maximal proper submodule.

False!!! Exercise.

\begin{exercise}
    \begin{enumerate}[label=\alph*)]
        \item If \(M \triangleleft F\) proper and \(M\) maximal, then \(F / M\) is simple. 
        \item Find a ring \(R\), module \(M\) which does not have proper maximal submodules.
        \item If \(F\) is a finitely generated \(R\)-module, then it is contained in a proper maximal submodule.
    \end{enumerate} 
\end{exercise}

\begin{proof}
    [Proof of SS2 \(\implies\) SS3]

    Suppose \(F \triangleleft E = \bigoplus_{i\in I}E_i\) with \(E_i\) simple. Let \(J \subset I\) be maximal such that:

    \[
        F + \bigoplus_{j\in J} E_j = F \oplus \bigoplus_{j\in J} E_j
    \]

    Take any \(i \in I - J\). Then, \(E_i \cap \left[ F \oplus \bigoplus_{j\in J} E_j \right] \neq 0\) by maximality of \(J\).

    Since \(E_i\) is simple, \(E_i \subset F \oplus \bigoplus_{j\in J}E_j\).

    Therefore, \(E = F \oplus \underbrace{\bigoplus_{j\in J}E_j}_{F^{\prime}}\).

    We have found \(F^{\prime}\), which proves SS3.

\end{proof}

\begin{proof}
    [Proof of SS3 \(\implies\) SS1]

    \begin{lemma}
        \(0 \neq F \triangleleft E\) and \(E\) satisfies SS3. Then, there exists simple finitely generated \(S \triangleleft F\).
    \end{lemma}
    
    \underline{Plan}: \(M \underset{\not-}{\triangleleft} \underset{\text{f.g.}}{F_0} \triangleleft F \triangleleft E\).

    Then, choose \(0 \neq v\in F\). Let \(F_0 = Rv\).

    \begin{exercise}
        \(M\) exists. [Zorn's Lemma]
    \end{exercise}

    Let \(E = \sum_{\text{simple } S \triangleleft E} S\).

    Then, by SS3, \(E = E_0 \oplus E_0^{\prime}\).

    Lemma and definition of \(E_0\) implies: \(E_0^{\prime} = 0\). So, \(E\) is indeed a sum of simple \(R\)-modules. We're done!

\end{proof}

\begin{proposition}
    [2.2] Every quotient module and submodule of a semisimple modules is semisimple.
\end{proposition}

\begin{proof}
    Quotients: Suppose \(M = E / N\). We have surjective \(f: E \to M\) with \(E\) semisimple.

    SS1 implies \(E = \sum_{i\in I} S_i\) with \(S_i\) simple.

    Then, \(M = \sum_{i\in I} f(S_i)\)

    Schur's lemma implies \(f(S_i)\) is either \(0\) or simple, so \(M\) satisfies SS1.

    Submodules: Suppose \(F \triangleleft E\) with \(E\) semisimple. SS3 implies \(E = F \oplus F^{\prime}\). Thus \(E \cong E / F^{\prime}\), so it is semisimple by the quotient result.

\end{proof}

\underline{Preview}:

\begin{definition}

    A ring \(R\) is \underline{semisimple} if and only if all \(R\)-modules are semisimple.

    Lang defines semisimple differently: A ring \(R\) is semisimple if it is semisimple as an \(R\)-module.

\end{definition}

\begin{theorem}
    [Artin-Weddenburn Theorem]

    A ring is semisimple if and only if it is isomorphic to a finite product of matrix rings over division algebras:

    \[
        R \cong M_{n_1}(D_1) \times \cdots \times M_{n_k}(D_k)
    \]

\end{theorem}

\(\mathbb{C} G, \mathbb{R} G\) are semisimple. We also have the result:

\begin{theorem}
    [Maschke's Theorem] The group ring \(kG\) is semisimple if \(G\) is finite and \(k\) is a field of characteristic prime to \(G\).

    This also works with \(\operatorname{char} k = 0\). It is in fact an if and only if.
\end{theorem}

So \(\mathbb{F}_p G\) is also semisimple given \(p \nmid \vert G \vert\)

\begin{proof}
    Outline: let \(\vert G \vert = n\). We will verify SS3.

    Let \(F \triangleleft E\) be \(kG\) modules.
    
    \(k\) is a field, so there exists a \(k\)-linear projection \(\pi : E \to F\) such that \(\pi(f) = f\) for \(f\in F\) [take a basis of \(F\) as a \(k\)-vector space, complete it to a basis of \(E\)].

    Now, define an `average'.
    \[
        \pi ^{\prime}(e) = \frac{\sum_{g\in G} g \pi (g ^{-1} e)}{n}
    \]

    Then, \(\pi ^{\prime} : E \to F\) is a \(kG\)-linear projection, meaning \(\pi ^{\prime} (ge) = g \pi ^{\prime} (e)\).

    Then \(E = \underset{F}{\im \pi ^{\prime}} \oplus \underset{F^{\prime}}{\ker \pi ^{\prime} }  \) 

\end{proof}


\section*{Friday, 9/6/2024}

\subsection*{Lang XVII, Sectiion 3}

``Density Theorem''

Suppose \(R\) is a ring and \(E\) is a \(R\)-module. Then we have maps \(R \times E \to E\) by mutliplication on the left.

\begin{definition}
    [Commutant]

    \(R^{\prime} = R^{\prime} (E) = \End_R(E)\) is a ring. 

    \(\phi \in R^{\prime} \iff \phi : E \to E\) such that \(\phi(re) = r \phi (e)\). It `commutes with \(E\)'.

    Note that \(E \) is also an \(R^{\prime}\)-module, with \(R^{\prime} \times E \to E\) given by \((\phi , e) = \phi(e)\). 
\end{definition}

\begin{definition}
    [Double Commutant]

    We can iterate on the previous definition.

    \[
        R^{\prime\prime} = R^{\prime} (R^{\prime} E) = \End_{R^{\prime}}(E)
    \]
\end{definition}

Therefore,

\[
    R^{\prime\prime} = \End_{R^{\prime}}(E) = \End_{\End{R}(E)}(E)
\]

This means, \(f\in R^{\prime\prime} \iff f: E\to E, \forall \phi\in R^{\prime} , f \circ \phi = \phi \circ f\). So, things in \(R^{\prime\prime}\):

\begin{center}
    \underline{commute} with things which commute with \(r\in R\).
\end{center}

\begin{example}
    Suppose \(R = \mathbb{R}\) and \(E = \mathbb{R}^n\). Then,

    \[
        \mathbb{R} ^{\prime}  = \End_{\mathbb{R}}(\mathbb{R}^n) = M_n(\mathbb{R})
    \]

    \[
        \mathbb{R}^{\prime\prime} = \underset{rI}{\End_{M_n(\mathbb{R})}(\mathbb{R}^n)} \underset{\leftarrow}{=} \underset{r}{\mathbb{R}}
    \]

\end{example}

Suppose \(V\) = vector space.

\(V^{\ast} = \Hom(V,\mathbb{R})\) 

Then we have evaluation map \(ev: V \to V^{\ast} \) given by \(v \mapsto (\phi \mapsto \phi(v))\).

\(ev\) is 1-1.

\(ev\) is onto iff \(\dim V < \infty\).

With inspiration from this, we define,

\begin{definition}
    [Evaluation map] \(ev: R \to R^{\prime\prime}\) given b \(r \mapsto (e \mapsto re)\) 

    We define \(f_r : E \to E\) given by \(f_r = ev(r)\)
\end{definition}

\begin{proposition}
    \begin{enumerate}[label=\alph*)]
        \item \(f_r \in R^{\prime\prime} \)
        \item \(ev\) is a ring homomorphism. 
    \end{enumerate} 
\end{proposition}

\begin{proof}
    \begin{enumerate}[label=\alph*)]
        \item \(f_r(\phi(e))= r\phi(e) = \phi(re) \phi(f_r(e))\)
        \item \(ev(r + r^{\prime}) = ev(r) + ev(r^{\prime}), ev(1)=1\).
        
        \((ev(r))(ev(r^{\prime}))e = ev(r)(r^{\prime} e) = r r^{\prime} e = ev(r r^{\prime})e\) 
    \end{enumerate} 
\end{proof}

\begin{lemma}
    [3.1] Suppose \(E\) is semisimple over \(R\), \(e\in E\) and \(f\in R^{\prime\prime}\) 

    Then \(\exists r\in R\) such that \(re = f(e)\) [i.e. \(f(e) = ev(r)(e)\)]
\end{lemma}

\begin{proof}
    \(E\) is semisimple, and \(Re\) is a submodule. Therefore, we can write \(E = R e \oplus F\).

    Define \(\pi : E \to E\) be projection to \(R e\).

    Then \(\pi \in E^{\prime} \implies f\circ\phi = \pi \circ f \implies f(e)=f(\pi(e))=\pi(f(e))= re\) for some \(r\in R\).
\end{proof}

We will prove a stronger version of this lemma called the \underline{Jacobson Density Theorem}.

\begin{theorem}
    [3.2, Jacobson Density Theorem]

    Suppose \(E\) is semisimple over \(R\)
    
    \(e_1, \cdots e_n \in E\)

    \(f\in R^{\prime\prime} \)

    Then, \(\exists r\in R\) such that \(r e_i = f(e_i) \forall i\).

    Therefoe, if \(E\) is finitely generated over \(R^{\prime}\), then \(R \to R^{\prime\prime} \) is onto. 
\end{theorem}

\begin{proof}
    We use a \underline{diagonal trick}.

    \underline{Special Case}: \(E\) is simple.

    Idea: Apply the lemma on \(E\) with \(\underbar{e} = (e_1, \cdots, e_n)\) and \(f^n : E^n \to E^n\) such that \(f(y_1,\cdots, y_n) = (f(y_1),\cdots, f(y_n))\). 

    We need to check that \(f\in R^{\prime} (R^{\prime} (E))\) to apply it.
    
    This would imply that \(f^n \in R^{\prime}(M_n R) \underset{E\text{ simple}}{=} R^{\prime} (R^{\prime} (E^n))\)
    
    Therefore, \(\exists r\) such that \(r \underbar{e} = f^n(\underbar{e})\). This finishes the proof.

    For \(E\) semisimple, key idea is \(f^n \in R^{\prime} (R^{\prime} (E))\) as above. [Complicated for infinite sums. We avoid.]

\end{proof}

Application:

\begin{theorem}
    [Burnside's Theorem] Suppose \(k\) is an algebraically closed field.

    Take subring \(R\) such that \(k \subset R \subset M_n(k)\)
    
    If \(k^n (=E)\) is a simple \(R\)-module, then prove that:

    \[
        R = M_n(k)
    \]
\end{theorem}

\begin{exercise}
    Suppose \(D_{2n}\) is the dihedral group of order \(2n\), aka

    \[
        D_{2n} = \langle r, s \mid r^n = 1, s^2 = 1, s r s ^{-1} = r ^{-1} \rangle 
    \]

    Let \(\zeta _n = e^{2 \pi i / n} \in \mathbb{C}\)

    Then we can define a homomorphism \(D_{2n} \to GL_2(\mathbb{C})\) given by:

    \[
        r \mapsto \begin{bmatrix}
            \zeta _n &  0 \\
            0 &  \zeta_n ^{-1} \\
        \end{bmatrix}
    \]

    \[
        s \mapsto \begin{bmatrix}
            0 &  1 \\
            1 &  0 \\
        \end{bmatrix}
    \]

    This gives us a ring map \(\pi:\mathbb{C} D_{2n} \to M_2\mathbb{C}\)

    Prove the following:

    \begin{enumerate}[label=\alph*)]
        \item Prove that \(\mathbb{C}^2\) is a simple \(\mathbb{C} D_{2n}\) module [can be done without technology]
        \item Use Burnside's theorem to show that \(\pi\) is onto.
    \end{enumerate} 
\end{exercise}

Note that Burnside's theorem doesn't work if \(k\) is not algebraically closed.

We have:

\[
    \mathbb{R} \subset \mathbb{C} \subset M_2\mathbb{R}
\]

since we can embed \(\mathbb{C}\) into \(M_2\mathbb{R}\).

\(\mathbb{C}\) is a simple \(R\) module, but \(\mathbb{C} \neq M_2\mathbb{R}\) 

\begin{proof}
    [Proof of Burnside's Theorem]

    \underline{Step 1}: We show that \(\End_R(E) = k\)

    Note that, \(k \underset{\text{central}}{<}  \underset{\text{skew field}}{\End_R(E)} \subset \underset{\text{finite dim} / k}{\End_k(E)} \) 

    \(\forall \alpha \in \End_R(E), k(\alpha)\) is a field and finite dimensional \(/ k\).

    Therefore, \(k(\alpha) = k\) since \(k\) is algebraically closed.

    Thus, \(\alpha \in k\). This finishes Step 1.

    \underline{Step 2}: We show that \(R = \End_k(E)\).

    \(R \subset \End_k(E)\) by hypothesis.

    Suppose \(A\in \End_k(E)\). Let \(e_1, \cdots , e_n\) be a \(k\)-basis for \(E = k^n\).

    Density theorem implies: \(\exists r\in R\) such that \(A e_i = re_i\) for all \(i\).

    Therefore, \(A = r \in R\).

\end{proof}

\section*{Monday, 9/9/2024}

Today:

Density Theorem

Characters determine representation

Artin-Wedderburn Theorem

Homework due Monday 9/16, Exercises 1-7

Recall Jacobson Density Theorem:

If \(E\) is semisimple over \(R\), \(e_1, \dots, e_n \in E\) and \(f\in R^{\prime\prime}\) then,

\[
    \exists r\in R \text{ s.t. } f(e_i) = re_i \forall i
\]

Recall that \(R^{\prime\prime} \) is defined as follows:

\[
    f\in R^{\prime\prime} \iff f:E \to E \text{ s.t. } \forall \phi\in R^{\prime} =\End_{R} E, f \circ \phi = \phi \circ f
\]

Also recall Burnside's Theorem:

Suppose \(k\) is an algebraically closed field, and \(k \subset R \subset M_n(k)\) are subrings

If \(k^n\) is a simple \(R\)-module, then

\(R = M_n(k)\)

\subsection*{3.7 Existence of Projection Operators}

\begin{theorem}

    Suppose \(E = V_1 \oplus \cdots \oplus V_m\), simple non-isomorphic \(R\)-modules. Then, for any \(i\), there exists \(r_i\in R\) such that,

    \[
        r_i v = \begin{dcases}
            v, &\text{ if } v\in V_i ;\\
            0, &\text{ if } v\in V_j, i \neq j
            \end{dcases}
    \]

    So, each projection map is just multiplication.

\end{theorem}
    
\begin{proof}
    This is a consequence of the density theorem.

    Choose nonzero \(e_k \in V_k\).

    Let \(f = \pi_i : E\to E\) which is a projection on \(V_i\).

    Note that \(f\in R^{\prime\prime} \) since for all \(\phi \in R^{\prime} , \phi(V_k) \subset V_k\) [Schur's Lemma, non-isomorphic].

    Density theorem \(\implies \exists r_i \in R\) such that \(r_i e_k = \pi_i (e_k)\).

    Note that \(V_k = Re_k\) so \(\forall v\in V_k, v = r e_k\). 

    So, \(r_i v = r_i r e_k = r\pi_i(e_k) = \pi_i (r e_k) = \pi_i(v)\) 

    Which is what we wanted.

\end{proof}

\underline{Correction to the Existence of Projection Operators}

Suppose \(k\) is a field, \(R\) is a \(k\)-algebra so that \(R\) is semisimple. Suppose \(R\)-module \(E = V \oplus V^{\prime}, \dim_k E < \infty\).

For all simple \(L \triangleleft V, \forall L^{\prime} \triangleleft V^{\prime}\) then \(L \cong L^{\prime}\) 

Then, \(\exists r\in R\) such that for all \(e \in E\),

\[
    r e = \begin{dcases}
        e, &\text{ if } e \in V ;\\
        0, &\text{ if } e\in V^{\prime}  ;
    \end{dcases}
\]

\begin{proof}
    We apply density theorem. Since we have finite dimension, we have:

    \[
        \{ e_1, \cdots , e_n \} = (k\text{-basis of } V) \cup (k\text{-basis of } V^{\prime})
    \]

    Let \(\pi_V : E \to E\) be the projection on \(V\).

    Then, \(\pi_V \in R^{\prime\prime}\) [the second commutant] since \(\forall \phi \in R^{\prime}, \phi (v) \subset V, \phi(v^{\prime}) \subset V^{\prime}\).

    Density theorem implies \(\exists r\) such that \(r e_i = \pi_v(e_i)\).

    Then \(\forall a\in k \subset \) center \(R\),

    \(r (a e_k) = a (r e_k) = a \pi_v (e_k) = \pi_v (a e_k)\) 

    Therefore, \(r e = \pi _v(re)\).
\end{proof}

\underline{Question}: What is a \(k\)-algebra?

Following Atiyah-McDonald, let \(k\) be a commutative ring [often but not always a field]. Then,

\[
    R \text{ is a } k \text{-algebra } \overset{\text{def}}{\iff} \text{homomorphism } h: k \to R, h(k) \subset \text{center}(R)
\]

\begin{example}
    Any ring is a \(\mathbb{Z}\)-algebra, homomorphism sends \(n\) to \(1+1+\cdots+1\) 

    \(k\) field, \(R \neq 0 \implies k \hookrightarrow R\)
    
    \(k\)-algebra \(\iff\) \(k \subset \text{center}(R)\) 
\end{example}

\begin{corollary}
    [3.8]

    Suppose \(\Char k = 0\), \(R\) is a \(k\)-algebra, \(E,F\) semisimple over \(R\), finite dimensional over \(k\).
    
    For \(r\in R\), let:

    \(f_r^E : E \to E\) be \(f_r^E(e)=re\) 

    \(f_r^F : F \to F\) be \(f_r^F(f)=rf\) 

    If \(\Tr(f_r^E)=\Tr(f_r^F)\) for all \(r\in R\),
    
    Then \(E \cong F\) as \(R\)-modules.
\end{corollary}

\begin{proof}
    Let \(V\) be a simple \(R\)-module.

    Suppose \(E = V^n \oplus \text{direct sum of simple \(R\)-modules not isomorphic to \(V\)} \)

    \(F = V^m \oplus \text{direct sum of simple \(R\)-modules not isomorphic to \(V\)}\) 

    We want to show \(n = m\) 

    Let \(r_v \in R\) be the projection operation from 3.7.

    Then, \(\Tr(f_{r_v}^E) = \Tr(r_v\cdot: E \to E) = \dim_k V^n = n \dim_k V\) 

    Similarly, \(\Tr(f_{r_v}^F) = m \dim_k V \implies n = m\) 

\end{proof}

\begin{corollary}
    [Characters determine representations]

    Suppose \(k\) is a field and \(\Char k = 0\). Let \(G\) be a finite group. Suppose:

    \(\rho: G \to GL_n(k)\) 

    \(\rho ^{\prime} : G \to GL_m(k)\) 

    with \(kG\)-modules \(E = k^n\) over \(\rho\) and \(F = k^m\) over \(\rho ^{\prime}\) 

    If \(\Tr (\rho (g))= \Tr(\rho ^{\prime} (g))\) for all \(g\),

    Then \(E \cong F\) as \(kG\)-modules. 

    Note that, substituting \(g = 1\) gives us:

    \(\Tr(\rho(1))=\Tr(\rho^{\prime} (1)) \implies \Tr(I) = \Tr(I) \implies n = m\). 

\end{corollary}

\begin{definition}
    [(semi)simple rings]

    Note that if \(R\) is a ring, then \(R\) is a left module as well. We write \(_R R\) when we're considering it as a left module, and \(_R R _R\) when we are considering a two sided ideal.

    \(R\) is called a \underline{semisimple ring} if \(_R R\) is a semisimple \(R\)-module.

    \(R\) is called a \underline{simple ring} if \(R\) is a semisimple ring, and for all simple \(L, L^{\prime} \triangleleft_R R \implies L \cong L^{\prime}\) 

    This means, \(_R R = \oplus_{i\in I} L_i\) where \(L_i\) are simple (left) ideals such that \(L_i \cong L_j\) for all \(i,j\).

    Recall that an ideal is simple if it has no proper sub-ideals.

\end{definition}

\begin{example}
    \(M_2(\mathbb{H})\) is a simple ring. We can write it as direct sum of two ideals

    \[
        \begin{bmatrix}
            \ast &  0 \\
            \ast &  0 \\
        \end{bmatrix} \oplus \begin{bmatrix}
            0 &  \ast \\
            0 &  \ast \\
        \end{bmatrix}
    \]
\end{example}


\begin{example}
    \(M_2(\mathbb{H}) \times \mathbb{R}\) is semisimple.

    \[
        \begin{bmatrix}
            \ast &  0 \\
            \ast &  0 \\
        \end{bmatrix} \times 0 \oplus \begin{bmatrix}
            0 &  \ast \\
            0 &  \ast \\
        \end{bmatrix} \times 0 \oplus \begin{bmatrix}
            0 &  0 \\
            0 &  0 \\
        \end{bmatrix} \times \mathbb{R}
    \]
\end{example}

Artin-Wedderburn generalizes this.

\begin{theorem}
    [Artin-Wedderburn Theorem]

    \begin{enumerate}[label=\roman*)]

        \item   \(R\) simple \(\iff  R\cong M_n(D)\) where \(D\) is a skew-field.

        \item \(R\) semisimple \(\iff R \cong R_1 \times \cdots \times R_s\) simple rings.
    \end{enumerate}
\end{theorem}


\section*{Wednesday, 9/11/2024}

Today we discuss the Artin-Wedderburn Theorem.

\underline{Exercise}: \(C_2 = \{ 1, g \}\), prove that \(\mathbb{Q} C_2\) is a semisimple ring.

\(\mathbb{Q} C_2 = B_1 \oplus B_2\) 2-sided ideals

\(\mathbb{Q} C_2 \cong \mathbb{Q} \times \mathbb{Q}\). 

\begin{lemma}
    Suppose we have a ring \(R\) which is decomposed as a sum of (left) ideals:

    \[
        _R R = \bigoplus_{i\in I} L_i \quad \text{with \(L_i \neq 0\)} 
    \]

    Then \(\vert I \vert < \infty\).
\end{lemma}

\begin{proof}
    Suppose \(_R R = \bigoplus_{j\in J} L_j\) where \(L_j\) are ideals. We want to prove that only finitely many are non-zero.

    Note that, \(1 = \sum_{j\in J} e_j\). We use only finitely many elements here, so \(1 = \sum_{i\in I} e_i\) where \(e_i \neq 0, I \subset J, \vert I \vert < \infty\).

    For all \(r\in R\) we have \(r = r \cdot 1 = r \sum_{i\in I} e_i = \sum_{i\in I} r e_i \in \sum_{i\in I} L_i\).
    
    Therefore, \(_R R = \bigoplus_{i\in I} L_i\) a finite sum!
\end{proof}

Now we go to the theorem.

\begin{proof}
    [Proof of Artin-Wedderburn Theorem Part I]

    We want to prove: \(R\) simple ring \(\iff R \cong M_n D\) where \(D\) is a skew field.

    First, note that \(_R R \cong L^n\) where \(L\) is a simple ideal [so no proper sub-ideals]. Therefore,

    \[
        R^{op} \cong  \End_R (_R R) \cong \End_R (L^n) \cong M_n(\underbrace{\End_R L}_{\text{ division ring} })
    \]

    Taking transpose,

    \[
        R \cong M_n(\End_R L)^{op} \cong M_n((\End_R L)^{op}) =M_n(D)
    \]

    So we are done with one direction!

    The other direction is a exercise. Here are the steps:

    \underline{Step 1}: \(M_n D = \begin{bmatrix}
        \ast & 0 & \cdots &  0 \\
        \ast & 0 & \cdots &  0 \\
        \vdots & \vdots & \ddots &  \vdots \\
        \ast & 0 & \cdots &  0 \\
    \end{bmatrix} \oplus \cdots \oplus \begin{bmatrix}
        0 & 0 & \cdots &  \ast \\
        0 & 0 & \cdots &  \ast \\
        \vdots & \vdots & \ddots &  \vdots \\
        0 & 0 & \cdots &  \ast \\
    \end{bmatrix}\) 

    \underline{Step 2}: Each summand is isomorphic to \(D^n = \begin{bmatrix}
         \ast \\
         \ast \\
         \vdots \\
         \ast \\
    \end{bmatrix}\) 

    \underline{Step 3}: \(D^n\) is a simple module.

\end{proof}

\begin{remark}
    \(R\) simple \(\iff\) \(R\) artinian, \(R\) has no proper \(2\)-sided ideals. Some definitions forgo the artinian condition, in this case these are called artinian simple rings.
\end{remark}

\begin{lemma}
    [4.2] Suppose \(L\) is a simple ideal and \(M\) is a simple module so that \(L \not\cong M\). Then \(LM = 0\).
\end{lemma}

\begin{proof}
    This is a direct consequence of Schur's lemma. Consider the map \(\phi_m: L \to M\) given by \(l \mapsto lm\) for \(m\in M\). Since this can't be an isomorphism, it must be the zero map. Thus, \(lm = 0\).
\end{proof}

\begin{proof}
    [Proof of Artin-Wedderburn Theorem Part II]

    \underline{Idea}: Decompose \(R\) as direct sum of simple ideals. Partition the set of simple ideals so that members of a partition are isomorphic to each other, members of a partition are not isomorphic to members of  another partition. Direct sum of each partition gives us one \(R_j\). 

    Suppose \(R\) is semisimple.
    
    Let \(L_1, \cdots , L_s\) be a set of pairwise non-isomorphic simple ideals [meaning \(L_i \not\cong L_j\)]

    So that, for all simple \(L < _R R, L \cong L_i\) for some \(i\).

    Let \(B_i = \sum_{L \cong L_i} L\).

    \underline{Claim}: \(B_i\) is a \(2\)-sided ideal.

    \underline{Proof of Claim}: 

    \[
        B_i R \underset{4.2}{=} B_i B_i \subset R B_i \underset{B_i \text{ is a left ideal}}{=} B_i 
    \]

    Thus the claim is proven.

    \underline{Claim}: We have a `block decomposition of \(R\)', meaning,

    \underline{Proof of Claim}:

    \[
        _R R _R = B_1 \oplus \cdots \oplus B_s
    \]

    \underline{Subclaim}: \(B_i \cap \sum_{j \neq i} B_j = 0\)
    
    \underline{Proof of Subclaim}: Every \(r\in R\), we have that \(r \in L\) where \(L\) is simple. \(L \subset B_i \implies L \cong L_i\). \(L \subset \sum_{j\neq i} B_j \implies L \cong B_j\) for some \(j \neq i\) which is not possible.

    Now, we go back to the main proof.

    We can write \(1 = e_1 + \cdots + e_s\).
    
    Then, \(R_i \coloneqq (B_i, e_i)\) is a ring!

    We have \(R \cong (R_1, e_1) \times \cdots \times (R_s, e_s)\), so we're done.

    The other direction is an exercise.

\end{proof}


\section*{Friday, 9/13/2024}

Key idea:

\[
    _R R = L^n \implies \End_R R \cong M_n(\End_R L)
\]

Note that \(R^{op} \cong \End_R R\) [function composition is written in the opposite direction].

Suppose \(L_1, \cdots , L_s\) are non-isomorphic simple \(R\)-ideals.

\(L\) simple \(\implies L \cong L_i\). 

Define \(B = \sum_{\text{simple } L \cong L_i} L \triangleleft _R R _R\).

We can prove that it is a two sided ideals. 

Then we can write \(R \cong R_1 \times \cdots R_s\) simple, where 

\(R_i = (B_i, e_i)\) [\(e_i\) is the identity in \(B_i\)]. 

\begin{theorem}
    [4.4] Suppose \(E\) is a \(R\)-module.

    \[
        E_i \coloneqq \sum_{\stackrel{\text{simple } M \triangleleft E}{M \cong L_i}} M 
    \]

    Then, \(E = \bigoplus_{i=1}^s E_i\)
    
    \(E_i = e_i E = B_i M\).
\end{theorem}

\begin{corollary}
    [4.5] If \(R\) is semisimple, \(M\) a simple \(R\)-module, then \(M \cong L_i\) for some \(i\).
\end{corollary}

\begin{corollary}
    [4.6] All simple modules of a simple ring are isomorphic.

    \[
        M \cong \oplus L
    \]
\end{corollary}

\subsection*{External Product vs. Internal Product}

\begin{definition}[External Product]
    If we have [finite] rings \(R_1, \cdots , R_s\) we can construct the ring:

    \[
        R_1 \times R_2 \times \cdots \times R_s
    \]
\end{definition}

\begin{definition}
    [Internal Product]

    `Block Decomposition': If we have a ring \(R\) and we can write it as sum of 2 sided ideals:

    \[
        _R R _R = B_1 \oplus \cdots \oplus B_s
    \]

    Then we have \(e_j \in B_j\) so that:

    \[
        1 = e_1 + \cdots + e_s
    \]

    Then, each \(B_j\) has a ring structure with \(e_j\) as identity. Then,

    \[
        R \cong (B_1, e_1) \times \cdots \times (B_s, e_s)
    \]
\end{definition}

Just for clarity:

\begin{definition}
    [Direct Sum of Ideals]

    \[
        _R R _R = B_1 \oplus \cdots \oplus B_s
    \]

    If and only if for every \(r\in R\),

    \[
        r = b_1 + \cdots + b_s
    \]

    where \(b_j \in B_j\) and the expression is unique.
\end{definition}

\underline{Jim's Rant}: A subring has to have the same identity. So, \((B_j, e_j)\) is \underline{not a subring}. 

Block Decomposition is \underline{not a direct sum of rings}!

This is because in category theory, sum refers to the co-product. 

\begin{lemma}
    Let \(k\) be a field, and let \(D\) be a skew-field which is a \(k\)-algebra such that \(\dim_k D < \infty\). Then,

    \begin{enumerate}[label=\alph*)]
        \item \(\forall \alpha \in D\) we have \(k[\alpha]\) is a field.
        \item \(k\) algebraically closed \(\implies D = k\). 
    \end{enumerate} 
\end{lemma}

\begin{example}
    If \(k\in\mathbb{R}, D = \mathbb{H}, \alpha \in \mathbb{H} - \mathbb{R} \) then \(k[\alpha] \cong \mathbb{C}\). 
\end{example}

It is not completely obvious since \(k[i+j] \cong \mathbb{C}\) as well.

\begin{proof}
    \begin{enumerate}[label=\alph*)]
        \item \(D\) is a \(k\)-algebra. Therefore, \(k[\alpha]\) is commutative. We just need to find inverse.
        
        Let \(0 \neq \beta \in k[\alpha]\). It is enough to prove that for \(\beta \in k[\alpha]\), multiplication map \(\cdot\beta : k[\alpha] \to k[\alpha]\) is bijective.

        \(\cdot \beta\) is a finite dimensional linear transformation so those are true.

        \item For all \(\alpha \in D\) we have: \(k[\alpha]=k\) since \(k\) is closed. So, \(\alpha \in K\). Thus \(D = k\). 
    \end{enumerate} 
\end{proof}

\begin{corollary}
    Suppose \(G\) is finite. Then,

    \[
        \mathbb{C} G \cong \prod_{i=1}^s M_{n_i}(\mathbb{C})
    \]
\end{corollary}

\begin{proof}
    Artin-Wedderburn Theorem plus the previous lemma.
\end{proof}

\begin{example}
    Suppose \(C_n = \langle g \rangle \) cyclic and \(\zeta_n = e^{2\pi i / n}\). Then,

    \(\mathbb{Q} C_2 \cong \mathbb{Q}_+ \times \mathbb{Q}_-\) where \(g \mapsto (1,-1)\). 

    If \(p\) is prime we can write:

    \(\mathbb{Q} (C_p) \cong \mathbb{Q} \times \mathbb{Q}(\zeta_p)\) where \(g \mapsto (1,\zeta_p)\). 
    
    \(\mathbb{C} [C_n] \cong \mathbb{C}^n\) where:

    \(g \mapsto (1, \zeta_n, \cdots , \zeta_n^{n-1})\) 

    \(\mathbb{Q} [C_2 \times C_2] \cong \mathbb{Q}^4\) where:

    \[
        (1,g) \mapsto (1,1,-1,-1)
    \]

    \[
        (g,1) \mapsto (1,-1,1,-1)
    \]

    \(\mathbb{R}[Q_8] \cong \mathbb{R} \times \mathbb{R} \times \mathbb{R} \times \mathbb{R} \times \mathbb{H}\) where \(\mathbb{R} [Q_8] \rightarrowtail \mathbb{R} [C_2 \times C_2]\)
    
    Some other examples: \(\mathbb{Q} [C_n], \mathbb{C} [Q_8], \mathbb{Q} [D_{2n}], \mathbb{R} [D_{2n}], \mathbb{C} [D_{2n}]\) 
\end{example}

\subsection*{Representation Theory}

Here, \(G\) is a finite group and \(k\) is a field.

\begin{table}[H]
    \centering
    \begin{tabular}{c|c|c}
        \toprule
            Representations & Modules over \(k G\) &  Characters \\
        \midrule
            \(\rho : G \to GL(V)\) where  &  &   \\
            \(V\) is a finite dimensional & \(V\) is a \(kG\) module &  \(\chi : G \to k, \chi_{\rho}(g) \Tr \rho (g)\) \\
            vector space &  &   \\
             &  &   \\
        \bottomrule
    \end{tabular}
    \caption{Representations, Modules and Characters}
    \label{tab:reptable}
\end{table}


\section*{Monday, 9/16/2024}

We have:

\begin{center}
    representation \(\iff\) modules over \(kG \implies\) [\(\impliedby\) only if \(\Char k = 0\)] characters.
\end{center}

rep \(\to\) \(kG\)-module 

\(\rho \mapsto V_\rho\) by \((\sum_{g} a_g g)v \coloneqq \sum_{g} a_g \rho (g) v\) 

\(\rho_v \leftarrow V\)

\(\rho_V(g)v \coloneqq gv\) 

Recall the definition of character:

We have the trace map:

\[
    \Tr: M_n k \to k
\]

Where \(\Tr(a_{ij})=\sum_{j} a_{j j}\) [or the sum of eigenvalues]

We have \(\Tr(AB)=\Tr(BA)\) which implies \(\Tr(P A P ^{-1}) =\Tr(A)\). 

So, \(\Tr\) is basis independent. Thus,

\[
    \Tr:\End_k V \to k
\]

\begin{definition}[character]
    Trace is an endomorphism map. This gives us:

    \begin{center}
        \begin{tikzcd}
            G \ar[r,"\rho"] \ar[rr, bend right, "\chi_p",swap] & GL(V) \ar[r,"\Tr"] & k
        \end{tikzcd}
    \end{center}

    This is called the \underline{character of \(p\)} 
\end{definition}

There's a correspondence between \(kG\) modules and Representations concepts:

\begin{table}[H]
    \centering
    \begin{tabular}{c|c}
        \toprule
            Repesentations &  Modules over \(kG\)  \\
        \midrule
            irreducible &  simple \\
             &  isomorphism \\
             &  direct sum \\
             &  Hom \\
             &  dual \\
             &  tensor product \\
        \bottomrule
    \end{tabular}
    \caption{Rep and kG-mod}
    \label{tab:repmod}
\end{table}

\underline{Irreducible vs Simple} 

We say irreducible representation, when we on the other hand say simple modules. Same concept!

\underline{Isomorphism} 

Suppose we have two representations:

\(\rho : G \to GL(V)\) 

\(\rho ^{\prime} : G \to GL(V^{\prime})\) 

We say two representations are isomorphic when:

\[
    \rho \cong \rho ^{\prime} \overset{def}{\iff} V_{\rho} \overset{\phi}{\cong} \underset{\phi(gv)=g \phi(v)}{V_{\rho} \overset{\phi}{\cong} V^{\prime} _{\rho ^{\prime}} \iff \exists k \text{ isomorphism s.t.} } 
\]

\(\phi : V \to V^{\prime}\) s.t. \(\forall g\in G\) we have the following commutative diagram:

\begin{center}
    \begin{tikzcd}
        V \ar[r,"\rho(g)"] \ar[d, "\phi"] & V\ar[d,"\phi"] \\ V' \ar[r, "\rho^{\prime}(g)"] & V^{\prime}
    \end{tikzcd}
\end{center}

\(\phi\) is called the intertwining map. 

\begin{corollary}
    \(\rho \cong \rho ^{\prime} \implies \chi_{\rho} =\chi_{\rho^{\prime} }\) 
\end{corollary}

\underline{Direct Sum}

Suppose \(V \oplus W\) is a \(kG\)-module.

\[
    \rho_{V \oplus W} : G \to GL(V \oplus W)
\]

is given by:

\[
    \rho_{V \oplus W} = \begin{bmatrix}
        \rho_V &  0 \\
        0 &  \rho_W \\
    \end{bmatrix}
\]

We also have \(\chi_{V \oplus W} = \chi_V + \chi_W\).

\underline{Two Representations}

\begin{definition}
    [Trivial Representations]

    \[
        \rho : G \to GL(k)
    \]

    \[
        g \mapsto 1
    \]

    Is the trivial representation. Also, \(\chi_{\rho} \equiv 1\).
\end{definition}

\begin{definition}
    [Regular Representation]

    Consider the \(kG\)-module \(_{kG}kG\). We have:

    \[
        \rho_{kG} : G \to GL(kG)
    \]

    This is injective.
\end{definition}

Note that \(G \acts G\) by multiplication, this is a free action. For finite group \(G\) with \(\vert G \vert = n\),

\(G \rightarrowtail \text{ Bijection}(G,G)\) so \(G\) is a subgroup of \(S_n\). So we have:

\begin{center}
    \begin{tikzcd}
        G \ar[rr, bend left, dotted, "\text{regular rep.}"] \ar[r] & S_n \ar[r] & GL(k^n)
    \end{tikzcd}
\end{center}

With the action of `permuting the standard basis'.

\underline{Exercise}: Compute character of Regular Representation.

We have, in line of the previous theorem:

\begin{theorem}
    [Maschke's Theorem] If \(V \subset W\) as \(kG\)-modules and \(\Char k \nmid \vert G \vert\) then \(\exists V^{\prime}\) such that \(W = V \oplus V^{\prime}\) 
\end{theorem}

\begin{proof}
    First, find a \(k\)-linear map \(\pi : W \to V\) such that \(\pi (v) = v\) for all \(v\in V\).

    We average it to make it \(kG\)-linear:

    \(\pi ^{\prime} : W \to V\) given by:

    \[
        \pi ^{\prime} (w) \coloneqq \frac{\sum_{g} g \pi (g ^{-1} w)}{\vert G \vert }
    \]

    We have: \(\pi ^{\prime} \) is \(kG\)-linear and \(\pi ^{\prime} (v)=v\) 

    We can take \(V^{\prime} \coloneqq \ker\pi\) 
\end{proof}

Thus, for \(w\in W\) we can write \(w = \pi ^{\prime} (w) + (w-\pi^{\prime} (w))\).

Note that Maschke's theorem implies \(kG\) is semisimple. Artin Wedderburn implies semisimple \(kG\) module is a direct sum of irreducible modules.

\[
    V \cong \bigoplus_{i} n_i V_i
\]

\[
    \chi_V = \sum_{i} n_i \chi_i
\]

\underline{Homomorphisms}:

Suppose \(V,W\) are \(kG\)-modules, "representations". Then,

\(\Hom_{kG}(V,W)\) is a \(k\)-vector space.

\(\Hom_k(V,W)\) is a \(kG\)-module.

we define: \((gf)v \coloneqq g f(g ^{-1} v)\) 

i.e. \(((\sum_{g} a_g g)f)v = \sum_{g} a_g (g f (g ^{-1} v))\) 

The \(g ^{-1} \) inside is needed for associativity: \((g^{\prime} g)f = g^{\prime} (gf)\) 

Officially this is a functor.

\(\Hom_k(-,-):(kG\text{-mod})^{op} \times kG\text{-mod}\to kG\text{-mod}\) 

Special case:

Dual Representation: \(W = k\). Then,

\(V^{\ast} = \Hom_k(V,k)\).

So, \((gf)(v)=gf(g ^{-1} v) = f(g ^{-1} v)\)

\underline{Exercise}: \(\chi_{V^{\ast}}=?\) 


\section*{Wednesday, 9/18/2024}

\subsection*{Tensor Products}

\underline{Motivation}:

Product Structure: \(- \otimes -\): \(kG\)-mod \(\times kG\)-mod \(\to kG\)-mod given by \(V \otimes_k W\).

Group action works diagonally, \(g(x \otimes y) = (gx) \otimes (gy)\), extended linearly.

Extension of scalars:

\[
    \mathbb{R} G \otimes _ \mathbb{R} \mathbb{C} = \mathbb{C} G
\]

Product of Groups: \(k[G \times H] = kG \otimes _k kH\) 

When for \(k\) a field then modules are vector spaces \(k^m\) and \(k^n\) which are easy:

\[
    k^n \otimes_k k^m = k^{nm}
\]

\[
    \dim(k^n \otimes_k k^m) = mn
\]

\(\{ e_i \} \) a basis for \(k^n\) 

\(\{ f_j \} \) a basis for \(k^m\) 

Then \(\{ e_i \otimes f_j \} \) is a basis for \(k^n \otimes k^m\).

However, tensor product consists of more than `pure' tensors.

\begin{definition}
    [Tensor Product] Let \(R\) be a \underline{commutative} ring. Tensor product is a functor:

    \[
        - \otimes_R - : R-\text{mod} \times R-\text{mod} \to R-\text{mod} 
    \]

    \[
        (A,B) \mapsto A \otimes_R B
    \]

    [Functor meaning if we have homomorphism on the left we will have homomorphisms on the right]

    \underline{Construction}:

    Let \(F(A \times B)\) be the free \(R\)-module with basis \(A \times B\). Then a typical element of the basis is \((a,b)\in A \times B\). 

    Let \(S\) be the sub-module generated by the following:

    \begin{enumerate}[label=\arabic*)]
        \item \((a_1 + a_2, b) - (a_1,b) - (a_2,b)\)
        \item \((a, b_1 + b_2) - (a,b_1) - (a,b_2)\)
        \item \(r(a,b)-(ra,b)\) 
        \item \(r(a,b)-(a,rb)\) 
    \end{enumerate} 

    Then, we define:

    \[
        A \otimes _R B \coloneqq \frac{F(A\times B)}{S}
    \]

    and write \(a \otimes b\) for the image of \((a,b)\). 

    This means, a typical element of \(A \otimes _R B\) is:

    \[
        \sum_{i=1}^n a_i \otimes b_i \in A \otimes _R B 
    \]

    We also have the following relations:

    \((a_1 + a_2) \otimes b = a_1 \otimes b + a_2 \times b\) 

    \(a \otimes (b_1 + b_2) = a \otimes b_1 + a \otimes b_2\) 

    \(r(a \otimes b) = (a \otimes rb) = (ra \otimes b)\) 
\end{definition}

\begin{exercise}
    \(\mathbb{Z} / 2 \otimes _ \mathbb{Z} \mathbb{Z} / 3 = 0\) 
\end{exercise}

\begin{proposition}
    Suppose \(A,B,M\) are \(R\)-modules, and

    \[
        \phi : A \times B \to M \text{ is \(R\)-billinear} 
    \]

    Meaning,

    \begin{enumerate}[label=\arabic*)]
        \item \(\phi(a_1 + a_2,b) = \phi(a_1,b) + \phi(a_2,b)\)
        \item \(\phi(a,b_1 + b_2) = \phi(a,b_1)+\phi(a,b_2)\)
        \item \(r\phi(a,b) = \phi(ra,b) = \phi(a,rb)\)  
    \end{enumerate}
    
    Then, by definition,

    \[
        \pi: A \times B \to A \otimes _ R B
    \]

    is \(R\)-bilinear. 
\end{proposition}

\begin{proposition}
    [Universal Property of Tensor Product]

    \(\pi\) is initial in the category of bilinear maps with domain \(A \times B\). Meaning, every bilinear map from \(A \times B\) factors through \(\pi\). 


    \begin{center}

    \begin{tikzcd}
        A \times B \ar[r,"\forall\phi\text{ bilinear}"] \ar[d,"\pi"] & M \\ A \otimes_R B \ar[ur, dotted, "\exists!\overline{\phi}", swap]
    \end{tikzcd}

    This diagram commutes

    \end{center}

\end{proposition}

\begin{proof}
    For uniqueness, note that, \(\overline{\phi}(a \otimes b) = \overline{\phi}(\pi(a,b))=\phi(a,b)\) 

    For existence, define \(\hat{\phi} (a,b) = \phi (a,b)\) where \(\hat{\phi} : F(A \times B) \to M\). Then \(\overline{\hat{\phi}}(S) = 0\) so \(\overline{\phi}: A \otimes _R B \to M\) exists.  
\end{proof}

\begin{proposition}
    [Rephrasing Universal Property in Terms of Adjoint Functors]

    \[
        \Hom(A \otimes B, C) \cong \Hom(A, \Hom(B,C))
    \]

\end{proposition}

\begin{proof}
    \[
        f \mapsto (a \mapsto (b \mapsto f(a \otimes b)))
    \] 

    \[
        (a \otimes b \mapsto g(a)b) \leftarrow g
    \]

    \begin{center}

        \begin{tikzcd}
            R\text{-mod} \ar[rr, bend right, "\Hom(A\text{,}\Hom(-\text{,}C))", swap] & & R\text{-mod} \ar[ll, bend right, "\Hom(A\otimes-\text{,}C)", swap]
        \end{tikzcd}

    \end{center}

\end{proof}

\begin{proposition}
    \begin{enumerate}[label=\arabic*)]
        \item Commutative \(A \otimes _ R B \cong B \otimes _ R A\) 
        \item Identity \(R \otimes _ R B \cong B\) 
        \item Assocative \((A \otimes B) \otimes C \cong A \otimes (B \otimes C)\) 
        \item Distributive \(\left( \bigoplus_{\alpha} A_\alpha  \right) \otimes B \cong \bigoplus_{\alpha} (A_\alpha \otimes B)\)  
        \item Functorial \(\dbinom{f:A \to A^{\prime}}{g:B\to B^{\prime}} \implies  \quad f \otimes g : A \otimes B \to A^{\prime} \otimes B^{\prime} \) 
        \item Exactness Short Exact Sequence \(0 \to A \overset{f}{\to} B \to C \to 0 \implies\) Short Exact Sequence \(0 \to A \otimes M \overset{f \otimes 1_M}{\to} B \otimes M \to C \otimes M \to 0\) 
        \item Right Exactness \(M\) \(R\)-mod,\(0 \to A \to B \to C \to 0 \implies\) Exact Sequence \(A \otimes M \to B \otimes M \to C \otimes M \to 0\)  
    \end{enumerate} 
\end{proposition}

\section*{Friday, 9/20/2024}

\subsection*{Lang Section 2}

\underline{Tensor Product of Representation} 

Suppose \(V,W\) are \(k\)-vector spaces, then we have \(V \otimes _ k W\) is also a \(k\)-vector space. But they all are \(kG\)-modules as well:

\[
    g(v \otimes w) = gv \otimes gw
\]

\begin{proposition}
    The character is multiplicative:

    \[
        \chi _{v \otimes w} = \chi _v \chi _w
    \]
\end{proposition}

\begin{proof}
    Let \(\{ e_i \} \) be a basis for \(V\) and \(\{ f_j \} \) a basis for \(w\).

    Suppose \(g e_i = \sum_{k} a_{ki} e_k\) 

    And \(g f_j = \sum_{l} b_{lj} f_l\)
    
    Then, \(g(e_i \times f_j) = ge_i \times gf_j = \sum_{k,l} e_{ki}b_{lj} e_k \times f_l\)
    
    Take \((k,l) = (i,j)\).

    Then, \(\chi_{v \times w} (g) = \sum_{i,j} a_{i i} b_{j j} = \chi_v(g) \chi_w(g) \) 
\end{proof}

Consider \(f : G \to k\). We have:

\{1d chars\} \(\subset\) \{simple chars\} \(\subset\) \{chars\} \(\subset\) \{virtual chars\} \(\subset\) \{class functions\}

We explain these later.

\begin{definition}
    \(f\) is a \underline{character} if \(\exists \rho : G \to GL_k(V)\) such that \(f = \chi_{\rho} = \Tr \circ \rho\) 
\end{definition}

\begin{definition}
    \(f\) is a \underline{class function} if \(\forall g, h \in G\) we have \(f(h g h ^{-1}) = f(g)\)
\end{definition}

\begin{definition}
    \(f\) is a virtual character if \(\exists \rho , \rho ^{\prime} \) such that \(f = \chi _ \rho - \chi _ {\rho ^{\prime} }\) 
\end{definition}

\begin{definition}
    \(f\) is \underline{simple} (=irreducible) character if \(f = \chi _ V\) where \(V\) is a simple \(kG\)-module.
\end{definition}

\begin{definition}
    \(f\) is \underline{1-dimensional character} if \(f : G \to k^\times\) is a homomorphism. eg trivial character \(\chi_1 (g) \equiv 1\). 
\end{definition}

\begin{proposition}
    Class Functions are \(k\)-algebras. Virtual characters are a commutative ring.
\end{proposition}

Now, suppose \(\Char k \nmid \vert G \vert\). Then,

\[
    kG \cong M_{n_1}(D_1) \times \cdots \times M_{n_s}(D_s)
\]

Assume \(M_{n_1}(D_{n_1})=k\). Then we have the trivial representation: \(ga = a\).

If \(L_i = D_i^{n_i}\) is a simple \(kG\)-module, then

\(\chi_i = \chi_{L_i}\) is a simple characteristics.

We have \(1 = e_1 + \cdots + e_s\) [central non-trivial idempotents].

\(\chi_i(e) = \Tr(\operatorname{Id}_{L_i}) = \dim_k L_i = n_i \dim_k D_i\).

\begin{example}
    Consider \(Q_8 \hookrightarrow \mathbb{H} ^\times\). Then,

    \[
        \chi_{\mathbb{H}}(e) = 4
    \]
\end{example}

Now, consider \(_{kG}kG \cong \bigoplus_{i} n_i L_i\), the `regular representation'. \(e_j L_i = 0\) for \(i \neq j\). Then,

\[
    \chi_i(e_i) = \chi_i(1) = \chi_i(e) = \dim_k L_i
\]

So, char \(\chi: G \to k\) extends to \(\chi : kG \to k\) by \(\sum a_g g \mapsto \sum a_g \chi (g)\).  

If \(V\) is a finitely generated \(kG\)-module, we have

\[
    V \cong m_1 L_1 \oplus \cdots \oplus m_s L_s
\]

where \(m_i \geq 0\).

\begin{theorem}
    [2.2, 2.3] \(\chi _v = \sum_{i} m_i \chi_i : G \to k\) with \(m_i\) uniquely determined if \(\Char k = 0\).
\end{theorem}

\begin{theorem}
    [2.3] Characters Determine Representations: suppose \(\Char k = 0\). Then,

    \[
        V \cong V^{\prime} \iff \chi_V = \chi_{V^{\prime} }
    \]
\end{theorem}

\begin{proof}
    \(\implies\): Trace is independent of basis, so this is easy.

    \(\impliedby\): We already gave a proof using projection operators. Second Proof:

    Assume \(\chi_V = \chi _{V^{\prime}}\). We decompose:

    \[
        V \cong \oplus m_i L_i, V^{\prime} \cong m_i^{\prime} L_i
    \]

    Note that we have \(\chi_V(e_i) = m_i \dim_k L_i = m_i^{\prime} \dim_k L_i = \chi_{V^{\prime}}(e_i)\)
    
    Thus we must have \(m_i = m_i^{\prime}\).
\end{proof}

\subsection*{Representation Ring}

\(R_k(G) = (\text{virtual char},+,\times) \cong (\text{virtual rep}, \oplus, \otimes)\).

Example: \(R_{\mathbb{Q}}[C_2] \cong \frac{\mathbb{Z}[X]}{(X^2 - 1)}\) 


\section*{Monday, 9/23/2024}

\subsection*{Dual Characters}



Consider \(\rho : G \to GL_k(V)\) 

Dual \(V^{\ast} = \Hom_k (V,k)\) is also a representation.

\[
    (g\phi)(v) = \phi(g ^{-1} v)
\]

Inverse because we want it to be a left module.

\underline{Claim}: \(\rho: G \to GL(V) \to \rho ^{\ast} : G \to GL(V^{\ast})\)

\(\rho ^{\ast} (g) = (\rho (g)^{-1})^T\) 

\begin{proof}
    \(\rho ^{\ast} (g) = (\rho (g ^{-1}))^{\ast} = \rho (g ^{-1})^T\) 
\end{proof}

\begin{corollary}
    \begin{enumerate}[label=\alph*)]
        \item \(\chi _{V^{\ast} }(g) = \chi _v(g ^{-1})\) 
        \item \(\chi_{\Hom(V,W)}(g) = \chi _V (g ^{-1}) \chi _W(g)\)  
    \end{enumerate} 
\end{corollary}

\begin{proof}
    a follows from the claim.

    b: Consider the \underline{slant homomorphism}:

    \[
        V^{\ast} \otimes W \to \Hom(V,W)
    \]

    \[
        \sum_{i} \phi_i \otimes w_i \mapsto \left(v \mapsto \sum_{i} \phi _i(v)w_i \right)
    \]

    It is an isomorphism since \(V,W\) are both finite dimensional.

    \[
        \chi_{\Hom(V,W)} (g) = \chi_{V^{\ast} \otimes W} (g) = \chi_{V^{\ast}}(g) \chi_W(g) = \chi_V(g ^{-1}) \chi _W(g)
    \]
\end{proof}

\subsection*{1 Dimensional Characters}

\begin{definition}
    \(1\) D representation is a homomorphism \(\rho: G \to k^\times \) 
\end{definition}

\begin{center}
    \begin{tikzcd}
        G \ar[rr] \ar[rd] & & k^\times\\
        & G^{ab} \ar[ur] &
    \end{tikzcd}
\end{center}

\underline{Question}: What are the 1d representations for \(D_6\)?

\(D_6 \cong \mathbb{Z} /3 \rtimes \mathbb{Z}/2\) 

So, \(D_6^{ab} \cong \mathbb{Z} / 2\) 

So, we have \(k_T, k_-\) 

\(r \mapsto 1\) 

\(s \mapsto -1\) 

\underline{Exercise}: Trivial Representation / Idempotent

\[
    e_T = \frac{\sum_{g\in G} g}{\vert G \vert } \in kG
\]

\[
    e_T^2 = e_T
\]

\[
    g e_T = e_T = e_T g
\]

\[
    e_T \in Z(kG)
\]

\[
    kG = (kG)e_T \oplus (kG)(1-e_T)
\]

\[
    kG \cong k \times \frac{kG}{\langle e_T \rangle }
\]

\begin{lemma}
    [2] Any finite subgroup of \(k^\times\) is cyclic.
\end{lemma}

\begin{proof}
    Key Fact: \(x^e - 1 \in k[x]\) has at most \(e\) roots [proof: long division].

    Note: \(x^2 - 1 \in \mathbb{Z} / 8 [x]\) has 4 roots. This implies \(\mathbb{Z} / 8\) is not a field.

    Consider finite abelian \(A < k^\times\) 

    Consider \(e = \text{exponent } A = \inf \{ m \geq 1 \mid \forall a\in A, a^m = e \}\)
    
    Then, \(\forall a\in A, a^e - 1 = 0\). From the key fact, \(\vert A \vert \leq e \leq \vert A \vert\)
    
    Thus, \(e = \vert A \vert \) 
\end{proof}

\begin{corollary}
    \(\forall \hom \quad \rho : G \to k^\times ,\exists\) Cyclic \(C\) such that:

    \begin{center}
        \begin{tikzcd}
            G \ar[rr, "\rho"] \ar[rd, two heads] & & k^\times\\
            & C \ar[ur, tail] &
        \end{tikzcd}
    \end{center}

\end{corollary}

Recall only finite subgroup of \(\mathbb{Q}\) is \(\pm 1\).

\(1-d\) \(\mathbb{Q}\) reps of \(G\) \(\leftrightarrow\) trivial representation + index \(2\) subgroups

Now we suppose \(k\) is algebraically closed, eg \(k = \mathbb{C}\). Then,

\[
    kG \cong \prod_i M_{n_i}(k)
\]

If \(G\) is abelian, then,

\[
    kG \cong k \times \cdots \times k
\]

\begin{corollary}
    [3] \(k\) is algebraically closed and \(G\) is abelian \(\iff\) all irreducible representations are \(1\)-dimensional. 
\end{corollary}

\begin{corollary}
    Let \(\vert G \vert = n, k = \mathbb{C}\). 

    \begin{enumerate}[label=\alph*)]
        \item \(\forall V, \chi_V(G) \subset \mathbb{Q} (\zeta_n)\) 
        \item \(\forall V, \chi_{V^{\ast}}(g) = \overline{\chi_V(g)} \)
        \item \(\forall V, W, \chi_{\Hom(V,W)} (g) = \overline{\chi_V(g)} \chi_W(g)\)   
    \end{enumerate} 
\end{corollary}

\begin{proof}
    \begin{enumerate}[label=\alph*)]
        \item True for \(1\)d representation from the lemma.
        
        \(\implies\) True for \(G\) abelian (corollary 3)

        \(\implies\) True for cyclic \(G\)

        \(\implies\) always true: \(g\in G \implies \langle g \rangle \) cyclic.

        \[
            \chi_{\rho}(g) = \chi_{\rho|_{\langle g \rangle }} (g) 
        \]

        Then, \(\rho : G \to GL(V)\), consider \(g\in G\).
        
        Then \(\rho(g)^n = I \implies \Tr(\rho_V(g)) \in \mathbb{Q}(\zeta_n)\).
        
        \item Same as (a). 
        
        \(\rho ^{\ast} (g) = (\rho (g)^{-1})^t\) 
        
        For 1-dim, \(\rho ^{\ast} = \overline{\rho}\).

        \item \(\chi_{\Hom(V,W)}(g) = \chi_V(g ^{-1}) \chi_W(g) = \overline{\chi_V(g)} \chi_W(g)\) 
    \end{enumerate} 
\end{proof}

\subsection*{Two Bases for center \(kG\)}

\begin{definition}
    \(g\in G\) is conjugate to \(\sigma \in G\) if \(\exists \tau\) such that,
    
    \[
        \tau g \tau ^{-1} = \sigma
    \]

    Write \(g \sim \sigma\) 
\end{definition}

\(G = \coprod_{G / \sim}[g]\) 

\([g]= \{ \sigma \in G \mid g \sim \sigma \} \) conjugacy classes 

\begin{proposition}
    \(\{ \sum_{\sigma \in [G]} \sigma \}_{[g]\in G / \sim} \) is a \(k\)-basis for center of \(kG\). 
\end{proposition}

\begin{proof}
    Clearly these are linearly independent.

    \(\alpha = \sum_{\sigma \in G} a_\sigma \sigma \in\) center

    \(\iff \tau \alpha = \alpha \tau \iff \tau \alpha \tau ^{-1} = \alpha\) 

    \(\sigma a_\sigma \tau \sigma \tau ^{-1} = \sum  a_\sigma \sigma \implies (g \sim \sigma \implies a_g = a_\sigma)\) 
\end{proof}


\section*{Wednesday, 9/25/2024}

Lang XVIII, 4

Two bases for \(Z(kG)\)

conjugacy classes

primitive cental idempotents [\(k\) algebraically closed]

\begin{exercise}
    \(G \twoheadrightarrow Q\), prove that \(kG \cong kQ \times R\) 
\end{exercise}

\begin{proposition}
    [4.1] Suppose \(\{ \sum_{\sigma \in [g]}  \}_{[g] \in G / \sim} \) form a \(\{ \substack{k \\ \mathbb{Z}} \} \)-basis for \(\{ \substack{Z(kG) \\ Z(\mathbb{Z} G)} \} \)  
\end{proposition}

Consider a ring \(R\).

\begin{definition}
    \(e\in R\) is a \underline{primitive central idempotent} if:
    
    \(e\) is a central idempotent [\(e^2 = e, e\in Z(R)\)]

    \(e = e^{\prime} + e^{\prime\prime}\) with \(e^{\prime} , e^{\prime\prime} \) central idempotent \(\implies \{ e^{\prime} , e^{\prime\prime}  \} = \{ 0, e \} \) 
\end{definition}

Then, \(kG\ni 1 = e_1 + \cdots + e_s, kG \cong \prod M_{d_i}(D_i)\)

\(e_i \to (0, \cdots , 0, 1, 0, \cdots , 0)\) 

Now suppose \(n = \vert G \vert \) 

We have irreducible representations \(L_1, \cdots , L_s\) and degrees \(d_1, \cdots , d_s\) then \(L_i \cong D_i^{d_i}\). We have irreducible characteristics \(\chi _1, \cdots , \chi _s\) and primitive central idempotents (p.c.i.) \(e_1, \cdots , e_s\) 

\underline{Facts}: \((\ast)\): \(_{kG} kG = \bigoplus_{i} d_i L_i\) 

\((\ast\ast)\): \(\alpha \in kG, i \neq j\) then \(\chi_j(e_i \alpha) = 0\) since \(e_i L_j = 0, \chi_i (e_i \alpha) =\chi_i (1 \alpha) = \chi_i (\alpha )\)  

We have: \(\chi_{\text{reg}} = \sum_{i} d_i \chi_i\) 

\begin{proposition}
    [4.3] \(\chi_{\text{reg}}(g) = \begin{dcases}
        n, &\text{ if } g=e ;\\
        0, &\text{ if } g \neq e
    \end{dcases}\) 
\end{proposition}

\begin{proof}
    \(\chi_{\text{reg}}(g) = \Tr (\cdot g: kG \to kG)\)
    
    Thus, \(\chi_{\text{reg}}(e) = \Tr(I) = n\)
    
    If \(g \neq e\) note that \(G\) has \(\{ \sigma_1, \cdots , \sigma_n \} \) and \(\rho_{\text{reg}}(g) (\sigma_j) = g \sigma_j \neq \sigma_j\) for all \(j\). So, there is nothing in the diagonal matrix and trace is \(0\).
\end{proof}

\underline{Motivation for \(k\) algebraically closed}:

Consider \(\mathbb{Q}C_3 \cong \mathbb{Q} \times \mathbb{Q}(\zeta_3)\). We only have primitive central idempotents, \(1 = e_1 + e_2\). But the center has dimension \(3\): \(\dim_{\mathbb{Q}}(Z(\mathbb{Q}C_3)) = 3\).

Assume \(k\) is algebraically closed.

\underline{Claim}: \(k\) algebriacally closed, \(D\) skew field, \(k < Z(D)\), \(\dim_k D < \infty\) implies \(k = D\)

Now, \(kG \neq \prod M_{d_i}(k)\) 

Consider primitimve central idempotents \(e_1, \cdots , e_s\) for a basis.

\(n = \sum_{i=1}^s d_i^2\) 

e.g. \(S_3 = D_6\). \(s=?\) \(d_1, d_2, d_3 = ?\) 

We have represantatives of conjugacy classes: \((1), (12), (123)\).

\(s=3, 6 = 1^2 + 1^2 + 2^2\) 

Char. Table:

\begin{table}[H]
    \centering
    \begin{tabular}{c|c|c|c}
        \toprule
             & (1) & (12) &  (123) \\
        \midrule
            \(\chi_1\) & 1 & 1 & 1  \\
            \(\chi_2\)  & 1 & -1 & 1  \\
            \(\chi_3\)  & 2 & 0 & -1  \\
        \bottomrule
    \end{tabular}
    \caption{characteristic table}
    \label{tab:chartable}
\end{table}

We have \(\mathbb{C} S_3 = \mathbb{C} _+ \times \mathbb{C} _- \times M_2\mathbb{C}\) 

Our representatives are \((1), (12), (123), (1234), (12)(34)\) 

\(d_i = 1, 1, 2, 3, 3\) 

\underline{Goal}: Express the p.c.i basis in terms of conjugacy class basis.

\begin{corollary}[4.2]
    If \(k\) is algebraically closed,
    
    the number of conjugacy classes = \(\dim_k Z(G)\) = number of irreducible representation \(= s\)  
\end{corollary}

\begin{proposition}
    [4.4] \(k\) algebraically closed, then

    \[
        e_i = \frac{d_i}{n} \sum_{\tau \in G} \chi_i (\tau ^{-1} ) \tau  
    \]
\end{proposition}

\begin{proof}
    Let \(e_i = \sum_{\tau \in G} a_\tau \tau \).
    
    We compute \(\chi_{\text{reg}}(e_i \tau ^{-1})\) in two ways.
    
    1: \(\chi_{\text{reg}}(e_i \tau ^{-1} ) = \chi_{\text{reg}}(\sum a_\sigma \sigma \tau ^{-1}) = \sum a_\sigma \chi_{\text{reg}}(\sigma \tau ^{-1}) = a_{\tau}n\)
    
    2: \(\chi_{\text{reg}}(e_i \tau ^{-1}) \overset{(\ast)}{=} \sum_{j} d_j \chi_j (e_i \tau ^{-1}) \overset{(\ast\ast)}{=} d_i \chi_i (e_i \tau ^{-1}) = d_i \chi_i (\tau ^{-1}) \)
    
    Thus, \(a_\tau n = d_i \chi_i (\tau ^{-1}) \implies a_\tau = \frac{d_i}{n} \chi_i (\tau ^{-1})\) 
\end{proof}

Recall that \(\exp G\) is the smallest positive integer \(m\) such that \(g^m = \operatorname{id}\) for all \(g\).

\begin{corollary}
    [4.5] Let \(m = \exp G\). Then, 

    \[
        e_i \in \frac{1}{n} \left[ \mathbb{Z} [\zeta_m] G \right] \subset \frac{1}{n} \left[ \mathbb{Z} [\zeta_n] G \right]
    \]
\end{corollary}

\begin{corollary}
    [4.6] \(\Char k \nmid d_i\) 
\end{corollary}

\begin{proof}
    If not, \(\Char k \mid d_i\) then \(e_i = 0\) which is a contradiction. 
\end{proof}

\begin{corollary}
    [4.7] \(\chi_1, \cdots , \chi_s\) are linearly independent over \(k\). In fact they form a basis for the \underline{class functions} \(f: G \to k\). 
\end{corollary}

\begin{proof}
    Suppose \(0 = \sum a_i \chi_i\). 

    Then \(0 = \sum a_i \chi_i(e_j) = a_j \chi_j (e_j) = a_j d_j \implies a_j = 0\) 
\end{proof}

Then \(\dim_k(\text{class functions}) = \) number of conjugacy classes = \(s\). 


\section*{Friday, 9/27/2024}

Review:

\[
    e_i = \frac{d_i}{n} \sum_{\sigma \in G} \chi _ i (\sigma) \sigma ^{-1} \in k G \quad (\ast)
\]

Is a primitive central idempotent.

\[
    \chi _{\text{reg}} = \chi _{kG} = \sum_{i} d_i \chi _ i
\]

\(\sigma = 1, n = \sum_{i} d_i ^ 2\) 

\(d_i \mid n\) 

\[
    \sum_{\sigma \in G} \chi _i(\sigma) \chi_j(\sigma ^{-1}) = n \delta_{ij}
\]

\[
    \sum_{i=1}^s \chi_i(\sigma) \chi_i(\tau^{-1}) = \begin{dcases}
        \frac{n}{\vert \sigma \vert}, &\text{ if } \tau = \sigma ;\\
        0, `&\text{ otherwise} .
    \end{dcases} 
\]

If \(G = S_3\) then:

\begin{table}[H]
    \centering
    \begin{tabular}{c|c|c|c || c}
        \toprule
             & (1) & (12) &  (123) & \\
        \midrule
            \(\chi_1\) & 1 & 1 &  1 & 6 \\
            \(\chi_2\) & 1 & -1 &  1 & 6 \\
            \(\chi_3\) & 2 & 0 &  -1 & 6 \\
            \midrule
            & 6 & 2 & 3 \\
        \bottomrule
    \end{tabular}
    \caption{Characeristic Table of \(S_3\)}
    \label{tab:chartable2}
\end{table}

\(0 = \chi_{\text{reg}}(123) = 1 \chi_1 (123) +1 \chi_2(123) + 2 \chi_3(123)\)

\(k = \mathbb{C} , \chi(\sigma ^{-1}) = \overline{\chi(\sigma)} \) 

End of review

\(X(G) = \{ \text{class functions } f: G \to k \}\) so that \(f(\tau \sigma \tau ^{-1}) = f(\sigma)\).

\begin{definition}
    [Perfect Pairing]

    A perfect pairing of \(k\) vector space is a \(k\)-bilinear map \(\beta : V \times W \to k\) such that \(\exists\) basis \(\{ v_i \}, \{ w_j \} \) such that

    \[
        \beta (v_i, w_j) = \delta_{ij}
    \]

    \[
        \iff \operatorname{Ad}_{b}: V \to W^{\ast} 
    \]

    \[
        v \mapsto (w \mapsto \beta(v,w))
    \]
\end{definition}



\begin{theorem}
    [4.9] 

    \[
        X(G) \times Z(kG) \to k
    \]

    \[
        (f,\alpha) \mapsto f(\alpha)
    \]

    is a perfect pairing.
\end{theorem}


\begin{proof}
    Dual basis: \(\left\{ \frac{1}{d_i} \chi_i \right\}, \{ e_j \} \) 

    \[
        \frac{1}{d_i} \chi_i(e_j) = \delta_{ij}
    \]
\end{proof}

\begin{corollary}
    [4.8] Suppose \(k\) is algebraically closed, \(\Char k = 0\). Then \(d_i = \dim_K L_i \mid n\) 
\end{corollary}

We need integrality theory (M502)

See Lang p 334.

\(A\) subring of \(B\), \(\alpha \in B\).

\(\alpha\) is integral over \(A\) if \(\exists\) monic \(f(x)\in A[x]\) such that \(f(\alpha) = 0\).

\(\alpha \in \mathbb{Q} \implies \alpha \text{ int} / \mathbb{Z} \iff \alpha \in \mathbb{Z} \) 


Condition \((\ast\ast)\): \(\alpha\) being integral is equivalent to the existence of a faithful \(A[\alpha]\)-module \(M\) which is finitely generated as \(A\)-module.

Faithful means: \(\forall \beta \in A[\alpha], \beta M = 0 \iff \beta  = 0\).

In other words, \(A[\alpha]\hookrightarrow \operatorname{End}_{A[\alpha]}(M)\).

Condition \((\ast\ast) \iff \alpha \text{ int}/A \). This is proved by a determinant trick.

Applying \((\ast\ast)\) on \(A=\mathbb{Z}, \frac{n}{d_i}\in\mathbb{Q}\),

Multiplying \(e_i = \frac{d_i}{n} \sum_{\sigma \in G} \chi_i (\sigma) \sigma ^{-1} \in kG\) with \(e_i\),

\[
    e_i = e_i^2 = \frac{d_i}{n} \sum_{\sigma} \chi_i(\sigma) \sigma^{-1} e_i 
\]

\[
    \frac{n}{d_i} e_i = \sum_{\sigma} \chi_i(\sigma) \sigma ^{-1} e_i
\]

\[
    M = \mathbb{Z} \langle \zeta_n^j \sigma e_i \rangle_{j,\sigma \in G} \text{ is a } \mathbb{Z}\left[ \frac{n}{d_i}\right]\text{-module}  
\]

We are done by \((\ast\ast)\). \(d_i\mid n\). 

\subsection*{Orthogonality, Lang XVIII, 5, Serre 2.3}

\begin{theorem}
    Suppose we have \(\langle , \rangle : X(G) \times X(G) \to k\) by:

    \[
        \langle f,g \rangle = \frac{1}{n}\sum_{\sigma \in G} f(\sigma) g (\sigma ^{-1})
    \]

    is a nonsingular symmetric form and \(\{ \chi_1, \cdots , \chi_s \}\) forms an orthonormal basis.
\end{theorem}

\begin{proof}
    Symmetric form, \(k\)-bilinear \(\langle f,g \rangle = \langle g,f \rangle \) 

    Apply \(\chi_j\) to \((\ast)\)
    
    \[
        d_i \delta_{ij} = \chi_j(e_i) = \frac{d_i}{n} \sum_{\sigma} \chi_i(\sigma) \chi_j(\sigma ^{-1})
    \]
\end{proof}

\underline{Remark}: Irreducibility criterion: \(\langle \chi , \chi  \rangle = 1 \iff \chi\) irreducible.

\((\sum_{i} a_i \chi_i, \sum_{i} a_i \chi_i) = \sum_{i} a_i^2\) 


\begin{proposition}
    [I.7, Serre p20]

    \begin{enumerate}[label=\alph*)]
        \item \(\sum_{i=1}^{s} \chi_i (\sigma) \chi_i(\sigma^{-1}) = \frac{n}{\vert [\sigma] \vert } \)
        \item \([\sigma] \neq [\tau] \implies \sum_{i=1}^s \chi_i(\sigma) \chi_i(\tau ^{-1}) = 0 \)   
    \end{enumerate} 
\end{proposition}

\begin{proof}
    Consier the characteristic function for \([\sigma]\):

    \(f_\sigma = 1\) on \([\sigma]\) and \(0\) everywhere else. 

    \(f_\sigma = \sum_{i} \lambda_i \chi_i\).

    \(\lambda_j = \langle f_\sigma, \chi_j \rangle = \frac{1}{n} \sum_{\tau \in G} f_\sigma (\tau) \chi_j (\tau ^{-1}) = \frac{\vert [\sigma] \vert}{n} \chi_j (\sigma ^{-1})\) 

    \(f_\sigma (-) = \sum_{i} \frac{\vert [\sigma] \vert }{n} \chi_i(\sigma ^{-1} ) \chi_i(-)\) 
\end{proof}

This finishes the proof.


\section*{Monday, 9/30/2024}

\subsection*{Serre Ch 4}

What about representations of infinite groups?

\begin{tikzcd}
    & G \ar[dl] \ar [d] \\
    \text{compact groups } S^1 \subset \mathbb{C}^\times & \text{discrete groups } \mathbb{Z}(\cong C_\infty) \ar[dl] \ar[d]\\
     C^\ast G \text{ $C^\ast$-algebras} & \mathcal{N}(G) \text{ von Neumann Algebra} 
\end{tikzcd}

\begin{definition}
    [Topological Group] \underline{Topological Group} is a group \((G,\cdot)\) such that \(G\) has a topology so that:

    \[
        G \times G \to G
    \]

    \[
        (g,h) \mapsto g h ^{-1} 
    \]

    is continuous.
\end{definition}

\begin{definition}
    [Lie Group] \underline{Lie Group} is a topological lie group \(G\) where \(G\) is a smooth manifold and \((g,h) \mapsto g h ^{-1} \) is smooth.
\end{definition}

Compact Lie Groups:

Torus \(T^r = S^1 \times \cdots \times S^1\)

\(O(n) = \{ A \in M_n(\mathbb{R}) \mid A A^T = I \} \) 

\(U(n) = \{ A \in M_n(\mathbb{C}) \mid A A^{\ast} = I \}\) 

\underline{Exceptional}: \(G_2, F_4, E_6, E_7, E_8\) 

We also have compact groups are not lie groups>

\((\mathbb{Z} / p)^{\infty} = \prod \mathbb{Z} / p \mathbb{Z}\) 

\(p\)-adic \(\mathbb{Z}_p = \lim \mathbb{Z} / p^n \mathbb{Z}\) 

Serre Ch 4 says that:

\begin{center}

\textbf{Representation of compact groups is almost the same as finite group!}

\end{center}

We need \underline{Haar Measure}.

\begin{proposition}


    For locally compact Hausdorff topological group \(G\) there exists a unique Haar Measure:

    \[
        \begin{array}{rcl}
            \mathrm{d}t: \{ \text{Borel Subsets of $G$} \} & \to & [0,1] \\
            B & \mapsto & \int_B \mathrm{d}t = \int_G \chi_B(t) \mathrm{d}t \\
        \end{array}
    \]

    So that \(\int_G \mathrm{d}t = 1\) and \(\mathrm{d}t\) is translation invariant:

    \[
        \int_G f(t) \, \mathrm{d}t = \int_G f(gt) \, \mathrm{d}t = \int_G f(tg) \, \mathrm{d}t
    \]
    
\end{proposition}

\begin{example}
    If \(G\) is finite:

    \[
        \int_G f \, \mathrm{d}t = \frac{1}{\vert G \vert} \sum_{g\in G} f(g)
    \]

    \(G = S^1\) 

    \[
        \int_{S^1} \, \mathrm{d}t = 1 \quad \int_{\text{quarter circle}} \, \mathrm{d}t = \frac{1}{4}
    \]
\end{example}

\begin{theorem}
    [Maschke's Theorem, Peter-Weyl Theorem] Let \(G\) be a compact group, \(k = \mathbb{C}\). Let \(W \subset V\) be a subrepresentation of \(\rho : G \to GL(V)\). Then \(\exists\) subrepresentation \(W^{\prime}\) such that \(V = W \oplus W^{\prime}\).
\end{theorem}

\begin{proof}
    Let \(\langle , \rangle ^{\prime} : V \times V \to \mathbb{C}\) be any inner product.

    We define a new inner product by averaging this inner product.

    \[
        \langle v,w \rangle = \int_G \langle \rho (t) v, \rho(t) w \rangle^{\prime} \, \mathrm{d}t
    \]

    This gives us a \(G\)-invariant inner product.

    We take \(W^{\prime}\) to be orthogonal to \(W\) w.r.t.\ this inner product.
\end{proof}

\begin{corollary}
    Any representation is the direct sum of irreducible representation (unique upto multiplicity).
\end{corollary}

Consider the regular representation \(L^2(G) \cong ``\bigoplus_{i}" d_i L_i\). 

We don't have characteristic of regular representation

We don't have a group ring

Suppose \(G = S^1, n\in \mathbb{Z}\) 

\(\chi_n : S^1 \to \mathbb{C} ^\times\) 

\(\chi_n(z) = z^n\) gives us \(\mathbb{C}_n\)

\(L^2(S^1) = ``\oplus" \mathbb{C}_n\) 

\underline{Representation Ring}: \(R(S^1) \ni \rho - \rho ^{\prime} \) 

\(R(S^1) = \mathbb{Z} [\chi_1, \chi_1 ^{-1}], \chi_n = \chi_1 \otimes _G \cdots \otimes _G \chi_1\) 

Then, \(R(S^1 \times \cdots \times S^1) = \mathbb{Z}[\alpha_1, \alpha_1 ^{-1}, \cdots , \alpha_r, \alpha_r ^{-1}]\) where:


\begin{center}

    \begin{tikzcd}
        S^1 \times \cdots \times S^1 \ar[r, "\text{proj}"] \ar[rr, bend right] & S^1 \ar[r, hook] & \mathbb{C}^\times
    \end{tikzcd}

\end{center}

Consider \(T^n \subset U(n)\)

\(\Sigma_n = U(n) / T^n\) 

\(R(U(n))\hookrightarrow R(T^n)\).

image \(\mathbb{Z} [\sigma_1, \cdots , \sigma_{n-1}, \sigma_n, \sigma_n ^{-1}]\) where

\(\sigma_i\) is the \(i\)-th elementary symmetric function in \(\alpha_1, \cdots , \alpha_n\).

\subsection*{Infinite Discrete Groups}

\(C_\infty = \langle x \rangle \) 

\(\mathbb{Z} C_\infty = \mathbb{Z} [x, x ^{-1}]\) the Laurent Polynomial Ring.

We can think of it like the localization of \(\mathbb{Z} [x]\) at \(x\) [aka \(x ^{-1} \mathbb{Z} [x]\)] or \(\mathbb{Z} [x, x ^{-1}] \subset \mathbb{Q} (x)\) the rational function field.

This is not a super well behaved domain since it has dimension \(2\).

\(\mathbb{Q} [x,x ^{-1}]\) is a Euclidean domain and hence a PID. But not \(\mathbb{Z} [x, x ^{-1}]\).

\subsection*{Some Conjectures about Torsion-Free Groups}

Torsion free: If \(g \in G - \{ e \}, n > 0\) then \(g^n \neq e\).

\begin{proposition}
    [Farrell-Jones Conjecture] for \(R = \mathbb{Z}\) or a field, all finitely generated projective \(\mathbb{R} G\)-modules are stably-free.
\end{proposition}

Projective means it's a summand of a free module.

\(P\) is stably free if \(P \oplus \text{free}\) is free.

It has been proved for the torsion-free groups we care about, but not generally.

\begin{proposition}
    [Kaplansky Idempotent Conjecture]

    Suppose \(R\) is an integral domain. Then the only idempotents in \(RG\) are \(0\) and \(1\).
\end{proposition}

\begin{proposition}
    [Zero Divisor Conjecture] Suppose \(R\) is an integral domain. Then \(RG\) has no zero divisor.
\end{proposition}

\begin{proposition}
    [Embedding Conjecture] Suppose \(R\) is an integral domain. Then \(RG\) is a subring of a skew field.
\end{proposition}

We have Embedding Conjecture \(\implies\) Zero Divisor Conjecture \(\implies\) Kaplansky Idempotent Conjecture

\begin{proposition}
    [Unit Conjecture]

    Suppose \(k\) is a field. Then,

    \[
        (kG)^\times = \langle k^\times, G \rangle 
    \]
\end{proposition}

\section*{Wednesday, 10/2/2024}

\underline{Serre Chapter 5}

Examples

\(k = \mathbb{C}\): Use characters.

5.1: \(C_n = \langle r \rangle, \zeta_n = e^{2\pi i / n}\).

\(n = \#\text{conjugacy classes}\implies n=s\) irreducible representations.

\(C_n\) is abelian \(\implies\) all irreducible representation (=char) is one dimensional.

\[
    \chi: C_n \to \mathbb{C}^\times
\]

\[
    \chi(r)^n = \chi(r^n)=\chi(e)=1
\]

Irreducible representation \(\chi_h(r)=\zeta_n^h\). We have characters \(\chi_0, \chi_1, \cdots, \chi_{n-1}\).

\(\chi_h \chi_{h^{\prime}} = \chi_{h + h^{\prime} \pmod n}\) 

Representation Ring \(\mathbb{Z}[\text{characters}] = \mathbb{Z} [\chi_1] \cong \mathbb{Z}[x] / (x^n - 1)\).

Trivial character is \(1\) in \(R(G)\).

\[
    \begin{array}{cccc}
        \phi: & \mathbb{C}[C_n] & \to &  \mathbb{C} \times \cdots \times \mathbb{C}  \\
         & r & \mapsto &  (\rho^0, \rho^1, \cdots , \rho^{n-1}) \\
    \end{array}
\]

\[
    \Phi: \mathbb{Q}[C_n] \to \prod_{d\mid n} \mathbb{Q}(\zeta_d)
\]

a

Question: How to justify that \(\phi\) and \(\Phi\) are isomorhisms?

Answer: CRT 

For a non-abelian group \(G\), recall that:

\(\#\) of 1d rep = \(\vert G^{ab} \vert = \vert G / [G,G] \vert \) 

\(\#\) of irreducible rep = \(\#\) of conjugacy classes.

Suppose \(d_i = \dim_\mathbb{C} L_i\) then \(n = d_1^2 + \cdots + d_s^2\) and \(d_i\mid \vert G \vert \).

\underline{5.1 Dihedral Group \(D_{2n}\) (order \(2n\))}

Recal,

\[
    D_{2n} = \langle r,s \mid r^n = 1, s^2 = 1, s r s = r ^{-1} \rangle 
\]

isometries of a regular \(n\)-gon.

Here, \((srs ^{-1})^k = s r^k s ^{-1}\) so \(s r^k s ^{-1} = r^{-k}\). Also, \(r^k s r^{-k} = r^{2k} s\).

Conjugacy classes are given by the following:

\[
    \begin{array}{cc}
        \{ e \} &  \{ s \} \\
        \{ r, r ^{-1} \} &  \{ r^2 s \} \\
        \{ r^2, r^{-2} \}  &  \{ r^4 s \}  \\
                            & \{ r^6 s \} \\
    \end{array}
\]

We have split based on whether \(n\) is even or odd.

\[
    \begin{array}{cc}
        n \text{ odd} &  n \text{ even}  \\
        \{ e \}  &  \{ e \} \\
        \{ r, r ^{-1} \}  & \{ r, r ^{-1} \} \\
        \vdots &  \vdots \\
        \{ r^{\frac{n-1}{2}}, r^{-\frac{n-1}{2}} \}  & \{ r^{\frac{n-2}{2}}, r^{-\frac{n-2}{2}}\}  \\
        \{ s, rs, r^2 s, \cdots , r^{n-1}s \}  & \{ r^{\frac{n}{2}} \}  \\
        & \{ r, r^2 s, \cdots , r^{n-1} s \} \\
        & \{ rs, r^3 s, \cdots , r^{n-2}s \}  
    \end{array}
\]

So, for \(n\) odd:

\(\#\) of conjugacy class is \(\frac{n+3}{2}\)

\(D_{2n}^{ab} = \{ 1, \overline{s} \} \cong C_2 \)

\(Z(D_{2n}) = \{ e \} \) 

For \(n\) even,

\(\#\) of conjugacy classes is \(\frac{n+6}{2}\) 

\(D_{2n}^{ab} = \{ 1, \overline{s}, \overline{r}, \overline{r} \overline{s} \} \cong C_2 \times C_2\) 

\(1\)-dim representations:

\(n\) odd implies we have representations \(\mathbb{C}_+,\mathbb{C}_-\)

\(\chi_{\pm}(r) = 1, \chi_{\pm}(s) = \pm 1\) 

\(n\) even implies we have representations \(\mathbb{C}_{++},\mathbb{C}_{+-},\mathbb{C}_{-+},\mathbb{C}_{--}\) 

\(\varepsilon _r = \pm 1, \varepsilon_s = \pm 1\)

\(\chi_{\varepsilon_r \varepsilon_s} (r) = \varepsilon_r\) and \(\chi_{\varepsilon_r \varepsilon_s} = \varepsilon_s\)

\(2\)-dim representations:

\[
    \rho^h : D_{2n} \to GL_2(\mathbb{C})
\]

\[
    \rho^h(r) = \begin{bmatrix}
        \zeta_n^h &  0 \\
        0 &  \zeta_n^{-h} \\
    \end{bmatrix}
\]

\[
    \rho^h(s) = \begin{bmatrix}
        0 &  1 \\
        1 &  0 \\
    \end{bmatrix}
\]

[Induced from \(C_n\)-representation \(\mathbb{C}_h\) later]

For \(0 < h < \frac{n}{2}\) it is irreducible [homework].

\(\chi_h(r^k) = e^{2 \pi i h k/ n} + e^{- 2 \pi i h k / n} = 2 \cos \frac{2\pi hk}{n}\) 

\(\chi_h(r^k s) = 0\) 

Since characters determine representation, we have \(\rho_h \cong \rho_{-h} = \rho_{n-h}\). 

Also, for \(0 < h < \frac{n}{2}\) the repesentations are distinct.

We have all irreducible 2-dim representations.

\underline{Remark}: \(\exists\) real representations \(D_{2n} \to GL_2(\mathbb{R})\) [isometries in \(\mathbb{R}^2\)]. Then,

\[
    \hat{\rho}^h (r) = \begin{bmatrix}
        \cos \frac{2\pi h}{n} &  - \sin \frac{2 \pi h}{n} \\
        \sin \frac{2 \pi h}{n} &  \cos \frac{2\pi h}{n} \\
    \end{bmatrix}
\]

\[
    \hat{\rho}(s) = \begin{bmatrix}
        0 &  1 \\
        1 &  0 \\
    \end{bmatrix}
\]

We have \(\chi_h = \hat{\chi}_h\) and thus \(\rho_h \cong \hat{\rho}_h\) 

\section*{Friday, 10/4/2024}

\underline{Serre 5.4}

Suppose \(G = D_{2n} \times C_2\).

Then, \(\mathbb{C} G = \mathbb{C} D_{2n} \otimes _\mathbb{C} \mathbb{C} C_2 = (\mathbb{C} D_{2n})_+ \times (\mathbb{C} D_{2n})_{-}\).

Twice as many irreducible representation as \(D_{2n}\).
\underline{5.7 and 5.8}

We have the following exact sequence:

\[
    1 \to A_4 \to S_4 \overset{\text{sign}}{\to} \{ \pm 1 \} \to 1
\]

We have \(\vert S_4 \vert = 24 = 4!, \vert A_4 \vert = 12\).

\(\left\{ \begin{matrix}
     S_4 \\
     A_4 \\
\end{matrix} \right\} = \left\{ \begin{matrix}
      \\
     \text{o.p}  \\
\end{matrix} \right\}  \) isometries of a tetrahedron.

Conjugacy classes (c.c.) in \(\left\{ \begin{matrix}
    S_4 \\
    A_4 \\
\end{matrix} \right\}\) are \(\begin{matrix}(1),(12),(12)(34),(123),(1234) & s=5\\ (1),(12)(34),(123),(213) & s=4\end{matrix}\) 

Interestingly, not all \(3\)-cycles are conjugates in \(A_4\). For example, \((123) \not\sim (124)\). Intuition: we need to swap \(3\) and \(4\), but in \(A_4\) we need something else because swapping \(3\) and \(4\) is odd.

Also: \(A_4\) is not simple [even though \(A_5,A_6\) etc are].

\(S_4 = C_2 \times C_2 \rtimes S_3\) 

\(A_4 = C_2 \times C_2 \rtimes C_3\).

Also: \(S_4^{ab} = C_2\) 

\(A_4^{ab} = C_3\)

Then, \(24 = 1^2 + 1^2 + 2^2 + 3^2 + 3^2\) 

\(12 = 1^2 + 1^2 + 1^2 + 3^2\)

\(\mathbb{C} [A_4] = \underbrace{\mathbb{C} \times \mathbb{C} \times \mathbb{C}}_{C_3 \text{-quotient}} \times \underbrace{M_3(\mathbb{C})}_{\text{geometry}}\) 

\(\mathbb{C} [S_4] = \overbrace{\underbrace{\mathbb{C} \times \mathbb{C}}_{C_2 \text{-quotient}}\times M_2\mathbb{C}}^{D_6 \text{-quotient}} \times \underbrace{M_3\mathbb{C}}_{\text{geometry}} \times \underbrace{M_3 \mathbb{C}}_{\text{geom}\otimes_\mathbb{C}\mathbb{C}_{\text{sign}}}  \) 

\underline{Chapter 6} 

Suppose we have a finite group \(G\) and \((\Char k, \vert G \vert)=1\). Then \(kG\) is semisimple.

\begin{proposition}
    [10] Let \(A\) be semisimple ring. Suppose \(L_1, \cdots , L_s\) are simple, non-isomorphic \(kG\)-modules such that \(\forall\) simple \(L\) we have \(L\cong L_i\) for some \(i\). Then,

    \[
        A\underset{\text{mul}}{\longrightarrow}  \prod \operatorname{End}_A L_i
    \]

\end{proposition}

\underline{Corollary}: \(t < s\) implies:

\[
    A \to \prod_{i=1}^t \operatorname{End}_A L_i
\]

is onto.

\underline{6.5}: 

\underline{Review}: \underline{Corollary 2}: if \(k\) is algebraically closed and \(\Char k = 0\) and \(d = \dim_k L\) where \(L\) is a simple \(kG\) module, then

\[
    d \mid \vert G \vert 
\]

We strengthen this.

\begin{proposition}
    [17] Let \(Z = Z(G)\) be the center of \(G\). Then,

    \[
        d \mid \frac{\vert G \vert}{\vert Z \vert}
    \]
\end{proposition}

\begin{proof}
    Let \(\rho : G \to GL(L)\) be an irreducible representation and \(d = \dim\). Define homomorphism \(\lambda : Z \to k^\times\) such that:

    \[
        \rho (s) = \lambda (s) \operatorname{id}
    \]

    \(\forall m \geq 1\) let \(\rho^m : G \times \cdots \times G \to GL(L \otimes \cdots \otimes L)\) which is irreducible.

    Then we have \(\lambda^m : Z \times \cdots \times Z \to k^\times\) with:

    \[
        (s_1, \cdots , s_m) \mapsto \lambda (s_1 \cdots s_m)
    \]

    Let \(H = \{ (s_i) \in Z^m \mid s_1 \cdots s_m = 1 \} < Z^m < G^m\).

    \(H \cong Z^{m-1}\) and \(H \subset \ker \rho^m\).

    Then \(\overline{\rho^m} : G^m / H \to GL(L \otimes \cdots \otimes L)\) irreducible.

    Therefore, \(\forall m, d^m \mid \vert \frac{G^m}{H} \vert = \frac{\vert G \vert^m}{\vert Z \vert^{m-1}}\) which implies by taking \(m\) big enough that \(d \mid \frac{\vert G \vert}{\vert Z \vert}\).  

\end{proof}

\subsection*{Tensor Product for Non-Commutative Rings}

Suppose \(R\) is a non-commutative ring. Then, tensor product is a functor 

\[
    - \otimes_R - : \underset{\text{right mod}}{\operatorname{mod} R} \times \underset{\text{left mod}}{R\,\operatorname{mod}} \to \operatorname{Ab}  
\]

\[
    \begin{array}{ccc}
        A_R \otimes _R \,_R B & \ni &  a_1 \otimes b_1 + \cdots + a_k \otimes b_k \\
         &  &  (a + a^{\prime}) \otimes b = a \otimes b + a^{\prime} \otimes b \\
         &  &  a \otimes (b + b^{\prime}) = a \otimes b + a \otimes b^{\prime} \\
         &  &  ar \otimes b = a \otimes rb \\
    \end{array}
\]

\begin{exercise}
    Formulate adjoint proposition:

    \[
        \Hom_? (\overset{?}{A} \otimes \overset{?}{B} , \overset{?}{C}) \cong \Hom_? (A, \Hom_?(B,C))
    \]
\end{exercise}

\begin{definition}
    [Induced module]: Suppose \(k\) is a field and \(H < G\). Then,

    \[
        \operatorname{Ind}_H^G : kH\operatorname{-mod} \to kG \operatorname{-mod}
    \]

    \[
        \operatorname{Ind}_H^G W = kG \otimes_{kH} W
    \]
\end{definition}

eg. Suppose \(H = C_n = \langle r |r^n = 1 \rangle \) and \(G = D_{2n} = \langle r,s \mid r^n = 1 = s^2; srs = r ^{-1} \rangle \). 

If \(W = \mathbb{C}\) we have \(H \to \mathbb{C} ^\times \) by \(r \mapsto \zeta_n\).

\[
    V = \mathbb{C} D_{2n} \otimes_{\mathbb{C} [C_n]} \mathbb{C}_1 = (\mathbb{C} [C_n] \oplus s \mathbb{C} [C_n]) \otimes_{\mathbb{C}[C_n]} \mathbb{C} _1
\]

\(\mathbb{C}\)-basis of \(V\) is \(1 \otimes 1, s \otimes 1\).

Recall \(r \mapsto \begin{bmatrix}
    \zeta_n &  0 \\
    0 &  \zeta_n ^{-1} \\
\end{bmatrix}, s \mapsto \begin{bmatrix}
    0 &  1 \\
    1 &  0 \\
\end{bmatrix}\).

\(s(1 \otimes 1) = s \otimes 1\)

\(s(s \otimes 1) = s^2 \otimes 1 = 1 \otimes 1\) 

\(r(1 \otimes 1) = r1 \otimes 1 = \zeta_n \otimes 1 = \zeta_n(1 \otimes 1)\) 

\(r(s \otimes 1) = rs \otimes 1 = s r ^{-1} \otimes 1 = s \otimes \zeta_n ^{-1} 1 = \zeta_n ^{-1} (s \otimes 1)\) 

\section*{Monday, 10/7/2024}

\begin{exercise}
    Work out the representation theory of \(G = C_7 \underset{\cdot 2}{\rtimes} C_3 = \langle r,s \mid r^7 = 1, s^3 = 1, s r s ^{-1} = r^2 \rangle\).

    Meaning: find an isomorphism \(\mathbb{C} G \overset{\cong}{\to} M_{d_i}\mathbb{C}\) 
\end{exercise}

Suppose we have a (most likely non-commutative) ring \(R\) and

A tensor product functor \(- \otimes_R - : \text{mod-}R \times R \text{-mod} \to \operatorname{Ab} \) 

\begin{proposition}
    [Universal Property]

    Suppose \(A\) is a right \(R\)-module and \(B\) is a left \(R\)-module and \(G\) is an abelian group.

    \(\pi : A \times B \to G\) is \underline{\(R\)-balanced}. Meaning: \(\pi\) is \(\mathbb{Z}\)-bilinear and \(\pi(ar,b) = \pi(a, rb)\).
    
    There exists an \(R\)-balanced \(\pi : A \times B \to A \otimes_R B\) which is \underline{initial}.

    \begin{center}

    \begin{tikzcd}
        A \times B \ar[d,"\pi"] \ar[rd, "\forall R\text{-balanced}"] & \\
        A \otimes_R B \ar[r, dotted, "\exists!\mathbb{Z}\text{-hom}", swap] & G
    \end{tikzcd}

    \end{center}
\end{proposition}

\underline{Construction}: 

\[
    A \otimes_R B \coloneqq \frac{F(A \times B)}{T}
\]

Where \(F(A \times B)\) is the free abelian group with basis of \(A \times B\). We write \(F(A \times B) = \mathbb{Z} [A \times B]\).

\(T\) is the subgroup generated by \((a + a^{\prime} , b) - (a,b) - (a^{\prime} ,b), (a,b+b^{\prime})-(a,b)-(a,b^{\prime}), (ar,b)-(a,rb)\).

Main thing to remember:

\[
    \boxed{ar \otimes b = a \otimes rb}
\]

\begin{proposition}
    Suppose we have a \underline{ring homomorphism} \(f: R \to S\) of possibly non-commutative rings. We preserve addition, multiplicationand identity.

    We then have the \underline{restriction functor}

    \[
        f^{\ast} : S\text{-mod} \to R\text{-mod}
    \]

    \[
        f^{\ast} M = M \text{ (as abelian group)}
    \]

    \[
        \begin{array}{ccc}
            R \times f^{\ast} M & \to  &  f^{\ast} M \\
            (r,m) & \mapsto  &  f(r)m \\
        \end{array}
    \]
\end{proposition}

If we have inclusion \(\operatorname{inc}: kH \to kG\) then we have:

\[
    \operatorname{inc}^{\ast} = \operatorname{Res}_H^G : kG\text{-mod} \to kH\text{-mod} 
\]

We also have the \underline{left adjoint of \(f^{\ast}\)}.

\[
    f_{\ast} : R\text{-mod} \to S\text{-mod} \text{ ``base change"}
\]

\[
    f_{\ast} M = S \otimes _R M
\]

\(S\) is a right \(R\)-module. We have \(S \times R \to S\) given by \((s,r) \mapsto sf(r)\) which trns \(S\) to a \((S,R)\) -bimodule: \(_S S _R\). So we can take the tensor product.

\begin{proposition}
    \[
        \boxed{\operatorname{Hom}_S(f_{\ast} M, N) \cong \operatorname{Hom}_R (M, f^{\ast} N)}
    \]

    is an isomorphism of abelian groups.
\end{proposition}

So we can go back and forth between \(S\)-modules and \(R\)-modules.

\begin{center}

    \begin{tikzcd}
        S\text{-mod} \ar[rr, "f^\ast", bend left] & \perp & R\text{-mod} \ar[ll, "f_\ast", bend left]
    \end{tikzcd}

\end{center}

\(f_{\ast}\) is left adjoint.

\(f^{\ast}\) is right adjoint.

Adjoint of \(\operatorname{Id}_{f^{\ast} N}: \boxed{f_{\ast} f^{\ast} N \to N}\) is the counit.

Adjoint of \(\operatorname{Id}_{f_{\ast} M}: \boxed{M \to f^{\ast} f_{\ast} M}\) is the unit. 

We also have:

\[
    \operatorname{inc}_{\ast} = \operatorname{Ind}_H^G : kH \text{-mod} \to kG \text{-mod} 
\]

Which gives us:

\[
    \operatorname{Hom}_{kG}(\operatorname{Ind}_H^G W, V) \cong \operatorname{Hom}_{kH}(W, \operatorname{Res}_H^G V)
\]

\underline{Remark}: If we have a module, how do we know it is induced?

\begin{proposition}
    If \(V = \bigoplus_{i\in I} W_i\) and \(G\) permutes summands transitively and \(\exists W = W_{i_0}\) and \(H = \{ g\in G \mid g W = W \} \) then \(V\) is induced.
\end{proposition}

Example: \(\mathbb{C} D_{2n} \otimes_{\mathbb{C} C_n} \mathbb{C}_1 = 1 \mathbb{C} C_n \otimes \mathbb{C}_1 + s \mathbb{C} G_n \otimes \mathbb{C}_1\).


\begin{proposition}
    \(V\) is induced if \(\exists W < V\) invariant under \(H\):

    \[
        V = \bigoplus_{r\in R} rW
    \]

    \(R\) is a set of left coset representation for \(H\) in \(G\).
\end{proposition}

\subsection*{Character of Induced representation}

\begin{theorem}
    [12, p30] \(V = \operatorname{Ind}_H^G W\).

    \[
        \chi_V (u) = \sum_{\substack{r\in R \\ r ^{-1} u r \in H}} \chi_W (r ^{-1} u r) = \frac{1}{\vert H \vert} \sum_{\substack{g\in G \\ g ^{-1} u g \in H}} \chi_W (g ^{-1} u g) 
    \]
\end{theorem}

\begin{proof}
    Write \(V = \bigoplus_{r\in R} rW\). We care about when \(ur W = rW\), since otherwise we have non-diagonal terms so they don't contribute to the trace.

    \[
        u r W = rW \iff r ^{-1} u r W = W \iff r ^{-1} u r \in H 
    \] 

    \[
        \chi_V (u) = \Tr (u\cdot: V \to V) = \sum_{\substack{r\in R \\ r ^{-1} u r \in H}} \Tr(u\cdot: rW \to rW)
    \]

    \[
        = \sum_{\substack{r\in R \\ r ^{-1} u r \in H}} \Tr(r ^{-1} u r\cdot: rW \to rW) = \sum_{\substack{r\in R \\ r ^{-1} u r \in H}} \chi_W (r ^{-1} u r) = \frac{1}{\vert H \vert} \sum_{\substack{g\in G \\ g ^{-1} u g \in H}} \chi_W (g ^{-1} u g)
    \]

\end{proof}

\subsection*{Frobenius Reciprocity}

\[
    \langle \operatorname{Ind} \psi , \phi \rangle _G = \langle \psi , \operatorname{Res} \phi \rangle _ H
\]

\section*{Wednesday, 10/9/2024}

Recall: If 

\[
    V =\operatorname{Ind}_H^G W
\]

Then \(V\) as a \(k\)-vector space can be written as direct sum of \(k\)-vector spaces:

\[
    V = \bigoplus_{g\in G / H} g W
\]

And action of \(H\) \underline{permutes the summands}.

\[
    \operatorname{Stab} (W) \coloneqq \{ g\in W \mid gW = W \} = H 
\]

Also recall \underline{Class Functions}:

\[
    \operatorname{Cl}(G) = \{ f : G \to k \mid f(g \sigma g ^{-1}) = f(\sigma) \} 
\]

The charcters \(\chi_V\) are a basis of the vector space of class functions.

For \(H < G\) we have restriction:

\[
    \begin{array}{cccc}
        \operatorname{R es}: & \operatorname{Cl}(G) & \to &  \operatorname{Cl}(H) \\
         & f & \mapsto &  f | _ H \\
    \end{array}
\]

We also have induction:

\[
    \operatorname{Ind} : \operatorname{Cl}(H) \to \operatorname{Cl}(G)
\]

\[
    (\operatorname{Ind} f)(\sigma) \coloneqq \frac{1}{\vert H \vert} \sum_{\substack{g\in G \\ g ^{-1} \sigma g \in H}} f(g ^{-1} \sigma g)
\]

Last time we did:

\[
    \chi_{\operatorname{Ind} W} = \operatorname{Ind} \chi_W
\]

Also we had the following:

\[
    \operatorname{Hom}_{kG} (\operatorname{Ind} W, V) \cong \operatorname{Hom}_{kH}(W, \operatorname{R es} V)
\]

Today we give a character version of this.

\subsection*{Frobenius Reciprocity}

\begin{theorem}
    [Frobenius Reciprocity]

    Suppose \(k\) is algebraically closed. Then:

    \[
        \langle \operatorname{Ind} \psi , \phi \rangle _ G = \langle \psi , \operatorname{R es} \phi \rangle _H 
    \]

    where \(\psi \in \operatorname{Cl} (H)\) and \(\phi \in \operatorname{Cl}(G)\) with \(H < G\).

    Also, for review: if \(\alpha ,\beta \in \operatorname{Cl}(G)\) then,

    \[
        \langle \alpha , \beta \rangle _G = \sum_{g\in G} \alpha(g) \beta(g ^{-1}) \in k
    \]

    And irreducible characters are an orthonormal basis w.r.t.\ this inner product.
    
    \[
        \langle \chi_i, \chi_j \rangle _ G = \delta _{ij}
    \]
\end{theorem}

\begin{proof}
    Suppose 

    \[
        V \cong \bigoplus_{i} m_i V_i
    \]

    where \(V_1, \cdots , V_s\) are irreducible. We define \underline{multiplicity}: \(m^V_{V_i} \coloneqq m_i\). Then,

    \[
        \langle \chi_V, \chi_{V^{\prime}} \rangle = \sum_{i=1}^s m^V_{V_i} m^{V^{\prime}}_{V_i} \underset{\text{Schur}}{=} \dim_k \operatorname{Hom}_{kG}(V,V^{\prime}) 
    \]

    We finally start the proof.

    \[
        \operatorname{Cl}(G) = \operatorname{span} \{ \chi_i \}
    \]

    WLOG assume \(\psi , \phi\) ae characters of \(W\) and \(V\).

    \[
        \dim_k \operatorname{Hom}_{kG} (\operatorname{Ind} W, V)=\dim_k (\operatorname{Hom}_{kH}(W, \operatorname{R es} V))
    \]

    \[
        \implies \langle \operatorname{Ind}(\chi_W), \chi_V  \rangle _G = \langle \chi_W. \operatorname{R es} \chi_V  \rangle _H
    \]

    Since this is true for basis, it is true for general character.

\end{proof}

\subsection*{Mackey's Double Coset Formula}

Suppose \(G\) is a group with subgroups \(H,K\). aka \(H,K < G\). Let \(W\) be a \(kH\)-module.

\underline{Question}: What is \(\operatorname{R es}_K^G \operatorname{Ind}_H^G W\) as a \(kK\)-module? 

Let \(s = [K \backslash G / H]\) be the \underline{double coset representation}. Meaning:

\[
    G = \coprod_{s\in S} KsH
\]

i.e.

\begin{center}
    \begin{tikzcd}
        G \ar[r, "\pi", swap] & K \backslash G / H \ar[l, bend right, dotted]
    \end{tikzcd}
\end{center}

The above dotted map is \([\,]\). Then,

\[
    \pi \circ [\,] = \operatorname{Id} 
\]

We have:

\[
    H_s \coloneqq s H s ^{-1} \cap K < K
\]

\[
    \rho : H \to \operatorname{GL} (W)
\]

We thus have the \underline{twisted representation}:

\[
    \rho^s : H_s \to \operatorname{GL} (W)
\]

\[
    \rho^s(x) = \rho_W(s ^{-1} x s)
\]

\(W_s = W_{\rho^s}\) is a \(kHs\)-module.

\begin{proposition}
    [Mackey's Double Coset Formula, MDCF]

    \[
        \operatorname{R es}^G_K \operatorname{Ind}_H^G W \cong \bigoplus_{s\in [K \backslash G / H]} \operatorname{Ind}^K_{H_s} W_s 
    \]
\end{proposition}

\begin{proof}
    Suppose \(V \coloneqq \operatorname{Ind}_H^G W\). Then, from the definition of \(\operatorname{Ind} W\),

    \[
        V = \bigoplus_{x\in G / H} xW
    \]

    Where \(\operatorname{Stab} (W) = H\).

    \[
        V = \bigoplus_{x\in G / H} xW
    \]

    Then, as \(hK\)-module,

    \[
        V = \bigoplus_{s \in [K\backslash G / H]} KsW 
    \]

    Note that, since \(\operatorname{Stab}^K (sW) = H_s\),

    \[
        KsW = \bigoplus_{x\in K / H_s} xsW 
    \]

    \[
        = \operatorname{Ind}^K_{H_s} sW
    \]

    \[
        = \operatorname{Ind}_{H_s} ^K W_s
    \]

    Since 
    \[
        \begin{matrix}
            W_s \cong sW \\ w \mapsto sw
        \end{matrix}
    \]
        
    So we're done.

\end{proof}

\subsection*{Mackey's Irreducibility Criterion, MIC} Suppowe \(W = W_{\rho}\) is \(kH\)-module. TFAE:

\begin{enumerate}[label=\arabic*)]
    \item \(V = \operatorname{Ind} ^G _H W\) is irreducible 
    \item 
        \begin{enumerate}[label=\alph*)]
            \item \(W\) irreducible
            \item \(\forall s \in G \setminus H\), \(\rho^s\) and \(\operatorname{R es}_{H_s}\rho \) are disjoint. 
        \end{enumerate} 
\end{enumerate} 

Recall: \(V, V^{\prime}\) are \underline{disjoint} if \(\operatorname{Hom}_{kG}(V,V^{\prime}) = 0\).

\begin{proof}
    We asssume \(k\) is algebraically closed.

    \[
        V \text{ irreducible} \iff \langle \chi_V, \chi_V \rangle _ G = 1
    \]

    \[
        \langle \chi_V, \chi_V \rangle _G = \langle \operatorname{Ind} \chi_W, \operatorname{Ind} \chi_W \rangle _ G
    \]

    \[
        = \langle \chi_W, \operatorname{R es} \operatorname{Ind} \chi_W \rangle _H \, [FR]
    \]

    \[
        = \langle W, \bigoplus_{s\in [K \backslash G / H]} \operatorname{Ind} ^H _{H_s} (\rho_s)  \rangle_H \, [MDCF] 
    \]

    \[
        = \sum_{s} \langle \operatorname{R es}_{H_s} \rho, \rho^s  \rangle _{H_s} \, [FR] 
    \]

    \[
        = \sum_{s} d_s
    \]

    \[
        d_s = \langle \operatorname{R es} \rho , \rho^s  \rangle _{H_s}
    \]

    \[
        d_1 = \langle \rho_W, \rho_W \rangle \geq 1
    \]

    Thus,

    \[
        1 = \langle V, V \rangle _G \iff \begin{matrix}
             d_1 = 1 \\
             d_s = 0 
        \end{matrix}
    \]

    So we're done.

\end{proof}

\underline{Example}: Suppose \(G = H \times K\) where \(H = C_3, G = D_6 = S_3, K = C_2\).

Then,

\[
    \mathbb{C} [C_3] = \mathbb{C}_0 \times \mathbb{C}_1 \times \mathbb{C}_2
\]

\[
    \mathbb{C}[D_6] = \mathbb{C}_+ \times \mathbb{C}_- \times M_2 \mathbb{C}
\]

\[
    \operatorname{R es} \mathbb{C}_+ = \mathbb{C}_0 
\]

\[
    \operatorname{R es} \mathbb{C}_- = \mathbb{C}_0 
\]

\[
    \operatorname{R es} \mathbb{C}^2 \overset{?}{=} \mathbb{C}_1 \times \mathbb{C}_2 
\]

\section*{Monday, 10/14/2024}

Exercises 8-13 due Friday

Wed, Chapter 9

Suppose \(K, H < G\) and \(\rho : H \to GL(W)\).

For \(s\in G\) consider \(H_s = s H s ^{-1} \cap K < K\)

Then \(\rho^s : H_s \to GL(W)\) 

\(\rho^s (x) \coloneqq \rho(s ^{-1} x s)\) 

MDCT:

\[
    \operatorname{R es}_K^G \operatorname{Ind}_H^G \rho \cong \sum_{s\in [K \backslash G / H]} \operatorname{Ind}_{H_s}^K \rho^s  
\]

Take \(K = H\).

MIC:

\[
    \operatorname{Ind}_H^G \rho \text{ is irreducible} 
\]

\(\iff\) \begin{enumerate}[label=\alph*)]
    \item \(\rho\) irredicuble
    \item \(\forall s\in G - H, \rho^s\) and \(\eval{\rho}_{H_s} \)  are disjoint.
\end{enumerate} 

Now take \(H = K \triangleleft G\) normal.

\underline{Corollary}: \(\operatorname{Ind} \rho\) is irreducible \(\iff\) \(\rho\) irredcible and \(\forall s\notin H\) \(\rho\) is not isomorpic to conjugate \(\rho^s\).

e.g. \(H = C_3 = \langle r \rangle \) 

\(G = D_6 = S_3 = \langle r, s \rangle \) 

\(\mathbb{C} H \cong \mathbb{C} _0 \times \mathbb{C} _1 \times \mathbb{C}_2\) 

\(r \mapsto (1, \zeta_3, \zeta_3^2)\) 

\(\mathbb{C} G \cong \mathbb{C}_+ \times \mathbb{C}_- \times M_2\mathbb{C}\) 

Only two dimensional irredicuble reps are \(\mathbb{C} _+ \times \mathbb{C}_-\) and \(\mathbb{C}^2\) 

\(\operatorname{Ind}_H^G \mathbb{C}_0 \cong \mathbb{C}_+ \times \mathbb{C}_-\) 

\(\operatorname{Ind}_H^G \mathbb{C}_1 \cong \mathbb{C}^2\) 

\underline{Corollary?}: \(\operatorname{Ind} \mathbb{C}_0\) is real since \(\rho  \cong \rho^s, \rho^s = \rho (s ^{-1} x s)\) 

\(\operatorname{Ind} \mathbb{C}_1\) is [], (\(\rho: H \to \mathbb{C}\)), \(\underset{\mathbb{C}_1}{\rho} \not\cong \underset{\mathbb{C}_2}{\rho^s} \).

More on MCDF ``Mackey Functors''

\underline{Review}

Ring \(f: R \to S\)

\begin{center}
    \begin{tikzcd}
        S\text{-mod} \ar[rr,"f^\ast", bend right] & & R\text{-mod} \ar[ll,"f_\ast", bend right]
    \end{tikzcd}
\end{center}

``Res'' \(f^{\ast} N = N\) 

``Ind'' \(f_{\ast} M = S \otimes_R M\) 

\underline{MDCF}: \(H, K < G\) 

\(K^s = s ^{-1} K s\) 

\(^s H = s H s ^{-1}\) 

\(c_s : \underset{g}{K^s} \underset{\mapsto}{\to} \underset{s g s ^{-1}}{K}\)

\((\operatorname{Ind} c_s) M = kK \otimes_{kK^s} M\) 

\[
    \boxed{\operatorname{R es}_K^G \operatorname{Ind}_H^G = \sum_{s\in [K \backslash G / H]} \operatorname{Ind}^K_{K \cap ^s H} \operatorname{Ind} c_s \operatorname{R es}^H_{K^s\cap H}} 
\]

\begin{definition}
    A \underline{Mackey Functor} \(M\) is:

    \[
        M : \{ \text{subgroups of } G \} \to \operatorname{Ab}
    \]

    \(\forall H \leq K \leq G\), we have: 
    
    Induction map \(I_H^K : M(H) \to M(K)\)
    
    Restriction map \(R^H_K M(K) \to M(H)\) 

    Conjugation map \(\forall g\in K, c_g : M(K^s) \to M(K)\)

    Satisfies \(6\) axioms. Key one is MDCF.

    \[
        H, K \leq J \leq G
    \]

    \[
        R_K^J I_H^J =\sum_{K \backslash J / H} \cdots 
    \]
\end{definition}

\underline{Examples of Mackey Functors} 

\(M(H)=R_K(H)\) representations.

Homology groups \(M(H)  H_n(H;-)\) 

Cohomology groups \(M(H)=H^n(H;-)\) 

Stable Homotopy theory: \(M(H)\) equals \(X\) based \(G\)-space \(\Pi^H_n X\)

Number theory: if we have \(K /_{\text{finite galois}} L /_{\text{finite}} \mathbb{Q}\),

\[
    M(H) = \operatorname{Cl}(\mathcal{O}(K^H))
\]

\section*{Wednesday, 10/16/2024}

No class Friday

Homework due monday, 8-13

\subsection*{Representation Ring}

Representation \(R(G) = \mathbb{Z}[\chi_1, \cdots , \chi_n] \subset \operatorname{Cl}(G) = \{ f: G\to \mathbb{C} : f(\sigma \tau \sigma ^{-1}) = f(\tau) \} \)

where \(\chi_1, \cdots , \chi_h\) are irreducible \(\mathbb{C}\)-rep.

\begin{itemize}
    \item \((R(G),+) \cong \mathbb{Z}^n\) 
    \item \(R(G) \otimes _\mathbb{Z} \mathbb{C} = \operatorname{Cl}(G)\)  
\end{itemize} 

A basis of \(\mathbb{C} G\) can be found the following way: Fix \(\sigma\). Then \(\sum_{\tau \sim \sigma} \tau\) gives us the basis where \(\sim\) means they are in the same conjugacy class.

Another basis are \(\chi_1, \cdots , \chi_h\). So, \(h=\) the number of conjugacy classes.

\begin{theorem}
    [Artin Induction Theorem]

    \[
        \operatorname{Ind} : \mathbb{Q} \otimes \bigoplus_{\text{cyclic } C < G} R(C) \twoheadrightarrow \mathbb{Q} \otimes R(G)
    \]
\end{theorem}

\underline{Exercise}: Let \(\chi_T\) be the trivial characteristic of \(D_6\) Express \(a \chi_T\) as a subrepresentationm of characters \(a>0\) induced from cyclic subgroups.

\begin{proof}
    \[
        \operatorname{Res}: R(G) \rightarrowtail \bigoplus_{C} R(C) 
    \]

    \[
        \operatorname{Res}: R(G) \otimes \mathbb{C} \rightarrowtail \bigoplus_{C} R(C) \otimes \mathbb{C} \, \text{injective}
    \]

    \[
        \underset{\text{Frob. Reciprocity}}{\implies} \operatorname{Ind} : \bigoplus_{C} R(C) \otimes \mathbb{C} \twoheadrightarrow R(G) \otimes \mathbb{C} \, \text{surjective} 
    \]

    Why? in matrix terms, we can think of the matrices being transposed, \(A\) injective implies \(A^T\) is surjective. We can also think of dual maps, \(V \rightarrowtail W \iff W^{\ast} \twoheadrightarrow V^{\ast}\) 

    \[
        \implies \operatorname{Ind} : \bigoplus_{C} R(C) \otimes \mathbb{Q} \twoheadrightarrow R(G) \otimes \mathbb{Q}
    \]
\end{proof}

\subsection*{Another view of \(R(G)\)}

Let \(V\) be a representation, \([V]\) be its isomorphism class. Then,

\[
    R(G) \in [V] - [V^{\prime}]
\]

``virtual representation''

\subsection{Grothendieck Construction}

Define the category \(\operatorname{CMon}\), commutative monoids.

\[
    (M,+: M \times M \to M)
\]

\[
    \text{commutative, associative, identity} 
\]

The morphisms are homomorphism [preserves unity].

\begin{center}
    \begin{tikzcd}
        \operatorname{CMon} \ar[rr, bend left, "\operatorname{Gr}"] & & \operatorname{Ab} \ar[ll, bend left, "F\text{ Forgetful}"]
    \end{tikzcd}
\end{center}

\[
    \operatorname{Ab} (\operatorname{Gr} M, A) \cong \operatorname{CMon} (M, FA)
\]

\(\iff \) universal property:

\begin{center}
    \begin{tikzcd}
        M \ar[r] \ar[rd, "\forall \text{monoid map}", swap] & \operatorname{Gr}M \ar[d, dashed, "\exists!"] \\
        & \underset{\operatorname{ab}}A
    \end{tikzcd}
\end{center}

[Take \(A = \operatorname{Gr} M\)]

Note: \(\operatorname{Gr} (\mathbb{Z}_{\geq 0}, +) = (\mathbb{Z} , +)\) 

\(\operatorname{Gr} (\mathbb{Z}_{>0}, \cdot) = (\mathbb{Q}^\times_{>0} , \cdot)\) 

\(\operatorname{Gr}(\mathbb{Z}_{\neq 0}, \cdot) = (\mathbb{Q}^\times, \cdot)\) 

Consider a field \(k\) and a group \(G\).

\(\operatorname{Iso} (k,G) =\) isomorphism class of finite dimensional \(k\)-representations \(\rho : G \to GL(V)\) with \(\dim_k V < \infty\).

We define \(R_k(G) \coloneqq \operatorname{Gr}(\operatorname{Iso}(k,G), \oplus)\) 

Is is a group. We can make this a ring by defining the product as:

\[
    [V][W] \coloneqq V \otimes_k W
\]

the diagonal \(k\)-action.


Suppose \(X\) is a set of subgroups of \(G\).

\begin{definition}
    \(R_k G\) is \(\begin{Bmatrix}
         \text{detected} \\
         \text{generated} \\
    \end{Bmatrix}\) by \(X\) if:
    
    \(\begin{Bmatrix}
         \operatorname{Res}: R(G) \to \bigoplus_{H\in X} R(H) \\
         \operatorname{Ind} : \bigoplus_{H\in X} R(H) \to R(G) \\
    \end{Bmatrix}\) is \(\begin{Bmatrix}
         \text{injective} \\
         \text{surjective} \\
    \end{Bmatrix}\)   
\end{definition}

e.g. \(R(G)\) is detected by cyclics

\(R(G) \otimes \mathbb{Q}\) is generated by cyclics.

Consider:

\[
    \operatorname{hom} f: H \to G
\]

\[
    \operatorname{Res}: R_k G \to R_k H \text{ is a ring hom} 
\]

\[
    \operatorname{Ind}: R_k H \to R_k G \text{ is a \(R_kG\)-module map} 
\]

\(1 = [k] \in R(G)\).

\[
    \operatorname{Res} W \otimes_k f_{\ast} V \cong f_{\ast} (W \otimes_k V)
\]

\[
    W \otimes_k (kG \otimes_{kH} V) \cong kG \otimes_{kH} (W \otimes_k V)
\]

\[
    w \otimes (\alpha \otimes v) \underset{?}{\mapsfrom} \alpha \otimes (w \otimes v)
\]

\underline{Note}: Consider \(f: X \to Y\). Then \(f^{\ast} : H ^{\ast} Y \to H^{\ast} X\) is a ring map, \(f_{\ast}: H_{\ast} X \to H_{\ast} Y\) is a module map.

\section*{Monday, 10/21/2024}

\subsection*{Brauer Induction Theorem}

Let \(p\) be a prime.

\begin{definition}
    \(H\) is \underline{\(p\)-elementary} if

    \[
        H \cong P \times C
    \]

    where \(P\) is a \(p\)-group and \(C\) is a cyclic group with order prime to \(p\).
\end{definition}

\begin{definition}
    \(H\) is \underline{elementary} if \(H\) is \(p\)-elementary for some \(p\).
\end{definition}

\begin{example}
    \(Q_8 \times C_3\) is \(2\)-elementary.
\end{example}

\begin{theorem}
    [Brauer Induction Theorem]

    \(R(G)\) is generated by elementary subgroups. i.e.:

    \[
        \operatorname{Ind}: \bigoplus_{\text{elem } E < G} R(E) \twoheadrightarrow R(G)
    \]

    in other words,

    \[
        \forall \rho : G \to GL(V); \chi_{\rho} = \sum_{i} a_i \operatorname{Ind}_{E_i}^G \rho_i
    \]

    where \(E_i\) are elementary.
\end{theorem}

\begin{example}
    Consider \(D_6 = C_3 \rtimes C_2\). Elementary subgroups are \(1, C_3, C_2\).

    For \(p\) odd prime, \(D_{2p}\) has elementary subgroups \(1, C_2, C_p\).
\end{example}

\begin{remark}
    We can't always choose \(a_i \geq 0\) in \(\chi_{\rho}\).
\end{remark}

\begin{theorem}
    [\(18^{\prime}\)] Let \(\vert G \vert = p^k l\) with \((l,p)=1\).

    \([\mathbb{C}^l] = l[\mathbb{C}]=l\) is induced by \(p\)-elementary subgroups.

    \[
        l = \sum_{E_i, p \text{ elem}} a_i \operatorname{Ind}_{E_i}^G \rho_i
    \]
\end{theorem}

Note: Theorem \(18^{\prime} \implies\) Brauer Induction Theorem. Let \(\vert G \vert = p_1^{e_1} \cdots p_r^{e_r}\). Then \(\gcd\left( \frac{\vert G \vert}{p_1^{e_1}}, \cdots , \frac{\vert G \vert}{p_r^{e_r}} \right) \in \operatorname{image} \operatorname{Ind} \left( \bigoplus_{E < G} R(E) \right)  \implies \forall x\in R(G), x\in \operatorname{image}\) [\(\operatorname{Ind}\) is \(R(G)\)-module map] \(\implies\) Brauer Induction Theorem.

Proof of theorem \(18^{\prime} \) is ommitted.

\subsection*{Applications of Brauer Induction Theorem}

\begin{definition}
    A representation \(\rho: G \to GL(V)\) is a \underline{monomial} if

    \[
        \rho = \operatorname{Ind}_H^G \hat{\rho}
    \]

    where \(\hat{\rho} : H \to \mathbb{C}^\times\) is a \(1\)-dim  representation.

    In other words, ``\(\rho\) is induced by irreducible representation of \(G^{\operatorname{ab}}\).''

\end{definition}

Application (Brauer): Artin \(L\)-functions are meromorphic (on \(\mathbb{C}\)).

\subsection*{Chapter 8}

\underline{Goal}:

\begin{theorem}
    [20] Every \(\chi \in R(G)\) is a \(\mathbb{Z}\)-linear combination of monomial characters.

    This is stronger than Brauer Induction Theorem.
\end{theorem}

Why does Brauer induction theorem imply this?

We want to show: Every character of an elementary group is a monomial.

\begin{definition}
    \(G\) is \underline{supersolvable} if:
    
    \[
        \exists 1 = G_0 \subset G_1 \subset \cdots \subset G_n = G
    \]

    such that \(G_i \triangleleft G\) and \(G_i / G_{i-1}\) is cyclic.

    Sylow theorem \(\implies p\)-groups are super solvable.

    Hence elementary subgroups are super-solvable.
\end{definition}

\begin{remark}
    \(p\)-group \(\implies\) nilpotent \(\implies\) super-solvable \(\implies\) solvable.
\end{remark}

\begin{definition}
    \(R\)-module
\end{definition}

Our goal changes to proving: every character of super-solvable group is monomial.

\begin{definition}
    \(R\)-module \(M\) is isotypic if \(M\) is a direct sum of simple, isomorphic submodules.

    \[
        M \cong S \oplus \cdots \oplus S
    \]
\end{definition}

\begin{proposition}
    [24] Suppose \((\operatorname{char} k, \vert G \vert)=1\). Suppose \(V\) is an irreducible \(kG\)-module and \(A \triangleleft G\). Then either:

    \begin{enumerate}[label=\alph*)]
        \item \(\exists\text{ proper }  H < G\) such that \(A < H\) and there eixsts an irreducible \(kH\)-module \(W\) such that \(V \cong \operatorname{Ind}_H^G W\)
        \item \(\eval{\operatorname{Res}}_A V\) is isotypic.    
    \end{enumerate} 
\end{proposition}

\begin{proof}
    \(V = \bigoplus_{i=1}^h V_i\) 

    \(V_i\) isotypic and \(i \neq j \implies V_i\) and \(V_j\) are disjoint.

    \(\forall s\in G\),

    \[
        s V_i = sA V_i \underset{A \triangleleft G}{=} As V_i
    \]

    Thus, \(s V_i = V_j\) for some \(j\).

    Thus, \(s: V \to V\) permutes \(V_i\) transitively [since \(W\) is irreducible].

    \underline{Case b}: \(V = V_1\).  

    \underline{Case a}: \(H = \operatorname{Stab}(V_1) = \{ s\in G \mid s V_1 = V_1 \} < G\) proper \(\implies W = \operatorname{Ind}_H^G V_1\).

    \begin{remark}
        If \(A\) is abelian and \(k=\mathbb{C}\) then Case b \(\iff \rho(a) = \alpha I\, \forall a\in A\).
    \end{remark}

\end{proof}

\section*{Wednesday, 10/23/2024}

Goal: Theorem 20: \(R(G)\) is generated by \underline{monomial characters}

Recall: \(R\)-module \(M\) is isotypic if:

\[
    M \cong S \oplus \cdots \oplus S
\]

where \(S\) is simple.

We also have proposition 24: Suppose we are in the Maschke case \((\operatorname{char} k, G) = 1\) and \(V\) is an irreducible \(kG\)-module and \(A \triangleleft G\).

Then either:

\begin{enumerate}[label=\alph*)]
    \item \(\exists\) proper \(H < G\) containing \(A\) and irreducile \(kH\)-module \(W\) such that \(V \cong \operatorname{Ind}_H^G W\) or:
    
    \item \(\operatorname{Res}_A V\) is isotypic.
\end{enumerate} 

\begin{proof}
    \(\operatorname{Res} _A V = V_1 \oplus \cdots \oplus V_n\) isotypic, nonzero, disjoint (meaning no common irreducible subrepresentation).

    Then \(\forall s\in G, s V_i = V_j\) [use \(A\) normal \(\implies s V_i\) is isotypic]

    \(V\) irreducible \(\implies G\) permutes \(V_i\) transitively.

    Let \(H = \{ s\in G \mid s V_1 = V_1 \}\). Let \(W = V_1\).
    
    Then \(V = \operatorname{Ind}_H W\).

    \(n > 1\) puts us in case a, \(n = 1\) gives us case b.
\end{proof}

\begin{remark}
    If \(V\) is a \(\mathbb{C} A\) module and \(A\) is abelian, \(\rho: G \to GL(V)\) 
    
    Then \(V\) is isotypic \(\iff \forall a\in A, \exists \alpha \in \mathbb{C}^\times\) such that \(\rho(a)= \alpha I\). 

    Why \(\mathbb{C}\)? Then representation is \(1\)-dimensional since \(A\) is abelian.
\end{remark}

\begin{corollary}
    Consider abelian \(A \triangleleft G\). Let \(V\) be a simple \(\mathbb{C} G\) module and \(d = \dim_\mathbb{C} V\).

    Then \(d\mid (G:A) = \frac{\vert G \vert }{\vert A \vert}\).
    
    eg \(C_p \triangleleft D_{2p} \implies d = 1,2\).
    
    In \(C_7 \rtimes C_3\) since \(C_7\) is normal \(d\mid \frac{21}{7} = 3\) so \(d=1,3\). 
\end{corollary}

\begin{proof}
    Recall \(d\mid \vert G \vert\) [on page 52].

    We also have \(d\mid (G : Z(G))\) [on page 53].
    
    We use the second result to prove this. We use induction on \(\vert G \vert\).

    We use Proposition 24/77:

    Case a:

    \[
        d \underset{\text{induction hypothesis}}{\mid} (H:A) \mid (G:A)
    \]

    Case b: \(\operatorname{Res}_A \rho\) is isotypic.

    \(\rho: G \to GL(V), G^{\prime} = \rho(G), A^{\prime} = \rho(A)\).

    \(G / A \underset{\rho}{\twoheadrightarrow} G^{\prime} / A^{\prime} \) 

    \begin{remark}
        \(A^{\prime} \subset Z(G^{\prime})\)

        \[
            d \underset{p.53}{\mid} [G^{\prime} : Z(G^{\prime} )] \mid [G^{\prime} : A^{\prime}] \mid [G:A]
        \]
    \end{remark}

\end{proof}

Recall irreducible \(\mathbb{C} G\)-module \(V\) is monomial if it is induced from a \(1\)-dim representation.

\begin{definition}
    \(G\) is \(\begin{Bmatrix}
        \text{supersolvable}  \\
        \text{solvable}
    \end{Bmatrix}\) if \(\exists 1 = G_0 \subset G_1 \subset \cdots \subset G_n = G\) such that \(\begin{Bmatrix}
        G_i \triangleleft G \\
        G_i \triangleleft G_{i+1}
    \end{Bmatrix}\) and \(G_i / G_{i-1} \) is \(\begin{Bmatrix}
         \text{cyclic} \\ \text{abelian} 
    \end{Bmatrix}\) 
\end{definition}

\begin{theorem}
    Evey irreducible representation of a semsimple group is \underline{monomial}.
\end{theorem}

\begin{lemma}
    [4] Let \(G\) be a non-abelian supersolvable group. Then \(\exists\) abelian \(A \triangleleft G\) such that \(A \not\subset Z(G)\).
\end{lemma}

\begin{proof}
    \(H = G / Z(G)\) is supersolvable. \(\implies \exists\) cyclic normal \(1 \neq H_1 \triangleleft H\).

    Let \(A = \pi ^{-1} H_1\) where \(\pi : G \to G / Z(G)\).

    Claim:

    \[
        1 \to \underset{\text{central}}{A} \to B \to \underset{\text{cyclic}}{C} \to 1 \implies B \text{ abelian} 
    \]

    choose \(b\in B\) such that \(\langle \text{im} b \rangle = C\).
    
    Every element of \(B\) looks like \(a b^i\):

    \(a b^i \underline{a} \underline{b}^j = \underline{a}\underline{b}^j a b^i\).

    \(a \in Z(B)\).
\end{proof}

\begin{proof}
    [Proof of theorem 16] induction on \(\vert G \vert. \rho : G \to GL(V)\), irreducible, \(G\) supersolvable.

    Case 1: \(\rho \) not injective. \(\overline{\rho}: G / \ker \rho \to GL(V)\).

    \(\overline{\rho} = \operatorname{Ind}_{\overline{H}}^{\rho(G)}\) (\(1\)-dim) by induction hypothesis so \(\rho = \operatorname{Ind}_{\rho ^{-1} \overline{H}}\) is \(1\) dim.
    
    Case 2: \(G\) abelian then we're done.

    Case 3: irreducible \(\rho : G \rightarrowtail GL(V)\) and \(G\) not abelian.

    Lemma 4 \(\implies \exists\) abelian \(A \triangleleft G, A \not\subset Z(G) \implies \rho(A) \not\subset Z(\rho(G)) \implies \exists a\in A\) such that \(\rho(a)\not \subset Z(\rho(G)) \implies\) remarkin case a.
\end{proof}

\begin{corollary}
    Every irreducible representation of elementary group is monomial.
\end{corollary}

\begin{corollary}
    [using BIT] Theorem 20
\end{corollary}

\section*{Friday, 10/25/2024}

3 Applications of rep theory to group theory:

\underline{Exercise 8.6}:

\begin{theorem}
    [Burnside's Theorem] Let \(\#G = p^a q^b\) where \(p,q\) are primes. Then \(G\) is not simple (\(\exists 1 < N \triangleleft G\)), all proper.
\end{theorem}

\underline{Frobenius I (Exercise 7.3)} 

If \(G \acts X\) effectively, transitvely, \(\forall g\in G \setminus e, X^g\) is a point or empty. Then,

\[
    G \cong H \rtimes K
\]

\(H = \operatorname{Stab}(x_0)\) for some \(x_0\in X\). 


For example, \(D_6 \acts \triangle\) so \(D_6 = C_2 \rtimes C_3\).

\underline{Frobenius II (Corollary 2, page 83)}

Suppose \(n\mid \#G\). Then,

\[
    n \mid \#\{ x\in G \mid x^n - 1 \}
\]

\underline{Suggestion} 

Look at exercises for Chapter 12.

\subsection*{Chapter 12 Rationality}

\(\mathbb{Q} C_2 \cong \mathbb{Q}_+ \times \mathbb{Q}_-\)

\(\mathbb{Q} C_p \cong \mathbb{Q} \times \mathbb{Q}(\zeta_p)\)

\(D_{2p}\) has \(C_p\) inside of it.

\[
    \mathbb{Q} D_{2p} \cong \underbrace{\mathbb{Q}_+}_{r \mapsto 1, s \mapsto 1} \times \underbrace{\mathbb{Q}_-}_{r \mapsto 1, s \mapsto -1} \times M_2(\mathbb{Q}[\lambda_p])
\]

\[
    \mathbb{Q} Q_8 \cong \mathbb{Q}_{++} \times \mathbb{Q}_{+-} \times \mathbb{Q}_{-+} \times \mathbb{Q}_{--} \times \mathbb{Q}[i,j,k]
\]

\(\mathbb{R} C_2 \cong \mathbb{R}_+ \times \mathbb{R}_-\) 

\(\mathbb{R} C_p \cong \mathbb{R} \times \mathbb{C}^{\frac{p-1}{2}} = \mathbb{R} \times \mathbb{C}_1 \times \cdots \times \mathbb{C}_{\frac{p-1}{2}}\) 

\[
    \mathbb{R} D_{2p} \cong \mathbb{R}_+ \times \mathbb{R}_- \times M_2(\mathbb{R})^{\frac{p-1}{2}}
\]

\[
    \mathbb{R} Q_8 \cong \mathbb{R}^4 \times \mathbb{H}
\]

\[
    \mathbb{H} = \mathbb{R}(i,j,k)
\]

\(\mathbb{C} C_2 \cong \mathbb{C}_+ \times \mathbb{C}_-\) 

\(\mathbb{C} C_p \cong \mathbb{C}_0 \times \mathbb{C}_1 \times \cdots \times \mathbb{C}_{p-1}\)

Where we map to \(\zeta_p^k\) at \(\mathbb{C}_k\). 

\(\mathbb{C}_1 \cong \mathbb{C}_{p-1}\) as \(\mathbb{R} C_p\) modules \([z \mapsto \overline{z}]\)

\(\mathbb{C}_1 \not\cong \mathbb{C}_{p-1} \) as \(\mathbb{C} C_p\)-modules.

\[
    \mathbb{C} D_{2p} \cong \mathbb{C}_+ \times \mathbb{C}_- \times M_2(\mathbb{C})^{\frac{p-1}{2}}
\]

\[
    \mathbb{C} Q_8 \cong \mathbb{C}^4 \times M_2(\mathbb{C})
\]

\[
    D_{2p} \to GL(\mathbb{C}^2)
\]

\[
    r \mapsto \begin{bmatrix}
        \zeta_p &  0 \\
        0 &  \zeta_p ^{-1} \\
    \end{bmatrix}
\]

\[
    s \mapsto \begin{bmatrix}
        0 &  1 \\
        1 &  0 \\
    \end{bmatrix}
\]

\[
    D_{2p} \to GL(\mathbb{R}^2)
\]

\[
    r \mapsto \begin{bmatrix}
        0 &  -1 \\
        1 &  \lambda_p \\
    \end{bmatrix}
\]

\[
    s \mapsto \begin{bmatrix}
        0 &  1 \\
        1 &  0 \\
    \end{bmatrix}
\]

Note that the matrices that map from \(r\) are conjugate over \(\mathbb{C}\). Both have the same characteristic polynomial: \(x^2 - \lambda_p x + 1\).

\subsection*{12.1}

Suppose \(K\) is a subfield of \(\mathbb{C}\).

\[
    \{ kG\text{-mod} \} \to \{ \mathbb{C} G\text{-mod} \} 
\]

\[
    V \mapsto V_\mathbb{C} = \mathbb{C} G \otimes_{KG} V = \mathbb{C} \otimes_K V
\]

\[
    \begin{Bmatrix} \text{central} \\ \text{idempotents of } \\ KG \end{Bmatrix} \to \begin{Bmatrix} \text{central} \\ \text{idempotents of } \\ \mathbb{C} G  \end{Bmatrix}
\]

\underline{Question}: What about irreducible representation?

\(V\) irreducible \(\overset{?}{\implies} V_\mathbb{C}\) irreducible?

\(W\) irreducible over \(\mathbb{C} G \overset{?}{\implies} W \cong V_\mathbb{C}\) for some \(V\).

\underline{Question}: What about primitive central idempotents?

\begin{center}
    \begin{tikzcd}
        G \ar[r,"\rho"] \ar[rr, bend right, "\rho_\mathbb{C}"] & GL_K(V) \ar[r, "\operatorname{Id}\otimes -"] & GL_{\mathbb{C}}V_{\mathbb{C}}
    \end{tikzcd}
\end{center}

\(\chi_p = \Tr(\rho) = \Tr(\rho_\mathbb{C}) = G \to K\).

\begin{definition}
    \(\mathbb{C} G\)-module \(W\) is \underline{realizable} over \(K\) if \(W \cong V_\mathbb{C}\) for some \(kG\)-mod \(V\).
\end{definition}

Consider the \underline{Representation Ring} \(RG = R_\mathbb{C} G\).

\(R_K G =\) subring of class function \(f: G \to K\), generated by the characters of \(K\)-representation.

\(R_K G\) is a subring of \(RG\).

\[
    = \operatorname{Gr}(\operatorname{Isom}(\text{f.g. }KG \text{-mod}), \oplus)
\]

``virtual representations"

Let \(\chi_1, \cdots , \chi_n\) be distinct irreducible character of \(KG\).

\(R_K(G) = \mathbb{Z} \chi_1 \oplus \cdots \oplus \mathbb{Z} \chi_n\) additively.

\(\{ \chi_i \} \) are orthogonal [but not orthonormal] under the usual bilinear form:

\[
    \langle f,g \rangle = \frac{1}{\# G} \sum f(\sigma)g(\sigma^{-1})
\]

\begin{theorem}
    [12.3] Every \(\mathbb{C}\)-rep of \(G\) is realizable over \(\mathbb{Q}(\zeta_{\vert G \vert})\).

    In fact let \(m = \text{l.c.m}\{ \text{order}(g) \mid g \in G \} \mid \#G\).

    Every \(\mathbb{C}\) representation of \(G\) is realizable over \(\mathbb{Q}(\zeta_m)\).
\end{theorem}


\section*{Monday, 10/28/2024}

\begin{proof}
    Special case: \(G\) abelian.

    Follows since irreducible rep \(G \to \mathbb{C}^\times\).

    General case: Let \(\chi\in R(G)\).

    Monomial representations generate \(R(G)\).

    \[
        \chi = \sum_{i} n_i \operatorname{Ind}_{H_{i}}^G(\phi_i) \quad \phi_i \text{ 1-dim}.
    \]

    Then \(\phi_i : H \to \mathbb{C}^\times\)

    \(\phi_i(H) \subset \mathbb{Q}(\zeta_m)\) 

    Thus \(\operatorname{Ind}_{H_i}^G(\phi_i) \subset \mathbb{Q}(\zeta_m)\).

    Therefore \(\chi\in R_{\mathbb{Q}(\zeta_m)}G\).
\end{proof}

\subsection*{12.2 Brauer Groups}

\begin{definition}
    A \underline{central simple algebra} over \(K\) is:

    A simple ring \(A\).

    \(K = Z(A)\).

    \((A : K) < \infty\).
\end{definition}

\begin{example}
    \(\mathbb{H}\) is a CSA over \(\mathbb{R}\).
\end{example}

Recall that a simple ring is simply a matrix ring over a division algebra.

Artin Wedderbern \(\implies A \cong M_n(D)\) where \(D\) is a  central simple \underline{division} algebra over \(K\).

Facts:

\begin{enumerate}[label=\arabic*)]
    \item \(A,B\) csa \(/ K \implies A \otimes_K B\) is csa \(/ K\).
    \item \(K\) subfield of \(L\) and \(A\) case \(/ K \implies  L \otimes_K A\) is csa \(/ L\).
    \item \(K\) alg. closed and \(A\) csa \(/ K \implies A \cong M_n(K)\). 
\end{enumerate} 

\begin{definition}
    \(L\) is a splitting field for csa \(A\) if

    \[
        L \otimes_K A \cong M_n L
    \]
\end{definition}


Facts \(\implies\) Algebraically closed is splitting field for \(A\).

3 \(\implies (A:K) = m^2\) since \((A:K) = (A_L : L)\) where \(L\) is splitting field which has dimension \(m^2\) since it is isomorphic to \(M_m L\). \(m = \sqrt{A:K} \) is the \underline{Schur Index} 

\underline{Harder Fact}: maximal subfield of \(A\) is splitting field for \(A\).

e.g. \(\mathbb{C} \otimes _\mathbb{R} \mathbb{H} \cong M_2\mathbb{C}\).

If \(D\) is a skew field CSA \(/ K\) then \((D:K) = m^2\) where \(m =\) Schur index of \(D\).

\(A\) case \(/ K\) so schur index of \(A\) is divisible by schur index of \(D\).

\begin{definition}
    [Brauer Group] Let \(K\) be a field.

    \[
        \operatorname{Br}(K) = \left( \frac{\text{csa} / K}{M_n(D) \sim D} \right) , \otimes_K
    \]
\end{definition}

eg \(\operatorname{Br} \mathbb{C} = 1\) 

\(\operatorname{Br}\mathbb{R} = C_2 = \langle \mathbb{H} \rangle \). \(\mathbb{H} \otimes_\mathbb{R} \mathbb{H} \cong M_4(\mathbb{R})\)

\(\operatorname{Br}(K) = H^2 (\operatorname{Gal}(\overline{K} / K); \mathbb{Z} / 2)\) 

\subsection*{12.2 Schur Indices}

\[
    \mathbb{R} Q_8 \cong \mathbb{R}^4 \times \mathbb{H} 
\]

\[
    \mathbb{C} Q_8 \cong \mathbb{C}^4 \times M_2\mathbb{C}
\]

\[
    i \mapsto \begin{bmatrix}
        0 &  -1 \\
        1 &  0 \\
    \end{bmatrix}
\]

\[
    j \mapsto \begin{bmatrix}
        i &  0 \\
        0 &  -i \\
    \end{bmatrix}
\]

Consider \(\mathbb{R} Q_8\) module \(V = \mathbb{H}\) and \(\mathbb{C}Q_8\) module \(W = \mathbb{C}^2\) not realizable over \(\mathbb{R}\).

\(\chi_V(\pm 1) = \pm 4\)

\(\chi_V(\pm i, \pm j,\pm k)=0\) 

\(\chi_W(\pm 1) = \pm 2, \chi_W(\pm i, \pm j, \pm k)=0\)

We have:

\[
    kG \cong \prod M_{n_i} (D_i)
\]

\[
    K_i = \text{center} D_i
\]

\[
    \text{schur index} m_i = \sqrt{(D_i : K_i)} 
\]

eg \(G = Q_8, K = \mathbb{R} , m_5 = 2\).

\begin{definition}
    \(R_K(G) \subset \overline{R}_K G = \{ f\in R(G) \mid f(G) \subset K \} \subset R(G)\) 
\end{definition}

eg \(\chi_W = \chi_{\mathbb{C}^2} \in \overline{R}_{\mathbb{R}}(Q_8) - R_\mathbb{R} (Q_8)\) 

\begin{proposition}
    [35] \(\chi_1, \cdots , \chi_h\) are the irreducible characters of \(KG\). Then they are \(\mathbb{Z}\) basis for \(R_K G\). Then,

    \(\frac{\chi_1}{m_1}, \cdots , \frac{\chi_h}{m_h}\) are a \(\mathbb{Z}\)-basis for \(\overline{R}_K G\). 
\end{proposition}

\begin{corollary}
    \(R_K (G) \subset \overline{R} _K (G)\) finite index with equality iff all \(D_i\) are fields.
\end{corollary}

\section*{Wednesday, 10/30/2024}

\subsection*{12.4 Rank \(R_K G\)}

\(\mathbb{C} C_p \cong \mathbb{C}^p\) 

\(\mathbb{Q} C_p \cong \mathbb{Q} \times \mathbb{Q}(\zeta_p)\)

\(\zeta = \zeta_m = e^{2\pi i / m}\) where \(m\) is multiple of \(lcm(\operatorname{ord}(g))\) e.g. \(m = \vert G \vert\).

\begin{center}
    \begin{tikzcd}
        & & \mathbb{C} \\
        & L = K(\zeta) \ar[ru, no head] & \\
        K \ar[ru, no head] & &
    \end{tikzcd}
\end{center}

\(LG \cong \prod M_{n_i} (L)\)

\[
    \begin{array}{rcl}
        \rank RG & = & \# \text{ of irreducible } \mathbb{C} G \text{-modules}  \\
         & = &  \# \text{ of irreducible } LG \text{-modules} \\
         & = & \# \text{ of conjugacy classes of } G   \\
    \end{array}
\]

What about \(\#\) of irreducible \(KG\)-reps?

\[
    \Gamma = \Gamma_K \coloneqq \{ t\in (\mathbb{Z} / m)^\times \mid \exists \sigma \in \operatorname{Gal}(L / K) \text{s.t. } \sigma(\zeta) = \zeta^t\} < (\mathbb{Z} / m)^\times
\]

\[
    \Gamma = \operatorname{image} (\underset{\sigma_t}{\operatorname{Gal}(L / K)} \underset{\mapsto}{\rightarrowtail} \underset{t}{(\mathbb{Z} / m)^\times}) 
\]

where \(\sigma_t(\zeta) = \zeta^t\).

eg \(\Gamma_\mathbb{Q} = \operatorname{Gal}(\mathbb{Q}(\zeta_m) / \mathbb{Q}) \cong (\mathbb{Z} / m)^\times\)

\(\Gamma_\mathbb{C} = 1\)

\(\Gamma_\mathbb{R} = \begin{dcases}
    1, &\text{ if } m \text{ odd}  ;\\
    \pm 1, &\text{ if } m \text{ even}.
\end{dcases}\) 

\begin{definition}
    \(s, s^{\prime} \in G\) are \(\Gamma_K\)-conjugate if \(\exists \tau \in G, t \in \Gamma_K\) such that:

    \[
        \tau s^{\prime} \tau ^{-1} = s^t
    \]

    we write \(s^{\prime} \underset{K}{\sim} s\) 
\end{definition}

\begin{corollary}
    [page 96]

    \(\rank R_K G = \#\) of \(\Gamma_K\) conjugacy classes. 
\end{corollary}

If \(G = C_p\) then \(\Gamma_Q\) conjugacy classes are \(\{ 1 \}, \{ r^t \}_{t\not\equiv o(p)}\)

Recall that \(\mathbb{R} C_p \cong \mathbb{R} \times \mathbb{C}^{\frac{p-1}{2}}\)

\(G = C_p\) then \(\Gamma_\mathbb{R}\) conjugacy classes are \(\{ 1 \}, \{ r, r ^{-1} \}, \{ r^2, r^{-2} \}, \cdots ,\{ r^{\frac{p-1}{2}}, r^{ \frac{p-1}{2} } \} \) 

We have:

\[
    R G \to \operatorname{Cl}_L G = \{ f: G \to L \mid f(\tau s \tau ^{-1}) = f(s) \}
\]

We can take \(K\) linear combinations of this.

\[
    K \otimes_\mathbb{Z} R G \hookrightarrow \operatorname{Cl}_L G = \{ f: G \to L \mid f(\tau s \tau ^{-1}) = f(s) \}
\]

\begin{theorem}
    [25] Let \(f\in \operatorname{Cl}_L G\). TFAE:

    \begin{enumerate}[label=\alph*)]
        \item \(f\in K \otimes_\mathbb{Z} RG\) 
        \item \(\forall t \in \Gamma, \forall s\in G\) we have \(\sigma_t (f(s)) = f(s^t)\)  
    \end{enumerate} 
\end{theorem}

\begin{proof}
    a \(\implies\) b: It is enough to show it for characters. We want to show for \(\chi_{\rho}\) where \(\rho: G \to GL(\mathbb{C}^n)\). Let \(\lambda_1, \cdots , \lambda_n\) be eigenvalues of \(\rho(s)\). They must all be roots of unity. Then \(\chi_{\rho} (s) = \sum_{i} \lambda_i\).

    \[
        \sigma_t (\chi_{\rho} (s)) = \sigma_t \left(\sum_{i} \lambda_{i}\right) = \sum_{i} \lambda_i ^ t = \chi_{\rho} (s^t)
    \]

    b \(\implies\) a: Let \(f\in \operatorname{Cl}_L\).

    Irreducible characters form an orthonormal basis.

    \[
        f = \sum_{\chi \text{ irr}} \langle f, \chi \rangle \chi
    \]

    \(\forall t \in \Gamma_K\) we have:

    \[
        \langle f, \chi \rangle = \frac{1}{\vert G \vert}\sum_{s\in G} f(s) \chi(s ^{-1}) \underset{\text{reindex}}{=}  \frac{1}{\vert G \vert}\sum_{s\in G} f(s^t) \chi(s^{-t})
    \]
    
    \[
        = \frac{1}{\vert G \vert}\sum_{s\in G} \sigma_t(f(s)) \sigma_t (\chi (s ^{-1})) = \sigma_t(\langle f, \chi \rangle)
    \]

    Thus, \(\langle f, \chi \rangle\) are invariant under Galois therefore \(\langle f, \chi \rangle \in K\) which is what we wanted to prove.
\end{proof}

\begin{corollary}
    [1] Let \(f\in \operatorname{Cl}_K\).

    \(f \in K \otimes R_K G \iff f\) is constant on \(\Gamma_K\) conjugacy classes.
\end{corollary}

\begin{proof}
    \(\implies\): WLOG \(f = \chi_{\rho}\) where \(\rho: G \to GL(K^n)\).

    \(\tau s^{\prime} \tau ^{-1} = s^t\)

    \(\implies \chi_{\rho} (s^{\prime}) = \chi_{\rho} (s^t) \underset{25b}{=} \sigma_t \chi_{\rho} (s) \underset{\chi_{\rho}(s) \in K}{=} \chi_{\rho} (s)\).
    
    \(\impliedby:\) \(f: G \to K\) is constant on \(\Gamma_K\) conjugacy classes.

    Thus, 25b holds for \(f\).

    Thus, 25a holds for \(f\).

    Thus, \(f\in K \otimes_\mathbb{Z} RG\).

    \[
        f = \sum_{\text{irr } \chi \in R_L(G)} \langle f, \chi \rangle \chi
    \]

    We need to take \(L\) representations to \(K\) representations.

    \[
        f = \frac{1}{\vert \Gamma_K \vert} \sum_{\text{irr } \chi \in R_L(G), t \in \Gamma_K} \langle f, \sigma_t \circ \chi \rangle (\sigma_t \circ \chi) 
    \]

    \[
        f = \frac{1}{\vert \Gamma_K \vert} \sum_{\text{irr } \chi \in R_L(G), t \in \Gamma_K} \langle \sigma_{t ^{-1}} \circ f, \chi \rangle (\sigma_t \circ \chi) 
    \]

    \[
        f = \frac{1}{\vert \Gamma_K \vert} \sum_{\text{irr } \chi \in R_L(G), t \in \Gamma_K} \underbrace{\langle f, \chi \rangle}_{\in K} (\sigma_t \circ \chi) 
    \]

    \[
        f = \frac{1}{\vert \Gamma_K \vert} \sum_{\text{irr } \chi \in R_L(G)} \langle f, \chi \rangle \sum_{t}  (\sigma_t \circ \chi) 
    \]

    \[
        f = \frac{1}{\vert \Gamma_K \vert} \sum_{\text{irr } \chi \in R_L(G)} \langle f, \chi \rangle (\Tr \chi)
    \]

    Last equality is due to the fact:

    \(G \overset{\rho}{\to} GL_L (L^n) \overset{\Tr}{\to} GL_K(L^n)\)

    \(\chi_{\Tr \circ \rho} = \sum \sigma_t \circ \chi_{\rho}\)
\end{proof}

\section*{Friday, 11/1/2024}

Recap:

\begin{center}
    \begin{tikzcd}
        & & \mathbb{C} \ar[dl, no head] \\
        & L = K(\zeta_m) \ar[dl, no head] & m = \exp G \\
        K & &
    \end{tikzcd}
\end{center}

\[
    R_L G \overset{\cong}{\to} RG \coloneqq R_\mathbb{C} G
\]

\[
    \Gamma_K = \operatorname{image} \left( \operatorname{Gal}(L / K) \to (\mathbb{Z} / m)^\times \right) 
\]

\[
    \sigma_t \mapsto t
\]

\[
    \sigma_t(\zeta_m) = \zeta_m^t
\]

\[
    s^{\prime} \underset{K}{\sim} s \, (s^{\prime} \text{ is } K \text{-conjugate to } s)
\]

If \(\exists \tau \in G, t \in \Gamma_K\) such that:

\[
    \tau s^{\prime} \tau ^{-1} = s^t
\]

Corollary 2, page 96: \(\rank R_K G = \#\) of \(K\)-conj classes.

13.1: \(K = \mathbb{Q}\). Then,

\[
    \operatorname{Gal}(\mathbb{Q}(\zeta_m) / \mathbb{Q}) \overset{\cong}{\to} (\mathbb{Z} / m)^\times
\]

Thus,

\[
    s^{\prime} \underset{\mathbb{Q}}{\sim} s \iff \exists \tau \in G \text{ s.t. } \tau \langle s^{\prime} \rangle \tau ^{-1} = \langle s \rangle 
\]

Corollary 1: \# of \(\mathbb{Q} G\)-reps = \# of conjugacy classes of cyclic subgroups.

Corollary 2: \(G\) finite, following TFAE:

\begin{enumerate}[label=\roman*)]
    \item \(\langle s \rangle = \langle s^{\prime} \rangle \implies s\) is conjugate to \(s^{\prime}\).
    \item \# of conjugacy classes = \# of conjugacy classes of cyclic subgroups.
    \item \# of p.c.i in \(\mathbb{Q} G\) = \# of p.c.i in \(\mathbb{C} G\)
    \item \(\forall \rho: G \to GL(\mathbb{C}^n), \forall s\in G, \chi_{\rho} (s) \in \mathbb{Q}\) [characters are rational valued].
    \item \(\forall \rho : G \to GL(\mathbb{C}^n), \forall s\in G, \chi_{\rho}(s) \in \mathbb{Z}\).
\end{enumerate} 

\begin{proof}
    ``Think about it''
\end{proof}

eg Symmetric grouo \(S_n\) satisfies (i).

Fact [stronger than this] \(\mathbb{Q} S_n \cong \prod M_{n_i} (\mathbb{Q})\) 

eg \(\mathbb{Q} S_3 = \mathbb{Q} D_6 \cong \mathbb{Q} \times \mathbb{Q} \times M_2(\mathbb{Q}[\lambda_3]) = \mathbb{Q} \times \mathbb{Q} \times M_2(\mathbb{Q})\).

All \(\mathbb{C}\)-rep of \(S_n\) are realizable over \(\mathbb{Q}\).

``Young diagrams''.

\(G = Q_8\) also satisfies (i).

\(\mathbb{Q} Q_8 \cong \mathbb{Q}^4 \times \mathbb{H}_\mathbb{Q}\) 

\(\mathbb{C} Q_8 \cong \mathbb{C}^4 \times M_2(\mathbb{C})\)

But irreducible representation \(\mathbb{C}^2\) not realizable over \(\mathbb{Q}\).

\subsection*{12.5}

\(\begin{matrix}
     &  &  \mathbb{C} \\
     & / &   \\
    K &  &   \\
\end{matrix}\) 

\begin{theorem}
    [Artin's Theorem]

    \[
        \bigoplus_{\text{cyclic } C < G} R_K C \otimes \mathbb{Q} \twoheadrightarrow R_K G \otimes \mathbb{Q}
    \]

    Same proof as for \(K = \mathbb{C}\).

    Characters are determined by cyclics.
\end{theorem}

\begin{theorem}
    [Brauer's Theorem]

    \[
        \bigoplus_{\text{elem } E < G} RE \twoheadrightarrow RG
    \]
\end{theorem}

\begin{definition}
    \(E\) is \underline{elementary} if \(E = P \times C\) where \(P\) is \(p\)-group, \(C\) is cyclic, \((\vert P \vert, \vert C \vert) = 1\) 
\end{definition}

\begin{theorem}
    [Brauer's Theorem]

    \[
        \bigoplus_{\Gamma_{K}\text{-elem } E < G} R_K E \twoheadrightarrow R_K G 
    \]
\end{theorem}

\begin{definition}
    \(E\) is \(\Gamma_K\)-elementary if \(E = C \rtimes_{\phi} P, P\) \(p\)-group, \(C\) cyclic, \((\vert P \vert, \vert C \vert)=1\)

    If \(\phi\) factos as

    \begin{center}
        \begin{tikzcd}
            P \ar[r] \ar[rrr, bend right, "\phi"] & \Gamma_K \ar[r, hook] & (\mathbb{Z}/m)^\times \ar[r, two heads] & \operatorname{Aut}(C)
        \end{tikzcd}
    \end{center}
\end{definition}

\subsection*{13.2 \(K = \mathbb{R}\)}

Fact: Only finite dimensional division algebras \(/\mathbb{R}\) are \(\mathbb{R}, \mathbb{C}\) and \(\mathbb{H}\).

``Proof'': \(\operatorname{Br} \mathbb{R} = H^2(\operatorname{Gal}(\mathbb{C} / \mathbb{R}); \mathbb{Z} / 2) = \{ \mathbb{R} , \mathbb{H} \}\).

\(\begin{matrix}
     &  & \mathbb{C} &  \text{alg closed} \\
     & / & \deg 2 &   \\
    \mathbb{R} &  &  &   \\
\end{matrix}\)

\(\operatorname{Br} \mathbb{C} = 1\).

Thus only \(\mathbb{R} ,\mathbb{C} ,\mathbb{H}\) are possible.

We achieve all:

\(\mathbb{R} C_2 \cong \mathbb{R} \times \mathbb{R}\).

\(\mathbb{R} C_3 \cong \mathbb{R} \times \mathbb{C}\)

\(\mathbb{R}Q_8\cong \mathbb{R}^4 \times \mathbb{H}\)

3 types of finite dimensional simple reps over \(\mathbb{R}\).

3 types of irreducible \(\mathbb{R}G\) reps

3 types of irreducible \(\mathbb{Q} G\) reps

Let \(\chi_0\) be char of irreducible \(\mathbb{R}G\) module.

\(\chi =\) char of irreduible \(\mathbb{C} G\) module

such that \(\chi\) is a component of \(\mathbb{C} \otimes_\mathbb{R} V_0 \iff \chi_0\) is a component of \(\operatorname{res} \chi\).

\underline{Type O}: \(\chi = \chi_0\). Complexification gives you the same representation.

\(\mathbb{R} = \operatorname{Hom}_{\mathbb{R} G}(V_0, V_0)\) by Schur.

\underline{Type U}: \(\chi \neq \overline{\chi}\). Then \(\chi_0 = \chi + \overline{\chi}\).

\(\mathbb{C} = \operatorname{Hom}_{\mathbb{R} G}(V_0, V_0)\)

\underline{Type \(S_P\)}: \(\chi = \overline{\chi}, \chi = 2 \chi_0\).

\(\mathbb{H} = \operatorname{Hom}_{\mathbb{R} G}(V_0, V_0)\) 

\begin{exercise}
    \(G\) odd order \(\implies\) all nontrivial irreducible representation have type \(U\).
\end{exercise}

\section*{Monday, 11/4/2024}

\subsection*{\(K = \mathbb{R}\)}

\(\mathbb{R} C_3 = \mathbb{R} \times \mathbb{C}\) 

\(\mathbb{R} Q_8 = \mathbb{R}^4 \times \mathbb{H}\)

\(\mathbb{C} C_3 = \underset{O}{\mathbb{C}_0} \times \underset{U}{\mathbb{C}_1} \times \underset{U}{\mathbb{C}_2} \) 

\(\mathbb{C} Q_8 = \underset{O}{\mathbb{C}^4}  \times \underset{S_P}{M_2(\mathbb{C})}\)

\(\chi\) type \(O\) if \(\chi\) is realizeable over \(\mathbb{R}\).

\(\chi\) is type \(U\) if \(\chi \neq \overline{\chi}\) 

\(\chi\) is type \(S_P\) if \(\chi =\overline{\chi} \) and \(\chi\) is not realizable \(/\mathbb{R}\).

Let \(i = \mathbb{R} G \hookrightarrow \mathbb{C} G\).

Let \(\chi_0\) be irreducible component of \(i^{\ast} x [=x \circ i]\).

\(\chi\) type \(O \iff \chi = \chi_0\) 

\(\chi\) type \(U \iff \chi_0 = \chi + \overline{\chi}\) 

\(\chi\) type \(S_P \iff \chi_0 = 2 \chi\) 

Goal: Propoistion 39:

\[
    \sum_{g\in G} \chi(g^2) = \begin{dcases}
        \vert G \vert , &\text{ if $\chi$ has type } O ;\\
        0, &\text{ if $\chi$ has type } U ;\\
        - \vert G \vert, &\text{ if $\chi$ has type } S_P.
    \end{dcases}
\]

Let \(V\) be finite dimensional vecto space over \(F\).

A bilinear \(B: U \times V \to F\) is nonsingular if:

\[
    \operatorname{Ad} B : V \overset{\cong}{\to} V^{\ast}
\]

given by

\[
    x \mapsto (y \mapsto B(x,y))
\]

\(\iff \forall\) basis \(\{ e_i \} \) for \(V,\) 

\[
    \det (B(e_i, e_j)) \neq 0
\]

\(V\) is a \(FG\)-module, so is \(V^{\ast} = \operatorname{Hom}_F (V,F)\). Action is like:

\[
    (g \phi)(v) = \phi (g ^{-1} v)
\]

\(F =\mathbb{C}\) then,

\[
    \chi^{\ast} (g) = \overline{\chi(g)} = \chi(g ^{-1})
\]

\begin{theorem}
    [31, FS]

    \(\rho: G \to GL_\mathbb{C} V, \chi = \chi_{\rho } : G \to \mathbb{C}\).

    \begin{enumerate}[label=\roman*)]
        \item \(\chi = \overline{\chi} \iff \exists\) nonsingular \(G\)-invariant form \(B: V \times V \to \mathbb{C}\).
        \item \(\chi\) realizable over \(\mathbb{R} \iff \exists\) nonsingular symmetric \(G\)-invariant \(B: V \times V \to \mathbb{C}\). 
    \end{enumerate} 
\end{theorem}

\begin{proof}
    \begin{enumerate}[label=\roman*)]
        \item \(\chi = \overline{\chi} (=\chi^{\ast}) \iff V \cong V^{\ast} \iff \exists G\)-invariant nonsingular bilinear \(V \times V \to \mathbb{C}\) 
        \item \(\implies\): Let \(V\) real / \(\mathbb{R}\).
        \(V = \mathbb{C} \otimes_\mathbb{R} V_0\) where \(V_0\) is \(\mathbb{R} G\) module.

        \(\exists\) symmetric, positive definite \(B: V_0 \times V_0 \to \mathbb{R}\).
        
        \(\implies\) symmetric, positive definite, \(G\)-invariant \(B_1: V_0 \to V_0\):

        \[
            B_1(x,y) = \frac{1}{\vert G \vert} \sum_{g\in G} B(gx, gy) 
        \]

        Extension of scalars: Define \(B_\mathbb{C}: V \times V \to \mathbb{C}\) by:

        \[
            B_\mathbb{C} (z \otimes v, z^{\prime}, z^{\prime} \otimes v^{\prime}) = z z^{\prime} B_\mathbb{C} (v, v^{\prime})
        \]

        \(\impliedby:\) (outline)

        Suppose we have nonsingular symmetric \(G\)-invariant \(B: V \times V \to \mathbb{C}\).

        Step 1: Choose \(G\)-invariant inner product:

        \[
            \langle-,-\rangle : V \times V \to \mathbb{C}
        \]

        [average any inner product]

        Step 2: Definea bijection \(\varphi: V \to V\):

        \[
            B(x,y) = \overline{\langle \varphi(x), y \rangle } 
        \]

        \(\varphi\) is conjugate linear. 

        Step 3: \(\varphi^2: V \to V\) is \(\mathbb{C}\)-linear, hermitian w.r.t.\ \(\langle -,- \rangle \) and has positive eigenvales.
        
        \[
            \langle \varphi^2 x, y \rangle = \langle x, \varphi^2 y \rangle
        \]

        Then \(\varphi^2\) has positive eigenvalues.

        Step 4: Spectral theorem \(\implies \exists !\) square root \(v: V \to V\) of \(\varphi^2\).

        \(v: V \to V\) of \(\varphi^2\).

        \(v\) is \(\mathbb{C}\)-linear, and \(v^2 = \varphi^2\) where \(v\) is hermitian, positive eigenvalues.

        Step 5: Let \(\sigma = \varphi \circ v ^{-1}\).

        \(\sigma: V \to V\) is the conjugate linear with \(\sigma^2 = \operatorname{Id}\).

        Step 6: \(\sigma\) eigenvalues are \(1\) and \(-1\). So we split into two eigenspaces: \(V = V_+ \oplus V_-\).

        \(i V_+ = V_- \implies V = \mathbb{C} \otimes_\mathbb{R} V_+\) (since \(V_+ = V_-\)).

    \end{enumerate} 
\end{proof}

\begin{corollary}
    Let \(V\) be an irreducible \(\mathbb{C} G\)-module.

    \begin{enumerate}[label=\alph*)]
        \item If \(\nexists\) non-zero \(G\)-invariant bilinear form \(V \times V \to \mathbb{C}\) then \(V\) has type \(U\).
        \item A non-zero \(G\)-invariant bilinear form \(V \times V \to G\) is unique up to a multiple.
        
        \(B\) symmetric \(\iff V\) has type \(O\).
        
        \(B\) alternating [\(B(x,y)=-B(y,x)\) ] \(\iff V\) has type \(S_P\).
    \end{enumerate} 

\end{corollary}

\begin{proof}

    Note that in irreducible, by Schur, nonsingular iff nonzero. This also gives us the uniqueness upto a multiple in ii.

    a \(\iff\) i: Contrapositive.

    ii: \(B(x,y) = \frac{B(x,y)+B(y,x)}{2} + \frac{B(x,y)-B(y,x)}{2} = B_+ + B_-\).

    Uniqueness \(\implies B_+ = 0\) or \(B_- = 0\).

    \(B\) symmetric \(\iff V\) type \(O\).

    \(V\) type \(S_P \iff\) not type \(O\) on \(V \iff B\) alternates.


\end{proof}

\section*{Wednesday, 11/6/2024}

\begin{proposition}
    [39] Let \(\chi = \chi_V\) be irreducible \(/ \mathbb{C} G\).

    \[
        \sum_{g\in G} \chi(g^2) = \begin{dcases}
            \vert G \vert & \text{ if \(\chi\) has type } O\\
            0 & \text{ if \(\chi\) has type } U\\
            - \vert G \vert & \text{ if \(\chi\) has type } S_P
        \end{dcases}
    \]
\end{proposition}

\begin{proof}
    Use sym and alt squares 1.6, 2.1, 13.2.

    \[
        sw: V \otimes_\mathbb{C} V \to V \otimes_\mathbb{C} V
    \]

    \[
        a \otimes b \mapsto b \otimes a
    \]

    \[
        sw^2 = \operatorname{id}
    \]

    We know that \(V \otimes_\mathbb{C} V = S(V) \oplus \Lambda(V) = V_\sigma \oplus V_a\) 

    \(S(V)\) is symmetric, \(+1\) eigenspace containing \(a \otimes a\) and \(a \otimes b + b \otimes a\).

    \(\Lambda(V)\) is altrnating, \(-1\) eigenspae containing \(a \otimes b - b \otimes a\).

    Then \((V_\sigma)^{\ast} = G\)-invariant symmetric \(V \times V \to \mathbb{C}\).

    \((V_a)^{\ast} = G\)-invariant alternating \(V \times V \to \mathbb{C}\).


    \[
        \begin{array}{cccc}
            \text{type} &  & \dim_\mathbb{C} \operatorname{Hom}_{\mathbb{C} G}(\mathbb{C} , V_\sigma ^{\ast}) &  \dim_\mathbb{C} \operatorname{Hom}_{\mathbb{C} G}(\mathbb{C} , V_a^{\ast}) \\
            O & \overset{Thm 35}{\iff} & 1 &  0 \\
            U &  & 0 &  0 \\
            S_P & & 0 & 1
        \end{array} \,(\ast)
    \]

    \[
        \dim_\mathbb{C} \operatorname{Hom}_{\mathbb{C} G} (\mathbb{C}, V_\sigma^{\ast}) = \langle 1, \overline{\chi}_\sigma  \rangle = \frac{1}{\vert G \vert} \sum_{g\in G} \overline{\chi}_\sigma (g ^{-1}) = \frac{1}{\vert G \vert} \sum_{g\in G} \chi_\sigma (g)  
    \]

    \[
        \dim_\mathbb{C} \operatorname{Hom}_{\mathbb{C} G}(\mathbb{C}, V_a^{\ast}) = \frac{1}{\vert G \vert} \sum_{g\in G} \chi_a(g)
    \]

\end{proof}

\begin{proposition}
    [3] \(\chi_\sigma (g) = \frac{\chi(g)^2 + \chi(g^2)}{2}, \chi_a(g) = \frac{\chi(g)^2 - \chi(g^2)}{2}\). 
\end{proposition}

\begin{proof}
    \(\rho_v(g)\) is diagonalizable with eigenvalue \(\lambda_i \implies \chi_v(g) = \sum_{i} \lambda_i\) with eigenvector \(e_i\).

    \(V_\sigma\) has eigenvectors \(e_i \otimes e_j + e_j \otimes e_i \, i \leq j\).

    \(V_a\) has eigenvectors \(e_i \otimes e_j - e_j \otimes e_i\, i < j\).

    \[
        \chi_\sigma (g) = \sum_{i \leq j} \lambda_i \lambda_j = \frac{\left( \sum_{i} \lambda_i \right)^2 + \sum_{i} \lambda_i^{2}}{2} = \frac{\chi(g)^2 + \chi(g^2)}{2}
    \]

    \[
        \chi_a(g) = \frac{\chi(g)^2 - \chi(g^2)}{2}
    \]

    Proposition 3 + Table \((\ast)\) + \((\ast \ast)\) implies Proposition 39.

    \[
        \chi_{V \otimes V}(g) = \chi^2(g) = \chi_\sigma (g) + \chi_a(g)
    \]

\end{proof}

\subsection*{Research Project?}

Consider ring \(R\) and nonzero divisor \(\Delta = \Delta_R = \left\{ r \in R \mid \forall r^{\prime} \in R - 0, \substack{r r^{\prime} \neq 0 \\ r^{\prime} r \neq 0} \right\} \).

\begin{definition}
    [Ore] A left classical ring of quotient (q.r. = quotient ring) of \(R\) is a ring homomorphism \(i:R \to A\):

    \(\forall a\in A, \exists r\in R, \exists \delta \in \Delta\) such that \(a = i(\delta)^{-1} i(r)\).

    We write:

    \[
        A = \Delta ^{-1} R
    \]
\end{definition}

eg if \(R\) is a commutative domain then \(\Delta ^{-1} R = \operatorname{Frac}(R)\).

Question: What rings have q.r.?

Question: For what group \(G\) does \(\mathbb{Z} G\) have a q.r.?

\(R\) commutative ring \(\implies \exists\) q.r. by localization.

\(G\) finite \(\implies \mathbb{Z} G\) has quotient ring, \(\Delta ^{-1} \mathbb{Z} G = \mathbb{Q} G\).

We don't know a lot about infinite groups.

\(\mathbb{F}_2\langle x,y \rangle\) non-commutative polynmials and \(\mathbb{Z} [F(2)]\) have no q.r.s.

\begin{proposition}
    \[
        R \text{ has q.r.} \iff \text{``Ore Conditions hold''}:
    \]

    \(\forall r\in R, \forall \delta \in \Delta,\)
    
    \[
        \Delta r \cap R \delta \neq \varnothing
    \]
\end{proposition}

\begin{definition}
    \(G\) is \underline{virtually abelian} if \(\exists \):

    \[
        1 \to \underset{abel}{A} \to G \to \underset{finite}{F} \to 1 
    \]

    \(G\) virtually abelian \(\implies \) q.r. for \(G\).

    \[
        \Delta_{\mathbb{Z} G}^{-1} G = \left( \Delta_{\mathbb{Z} A}^{-1} \mathbb{Z} A \right) \otimes_{\mathbb{Z} A} \mathbb{Z} F
    \]
\end{definition}

Now assume \(A = \mathbb{Z}^n\).

\[
    1 \to \mathbb{Z}^n \to G \to \underset{finite}{F} \to 1
\]

\begin{remark}
    \(G\) is classified by \(2\) invariants.

    \(F \to GL_n(\mathbb{Z})\)

    and an extension class \(\in H^2(F;\mathbb{Z}^n)\).
\end{remark}

\begin{theorem}
    \(\Delta ^{-1} \mathbb{Z} G\) is \underline{semisimple}.
\end{theorem}

\[
    \Delta ^{-1} \mathbb{Z} G \cong M_{d_i}(D_i)
\]

Research project: Redo Parts I and II of Serre. \(h = ?\) divisibility for \(d_i\)? types? Splitting fields? \(\mathbb{Q}(\zeta_{\vert F \vert}) \otimes_\mathbb{Z} \Delta ^{-1} \mathbb{Z} G \overset{?}{=} \prod M_j\) (fields)? induction theorem?

Warm up: \(G = \mathbb{Z}^n \rtimes S_n\).

Q: \(\Delta ^{-1} \mathbb{Z} G = ??\) 

\section*{Friday, 11/8/2024}

\subsection*{Modular Representation Theory}

Recall Maschke's theorem:

\(kG\) semisimple \(\iff (\operatorname{char} k, \vert G \vert)=1\).

We ask the question: what happens if \(\operatorname{char} k \mid \vert G \vert\)?

eg \(\mathbb{F}_p G\) where \(p\mid \vert G \vert\).

It is not semisimple, but it is not BAD. For example, they're Artinian.

Motivation:

1. (Jim) study \(\mathbb{Z} G\) modules.

\(G \acts \widetilde{X}\to X\), \(\pi_1 X = G\).

\(H_n \overline{X}, \pi_n \widetilde{X}\) are \(\mathbb{Z} G\) modules.

We can consider:

\begin{center}
    \begin{tikzcd}
        \mathcal{O}_K \ar[r, no head] \ar[d, no head] & K \ar[d, no head, "\text{Galois}"] \\ \mathbb{Z} \ar[r, no head] & \mathbb{Q}
    \end{tikzcd}
\end{center}

\(\mathcal{O}_K\) is \(\mathbb{Z}[\operatorname{Gal}(K / \mathbb{Q})]\).

2. Classification of (simple) groups.

3. Algebraic \(K\)-theory: \(K_{\ast} (\mathbb{F}_p)\). eg \(G = GL_2(\mathbb{F}_p)\).

4. Non-abelian class field theory: \(\operatorname{Gal} \to GL_n(\mathbb{Z}_p)\). Here we want to deal with \(\mathbb{Z}_p G\)-modules.

Technique: Use \(p\)-adic integers \(\mathbb{Z}_p\) to interpolate between \(\mathbb{Q}\) and \(\mathbb{F}_p\).

Now we start studying \(\mathbb{F}_p G\).

\begin{example}
    \underline{Exercise}: Let \(p,q\) be distinct primes. Then,

    \[
        \mathbb{F}_p C_q = \prod_{i=1}^h \mathbb{F}_{p^{f_i}}
    \]

    What is \(h\) and \(f_i\)?

    eg \(\mathbb{F}_p C_2 \cong\) trivial rep and sign rep \(\cong \mathbb{F}_p \times \mathbb{F}_p\)

    \(\mathbb{F}_2 C_q = ?\)

    Hint: Multiplicative group of a finite field \((\mathbb{F}_p^\times)\) is cyclic.
    \(\mathbb{F}_2 \times C_3 \cong \mathbb{F}_2 \times \mathbb{F}_4\) since \(\mathbb{F}_4^\times \cong \mathbb{Z} / (4-1) = \mathbb{Z} / 3\).

    It is given by \(r \mapsto (1, \zeta_3)\).

    \(\mathbb{F}_2 C_5 = ?\)

    We have \(\zeta_5 \in \mathbb{F}_{16}^\times \cong \mathbb{Z} / 15\) so:

    \(\mathbb{F}_2 C_5 \cong \mathbb{F}_2 \times \mathbb{F}_{16}\).

    Actually we can say \(\mathbb{F}_2 C_5 = \mathbb{F}_2 \oplus \mathbb{F}_{16}\).

    \(\mathbb{F}_2 C_7 \cong \mathbb{F}_2 \times \mathbb{F}_8 \times \mathbb{F}_8\).
    
    \(r \mapsto (1 , \zeta_7, \zeta_7^3)\) or \(r \mapsto (1, \zeta_7, \zeta ^{-1})\) 
    
    Minimal polynomial: \(\Phi_7(x) = x^6 + x^5 + x^4 + x^3 + x^2 + x + 1\) 

    \(\Phi_7(x) = f(x) g(x) \in \in \mathbb{F}_2[x]\).

    \(\mathbb{F}_2 C_7 = \frac{\mathbb{F}_2[x]}{(x^7 - 1)} = \frac{\mathbb{F}_2[x]}{(x-1)f(x) g(x)} \cong \frac{\mathbb{F}_2(x)}{x-1} \times \frac{\mathbb{F}_2[x]}{f(x)} \times \frac{\mathbb{F}_2[x]}{g(x)} \cong \mathbb{F}_2 \times \mathbb{F}_8 \times \mathbb{F}_8\).

    Now, we deal with \(p\neq 3\) and \(\mathbb{F}_p C_3\).

    \[
        \mathbb{F}_p C_3 = \begin{dcases}
            \mathbb{F}_p \times \mathbb{F}_p \times \mathbb{F}_p, &\text{ if } p \equiv 1 (3) ;\\
            \mathbb{F}_p \times \mathbb{F}_{p^2}, &\text{ if } p \not\equiv 1 (3).
        \end{dcases}
    \]

    How do we know \(\mathbb{F}_2 C_7 \cong \mathbb{F}_2 \times \mathbb{F}_8 \times \mathbb{F}_8\) and not \(\mathbb{F}_2 \times \mathbb{F}_{64}\)?

    The image of \(r\) lies in \(\mathbb{F}_8\) so it is actually in \(\mathbb{F}_2 \times \mathbb{F}_8\)!

    We look for the minimal field where the cyclotimic polynomial splits.

\end{example}

\subsection*{Modular Case}

Complete list of ideals in \(\mathbb{F}_2 C_2\).

\(O \subset \langle 1 - r \rangle \subset \mathbb{F}_2 C_2\).

\(\langle 1-r \rangle\) is isomorphic to \(\mathbb{F}_2\), simple, not projective [not summand of free modules].

Why is it not projective?

Consider the \underline{augmentation map}:

\[
    \varepsilon : \underset{\sum_{i} r_i g_i}{RG} \underset{\mapsto}{\to} \underset{\sum_{i} r_i}{R}
\]

It is a ring map.

\underline{Augmentation ideal} \(I = \ker(\varepsilon) \subset RG\).

We have Norm element \(N = \sum_{g\in G} g \in RG\).

If \(G\) is a \(p\)-group then \(N \in \ker (\varepsilon : \mathbb{F}_p G \to \mathbb{F}_p)\).

Aug map \(\varepsilon : \mathbb{F}_2 C_2 \to \mathbb{F}_2\) as \(\mathbb{F}_2 C_2\) module.

Therefore \(\mathbb{F}_2\) is not projective over \(\mathbb{F}_2 C_2\).

Complete list of finitely generated \(\mathbb{F}_2 C_2\)-modules (up to isomorphism):

\[
    (\mathbb{F}_2)^a \oplus (\mathbb{F}_2 C_2)^b
\]

Complete list of \(\mathbb{F}_p C_p\)-ideals:

\[
    0 \subset \underset{\langle N \rangle}{\langle 1-r \rangle^{p-1}} \subset \cdots \subset \underset{\ker \varepsilon}{\langle 1-r \rangle} \subset \mathbb{F}_p C_p
\]

Thus \(\mathbb{F}_p C_p\) is local.

It is simple, not projective.

Complete list of finitely generated \(\mathbb{F}_p C_p\)-modules up to isomorphism: direct sum of ideals.

\begin{definition}
    Ring \(R\) is \underline{semilocal} if \(R / J(R)\) is semisimple.
\end{definition}

eg \(kG\) is always semilocal.

Serre p 163

\begin{definition}
    [Artinian Ring] \(R\) is \underline{artinian} if:

    \begin{enumerate}[label=\roman*)]
        \item Every decreasing sequence of ideals is stationary.
        \item \(\iff\) every f.g. \(R\)-module has finite length.
    \end{enumerate} 
\end{definition}

eg \(\mathbb{Z}\) is not artinian, but \(kG\) is artinian.

This is because f.d. \(k\)-algebra is artinian.

\begin{remark}
    If \(R\) is artinian then every finitely generated module has a minimal submodule and hence simple.
\end{remark}

\begin{theorem}
    If \(R\) is artinian then \(\exists\) unique minimal 2-sided ideal \(J(R)\) so that \(R / J(R)\) is semisimple.
\end{theorem}

Here, \(R / J(R)\) is the maximal semisimple quotient. \(J(\mathbb{F}_p C_p) = \langle 1 - r \rangle\) since the quotient is \(\mathbb{F}_p\).

For a general ring \(R\) we have:

\[
    J(R) = \bigcup_{\substack{\text{max left} \\ \text{ideals}}} M
\]

Despite having a one-sided definition it is a two sided ideal.

Then, \(J(R)S = 0\) when \(S\) is a simple module.

\(R\) artinian:

Simple modules over \(R\) \(\leftrightarrow\) simple modules over \(R / J(R)\).

\section*{Monday, 11/11/2024}

\subsection*{Simple vs Indecomposable}

Simple and Indecomposable are not the same thing.

We have \underline{Jordan-H\"older Theorem} and \underline{Krull-Schmidt Theorem}.

Let \(R\) be a ring and \(M\) be a module. Then,

\(l(M) = n\) if chain \(0 = M_0 \subsetneq M_1 \subsetneq \cdot \subsetneq M_n = M\) and \(n\) is maximal.

\begin{definition}
    Composition series for \(M\) is maximal chain \(\iff\) all the quotient modules \(M_i / M_{i-1}\) are simple.
\end{definition}

\begin{definition}
    Module \(M\) is \underline{indecomposable} if \(M = A \oplus B\implies A = 0\) or \(B = 0\).
\end{definition}

Let \(M\) be of finite length. 

\begin{theorem}
    [Jordan-H\"older Theorem] If \(M\) has finite length, then \(M\) has a composition series. Any two composition series have the same simple quotients.
\end{theorem}

\begin{theorem}
    [Krull-Schmidt Theorem] If \(M\) has finite length then \(M = I_1 \oplus \cdots \oplus I_k\) with \(I_j\) indecomposable and if \(M = I_1^{\prime} \oplus \cdots \oplus I_{k^{\prime}}^{\prime} \) with \(I_j^{\prime}\) independent then \(k = k^{\prime}\) and \(I_j = I^{\prime}_{\sigma(j)}\) for \(\sigma \in S_k\).
\end{theorem}

Works for abelian categories, works for groups.

Group Ring where the ring is a field has finite length.

Consider \(S_3 \cong D_6 = \langle r, s \mid r^3 = 1, s^2 = 1, s r s = r ^{-1} \rangle = C_3 \rtimes C_2\).

\(\mathbb{Q}D_6 = \mathbb{Q} \oplus \mathbb{Q} \oplus M_2(\mathbb{Q})\)

\[
    r \mapsto \left( 1, 1, \begin{bmatrix}
        0 &  1 \\
        -1 &  -1 \\
    \end{bmatrix} \right) 
\]

\[
    s \mapsto \left( 1, -1, \begin{bmatrix}
        0 &  1 \\
        1 &  0 \\
    \end{bmatrix} \right) 
\]

\(\mathbb{F}_2 D_6 = ?\) 

We have: \(\frac{1}{3}(1 + r + r^2)\) a central idempotent.

\[
    \mathbb{F}_2 D_6 = \mathbb{F}_2 C_2 \oplus M_2 \mathbb{F}_2
\]

\(\mathbb{F}_2 C_2\) is projective, not simple.

\[
    \mathbb{F}_2 D_6 = \mathbb{F}_2 C_2 \oplus \begin{bmatrix}
        \ast &  0 \\
        \ast &  0 \\
    \end{bmatrix} \oplus \begin{bmatrix}
        0 &  \ast \\
        0 &  \ast \\
    \end{bmatrix}
\]

\(JH \implies \mathbb{F}_2, \mathbb{F}_2, (\mathbb{F}_2)^2, (\mathbb{F}_2)^2\).

Maximal semisimple quotient \(\mathbb{F}_3 D_6 / J = \mathbb{F}_3 C_2 = \mathbb{F}_3 \times \mathbb{F}_3\).

Jacobson Radical \(J = \langle 1-r \rangle\).

We have a (not central) idempotent: \(e = \frac{1 + s}{2}\). So we don't have block decomposition.

\(\mathbb{F}_3 D_6 = \frac{\mathbb{F}_3 D_6}{1-e} \oplus \frac{\mathbb{F}_3 D_6}{e}\) not block decomposition.

Now we go back to Serre.

Let \(R\) be semisimple. Then Projective \(\iff\) \(\oplus\) simple.

If \(R\) is Artinian, \underline{which is better}? Both

\subsection*{Serre 14.1 Simple}

The abelian group \(R_k G\) is \(\mathbb{Z}[T] / R\) with generator set \(T\) where:

\(T =\) isomorphism classes of finitely generated \(kG\)-modules \([M]\).

We have following relations \(R\):

\([M] = [M^{\prime}] + [M^{\prime\prime}]\) if there exists a short exact sequence:

\[
    0 \to M^{\prime} \to M \to M^{\prime\prime} \to 0
\]

In the Maschke case the short exact sequence splits and so \(M = M^{\prime} \oplus M^{\prime\prime}\). 


Ring with \(- \otimes_k -\).


\(S_k = S_k G =\) isomorphism classes of simple \(kG\)-modules.

\((R_{\mathbb{F}_2} D_6, +) \cong \mathbb{Z}^2, [\mathbb{F}_2 D_6] = [S_1] + [S_1] + [S_2] + [S_2]\).

\(S_1 = \mathbb{F}_2, S_2 = \begin{bmatrix}
    \ast \\ \ast 
\end{bmatrix}\) 

\((R_{\mathbb{F}_3}D_6, +) \cong \mathbb{Z}^2\).

\([\mathbb{F}_3 D_6] = S_1^{\prime} + S_1^{\prime} +S_1^{\prime} + S_2^{\prime} + S_2^{\prime} + S_2^{\prime}\)

We want to prove proposition 40:

\begin{proposition}
    [Serre 40] \(S_k\) is \(\mathbb{Z}\)-basis for the representation ring \(R_k(G)\) additively. \([s] \mapsto [s]\).
\end{proposition}

\begin{proof}
    \[
        \mathbb{Z}[S_k] \leftrightarrow R_k G
    \]

    \[
        \sum [M_i / M_{i-1}] \mapsfrom M
    \]
\end{proof}

\subsection*{Projective Module Review}

Let \(R\) be a ring.

\begin{lemma}
    \(R\)-module \(P\). TFAE:

    \begin{enumerate}[label=\roman*)]
        \item \(\exists Q\) such that \(P+Q =\) free [has a basis].
        \item We have the following:
        \begin{center}
            \begin{tikzcd}
                & P \ar[dl, dotted, "\exists"] \ar[d] \\ M \ar[r, two heads] & N
            \end{tikzcd}
        \end{center}

        \item A surjection to \(P\) splits.
        
        \begin{center}
            \begin{tikzcd}
                M \ar[r, two heads] & P \ar[l, bend right, dotted]
            \end{tikzcd}
        \end{center}

        \item SES
        
        \begin{center}
            \begin{tikzcd}
                0 \ar[r] & M \ar[r] & N \ar[r] & P \ar[l, dotted, bend right] \ar[r] & 0
            \end{tikzcd}
        \end{center}

        splits.

        \item \(P\) isimage of projection.
        
        \[
            \exists \pi \circ \pi = \pi : R^s \to R^s \, \text{s.t.} P \cong \pi (R^s)
        \]

        eg \(R = \mathbb{R} \times  M_2\mathbb{R}, \begin{pmatrix}
            \ast \\ \ast
        \end{pmatrix} \cong \begin{pmatrix}
            \ast  &  0 \\
            \ast &  0
        \end{pmatrix}\) is projective, not free. 

    \end{enumerate} 

\end{lemma}

Let \(R\) be a ring.

\(K_0 R = \operatorname{Gr} (\text{iso chlass of f.g. projective \(R\)-modules,} \oplus)\).

Serre writes \(P_A(G) = K_0(AG)\) for ring \(A\).

\(K_0(kG)\) is module over \(R_k G\). [Not ring since we don't have identity].

Key point: \(M \otimes_k kG \cong i^{\ast} M \otimes_k kG\) where \(i : k \hookrightarrow kG\) is free.

\(m \otimes g \mapsto g ^{-1} m \otimes g\).

Note that \(M \otimes_k \text{proj}\) is proj.

\section*{Wednesday, 11/13/2024}

\subsection*{Serre 14.3}

We are looking at \(kG\), character possibly dividing \(\#G\).

\[
    \begin{array}{ccc}
        \text{indecomposable} &  &  \text{simple} \\
        K_0(kG) &  &  R_k G \\
        P & \mapsto &  P / J(R)P \\
        P_S & \mapsfrom & S \\
        \text{projective cover} & &
    \end{array}
\]

\begin{definition}
    \(f: M \to M^{\prime}\) is \underline{essential} if:
    
    \begin{itemize}
        \item \(f\) onto.
        \item \(\forall M^{\prime\prime} \subsetneq M^{\prime}, \eval{f}_{M^{\prime\prime}}\) not onto.  
    \end{itemize} 

    The idea is \(f\) is essential if it is `barely onto'.
\end{definition}

\begin{definition}
    \(f: P \to M\) where \(P\) is projective and \(f\) is essential is a \underline{projective cover}.  

    Note: \(P\) is the projective cover of \(M\).
\end{definition}

\begin{proposition}
    [4.1] If \(l(M) < \infty\) there exists projective cover, unique upto isomorphism.

    If \(P\) is projective and \(E\) is maximal semisimple quotient, then \(P \to E\) is a projective cover.

    eg if \(R\) is \underline{artinian}, then \(l(M) < \infty \iff M\) finitely generated.

    \(P\) projective implies \(P \to P / JP\) is projective cover. \(P / JP\) is semisimple.

    eg \(\mathbb{F}_2 C_2 \to \mathbb{F}_2\) is a projective cover.

    \(e = \frac{1+s}{2}\), \(\mathbb{F}_3 D_6 e \to \mathbb{F}_3\) is a projective cover.

    \(\overset{\text{proj}}{\mathbb{F}_3 D_6} \underset{\text{essential}}{\twoheadrightarrow} \overset{\text{s.s.}}{\mathbb{F}_3 C_2}\).

\end{proposition}

\begin{proof}
    Existence:

    \begin{itemize}
        \item Choose SES (choice in blue):
        
        \[
            0 \to R \to {\color{blue}\overset{\text{proj}}{L}} \to M \to 0
        \]

        \item Choose \(N \subset R\) minimal such that:
        
        \[
            {\color{blue} L / N \overset{\text{ess}}{\to} M}
        \]

        Let \(P \coloneqq L / N\).

        \item Let \(Q \subset L\) minimal such that:
        
        \begin{center}
            \begin{tikzcd}
                & L \ar[d, two heads] \\ {\color{blue} Q} \ar[ur, hook, blue] \ar[r, two heads, "\text{onto}"] & P
            \end{tikzcd}
        \end{center}

        \item Choose lift
        
        \begin{center}
            \begin{tikzcd}
                & L \ar[dl, dotted, "q", blue, swap] \ar[d, two heads] \\ Q \ar[r] & P
            \end{tikzcd}
        \end{center}

    \end{itemize} 

    2nd choice and 3rd choice implies:

    \[
        0 \to N \to L\overset{q}{\to} Q \to 0
    \]

    SES \(\implies P \cong Q\).

    3rd choice and 4th choice \(\implies\)\begin{tikzcd}  L \ar[r, "q", swap]  & Q \ar[l, bend right, dashed, "i", swap]   \end{tikzcd} split.

    \(L \cong N \oplus Q \cong N \oplus P\), P projective. 

    Uniqueness:

    \begin{center}
        \begin{tikzcd}
            & P \ar[dl, "\text{lift } q \, \cong", red, swap] \ar[d] \\ P^\prime \ar[r, two heads] & M
        \end{tikzcd}
    \end{center}

    \(P^{\prime} \to M\) essential so \(q\) onto.

    \(P \to M\) essential so \(q\) is 1-1.
    
\end{proof}

Suppose \(R\) is artinian eg \(R = kG\).

\begin{corollary}
    [1]

    \[
        \text{proj. indecomposable } R\text{-mod} \leftrightarrow \text{simple} R\text{-mod}
    \]

    \[
        P \mapsto P / JP
    \]

    \[
        P_E \mapsfrom E
    \]
\end{corollary}

\begin{corollary}
    Let \(\$\) be isomorphism classes of simple \(R\)-modules.

    \(\{ P_E \}_{E \in \$}\) form a basis of \(K_0 R\).
\end{corollary}

\begin{corollary}
    f.g. projective \(R\)-modules \(P\) and \(P^{\prime}\), \([P]=[P^{\prime}] \in K_0 R \iff P \cong P^{\prime}\).

    No stabilization required!
\end{corollary}

\begin{proof}
    ?: Suppose \([P] = [P^{\prime}] \in K_0(kG)\).

    \(\iff [s]=[s^{\prime}] \in R_k G\) [\(s = P / JP\)]

    \(\iff s \cong s^{\prime}\) 

    \(\iff P \cong P^{\prime}\).
\end{proof}

\subsection*{Setting of Chapter 14, p-adics}

Consider \(((K,\nu),A, \mathfrak{m}, k)\).

Example: \((\mathbb{Q}_p, \nu_p),\mathbb{Z}_p, p \mathbb{Z}_p, \mathbb{F}_p\).

\begin{definition}
    [p164] A discrete valuation \((K,v)\) is a field \(K\) and a homomorphism \(\nu: K^\times \twoheadrightarrow \mathbb{Z}\) such that \(\nu(x + y) \geq \min(\nu(x), \nu(y))\).
\end{definition}

Basic example: \(K = \mathbb{Q}\) then \(\nu_p\) is the power of \(p\) in the factorization.

Generalize: If \(A\) is a PID and we have prime \(P \triangleleft A\) we have a discrete valuation \((\operatorname{Frac}(A), \nu_P)\).

Let \((K,\nu)\) bea discrete valuation.

\begin{definition}
    Valuation ring of \((K,\nu)\) is:

    \[
        A = \nu ^{-1} \mathbb{Z}_{\geq 0}
    \]

    This is a DVR (discrete valuation ring) (\(=\) PID with unique maximal ideal).

    Maximal ideal is

    \(\mathfrak{m} = \nu ^{-1} \mathbb{Z}_{> 0}\)

    eg for \((\mathbb{Q}, \nu_P)\) we have \(A = \mathbb{Z}_{(P)}\).

    For \((K, \nu)\) we have an absolute value on \(K\) which gives us a metric on \(K\).

    \(\vert x \vert = e^{-\nu(x)}\).

    metric: \(d(x,y) = \vert x - y \vert\).
\end{definition}

Fact: Completion of \(K\) (use Cauchy sequences) \(\widehat{K}_\nu\) is also a field with discrete valuation \(\nu\).

\(K\) is \underline{complete} if \(K = \widehat{K}_\nu\).

\section*{Friday, 11/15/2024}

Basic plan for learning \(p\)-adic: Suppose we want to study \(\mathbb{F}_p G\). If \(p\mid \vert G \vert\) then Maschke doesn't work. So we mod out the Jacobson RadicaL \(\mathbb{F}_p G /\).

Our setting:

\[
    (\underset{\text{complete D.V}}{(K,\nu)}, \underset{\text{valuation ring}}{A}, \underset{\text{maximal}}{\mathfrak{m}}, \underset{\text{residue field} }{k} )
\]

eg \(((\mathbb{Q}_p, \nu_p),\mathbb{Z}_p, p \mathbb{Z}_p, \mathbb{F}_p)\) 

In \(\mathbb{Q}, \nu_p\), \(\nu_p(p^n \frac{a}{b}) = n\).

Renormalize: \(\lVert x \rVert = p^{-\nu(x)}\)

\[
    \lim_{n \to \infty} p^n = 0
\]

\(\mathbb{Q}_p\) is completion of \(\mathbb{Q}\) under \(\lVert x-y \rVert _p\)

\[
    \mathbb{Q}_p= \left\{\sum_{i=-k}^{\infty} a_i p^i \mid 0 \leq a_i \leq p-1 \right\}
\]

\[
    \mathbb{Z}_p= \left\{\sum_{i=0}^{\infty} a_i p^i \mid 0 \leq a_i \leq p-1 \right\}
\]

\[
    \mathfrak{m} = \left\{\sum_{i=1}^{\infty} a_i p^i \mid 0 \leq a_i \leq p-1 \right\}
\]

\subsection*{Better Approach}

We use the inverse limit to define it.

\[
    \mathbb{Z}_p \coloneqq \lim_{\leftarrow} \mathbb{Z} / p^n = \left\{ (b_n) \in \prod \mathbb{Z} / p^n \mid b_{n+1} \equiv b_n \pmod{p^n}\right\}  
\]

Compact by Tychonoff.

\(\mathbb{Q}_p = \operatorname{Frac}(\mathbb{Z}_p)\).

\subsection*{The case \(p=2\)}

\(p=2\) consider binary expansion.

In \(\mathbb{Z}\), \(11011\) is finite.

In \(\mathbb{R}\) we can have \(\underbrace{11011}_{\text{finite}}.\underbrace{101110110\cdots}_{\text{infinite}}\) 

In \(\mathbb{Q}_2\) we ca have \(\underbrace{\cdots 1011011}_{\text{infinite}}.\underbrace{01101}_{\text{finite}}\) 

Thus we can have algorithms for adding and other stuff.

\subsection*{Serre 14.4}

\begin{lemma}
    [Lemma 20] Let \(\Lambda\) be a commutative ring and \(P\) be a \(\Lambda G\)-module.

    \(P\) projective \(/ \Lambda G\implies P\) projective \(/ \Lambda\) and \(\exists \Lambda\)-map \(u: P \to P\) so that:
    
    \[
        \sum_{s\in G} s u (s ^{-1} x) = x \, \forall x\in P
    \]

    Serre writes it as:

    \[
        \sum_{s\in S} s u s ^{-1} = 1
    \]
\end{lemma}

\begin{proof}
    Ommitted. Just computation
\end{proof}

\begin{lemma}
    [Lemma 21] Let \(\Lambda\) be local ring, \(k = \Lambda / \mathfrak{m}\).

    \begin{enumerate}[label=\alph*)]
        \item Let \(P\) be a \(\Lambda G\)-module free \(/ \Lambda\)
        \[
            P \text{ proj.} / \Lambda G \iff \overline{P} = P \otimes_\Gamma k \text{ proj}/kG 
        \]
        \item Projectives \(P, P^{\prime}\) implies \(P \cong P^{\prime} \iff \overline{P} \cong \overline{P^{\prime}}^k \) 
    \end{enumerate} 
\end{lemma}

\begin{proof}

    Idea: the maps are matrices, we show their determinants are invertible. Local means we need to show dets are not in max ideal.

    \begin{enumerate}[label=\alph*)]
        \item \(\implies\) part is clear. We do \(\impliedby:\)
        
        \(\overline{P}\) projective. Lemma 20 implies \(\exists \overline{u}: \overline{P} \to \overline{P}\) \(k\)-map so that:
        
        \[
            \sum s \overline{u} s ^{-1} = 1 
        \]

        We ``lift \(\overline{u}\)''.

        \begin{center}
            \begin{tikzcd}
                & P \ar[ddl, dashed, "u"] \ar[d] \\ & \overline{P} \ar[d, "\overline{u}"] \\ P \ar[r, two heads] & \overline{P} 
            \end{tikzcd}
        \end{center}

        Then \(u^{\prime} = \sum s u s ^{-1} \equiv 1 \mod \mathfrak{m}\).
        
        Thus \(u^{\prime}\) is \(\Lambda G\)-map, \(\det u^{\prime} \notin \mathfrak{m} \implies \det u^{\prime} \in \Gamma ^\times \implies u^{\prime}\) invertible.

        \[
            \sum s u (u^{\prime})^{-1} s ^{-1} = u^{\prime} (u^{\prime})^{-1} = 1 \overset{L20}{\implies} P \text{ proj}
        \]

        \item Let \(\overline{w}: \overline{P} \overset{\cong}{\to} \overline{P}^{\prime}\). Lif \(w: P \to P^{\prime}\). Then \(\det w\notin \mathfrak{m} \implies w\) is invertible and thus is isomorphism. 

    \end{enumerate} 
\end{proof}

\begin{proposition}
    [42] Let \(A\) be a complete local ring.

    \begin{enumerate}[label=\alph*)]
        \item \(E\) is \(AG\)-module. Then \(E\) proj / \(AG \iff E\) free \(/A\) and \(\overline{E}\) projective \(/ kG\).
        \item If \(F\) is projective \(kG\)-module, \(\exists !\) projective \(P / A G\) such that \(\overline{P} \cong F\).
    \end{enumerate} 
\end{proposition}

\begin{corollary}
    There exists bijection:

    \[
        \begin{matrix}
             \text{proj indecom} & & \text{proj. indecom} & & \text{simple} \\
             AG\text{-mod} & \to & kG\text{-mod} & \to & kG/J\text{-mod} \\ - - - - - - & & - - - - - - & & - - - - - - \\
             \text{iso} & & \text{iso} & & \text{iso}
        \end{matrix}
    \]
\end{corollary}

Now we go back to proposition 42.

\begin{proof}
    [Proof of Lemma 21] Lemma 21 \(\implies\) a and uniqueness.Question: existence?

    \(F\) projective \(kG\)-module.

    \[
        A = \lim A / \mathfrak{m}^n
    \]

    \((A / \mathfrak{m}^n)G\) is Artinian.
    
    \(\exists \) projective cover \(P_n \to F\) of \((A / \mathfrak{m}^n)G\)-modules.
    
    \begin{center}
        \begin{tikzcd}
            & P_{n+1} \ar[dl, dashed] \ar[d] \\ P_n \ar[r] & F
        \end{tikzcd}
    \end{center}

    We have \(\cdots \to P_3 \to P_2 \to P_1 \to P_0\)
    
    Let \(P = \lim_{\leftarrow} P_n\), detailed ommitted. \(P\) projective \(AG\)-module, \(\overline{P} = P \otimes_A k\).  
\end{proof}

\section*{Monday, 11/18/2024}

\subsection*{14.3 and 14.4 Review}

In \((A,k)\) [eg \(\mathbb{Z}_p, \mathbb{F}_p\)] we say \(A\) is a \underline{complete local ring} where valuation ring is complete \((K,\nu)\). \(k = A / \mathfrak{m}\) is the residue field.

Suppose we have our finite group \(G\). We have the `reduction mod \(\mathfrak{m}\)' homomorphism:

\[
    A G \overset{\pi}{\to} \mathbb{F}_p G
\]

Then we have:

\[
    AG \overset{\pi}{\to} \mathbb{F}_p G \overset{p}{\to} \mathbb{F}_p G / J(\mathbb{F}_p G)
\]

\(J\) indicates the Jacobson Radical.

We have bijections.

\[
    \begin{matrix}

        \text{basis } K_0(AG) & & \text{basis } K_0(\mathbb{F}_p G) & & \text{basis } R_k G \\ \\
        \text{proj indecom} & & \text{proj. indecom} & & \text{simple} \\
        AG\text{-mod} & \to & kG\text{-mod} & \to & kG/J\text{-mod} \\ - - - - - - & \pi_{\ast} & - - - - - - & p_{\ast} & - - - - - - \\
        \text{iso} & & \text{iso} & & \text{iso}
    \end{matrix}
\]

If \(M\) is an \(AG\)-module then \(\pi_{\ast} M =\mathbb{F}_p G \otimes_{AG} M\).

We have \(\underset{\text{essential}}{P_E \to E} \mapsfrom E\)

Recall that essential maps are maps that are `barely surjective'.

We have \(P = \lim_{\leftarrow} P_n \mapsfrom \overline{P}\) 

\(P_n \to \overline{P}\) projective cover of \((A / \mathfrak{m}^n)G\)-modules. 

Now we deal with the case \(\operatorname{char} K =0, \operatorname{char} k = p\). Recall that \(K\) has a valuation ring \(A\) with unique maximal ideal \(\mathfrak{m}\) and \(k = A / \mathfrak{m}\).

\begin{definition}
    \(\begin{Bmatrix}
        K \\ k
    \end{Bmatrix}\) is a \underline{splitting field} for \(G\) if:

    \[
        KG \cong \prod M_{n_i} K
    \]

    \[
        kG / J \cong \prod M_{l_i} (k)
    \]
\end{definition}

\begin{definition}
    \(\begin{Bmatrix} K \\ k \end{Bmatrix}\) is \underline{sufficiently large} if \(\begin{Bmatrix} K \\ k \end{Bmatrix}\) contains all \(\begin{Bmatrix} m \\ m^{\prime} \end{Bmatrix}\).
    
    Where \(m = \operatorname{lcm}\{ \operatorname{ord}(G) \mid g\in G \} = \exp G\) where \(m^{\prime} = m / p^a\) where \((p, m^{\prime})=1\). 
\end{definition}

Fact: sufficiently large \(\implies\) splitting fields.

\(K\) due to Brauer, \(k\) see remark in 14.5.

\begin{example}
    \(\mathbb{F}_5 [C_3] \cong \mathbb{F}_5 \times \mathbb{F}_{25}\). So \(\mathbb{F}_5\) is not splitting field.

    \(\mathbb{F}_{25} [C_3] \cong \mathbb{F}_{25}^3\) so \(\mathbb{F}_{25}\) is splitting field for \(C_3\).
\end{example}

\begin{definition}
    \(E\) is absolutely simple if \(\dim_{\begin{Bmatrix} K \\ k \end{Bmatrix}} \operatorname{Hom}_{\begin{Bmatrix} KG \\ kG \end{Bmatrix}}(E,E) = 1\). 
\end{definition}

\subsection*{14.5 Dualities}

Suppose \(\operatorname{char} K = 0\).

If \(E,F\) are \(KG\)-modules, we can define:

\[
    \langle E, F \rangle = \dim_K \operatorname{Hom}_{KG} (E,F) = \langle E,F \rangle = \langle \chi_E, \chi_F \rangle 
\]

We thus have bilinear \(\langle , \rangle : R_k G \times R_k G \to \mathbb{Z}\).

Simples \([E]\) are orthogonal basis.

Orthonormal iff \(K\) is a splitting field for \(G\).

Now suppose \(\operatorname{char} k = p \mid \#G\).

\(\langle , \rangle : R_k G \times R_k G \to \mathbb{Z}\) is \underline{not bilinear}! This is because SES don't split.

Take \(0 \to \mathbb{F}_2 \to \mathbb{F}_2 C_2 \to \mathbb{F}_2 \to 0\) . But if we take \(\operatorname{Hom}_{\mathbb{F}_2 C_2}(\mathbb{F}_2 C_2, \mathbb{F}_2)\) but \(\langle \mathbb{F}_2 C_2, \mathbb{F}_2 \rangle \neq \langle \mathbb{F}_2, \mathbb{F}_2 \rangle + \langle \mathbb{F}_2, \mathbb{F}_2 \rangle\).

But the following is bilinear:

\[
    \langle , \rangle : K_0(kG) \times R_k G \to \mathbb{Z}
\]

If \(k\) is a splitting field then \(\{ P_E \}\) and \(\{ E \}\) are dual bases.

\(\operatorname{Hom}_{kG}(P_E, E^{\prime}) \cong \operatorname{Hom}_{kG}(E,E^{\prime})\) for \(E,E^{\prime}\) simple.

\subsection*{14.6}

Consider \(K^{\prime} / K\). Then we have \(R_K G \hookrightarrow R_{K^{\prime}}G\).

This is an injection.

This is infact a \underline{split injection} [so there's a map backwards] iff \(\forall\) simple \(E, \langle E, E \rangle = 1\) [so the schur index \(=1\)].

Isomorphism \(\iff K\) is a splitting field.

All follow from \(KG\) semisimple:

\[
    M_n(D) \otimes_K K^{\prime} = M_n(D \otimes_K K^{\prime})
\]

\begin{example}
    \(R_\mathbb{R} (Q_8) \to R_\mathbb{C}(Q_8)\):

    We have the matrix:

    \[
        \begin{bmatrix}
            1 &  &  &  &   \\
             & 1 &  &  &   \\
             &  & 1 &  &   \\
             &  &  & 1 &   \\
             &  &  &  & 2  \\
        \end{bmatrix}
    \]

    Since \(\mathbb{H} \otimes_\mathbb{R} \mathbb{C} \cong M_2(\mathbb{C})\) as rings and \(\cong \mathbb{C}^2 \oplus \mathbb{C}^2\) as module and also \(\langle \mathbb{H} , \mathbb{H}  \rangle_{\mathbb{R} Q_8} = 4\).

    So not split injection.
\end{example}

\begin{theorem}
    [Wedderburn] Finite \(\begin{Bmatrix} \text{integral domain} \\ \text{skew field} \end{Bmatrix}\) is a field.
\end{theorem}

Consider \(k^{\prime} / k, R_k(G) \to R_{k^{\prime}}G, K_0(kG) \to K_0(k^{\prime} G)\).

These are split injection.

Isomorphism iff \(k^{\prime}\) is spliting field for \(G\).

``Setting'':

\begin{center}
    \begin{tikzcd}
        k^\prime \ar[d] & A^\prime \ar[l] \ar[d, no head] \ar[r] & K^\prime \ar[d, no head, "\text{finite}"] \\ k & A \ar[l] \ar[r] & K
    \end{tikzcd}
\end{center}

Here \(A^{\prime} =\) integral closure of \(A\) in \(K^{\prime}\) 

We have:

\begin{center}
    \begin{tikzcd}
        K_0(AG) \ar[r] \ar[d,"\pi_\ast\, \cong"] & K_0(A^\prime G) \ar[d, "\cong"] \\ K_0(kG) \ar[r] & K_0(k^\prime G)
    \end{tikzcd}
\end{center}

\(K_0 AG \to K_0 A^{\prime} G\) is splitting.

Isomorphism if \(K\) is sufficiently large.

\section*{Wednesday, 11/20/2024}

\subsection*{CDE Triangle}

Recall:

\(A =\) completely local ring

\(K =\) field of fractions

\(k =\) residue field.

\begin{center}
    \begin{tikzcd}
        A \ar[r, hook] \ar[d, two heads] & K \\ k
    \end{tikzcd}
\end{center}

The CDE triangle is the following:

\begin{center}
    \begin{tikzcd}
        K_0(kG) \ar[rr,"c","\text{forgetful}"'] \ar[rd,"e","\text{lift}"'] && R_k G \\ & R_KG \ar[ur,"d","\text{lattice}"'] &
    \end{tikzcd}
\end{center}

Each group has a canonical basis.

Therefore, we have matrice \(C,D,E\).

\begin{exercise}
    Compute \(C,D,E\) for \(k = \mathbb{F}_2, G = C_6, D_6\). 
\end{exercise}

15.1: \(c[P]=[P]\)

\(S =\) isomorphism classes of simple \(kG\) modules.

\begin{center}
    \begin{tikzcd}
        K_0(kG) \ar[r,"c"] & R_k G \\ \{ P_E \}_{E\in S} & \{ E \}_{E\in S} \\ C \text{ is square} & C = (C_{FE})
    \end{tikzcd}
\end{center}

\(c[P_E] = \sum_{F\in S} C_{FE} [F] \) 

\(C_{FE} = \#\) of \(F\) factor in composition series for \(P_E\). 

\(d: R_K G \to R_k G\)

Let \(E\) be finitely generated \(KG\)-module.

\begin{definition}
    A \underline{\(G\)-lattice} in \(E\) is a finitely generated \(AG\)-submodule of \(E\).
\end{definition}

\begin{remark}
    Existence: If \(\{ e_1, \cdots , e_n \}\) generates \(E\), then \(E_1 = \sum_{i=1}^n AG e_i \subset E\) is \(G\)-lattice.  
\end{remark}

\(E_1\) is \(G\)-lattice in \(E\).

\[
    \overline{E_1} \coloneqq E_1 / \mathfrak{m} E_1 (= k \otimes_A E_1)
\]

Define \(d[E] = [\overline{E_1}]\) 

Is \(d\) well defined? Proof later!

\(e: K_0(kG) \to R_K G\):

\begin{center}
    \begin{tikzcd}
        K_0(kG) \ar[rr, "e", bend left] & K_0(AG) \ar[l, "\cong"] \ar[r] & K_0(KG) = R_K G \\ \overline{p} \ar[r,mapsto] & p & KG \otimes_{AG} P \ar[d,phantom, sloped, "="] \\ & & K \otimes_A P
    \end{tikzcd}
\end{center}

\begin{remark}
    \begin{enumerate}[label=\roman*)]
        \item \(c\) is defined for any field \(k\).
        \item \(d\) is defined when \(A\) is a local ring 
        \item \(e\) is defined wen \(A\) is a complete local ring
    \end{enumerate} 
\end{remark}

\begin{remark}
    The triangle commutes: \(c = d \circ e\).
\end{remark}

\begin{lemma}
    \(d\) and \(e\) are adjoints.

    \[
        \langle x, d(y) \rangle_k = \langle e(x), y \rangle_K
    \]

    for all \(x \in K_0(kG)\) and \(y\in R_K G\)
\end{lemma}

\begin{proof}
    \(x = [\overline{X}]\) where \(X\) is a projective \(AG\)-module.

    \(y = [K \otimes_A Y]\) where \(Y\) is \(AG\)-module which is \(A\)-free.
    
    \(\operatorname{Hom}_{AG}(X,Y)\) is projective \(A\)-module. Thus it is a free \(A\)-module.
    
    Let \(r\) be the rank.

    \(\langle -,- \rangle_k : K_0(kG) \times R_k G \to \mathbb{Z}\)

    \(\langle A, B \rangle = \dim_k \operatorname{Hom}_{kG} (A,B)\)
    
    \(\langle x, d(y) \rangle_k = \dim_k \operatorname{Hom}_{kG} (\overline{X},\overline{Y}) = \dim_k (k \otimes_{A} \operatorname{Hom}_{AG} (X,Y)) = r\) 

    \(\langle e(x),y \rangle_K = \dim_K \operatorname{Hom}_{KG}(K \otimes_{A} X, K \otimes_{A} Y) = \dim_K K \otimes_{A} \operatorname{Hom}_{AG}(X,Y) = r\) 

\end{proof}

\begin{remark}
    For \(K\) sufficiently large [\(\zeta_m \in K, m = \exp(G)\)] implies \(K, k\) are both splitting fields.

    Thus, bases of \(K_0(kG)\) and \(R_k G\) are duals. Basis of \(R_K G\) is orthonormal. So, \(\langle -,- \rangle_k\) are perfect parings.
    
    Therefore, \(E = D^T\).

    Then \(C = DE = D D^T \implies C\) is symmetric.
    
\end{remark}

We now prove that \(d\) is well-defined.

\section*{Friday, 11/22/2024}

\underline{\(G\)-lattice} in f.g. \(KG\)-module \(E\) is f.g. \(AG\)-submodule \(E_1\) such that \(E = K E_1\).

\[
    \overline{E}_1 = E_1 / \mathfrak{m} E_1
\]

\[
    d[E] = [\overline{E}_1]
\]

We want to show this is well defined.

\begin{lemma}
    If \(E_1\) and \(E_2\) are \(G\)-lattices in \(E\), then \([\overline{E}_1] = [\overline{E}_2]\). 
\end{lemma}

\begin{proof}
    Recall: \(d[E] = [\overline{E_1}]\) where \(E_1 \subset E\) is finitely generated \(AG\)-submodule and \(\overline{E_1} = E_1 / \mathfrak{m} E_1\).

    Case A: \(\mathfrak{m} E_1 \subset E_2 \subset E_1\) 

    Consider:

    \[
        0 \to E_2 \to E_1 \to E_1 / E_2 \to 0
    \]

    Third isomorphism theorem:

    \[
        \implies 0 \to E_2 / \mathfrak{m} E_1 \to E_1 / \mathfrak{m} E_1 \to E_1 / E_2 \to 0
    \]

    Thus,

    \[
        (\ast) 0 \to E_2 / \mathfrak{m} E_1 \to \overline{E_1} \to E_1 / E_2 \to 0
    \]

    We also have:

    \[
        0 \to \mathfrak{m} E_1 \to E_2 \to E_2 / \mathfrak{m} E_1 \to 0
    \]

    Then,

    \[
        0 \to \frac{\mathfrak{m} E_1}{\mathfrak{m} E_2} \to \frac{E_2}{\mathfrak{m} E_2} \to E_2 / \mathfrak{m} E_1 \to 0
    \]

    \[
        \implies (\ast \ast) 0 \to E_1 / E_2 \to \overline{E_2} \to E_2 / \mathfrak{m} E_1 \to 0
    \]


    Splicing \((\ast)\) and \((\ast \ast)\) we get:

    \begin{center}
        \begin{tikzcd}
            0 \ar[r] & E_2 / \mathfrak{m}E_1 \ar[r] & \overline{E_1} \ar[rd, two heads] \ar[rr] && \overline{E_2} \ar[r] & E_2 / \mathfrak{m}E_1 \ar[r] & 0 \\ & & & E_1 \chi E_2 \ar[ru,tail]
        \end{tikzcd}
    \end{center}

    \(\implies [\overline{E_1}] = [\overline{E_2}]\) 

    Case B: \(E_2 \subset E_1\) \(\exists n\) such that \(\mathfrak{m}^n E_1 \subset E_2 \subset E_1\).

    We show that \([\overline{E}_1] = [\overline{E}_2]\) by induction on \(n\). Case A was our base case.
    
    Let \(E_3 = \mathfrak{m}^{n-1} E_1 + E_2\).

    \(\mathfrak{m}^{n-1} E_1 \subset E_3 \subset E_1\) and \(\mathfrak{m} E_3 \subset E_2 \subset E_3\).

    Induction hypothesis \(\implies [\overline{E}_1] = [\overline{E}_3] = [\overline{E}_2]\).
    
    \underline{General Case}: \(G\)-lattices \(E_1, E_2\) then \(\exists l\in A \setminus \{ 0 \} \) such that \(l E_2 \subset E_1\). 

\end{proof}

\subsection*{15.5 \(p^{\prime}\) group}

i.e. \(p\nmid  \#G\)

\(\mathbb{F}_p G\) semisimple.

central idempotents of \(\mathbb{Q} G \subset \frac{1}{\vert G \vert} \mathbb{Z} G \subset \mathbb{Z}_{(p)} G\subset \mathbb{Z}_p G\) 

\begin{proposition}
    [43] Premise is as before. Then,

    \begin{enumerate}[label=\roman*)]
        \item All \(kG\)-modules are projective.
        
        All \(A\)-free \(AG\)-modules are projective.

        \item \(\begin{matrix}
            S_K & \to &  S_k \\
            E & \mapsto &  \overline{E}_1 \\
        \end{matrix}\) is bijective.

        \item \(C = D = E = I\).
    \end{enumerate} 
\end{proposition}

\begin{proof}
    \begin{enumerate}[label=\roman*)]
        \item \(kG\) semisimple from Maschke.
        
        Let \(P\) be an \(A\)-free \(AG\)-module.

        We will prove that any epomorphism to \(P\) splits.

        Consider \(M \overset{\pi}{\twoheadrightarrow} P\) 

        \(P\) is \(A\)-free, \(\exists A\)-splitting \(M \overset{s}{\leftarrow} P\).
        
        Then we `average':

        \[
            \widehat{s}(p) = \frac{1}{\vert G \vert} \sum_{g\in G} g s (g ^{-1} p)
        \]

        \(\implies \widehat{s}\) is \(AG\)-map.

        \(\implies \widehat{s}\) is splitting. So we are done.
    \end{enumerate} 

    ii and iii:

    \begin{center}
        \begin{tikzcd}
            \dfrac{\text{f.g. }KG\text{-module}}{\cong} \ar[r,bend left, "G\text{-lattice}"] & \dfrac{A\text{-free f.g. }AG\text{-mod}}{\cong} \ar[l,"K\otimes_A","\cong"'] \ar[d, phantom, sloped, "="] & \dfrac{\text{f.g. }kG\text{-mod}}{\cong} \ar[l, dotted, "\text{proj. cover}"] \\

            & \text{f.g. proj }AG\text{-mod} \ar[ur,"\cong"', "k \otimes_A",swap] \\
            
            S_K & S_A \ar[r,"\cong"] \ar[l,"\cong",swap] & S_k \\

            \text{simple }KG\text{-mod} & \text{proj }AG\text{-mod} & C=D=E=I
        \end{tikzcd}
    \end{center}


\end{proof}

\subsection*{Jacobson Radical}

Suppose \(\operatorname{char} k = p\)

\begin{theorem}
    [Davis Thesis] Suppose we have a \(p\)-group \(P \triangleleft G\). \(\forall p\in P, p-1 \in J(kG)\)
\end{theorem}

\begin{corollary}
    [1]

    \[
        1 \to P \to G \to Q \to 1 \implies G = P \rtimes Q.
    \]

    Here \(Q\) is a \(p^{\prime}\)-group.

    \(kG / J(kG) \cong kQ\) is ``largest semisimple quotient''.
\end{corollary}

\begin{corollary}
    \(1 \to P \to G \to Q \to 1\) 

    \(kG / J(kP) \cong kQ / J(kQ)\).
\end{corollary}

We redefine Jacobson Radical:

\underline{Old def}: \(J(R) = \bigcap_{M \text{ max left} }^{} M\) 

\underline{New Def}: \(J(R) = \bigcap_{\text{simple } E}^{} \operatorname{Ann} (E)\).

Recall:

\[
    \operatorname{Ann} E = \{ r\in R \mid rE = 0 \} 
\]

\(\operatorname{Ann} E\) is 2 sided ideal.

\(JE = 0\).

\subsection*{P64 Serre}

\begin{theorem}
    [L1] Suppose a \(p\)-group \(P \acts X\) finite set.

    \[
        \vert X^G \vert \equiv \vert X \vert \pmod{p}
    \]
\end{theorem}

\begin{proof}
    \(X - X^G = \sqcup \text{orbits} = \sqcup G x \cong \sqcup G / G_x \) 
\end{proof}

\begin{theorem}
    [L2] If \(M\) is f.g. \(kP\)-module, then \(M^P \neq 0\)
\end{theorem}

\begin{proof}
    Can assume \(k\) finite \(\implies \# M\) finite.

    \[
        0 \equiv \vert M \vert \equiv \vert M^p \vert \pmod{p}
    \]
\end{proof}

Now we prove that \(p-1\in J(kG)\).

\begin{proof}
    Let \(E\) be a simple \(kG\)-module.

    \(E^p \subset E\) is a \(kG\)-submodule (use \(P \triangleleft G\)).

    \(L2 \implies 0 \neq E^p \implies E^p = \) 
    
    Thus, \(\forall p\in P, p-1 \in \operatorname{Ann} E \implies p-1\in J(kG)\) 
\end{proof}

\end{document}