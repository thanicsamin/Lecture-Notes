\documentclass{article}
\usepackage{amsmath, amsthm, amssymb, amsfonts, mathtools,enumitem, stmaryrd,physics, cancel, tikz-cd, graphicx, float, booktabs}
\usetikzlibrary{arrows}
\usepackage{geometry}
    \geometry{
        a4paper,
        left = 40mm,
        top = 20mm,
        right = 40mm,
        bottom = 30mm
    }
\setlength{\parindent}{0pt}

\theoremstyle{definition}
\newtheorem{problem}{Problem}
\newtheorem{solution}{Solution}
\newtheorem*{example}{Example}
\newtheorem*{exercise}{Exercise}
\newtheorem*{definition}{Definition}
\newtheorem{theorem}{Theorem}
\newtheorem*{theorem*}{Theorem}
\newtheorem{proposition}[theorem]{Proposition}
\newtheorem*{proposition*}{Proposition}
\newtheorem{lemma}[theorem]{Lemma}
\newtheorem*{lemma*}{Lemma}
\newtheorem{corollary}[theorem]{Corollary}
\newtheorem*{corollary*}{Corollary}
\newtheorem*{remark}{Remark}

\title{M623 Geometric Topology I}
\author{Thanic Nur Samin}
\date{}


\begin{document}
    \maketitle

    \section*{Monday, 8/25/2025}

    Textbook: \textit{Characteristic Classes} by Milnor and Stasheff. Hereafter referred by MS.

    \underline{Read Chapter 1 and 2 of MS.}

    \begin{definition}
        [\(n\)-manifold]

        Two different variants: embedded and abstract.

    

        Abstract: \((M, \mathcal{A})\) where \(\mathcal{A}\) is an atlas.
    \end{definition}

    Embedded: \(M \subset \mathbb{R}^A\). Here, \(A =\) index set, \(\mathbb{R}^A = \text{func}(A,\mathbb{R})\) with the product topology.

    \(M\) Hausdorff space, \(U \subset M\) open, \(V \subset \mathbb{R}^n\) open.
        
    Chart \(\phi : U \xrightarrow{\approx} V\) homeomorphism.

    Parameterization (ptz) \(h: V \xrightarrow{\approx} U\)

    We want some calculus.

    Let open \(V \subset \mathbb{R}^n\).

    A function \(f: V \to \mathbb{R}\) is \textit{smooth} if all partials of all orders exist: \(\frac{\partial^k f}{\partial x_{i_1} \cdots \partial x_{i_{p}}}\).

    \(f: V \to \mathbb{R}^A\) is \textit{smooth} if \(f_\alpha\) smooth \(\forall \alpha \in A\).

    \[
        \begin{tikzcd}
            V \ar[rr, bend right,"f_\alpha"'] \ar[r,"f"] & \mathbb{R}^A \ar[r,"pr_{\alpha}"] & \mathbb{R}
        \end{tikzcd}
    \]

    We can go from abstract manifold to embedded manifold.

    Let \(A = C^{\infty} (M,R)\).

    \(M \xrightarrow{i}\mathbb{R}^A\) where \(i(x) = (f \mapsto f(x))\).

    We can go to the reverse direction easily once we have all the definitions.

    \begin{definition}
        Two charts \((\phi_1: U_1 \to V_1)\) and \((\phi_2: U_2 \to V_2)\) are compatible (or smoothly compatible) if \(\phi_2 \circ \phi_1 ^{-1}\) is smooth. Explicitly,

        \(\phi_1(U_1 \cap U_2) \xrightarrow{\phi_2 \circ \phi_1 ^{-1}} \phi_2(U_1 \cap U_2)\) needs to be smooth. 
    \end{definition}

    \begin{definition}
        Parameterization \(h: V \to U\) is \textit{smooth} (assume \(M \subset \mathbb{R}^A\)) if:

        \[
            \begin{tikzcd}
                V \ar[r,"h"] \ar[rrr, bend right,"h"] & U \ar[r,hook] & M \ar[r,hook] & \mathbb{R}^n
            \end{tikzcd}
        \]

        is smooth.

        and has rank \(n\). ie, \(\forall v\in V\) the Jacobian:

        \[
            dh = \left( \frac{\partial_\alpha h}{\partial x_j}(v) \right)
        \]
    
        has rank \(n\).

        eg \(x \mapsto x^3\) is a parameterization which is not smooth, since the Jacobian has rank \(0\) at \(0\).
    \end{definition}

    Now we can properly define manifolds.

    \begin{definition}
        [Embedded Smooth \(n\)-Manifold] \(M \subset \mathbb{R}^A\) so that \(\forall x\in M\) there exists a smooth rank \(n\) parameterization \(h : V \to U \ni x\).
    \end{definition}

    We can now define a Cateogry of Embedded Manifolds.

    \begin{definition}
        [Category of Embedded Manifolds] \(\operatorname{Embmfld}\).

        \underline{Objects}: embedded \(M \subset \mathbb{R}^A\) of \(\dim n\) for some \(n\).

        \underline{Morphisms}: Smooth Maps (has to be defined carefully, restricting in Euclidean space).

        Diffeomorphism = invertible morphism.
    \end{definition}

    Let \((M \subset \mathbb{R}^A), (N \subset \mathbb{R}^B)\). \(f: M \to N\) is smooth if \textit{locally smooth}, meaning \(\forall x\in M, \exists\) smooth parameterization \(h: V \to U \ni x\) such that \(V \to U \hookrightarrow M \xrightarrow{f} N \to \mathbb{R}^B\) is smooth.

    Now we can define abstract manifold independend of embedded manifolds.

    \begin{definition}
        [Abstract Manifold] Let \(M\) be Hausdorff. An \(n\)-\textit{atlas} on \(M\) is a set \(\mathcal{A}=\left\{ \phi_\alpha : U_\alpha \xrightarrow{\approx} V_\alpha \subset \mathbb{R}^n \right\}\) of compactible \(n\)-charts such that \(\{ U_\alpha \} \) covers \(M.\)

        Atlas \(\mathcal{A}\) and \(\mathcal{A}^{\prime}\) are \underline{compatible} if all charts are.

        Fact: Every atlas is contained in a unique maximal atlas.

        Then an abstract manifold is \((M,\mathcal{A})\) with a maximal \(n\)-atlas.
    \end{definition}

\end{document}