\documentclass{article}
\usepackage{amsmath, amsthm, amssymb, amsfonts, mathtools, mathrsfs, enumitem, stmaryrd,physics, cancel, tikz-cd, graphicx, float, booktabs}
\usetikzlibrary{arrows}
\usepackage{geometry}
    \geometry{
        a4paper,
        left = 40mm,
        top = 20mm,
        right = 40mm,
        bottom = 30mm
    }
\setlength{\parindent}{0pt}

\theoremstyle{definition}
\newtheorem{problem}{Problem}
\newtheorem{solution}{Solution}
\newtheorem*{example}{Example}
\newtheorem*{exercise}{Exercise}
\newtheorem*{definition}{Definition}
\newtheorem{theorem}{Theorem}
\newtheorem*{theorem*}{Theorem}
\newtheorem{proposition}[theorem]{Proposition}
\newtheorem*{proposition*}{Proposition}
\newtheorem{lemma}[theorem]{Lemma}
\newtheorem*{lemma*}{Lemma}
\newtheorem{corollary}[theorem]{Corollary}
\newtheorem*{corollary*}{Corollary}
\newtheorem*{remark}{Remark}

\title{NT Reading 2}
\author{Thanic Nur Samin}
\date{\vspace{-5ex}}

\begin{document}
    \maketitle

    \tableofcontents

    \setcounter{section}{8}

    \section{Thursday, 1/16/2025, Cartan Subalgebras (CSA) by Rostyslav}

    \begin{corollary}
        [15.3] Let \(L\) be semisimple. CSA's of \(L\) are precisely the maximal toral subalgebras of \(L\).
    \end{corollary}

    \begin{proof}
        \(\implies\): Let \(H\) be a maximal toral subalgebra.

        \(\implies\) \(H\)-abelian \(\implies H\)-nilpotent, \(N_L(H) = H\) since \(L = H + \bigsqcup_{\alpha\in \Phi} L_\alpha\) with \([H,L_\alpha] = L_\alpha\) for \(\alpha \in \Phi \implies H\)-CSA.
        
        \(\impliedby\): Let \(H\)-CSA. \(x = x_s + x_h\) by Jordan decomposition.

        \(\implies L_0(\operatorname{ad} x_s) \subset L_0(\operatorname{ad} x)\) for \(x\in L\) semisimple.

        \(L_0(\operatorname{ad} x_s) = C_L(x_s)\) since \(\operatorname{ad} x\) is diagonal.
        
        \(H\)- minimal Engel.

        By defintion, \(\implies L_0(\operatorname{ad} x_s) = C_L(x_s) = H\).
        
        But \(C_L(x_g)\) contains maximal toral subalgebra which is CSA. Thus it itself is minimal Engel. \(H\)-maximal toral.

        Detaills:

        \(L \subset H\). \(H\) is CSA \(\iff H\) is nilpotent, \(N_L(H)=H\).


        \(L_0(\operatorname{ad}(x_s)) \subset L_0(\operatorname{ad}(x)).\)
        
        \(L_0(\operatorname{ad}(x_s)) = \{ y\in L \mid \operatorname{ad}(x_s)(y) = 0\} = \{ y\in L \mid [x_s,y] = 0 \} \).
        
        \([x,y] = \underbrace{[x_s,y]}_{=0} + [x_n,y]\).

        \(\operatorname{ad}(x)^m(y) = \operatorname{ad}(x_n)^m(y) = 0\) for \(m \gg 0\).
        
    \end{proof}

    \begin{lemma}
        [15.4.B] Let \(\phi: L \to L^{\prime}\) [epimorphism]. Let \(H^{\prime}\) be CSA of \(L^{\prime}\). Then, any CSA of \(\phi ^{-1} (H^{\prime})\) is also a CSA of \(L\).
    \end{lemma}

    \begin{definition}
        \(x\in L\) is callled \underline{strongly ad-nilpotent} if \(\exists y\in L\) and \(\exists a\neq 0\) eigenvalue of \(\operatorname{ad} y\) such that \(x\in L_a(\operatorname{ad} y)\).

        \[
            [L_a(\operatorname{ad} y), L_b(\operatorname{ad} y)] \subset L_{a,b}(\operatorname{ad} y)
        \]
    \end{definition}

    \begin{remark}
        By \underline{Lemma 15.1} if \(x\) is strongly ad-nilpotent then \(x\) is \(\operatorname{ad}\)-nilpotent.  

        Recall: \(x\in L_a(\operatorname{ad} y)\) means \(\exists m > 0 : (\operatorname{ad} y - a \cdot \operatorname{id}_{})^m (x) = 0\).
    
        This implies that \(\operatorname{ad} x\) is nilpotent after some calculation.
    \end{remark}

    \begin{definition}
        \(\mathcal{N} (L)\) is the set of strongly ad-nilpotent elements. 
    \end{definition}

    \begin{definition}
        \(\mathscr{E} (L) < \operatorname{Int} L \coloneqq \langle \exp (\operatorname{ad} x) \mid x \text{ is ad-nilpotent}  \rangle \) generated by \(\forall \text{exp} \operatorname{ad} x\) where \(x\in \mathcal{N} (L)\).
    \end{definition}

    \begin{remark}
        \(\mathcal{N}(L)\) is stable under \(\forall x \in \operatorname{Aut}(L)\).
        
        Thus, \(\mathscr{E}(L) \trianglelefteq \operatorname{Aut}(L)\).

        \(K \subset L \implies \mathcal{N}(K) \subset \mathcal{N}(L)\).

        Then,
    \end{remark}

    \begin{definition}
        \(\mathscr{E} (L,K)\) is generated by \(\exp \operatorname{ad} x \forall x\in \mathscr{E}(K)\) 
    \end{definition}

    Then, \(\mathscr{E}(K) = \mathscr{E}(L,K)\).

    If \(\phi : L \to L^{\prime}\) is an epimorphism then \(\phi(L_a(\operatorname{ad} y)) = L^{\prime}_a(\operatorname{ad} \phi(y))\).
    
    \(\implies  \phi (\mathcal{N}(L)) = \mathcal{N}(L^{\prime})\).

    \begin{lemma}
        [16.1] Let \(\phi: L \to L^{\prime}\) be an epimorphism. If \( \sigma'^{\prime} \in \mathscr{E}(L^{\prime}) \implies \exists \sigma\in \mathscr{E}(L)\) such that:

        \begin{center}
            \begin{tikzcd}
                L \ar[r,"\phi"] \ar[d,"\sigma"] & L^\prime \ar[d,"\sigma^\prime"] \\ L \ar[r,"\phi"] & L'
            \end{tikzcd}
        \end{center}

    \end{lemma}

    \begin{proof}
        If \(\sigma ^{\prime} = \exp \operatorname{ad} x^{\prime}\) \(x^{\prime} \in \mathcal{N}(L^{\prime})\) \(\exists x\) s.t. \(\phi(x)=x^{\prime}\).
        
        \(\forall z\in L, (\phi \circ \exp \operatorname{ad}_L x)(z) = \phi(z + [x,z] + [x,[x,z]]) = \phi(z) + [x^{\prime}, \phi(z)] + [x^{\prime}, [x^{\prime},\phi(z)]]\)
        
        \(= (\exp \operatorname{ad}_L x^{\prime})(\phi(z)) \implies\) QED. 
        
    \end{proof}

    \begin{theorem}
        [16.2] Let \(L\) be solvable. Let \(H_1, H_2\) be CSA's of \(L\). 
        
        Then, \(H_1\) is conjugate with \(H_2\) by an element of \(\mathscr{E}(L)\).
    \end{theorem}

\end{document}