\documentclass{article}
\usepackage{amsmath, amsthm, amssymb, amsfonts, mathtools, mathrsfs, enumitem, stmaryrd,physics, cancel, tikz-cd, graphicx, float, booktabs}
\usetikzlibrary{arrows}
\usepackage{geometry}
    \geometry{
        a4paper,
        left = 40mm,
        top = 20mm,
        right = 40mm,
        bottom = 30mm
    }
\setlength{\parindent}{0pt}

\theoremstyle{definition}
\newtheorem{problem}{Problem}
\newtheorem{solution}{Solution}
\newtheorem*{example}{Example}
\newtheorem*{exercise}{Exercise}
\newtheorem*{definition}{Definition}
\newtheorem{theorem}{Theorem}
\newtheorem*{theorem*}{Theorem}
\newtheorem{proposition}[theorem]{Proposition}
\newtheorem*{proposition*}{Proposition}
\newtheorem{lemma}[theorem]{Lemma}
\newtheorem*{lemma*}{Lemma}
\newtheorem{corollary}[theorem]{Corollary}
\newtheorem*{corollary*}{Corollary}
\newtheorem*{remark}{Remark}

\title{M522 Topology II}
\author{Thanic Nur Samin}
\date{\vspace{-5ex}}

\begin{document}
    \maketitle

    \section*{Monday, 1/13/2025}
    
    \section*{Singular Homology and CW Complexes}

    We want to talk about the \underline{Homology} of a space \(X\).

    \begin{definition}
        [Homology] Let \(X\) be a topological space. Consider the sequence of abelian groups:

        \[
            H_0 X, H_1 X, H_2 X, \cdots 
        \]

        These are the homemorphism invariants.

    \end{definition}

    For example, consider the \(2\)-torus \(T^2\) and the \(2\)-sphere \(S^2\). They are not homeomorphic, we can see that from their fundamental groups.

    % insert

    \(H_1 T^2 = \mathbb{Z} \oplus \mathbb{Z}\).

    \(H_1 S^2 = 0\).

    \(\therefore S^2 \not\cong T^2\) 

    Do note that, even if all elements from the sequence are isomorphic the spaces might not be isomorphic!

    Some application: see Davis and Kirk ``Homology Greatest Hits''.

    Knot theory seems very intuitive but proving statements is very troublesome. For example, how do you prove that the trefoil and the unknot are not the same?

    \begin{theorem}
        [Brouwer's Fixed Point Theorem] Every \(f:D^n \to D^n\) has a fixed point.
    \end{theorem}

    \begin{theorem}
        [Euler's Formula] For every `triangulation' of \(S^2\) we have:

        \[
            v - e + f = 2 = \chi(S^2)
        \]

        \(\chi\) denotes the \underline{Euler Characteristic}.

        eg pyramid \(4-6+4=2\), triangulated bipyramid \(5-9+6 = 2\), cube \(8-12+6=2\).

        % insert
    \end{theorem}

    \begin{theorem}
        [Hairy Ball Theorem]

        % insert

        \(\not\exists f\colon S^2 \to S^2\) s.t. \(\forall x\in S^2, x \cdot f(x) = 0\).
        
        So you can't comb the hairy ball.
    \end{theorem}
    
    \begin{theorem}
        [Jordan Curve Theorem]

        % insert

        The complement of a closed curve in plane has two components.


    \end{theorem}

    \begin{theorem}
        [Brouwer's Theorem on Invariance of Domain]

        \(m \neq n \implies \mathbb{R}^m \not\cong \mathbb{R}^n\).

        Consider open \(U \subset \mathbb{R}^n\) [a domain] and let \(f\) be a continuous injection \(f\colon U \to \mathbb{R}^n\).

        Then \(f(U)\) is open in \(\mathbb{R}^n\).
    \end{theorem}

    \subsection*{Variants of Homology}

    \begin{table}[H]
        \centering
        \begin{tabular}{c|c|c}
            \toprule
                 & defined for &  \\
            \midrule
                Singular Homology &Top Spaces & Easy to define  but \\ & & hard to compute \\ \midrule
                Simplicial homology & simplicial complexes and  & Easy to define and compute \\ & \(\Delta\)-complexes & but difficult to show \\ & & homeo inv. \\ \midrule
                Cellular homology & CW-complexes & hard to define, \\ & & easy to compute, flexible. \\
            \bottomrule
        \end{tabular}
        \caption{Variants of Homology}
    \end{table}

    \section*{Definition of Singular Homology}

    \begin{definition}
        [Standard \(n\)-simplex]

        \[
            \Delta^n = \left\{ (t_0, \cdots , t_n) \mid \sum_{i} t_i = 1, 1 \geq t_i \geq 0 \right\} \subset \mathbb{R}^{n+1}
        \]

        \[
            = \left\{ \sum_{i} t_i \underline{e}_i \mid 1 \geq t_i \geq 0, \sum_{i} t_i = 1 \right\} 
        \]

        \[
            = \text{convex hull of } \{ \underline{e_0}, \cdots , \underline{e_n} \} 
        \]

        % insert
    \end{definition}

    Recall that convex hull is the intersection of all convex sets containing the original set.

    \section*{Singular \(n\)-simplex in \(X\)}

    \(n\)-simplices are images of standard simplices under continuous maps.

    They are defined by a continuous map \(\sigma : \Delta^n \to X\).

    % insert

    We define singular \(n\)-chains \(S_n X\). These are free abelian groups with \(\mathbb{Z}\)-basis the singular \(n\)-simplicies in \(X\).

    A typical element will be a finite sum:

    \[
        n_1 \sigma_1 + \cdots + n_k \sigma_k \in S_n X
    \]

    Where \(\sigma_i : \Delta^n \to X\).

    Note: Davis and Kirk uses \(S_n X\), Hatcher uses \(C_n X\).

    For example, let \(X\) be the punctured plane \(X = \mathbb{R}^2 - \{ 0 \}\).

    \(\sigma_1 + \sigma_2 - \sigma_3 \in S_1 X\).

    % insert

    This is an example of a special \(1\)-chain callled the \(1\)-cycle.

    \underline{Goal}: Define a boundary map \(\partial_n : S_n X \to S_{n-1} X\) [Read Davis Kirk].
    
    Then, \(H_n\) is given by the quotient map:

    \[
        H_n = \frac{\ker \partial_n}{\operatorname{im} \partial_{n+1}} = \frac{n \text{-cycles}}{n \text{-boundaries}}
    \]

    \section*{Wednesday, 1/15/2025}
    
    \underline{Goal}: We want to define a homomorphism called a \underline{boundary map}.

    \[
        \partial_n : S_n X \to S_{n-1} X
    \]

    We start with the \(j\)'th face map.

    \[
        \delta_j = \delta_j^n: \Delta^{n-1} \to \Delta^n
    \]

    We have the map of barycentric coordinates:

    \[
        (t_0, \cdots ,t_{n-1}) \mapsto (t_0, \cdots , t_{j-1}, 0, t_j, \cdots , t_{n-1})
    \]

    The \(j\)'th face map of \(\sigma\) is given by precomposing \(\delta_j\):

    \[
        \sigma \circ \delta_j : \Delta^{n-1} \to X
    \]

    \begin{definition}
        The boundary \(\partial_n \sigma = \sum_{j=0}^n (-1)^j \sigma \circ \partial_j\).

        We can extend this definition tp \(S_n\) by linearity.

        \[
            \partial_n \left( \sum_{j} n_j \sigma_j \right) = \sum_{j} n_j \partial_n \sigma_j
        \]
    \end{definition}

    Let \(\sigma: \Delta^2 \to X\).

    Then, \(\partial_2 \sigma = \sigma \circ \delta_0 - \sigma \circ \delta_1 + \sigma \circ \delta_2\).

    % insert

    \begin{figure}[H]
        \centering
        \includegraphics[width=0.8\textwidth]{img/boundary.pdf}
        \caption{Boundary Map}
        \label{fig:boundary}
    \end{figure}

    \(\sigma : \Delta^1 \to X\)

    \(\partial \sigma = \sigma(e_1) - \sigma(e_0) = c_{\sigma(e_1)}=c_{\sigma(e_0)}\), endpoint - starting point.
    
    \begin{lemma}
        \(\partial_{n+1} \circ \partial_n = 0\).

        \begin{center}
            \begin{tikzcd}
                S_{n+1} X \ar[r,"\partial_{n+1}"] \ar[rr, bend right, "0"] & S_n X \ar[r,"\partial_n"] & S_{n-1} X
            \end{tikzcd}
        \end{center}

        This is the reason for \(-\) signs.
    \end{lemma}

    Then, we have,

    \[
        \begin{matrix}
            \operatorname{im} \partial_{n+1} & \subset & \ker \partial_n & \subset & S_n X \\
            n \text{-boundaries} &  & n \text{-cycles} &  & n \text{-chains}  \\
        \end{matrix} 
    \]

    \begin{definition}
        [Singular Homology]

        \[
            H_n X = \frac{\ker \partial_n}{\operatorname{im} \partial_{n+1}} = \frac{\text{cycles}}{\text{boundaries}}
        \]
    \end{definition}

    \begin{proof}
        We prove the lemma: \(\partial_{n-1} \circ \partial_n = 0\).

        \[
            \partial_{n-1} (\partial_n \sigma)
        \]

        \[
            = \partial_{n-1} \left( \sum_{j} (-1)^{j} \sigma(t_0, \cdots , 0, \cdots , t_{n-1}) \right) 
        \]

        \[
            = \sum_{k < j} (-1)^k (-1)^j \sigma(t_0, \cdots , 0, \cdots , 0, \cdots , t_n) \, \text{\(0\) s in \(k\)'th and \(j\)'th slots}
        \]

        \[
            + \sum_{k > j} (-1)^{k-1} (-1)^j \sigma(t_0, \cdots , 0, \cdots , 0, \cdots , t_n) \, \text{\(0\) s in \(k\)'th and \(j\)'th slots}
        \]

        \[
            = 0
        \]
    \end{proof}

    \begin{remark}
        \begin{enumerate}[label=\arabic*)]
            \item \(H_n X\) is defined for any topological space \(X\) and \(n \geq 0\).
            
            \item \(X \cong Y \implies H_n X \cong H_n Y\).
            
            \item Big and Formula Construction.
            
            \item Unclear how to compute.
        \end{enumerate} 
    \end{remark}

    Answer to the question: What is \(H_n X\):

    \[
        H_{\ast} X = \left\{ H_0 X, H_1 X, H_2 X , \cdots  \right\} 
    \]

    is a graded abelian group. \(H_k X\) individually are abelian groups.

    \begin{lemma}[Lemma 1]
        \[
            H_n(pt) \cong \begin{dcases}
                \mathbb{Z}, &\text{ if } n = 0 ;\\
                0, &\text{ otherwise} .
            \end{dcases}
        \]
    \end{lemma}

    \begin{lemma}[Lemma 2]
        If \(X\) has path-components \(\{ X_\alpha \}_{\alpha \in I}\), then,

        \[
            H_n X = \bigoplus_{\alpha \in I} H_n (X_\alpha)
        \]
    \end{lemma}

    \begin{lemma}
        [Lemma 3]
        \begin{enumerate}[label=\alph*)]
            \item \( H_0 X \cong \bigoplus_{I} \mathbb{Z} = \mathbb{Z}^{\# \text{ of path component}}\)
            \item \(X\) is path-connected, then \(H_0 X \cong \mathbb{Z}\).
        \end{enumerate} 
    \end{lemma}

    Recall:

    \begin{definition}
        \(X\) is path-connected if \(\forall a,b\in X, \exists \gamma: [0,1] \to X\) such that \(\gamma(0)=a, \gamma(1)=b\) 
    \end{definition}

    \begin{definition}
        A maximal path-connected subset of \(X\) is path-component.
    \end{definition}

    \begin{corollary}
        Homology of rational numbers is isomorphic to the homology of integers:

        \[
            H_{\ast} \mathbb{Q} \cong H_{\ast} \mathbb{Z} = (\mathbb{Z}^{\infty}, 0, 0, \cdots)
        \]

        But \(\mathbb{Q} \not\cong \mathbb{Z}\).
    \end{corollary}

    \section*{Friday, 1/17/2025}
    
    Recall:

    \(H_n X = \frac{\ker (\partial_n)}{\operatorname{im} (\partial_{n+1})} = \frac{\text{cycles}}{\text{boundary}} \in \text{homology class}\).

    We are looking for two cycles that belong to the same homology class.

    So, we want cycles \(z_1 \neq z_2\) which are homologous so that \(z_1 - z_2\) is a boundary. This implies their homology classes are equal: \([z_1] = [z_2]\).

    % insert 0 cycle example

    % insert torus example

    \section*{Algebra}

    \begin{definition}
        [Chain Complex] A \underline{chain complex} \(C_\bullet\) is a sequence:
        
        \[
            C_{\ast} = \{ C_0, C_1, C_2, \cdots \}
        \]
        
        of abelian groups with \(\partial_n : C_n \to C_{n-1}\) such that \(\partial_n \circ \partial_{n+1} = 0\).

        It looks like the following:

        \begin{center}
            \begin{tikzcd}
                \cdots \ar[r] & C_3 \ar[r, "\partial_3"] & C_2 \ar[r,"\partial_2"] & C_1 \ar[r,"\partial_1"] & C_0
            \end{tikzcd}
        \end{center}

        so that the composition of any two consecutive maps is \(0\). By conventions, \(\partial_0 = 0: C_0 \to 0\).
    \end{definition}

    Then, \(C_\bullet = \{ C_{\ast}, \partial_{\ast} \}\).

    \begin{definition}
        [Homology]

        \[
            H_n C_\bullet = \frac{\ker \partial_n}{\operatorname{im} \partial_{n+1}} = \frac{Z_n}{B_n}
        \]

        Here, \(C_n = n\)-chain.
        
        \(Z_n = \ker \partial_n, n\)-cycles
        
        \(B_n = \operatorname{im} \partial_{n+1}, n\)-boundaries.

        eg. \(S_\bullet X = \{ S_{\ast} X, \partial_{\ast} \}\) is a singular chain complex of \(X\).
    \end{definition}

    L1:

    \[
        H_n \text{pt} = \begin{dcases}
            \mathbb{Z}, &\text{ if } n=0 ;\\
            0, &\text{ otherwise} .
        \end{dcases}
    \]

    \[
        H_{\ast} (\text{pt}) = \{ \mathbb{Z},0,0,\cdots \}
    \]

    \begin{proof}
        \(\forall n, \exists ! \sigma_n : \Delta^n \to \text{pt}\).

        Then, \(\partial_1 \sigma_1 = \sigma_1 \circ \delta_0 - \sigma_1 \circ \delta_1\).

        \(\delta_0(t_0) = (0,t_0), \delta_1(t_0) = (t_0, 1)\).

        Thus, \(\partial_1 \sigma_1 = \sigma_1 \circ \delta_0 - \sigma_1 \circ \delta_1=(1-1)\sigma_0 = 0\)
        
        \(\partial_2 \sigma_2 = \sigma_2 \circ \delta_0 - \sigma_2 \circ \delta_1 + \sigma_2 \circ \delta_2 = (1-1+1)\sigma_1 = \sigma_1\).

        \(\partial_n \sigma_n = \begin{dcases}
            0, &\text{ if } n \text{ odd} ;\\
            1, &\text{ if } n \text{ even}.
        \end{dcases}\) 

        \(S_{\ast} X:\)
        
        \begin{center}
            \begin{tikzcd}
                & & \ar[r] & \mathbb{Z}\sigma_2 \ar[r] & \mathbb{Z}\sigma_1 \ar[r] & \mathbb{Z}\sigma_0 \\

                & & & \sigma_2 \ar[r,mapsto] & \sigma_1 \ar[r,mapsto] & 0 \\
                \cong & \ar[r,"1"] & \mathbb{Z} \ar[r,"0"] & \mathbb{Z} \ar[r,"1"] & \mathbb{Z} \ar[r,"0"] & \mathbb{Z}
            \end{tikzcd}
        \end{center}

        \(H_0 \text{pt} = \mathbb{Z} / 00 = \mathbb{Z}\).

        \(H_1 \text{pt} = \mathbb{Z} / \mathbb{Z} = 0\) 

        \(H_2 \text{pt} = 0 / 0 = 0\).
    \end{proof}

    L2: If \(\{ X_\alpha \}_{\alpha \in I}\) are path components of \(X\) then,

    \[
        H_n X = \bigoplus_{I} H_n X_\alpha
    \]

    \begin{proof}
        \(\sigma: \Delta^n \to X \overset{\Delta^n \text{p.c.}}{\implies} \sigma(\Delta^n)\) p.c.

        \(\implies \exists ! \alpha\) such that \(\sigma(\Delta^n) \subset X_\alpha\).

        Also \((\sigma \circ \delta_j)(\Delta^{n-1}) \subset X_\alpha\)
        
        \(S_{\ast} X = \bigoplus_{I} S_{\ast} X_\alpha\) 
    \end{proof}

    Augmentation

    \(\varepsilon : S_0 X \to \mathbb{Z}\)

    \(\varepsilon (\sum_{i} n_i \sigma_i) \coloneqq \sum_{i} n_{i}\)
    
    \(\varepsilon  \circ \partial_1 (\sigma) = \varepsilon  (\sigma \circ \delta_0 - \sigma \circ \delta_1) = 1-1 = 0\)

    Thus \(\operatorname{im} \partial_1 \subset \ker \mathscr{E}\)

    Thus, \(\exists \overline{\varepsilon}: H_0 X \to \mathbb{Z}\)

    \(\overline{\varepsilon} \left[\sum_{i} n_i \sigma_i\right] = \varepsilon \left( \sum_{i} n_i \sigma_i \right) = \sum_{i} n_i \). 

    L3:

    \begin{enumerate}[label=\arabic*)]
        \item If \(X\) is path connected then,
        
        \[
            \overline{\varepsilon}: H_0 X \xrightarrow{\cong}\mathbb{Z}
        \]

        \item If \(\{ X_\alpha \}_{\alpha \in I}\) are path components of \(X\) then,
        
        \[
            H_0 X = \bigoplus_{I} H_0 X_\alpha = \mathbb{Z}^{\text{\# of p.c. of } X} 
        \]
    \end{enumerate}
    
    \begin{proof}
        \begin{enumerate}[label=\arabic*)]
            \item Need to show \(\ker \varepsilon \subset \operatorname{im} \partial_1\).
            
            Choose base point \(x_0 \in X\).

            Suppose \(\varepsilon \left( \sum_{i} n_i \sigma_i \right) = 0\).

            Choose path \(\gamma_i: \Delta^1 \to X\) such that \(\gamma_i(\underline{e}_1) = \sigma_i(\underline{e}_0), \gamma_0(e_0) = x_0\).
            
            \(\partial_1 \left( \sum_{i} n_i \gamma_i \right) = \sum_{i} n_i \sigma_i - \sum_{i} n_i \text{cons}_{X_0} = \sum_{i} n_i \sigma_i \) 
        \end{enumerate} 
    \end{proof}

    \section*{\(\Delta\)-complex (p.102-104 of Hatcher)}

    eg torus

    \begin{center}
        \begin{tikzpicture}
            \draw[->] (0,0) -- (2,0);
        \end{tikzpicture}
    \end{center}

    % insert torus picture

    \(1 =\) \# of vertices
    
    \(3 =\) \# of edges

    \(2 =\) \# of faces

    \(\Delta_2 T \to \Delta_1 T \to \Delta_0 T\) 

    \(\cong \mathbb{Z}^2 \to \mathbb{Z}^3 \to \mathbb{Z}^1\) 

\end{document}