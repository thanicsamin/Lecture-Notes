\documentclass{article}

\usepackage{amsmath, amsthm, amssymb, amsfonts, mathtools,enumitem}
\usepackage{tikz-cd}
\usepackage{graphicx}
\usepackage{float}
\usepackage{booktabs}
\usepackage{geometry}
    \geometry{
        a4paper,
        left = 40mm,
        top = 20mm,
        right = 40mm,
        bottom = 30mm
    }
\setlength{\parindent}{0pt}

\theoremstyle{definition}
\newtheorem{problem}{Problem}
\newtheorem{solution}{Solution}
\newtheorem{example}{Example}
\newtheorem{definition}{Definition}
\newtheorem{theorem}{Theorem}
\newtheorem{proposition}{Proposition}

\newcommand{\Frac}{\operatorname{Frac}}
\newcommand{\im}{\operatorname{im}}


\title{Commutative Algebra MATH 502}
\author{Thanic Nur Samin}
\date{\vspace{-5ex}}

\begin{document}

\maketitle

Class 1: 01/08

\begin{table}[H]
    \centering
    \begin{tabular}{|c|c|}
        \toprule
            Algebraic Geometry &  Commutative Rings \\
        \midrule
            \(k^n\)  &  \(k[x_1, \dots, x_n ]\)  \\
            point \((p_1, \dots, p_n )\)  & maximal ideal \((x_1 - p_1,\cdots,x_n - p_n )\)  \\
            varities & ideals in \(k[x_1, \dots, x_n ]\)   \\
            Some shape defined by \(F=0\)  & \(k[x_1, \dots,x_n ]/(F)\)   \\
        \bottomrule
    \end{tabular}
    \caption{Relationship between Algebraic Geometry and Commutative Rings}
    \label{tab:algeocomring}
\end{table}

\(\text{ED} \implies \text{PID} \implies \text{UFD} \) 

\begin{theorem}
    Gauss Lemma: \(A \text{ UFD } \implies A[X] \text{ UFD} \) 
\end{theorem}

\begin{definition}
    Ring is a five-tuple \((A,+,\cdot,0,1)\) 
    \begin{itemize}
        
        \item \(A\) is a set.
        \item \(0,1\in A\) 
        \item \(+:A\times A \to A : (x,y) \mapsto (x+y)\)
        \item \(\cdot:A\times A\to A: (x,y)\mapsto xy\)
        \item \((A,+,0)\) abelian group
        \item \((xy)z=x(yz)\) associativity
        \item \((x+y)z=xz+yz\) distributivity
        \item \(x(y+z)=xy+xz\) distributivity
        \item \(x1=1x=x\)  
    \end{itemize}

    \(A\) is commutative if \(xy=yx\), in this course all rings are commutative.

    \begin{example}[Commutative Rings]
        \(\mathbb{Z},\mathbb{Q},\mathbb{R} ,\mathbb{C} ,\mathbb{F} _{p} ,\mathbb{F} _{p^r}, A[x],A/I,\Frac(A)\)         
    \end{example}

\end{definition}

\begin{definition}
    A homomorphism is a function \(f:A\to B\) so that,
    \[
        \begin{array}{ccc}
            f(x+y) & = &  f(x)+f(y) \\
            f(xy) & = & f(x)f(y)  \\
        \end{array}
    \] 
\end{definition}

\begin{definition}
    A subring \(R\) of a ring \(A\) is a subset so that \((R,\cdot,+,0_{A},1_{A})\) is a ring.

    \(x,y\in R\implies x+y,xy\in R\) which means R is `closed' under the operations.

    \(\mathbb{Z} \) is an initial ring. This means, for all ring \(A\), there exists a unique ring homomorphism \(\mathbb{Z} \to A\) that sends \(1_{Z} \to 1_{A}\)    
\end{definition}

\begin{definition}
    An ideal \(I\) of \(A\) is a subset so that \((I,0,+)\) is an abelian group and \(AI \subseteq I\).

    So we have, \(0\in I, x\in A,y\in I\implies xy\in I, x,y\in I\implies x+y\in I\) 

\end{definition}

\begin{example}
    Are all subrings ideals? NO

    Are all ideals subrings? NO
\end{example}

\(I\triangleleft A\) means \(I\) is an ideal. This gives an equivalence relation on \(A\).

\(a \sim a^{\prime} \iff a+I=a^{\prime} +I\iff a-a^{\prime} \in I\)

\([a]=a+I\) equivalence classes.

\(\{a^{\prime} |a^{\prime} \sim a\}\)

\begin{definition}
    Quotient ring \(A/I\) is defined by \([x]+[y]\coloneqq [x+y],[xy]\coloneqq [x][y]\) 
\end{definition}

\(A\to A/I\) is a ring homomorphism with \(\ker =I\)

\(f:A\to B\) is a ring homomorphism. This implies:

\begin{itemize}
    
    \item \(f(A) \text{ is a subring of} B\) 
    \item \(\ker f\triangleleft A\) 
    \item \(A/\ker f\) is isomorphic to  \(A/I\)  
\end{itemize}

\begin{definition}
    For a ring \(A\), an element \(x\in A\) is an unit of there exist \(y\) so that \(xy=1\).
    
    \(A^\times\) is the group of units.

    For example, \(\mathbb{Z}^\times\) is the cyclic group of order \(2\). 
\end{definition}

\begin{definition}
    \(x\in A\) is a zero divisor if there exists nonzero \(y\) so that \(xy=0\). 
\end{definition}

We have units, zero divisors and other elements.

\begin{table}[H]
    \centering
    \begin{tabular}{c|c|c|c}
        \toprule
            Ring & Zero divisors & units &  nonzerodivisors \\
        \midrule
            \(\mathbb{Z} /6\mathbb{Z}\)  & [0],[2],[3],[4] & [1],[5] &   \\
            \(\mathbb{Z} \)  & [0] & [1],[-1] & [2],[-2],...
              \\
        \bottomrule
    \end{tabular}
    \caption{Units and Zero divisors}
    \label{tab:unitzero}
\end{table}

\begin{definition}
    \(A\) is a domain if the only zero divisor is zero. For example, fields, subrings of domain, \(\mathbb{Z} [\sqrt 5]\) etc.
\end{definition}

If \(A\) is a domain then there exists \(\Frac(A)\), the field of fractions.

Now, suppose \(A\) is a subring of \(B\). let \(\beta \in B\).

\begin{definition}
    \(A[\beta ]=\text{smallest subring of \(B\) containing \(A\) and \(\beta \)  } \)
    
    \(=\{f(\beta )|f(x)\in A[x]\}\) 

    We have \(ev:A[x]\to A[\beta ]\) given by \(x\mapsto \beta \) 

\end{definition}

\hrule
\hfil

Class 2: 01/10

Let \(A\) be a domain. This means \(xy=0\implies x=0 \text{ or } y=0\)

Key property: Domain implies cancellation

\(xy=xz,x\neq 0\implies y=z\) as \(xy=xz\implies xy-xz=0\implies x(y-z)=0\implies y-z=0\implies y=z\)

\begin{definition}
    ED, Euclidean Domain is a domain with extra condition: a function \(f:A-\{0\}\to\mathbb{Z}_{>0}\) so that for all \(a\in A-0,b\in A\) we have \(b=aq+r\) with either \(r=0\) or \(f(r)<f(a)\).
    
    Alternative formulation (Dummit and Forte) we can have \(f(0)=0\). Also in some formulation \(f(0)=-\infty\).  
\end{definition}

\begin{example}
    if \(A\) is a field then \(f\) can be anything since \(r=0\) always. We take const.

    For \(\mathbb{Z}\) we have \(f(x)=|x|\).
    
    In \(k[x]\) we have \(f(p(x))=\deg p(x)\)

    Note that \(\mathbb{Z}[x]\) is not an ED.

    \(\mathbb{Z}[i]\) is an ED. \(f(a+ib)=a^{2}+b^{2}\)
    
    Lorentz polynomials \(k[x,x ^{-1} ]\) is an ED.

    \(f(ax^m + \dots + bx^{-n} )=m+n\)
\end{example}

\begin{definition}  
    \(I\triangleleft A\) is principal if there exists \(a\in A\) so that \(I=(a)\coloneqq Aa\) 
\end{definition}    

A domain \(A\) is a PID (Principal Ideal Domain) if all ideals are principal.

\begin{theorem}
    \(ED\implies PID\) 
\end{theorem}

\begin{proof}
    Suppose \(0\neq I\triangleleft A\) and \((A,f)\) is ED.
    
    Choose \(a\in I\) so that \(f(a)=\min f(I-0)\) 

    for all \(b\in I\) we have \(b=aq+r\) where \(r\in I\). We must have \(f(r)\geq f(a)\) which means \(r=0\) and thus \(b\in(a)\).   

\end{proof}

Note that \(\mathbb{Z}[x],\mathbb{C}[x,y]\) are not PID and thus they are not EDs.

\begin{theorem}
    By Gauss. Every \(n\in\mathbb{Z} -0\) factors into \(n=\pm p_1\cdots p_n\) primes unique upto reordering.
\end{theorem}

We generalize this.

\begin{definition}
    \(x\in A\) is irreducible if \(x\neq 0,\notin A^\times \) and \(x=ab\implies a\in A^\times \) or \(b\in A^\times\).
    
    \(x\in A\) is prime if \(x\neq 0,\notin A^\times ,x\mid ab\implies x\mid a\) or \(x\mid b\)
    
    \(x,y\in A\) are associates if \(x=yu\) where \(u\in A^\times \).   
\end{definition}

\begin{definition}
    A domain \(A\) is a UFD (unique factorization domain) if for any nonzero nonunit \(x\in A\) then:

    1. \(x=p_1\cdots p_r\) where \(p_j\) are irreducibles

    2. Decomposition is unique upto reordering and associates.
    
    This means if \(p_1\cdots p_n=q_1\cdots q_m\) then \(m=n\) and there is a permutation \(\sigma \) of indices and units \(u_i\in A^\times \) so that \(p_i=q_{\sigma(i)}u_i\)  
\end{definition}

\begin{theorem}
    All PIDs are UFDs.
\end{theorem}

\begin{proof}
    Assume \(A\) is a PID. Let \(x\in A-0,x\notin A^\times \)
    
    First we prove existence (1).

    If \(x\) is irreducible then we're done. If it is not, then since it is reducible we have \(x=x_1x_2\). If \(x_1,x_2\) are both irreducible then we're done. Otherwise \(x_1\) or \(x_2\) is reducible. WLOG \(x_1\) is reducible. Then \(x=x_{11}x_{12}x_2\). We continue. After reordering we have an infinite chain of ideals \((x)\subsetneq (x_1)\subsetneq (x_{11})\subsetneq (x_{111}\subsetneq)\).
    
    We claim that this terminates. Suppose otherwise.

    Then there exists an \(\infty \) number of ideals \(I_1\subsetneq I_2\subsetneq I_3\subsetneq\cdots\) 
    
    Take \(I=\bigcup_{i=1}^{\infty} I_i \). This is an ideal, but \(I=(x)\). \(x\in I_n\) for some \(n\) which means \(I_n=I_{n+1}=\cdots \) so there can't be any infinite ascending chain and thus we're done with te existence.

    [Take the tree. If it's unbounded by AOC we have an infinite chain. If \(x\) is not a finite product of irreducibles then we have an infinite chain.]

    Reread Dummit and Forte.

    Lemma: in a domain prime \(\implies \) irreducible, and in a PID prime \(\iff\) irreducible. We use this.

    \(p_1\cdots p_n=q_1\cdots q_m\)

    \(p_1|q_1\cdots q_m\)  implies \(p_1|q_i\) for some \(i\). Reorder so that \(p_1|q_1\). Then \(q_1=p_1u\). Since \(q_1\) is irreducible \(u\) is a unit.
    
    Now we have \(p_2\cdots p_n=uq_2\cdots q_m\). We keep going for the proof. 

\end{proof}

\hrule
\hfil

Class 3: 01/12

Today we prove \(A \text{ UFD} \implies A[x] \text{ UFD} \) 

Monday MLK day, Wednesday explicit Galois Theory \(\operatorname{Ga}(\mathbb{F}_{p^r}/\mathbb{F}_p)\) and \(\operatorname{Ga}(\mathbb{Q}(\xi_n)/\mathbb{Q} )\). 

Recall that \(A\) is a UFD (Unique Factorization Domain) if \(A\) is a domain and every \(x \in A - (0 \cup A^\times )\) is a product of irreducibles and this factorization is unique upto reordering and multiplication by units.

\underline{Notation}: \(x \overset{\bullet}{=} \) means \(x,y\) are associates, aka \(x=yu\) where \(u \in A^\times\)  

Two nice properties of UFD:

\begin{itemize}
    \item prime \(\iff \) irreducible
    \item gcd and lcm exists
\end{itemize}

\begin{proposition}
    If \(A\) is a UFD, \(x\in A\) is non-unit and non-zero, then \(x\) prime \(\iff \) \(x\) is irreducible.
\end{proposition}

\begin{proof}
    \(\implies \)
    
    Suppose \(x\) is a prime. Assume \(x\) is not irreducible. If \(x=ab\) we have,
    \(x \vert ab\) which implies \(x \vert a\) or \(x \vert b\).

    WLOG, \(x \vert a\). Then we have \(a = xy\).
    
    So, \(x = ab = xyb \implies yb = 1 \implies b\in A^\times\)
    
    So \(x\) is indeed irreducible.

    [Note that this is true for arbitrary domains]

    \(\impliedby \) 

    Assume \(x\) is irreducible. If \(x \vert ab\), we have \(ab=xy\)
    
    The uniqueness of factorization implies \(x\) is a factor of \(a\) and \(b\). So \(x\) is a prime.

    [We need uniqueness for this]
    
\end{proof}

\begin{definition}
    GCD. First, \(d\) is a \underline{common divisor} of \(a\) and \(b\) if \(d\vert a\) and \(d\vert b\). Equivalently, \((a),(b) \subset (d)\). \underline{To contain is to divide}.

    \(d\) is a gcd of \(a\) and \(b\) if \(d\) is a common divisor and for any other common divisor \(d^{\prime} \), we have \(d^{\prime} \vert d\) ir \((d)\subset (d^{\prime})\) . In other words, \((d)\) is the unique minimal principal ideal so that \((a),(b)\subset (d)\) 
\end{definition}

In a general ring, it may or may not exist. But in a UFD, gcd's exist and unique upto multiplication by units.

\(\forall a,b \in A\), if \(d,d^{\prime}\) are \(\gcd(a,b)\) then \(d\overset{\bullet}{=} d^{\prime} \).

So, \(2,-2=\gcd(4,6)\) 

Formula for gcd:

\(a=u p_1^{e_1}\cdots p_r^{e_r}\)

\(b=v p_1^{f_1}\cdots p_r^{f_r}\) 

Where \(u,v\in A^\times \), \(p_j\) are distinct primes and \(e_i,f_i\geq 0\) 

Thus, \(\gcd(a,b)=p_1^{\min(e_i,f_i)}\cdots p_r^{\min(e_r,f_r)}\) 

Now, we prove that,

\begin{theorem}
    \(A\) is a UFD \(\implies \) \(A[x]\) is a UFD
\end{theorem}

Note that, if \(A\) is a domain, then \(A[x]\) is a domain.

Proof is an exercise. Just work through the coefficients.

For the rest of the class, assume \(A\) is a UFD. Our canonical example is \(A=\mathbb{Z}\). Note that \(\mathbb{Z} [x]\) is not a PID.

\begin{definition}
    A polynomial \(f(x)=\sum_{j=0}^{n} a_{n-j} x^{n-j} \in A[x]\) is primitive
    
    if \(\gcd(a_n,\cdots,a_1,a_0) = 1\). 
\end{definition}

\begin{theorem}
    Gauss' Lemma: Suppose \(f,g\in A[x]\) are primitive. Then \(fg\) is primitive.
\end{theorem}

\begin{proof}
    Note that, \(h\in A[x]\) is primitive \(\iff \) \(\forall \) prime \(p\) of \(A\), \(\overline{h} \neq 0\in A[x]/(p)=(A/p)[x] \iff \forall\) prime \(p, p \not\vert \gcd(a_n, \dots ,a_0)\) 
    
    Now, since \(f,g\) are primitive, \(\overline{f},\overline{g}  \neq 0 \in (A / p) [x]\). Note that \((A / p)[x]\) is a domain. Now, \(\overline{f} \overline{g} =\overline{fg} \neq 0 \in (A / p)[x]\) and thus \(fg\) is primitive.
    
\end{proof}

\begin{definition}
    Suppose \(f\in A[x]\) Then, the content \(c(f)\) is the gcd of the coefficients \(\gcd(a_n,\cdots,a_0)\).  
\end{definition}

Note that \(\frac{1}{c(f)}f\) is primitive.

Also, \(\forall f,c\) we have \(\frac{f}{c}\in A[x]\) is primitive \(\iff c\overset{\bullet}{=} c(f)\) 

Note that division doesn't always make sense in a ring. \(\frac{x}{y}\) means if \(x=yq\) then \(q=\frac{x}{y}\) 

A corollary of Gauss' Lemma: Applying Gauss' Lemma to non-primitive polynomial gives us \(c(fg)\overset{\bullet}{=} c(f)c(g)\) 

\begin{proof}
    \(\frac{1}{c(f)}f,\frac{1}{c(g)}g\) are primitive, which means \(\frac{1}{c(f)c(g)}fg\) is primitive. This implies \(c(f)c(g)\overset{\bullet}{=} c(fg)\) 
\end{proof}

Suppose \(A\) is a domain. Let \(k=\Frac(A)\). EG \(\mathbb{Q} =\Frac(\mathbb{Z} )\) 

(i.e. \(A\subset k\) field, \(\forall x\in k^\times, x=\frac{a_1}{a_2}\) where \(a_1,a_2\in A\) )

For \(f\in k[X]\) we define \(c(f)\in k^\times \) by \(\frac{f}{c(f)}\in A[x]\) is primitive.

Corollary of Gauss Lemma: \(\forall f,g\in k[x]\) we have \(c(fg)=c(f)c(g)u\) where \(u\in A^\times\)

Finally we prove our theorem.

\begin{proof}
    ``Existence'': Suppose \(0\neq h\in A[x]\).

    We can write \(h=c(h)f\) where \(f\) is primitive.

    Factor \(c(h)=p_1\cdots p_k\) as product of primes in \(A\).

    Factor \(f=f_1\cdots f_r\) as a product of irreducible in \(k[x]\) since \(k[x]\) is an Euclidean Domain.

    We say that \(h = p_1\cdots p_k \frac{f_1}{c(f_1)}\cdots \frac{f_r}{c(f_r)}\) 

    Since \(f_j\) is irreducible in \(k[x]\) \(\implies\) \(\frac{f_j}{c(f_j)}\) is irreducible in \(A[x]\)  

\end{proof}

Addendum from me: Suppose primitive \(f\in A[x]\) such that \(f=g^{\prime} h^{\prime} \) in the field of fractions. Then, \(f=\frac{g}{u}\frac{h}{v} \implies fuv=gh\). So, \(c(g)c(h)=c(gh)=c(fuv)\overset{\bullet}{=} uv\). So \(f=\frac{g}{c(g)}\frac{h}{c(h)}\)  

\hfil
\hrule

Class 04: 01/17

We have A UFD \(\implies \) A[x] UFD

We use Gauss lemma and k[x] UFD.

Since we have \(A[x][y]=A[x,y]\) by induction we have \(A\) UFD implies \(A[x_1, \dots, x_n ]\) is a UFD.

We give a geometric definition.

\begin{definition}
    A \underline{variety} \(V \subset k^n\) is the zero set of \(\left\{ f_\alpha  \right\} \subset k[x_1,\dots, x_n]\)  
\end{definition}

For example, if \(f=x_1 x_2\) then the variety is the axes in euclidean space [insert pictures]

We actually have a correspondence between varieties in \(k^n\) and ideals \(I\) of \(k[x_1, \dots, x_n ]\) 

\(V \mapsto I(V)=\left\{ f\in k[x_1, \dots, x_n ] | f(V)=0 \right\} \) 

On the other hand \(V(I)=\left\{ \underline{x}\in k^n | \forall f\in I, f(\underline{x}) = 0 \right\} \)  

Is it a bijection? No, \(V(x)=V(x^2)\) but \((x) \neq (x^2)\)  

Is \(I(v)\) finitely generated? Yes, by a theorem of Hilbert

If so is it by \(\leq n\) polynomials? by \(\leq n\) irreducible polynomials?

Is \(V=\) union of irreducible varieties uniquely?

There are a lot of connections between algebraic geometry and ring theory.

\section*{Explicit Galois Theory}

eg. \(\mathbb{F} _{p^r}/\mathbb{F} _p \simeq \mathbb{Z} /r\) with correspondence between \(x\mapsto x^p\) and \(1\)   

eg, \(\mathbb{Q} (\xi _n)/\mathbb{Q} \simeq \mathbb{Z} /n\) with correspondence between \(\xi _n \mapsto \xi _n^k\) and \(k\)

\section*{Field Extensions}

Suppose \(0 \neq f \in k[x]\)

1. \(\exists K\) such that \(f\) has a root in \(K\) 

2. \(f(\alpha )=0\) for some \(\alpha \in K\)  then the ring \(k[\alpha ]\) is a field

3. \(f(x)\) has at most \(\deg f\) roots in \(k\) 

For proof of 1: Let \(f_1\) be irreducible factor of \(f\).

Then, \(K=\frac{k[x]}{(f_1(x))}\), so \(x+(f_1(x))\) is a root of \(f_1\) and thus \(f\).

2. Let \(\beta \neq 0\) be an element of \(k[\alpha]\) 

Then, \(\cdot \beta : k[\alpha] \to k[\alpha]\) 

It is injective since \(k[\alpha]\) is a domain

\(\dim_h k[\alpha ]\leq \deg f\) 

So it must be surjective and thus \(1\) is in the image.

3. \(\alpha \) is a root iff \(x-\alpha | f\)  then induct.

\begin{definition}
    Splitting Fields (Definition 13.4): \(K/k\) is a splitting field for nonzero \(f\in k[x]\) if \(f(x) = a \prod (x-\alpha _i)\in K[x]\)
    
    K----E----k \(\implies \) does not split in \(E\)
    
    For example, \(\mathbb{Q} [\sqrt{2} ]\) is splitting field for \(x^2-2\) but \(\mathbb{Q} (\sqrt[3]{2} )\) is not the splitting field for \(x^{3-2} \). We need \(\mathbb{Q} (\sqrt[3]{2},\xi _3 )\).

\end{definition}

We define \(\xi _n=e^{2\pi i/n} \) as the primitive root of \(1\)

\begin{theorem}
    \(f(x)\in k[x]\).

    1. There exists splitting field \(K/k\) for \(f(x)\)
    
    2. If \(K/k\) and \(L/k\) are splitting fields for \(f\) then there exists \(\phi : K \to L\) such that \(\phi |_{k}=Id_k \) 

\end{theorem}

Proof uses observation and induction on \(\deg f\)

Application 1:

Algebraic Closure:

\begin{definition}
    A algebraic closure \(\overline{k} \) over a field \(k\) is \(\overline{k} / k\) such that \(\alpha \in \overline{k} \) is algebraic over \(k\) which means \(\exists f(x)\in k[x]\) that is nonzero and \(f(\alpha )=0\) and also every nonzero \(f\in k[x]\) has a root \(\alpha \in \overline{k} \)  
\end{definition}

\begin{theorem}
    Suppose \(k\) is a field. Then,

    1: There exists an algebraic closure \(\overline{k} / k\)
    
    2: If there are two algebraic closures of \(k\) then there is an isomorphism which restricted to \(k\) is the identity.

\end{theorem}

Proof: Zorn's lemma

\(k\) is algebraically closed if \(\overline{k} = k\) 

Which means every polynomial has a root.

\(\overline{k} \) is algebraically closed which means \(\overline{\overline{k} } = \overline{k} \) 

\(\mathbb{C} \) is algebraically closed by Gauss Fundamental theorem of Arithmetic.

We have \(\mathbb{Q}  \subset \overline{\mathbb{Q}} \subset \mathbb{C} \) 

\(\overline{Q} \) is countable and algebraically closed.

\(\overline{\mathbb{F} _p} = \cup \mathbb{F} _{p^r}\) 

Application 2 of splitting field is to finite fields.

Let \(p\) be a positive prime number.

\(\mathbb{F} _p\coloneqq \mathbb{Z} / p\mathbb{Z} \) 

It is a field.

Suppose \(F\) is a finite field.

So, we have, \(\vert F \vert > 1\) from the definition \(1 \neq 0\)

There exists a unique prime \(p\) so that \(p 1_F = 0_F\) 

Proof: Consider the map \(\mathbb{Z} \to F\) that sends \(1\to 1_F\). The kernel is \(n\mathbb{Z} \). So, the domain \(\mathbb{Z} / n\mathbb{Z} \) has an injective homomorphism to \(F\) which means \(n\) is a prime

\(\operatorname{char}(F) = p \) means in \(F\) field, \(p 1_F = 0_f\)

Now, \(\mathbb{F} _p\) has injective homomorphism to \(F\) (sends \(1\to 1\) ) So we have \(F\) is a vector space over \(\mathbb{F} _p\) so we have \(\dim_{\mathbb{F} _p}=r\) 

\hfil
\hrule

Class 05: 01/19

\section*{Finite Fields and Galois Theory}

Suppose \(F\) is a field and \(\operatorname{char}(F) = p \implies p1 = 0\).

\begin{definition}
    Frobenius Endomorphism: \(\sigma : F \to F\) given by \(\sigma (x)=x^p\) 
\end{definition}

\begin{proposition}
    \(\sigma \) is a field morphism.
\end{proposition}

\begin{proof}
    \(\sigma (1)=1,\sigma (xy)=(xy)^p=x^p y^p = \sigma(x)\sigma (y)\)
    
    \(\sigma (x+y)=(x+y)^p=\sum_{k=0}^{p} \binom{p}{k}x^{p-k}x^k=x^p + y^p \) since \(\binom{p}{k}\) is divisible by \(p\) for \(1\leq k \leq p-1\) and hence \(0\) in \(F\) since \(\operatorname{char}(F) = p\) 
   
    Also, \(\ker \sigma \triangleleft F\) but \(F\) is a field which means \(\sigma \) is \(1-1\) aka injective
    
    Also \(\vert F \vert < \infty \implies \sigma(F) \simeq F\) 

\end{proof}

Consider the field of fractions \(\mathbb{F} _p(t)=\operatorname{Frac}(\mathbb{F} _p[t]) \) then we have \(\sigma(t)=t^p\), then \(\sigma\) is not onto as \(t\notin \operatorname{im}\sigma \)  

Let \(F\) be a finite field. Last time we showed \(\operatorname{char} F = p\) and \(\vert F : \mathbb{F}_p \vert = \dim_{\mathbb{F}_p}F = r \)  

\begin{proposition}
    Let \(q=p^r\). Then,
    
    \begin{itemize}
        \item \(\exists \) field of order \(q\) 
        \item \(\vert F \vert = q \implies F\) is splitting field of \(x^q - x\) 
        \item If \(\vert F \vert = q = \vert K \vert \) then \( F \simeq K\) 
    \end{itemize}

\end{proposition}

\begin{proof}
    For 1: Let \(E\) be a field where \(x^q - x\) splits. Let \(f(x)=x^q - x\). We claim that \(f\) has distinct roots. One way to look at it is \(\gcd(f,f^{\prime})\). Then, \(\gcd(x^q - x, qx^{q-1} - 1 ) = \gcd(x^q - 1, -1 )=1\) since we are working modulo \(p\) and thus \(q=0\). This implies \(f\) has \(q\) distinct roots in \(E\).
    
    Let \(F=E^{\sigma^q}=\left\{ x\in E : \sigma ^r(x) = x \right\} = \left\{ x\in E:f(x)=0 \right\} \) 

    Then \(\vert F \vert = q\)
    
    We claim that \(F\) is a field. We have multiplication and inverse easily. If \(\alpha ,\beta \in F\) then \((\alpha +\beta )^q = \alpha ^q + \beta ^q\) [Freshmen's dream] so we have addition too so we have proved that \(F\) is a field.  

    For 2: If \(F\) is a field of order \(q\) then for \(x\in F^\times\) since \(\vert F^\times \vert = q-1\) we have \(x^{q-1} = 1 \implies x^q - x = 0 \), and this is also true for \(x=0\). So, \(x^q - x = \prod_{\alpha \in F}(x-\alpha)\) and thus \(F\) must be a splitting field. 

    For 3: Splitting fields are unique.

\end{proof}

Note that \(\mathbb{F}_{p^r}\) means any field of order \(p^r\). All such fields are isomorphic, but there isn't any canonical `god given' construction of \(\mathbb{F}_{p^r}\)  

Warning: \(\mathbb{F}_{p^r} \not\simeq \mathbb{Z} / p^r\) 

There's another basic fact.

\begin{proposition}
    \(\mathbb{F} _p^\times \) is cyclic.
\end{proposition}

In Dummit and Foote there's a more general fact: \(\forall\) field \(F\) , any finite subgroup \(A\) of \(F^\times \) is cyclic. This lemma directly implies the above proposition.

\begin{proof}
    This uses the third observation from last time. Let \(n = \vert A \vert\) and let \(d = \exp A = \max \left\{ \operatorname{ord} a | a\in A \right\} \). This must divide \(n\).

    Then, \(x^d - 1\) has \(n\) roots [namely the elements of \(A\)] which means \(d \geq n\) but also \(d \vert n\) so \(d=n\) which means \(A\) is cyclic.
    
\end{proof}

Corollary: \(\mathbb{F} _{p^r}/\mathbb{F} _p\) is primitive, i.e. \(\mathbb{F}_{p^r}=\mathbb{F}_p [\theta]\) for some \(\theta\) and \(\langle \theta \rangle = \mathbb{F}_q^\times \)  

Corollary: \(\forall r \geq 1, \exists \) irreducible polynomial \(h(x)\) of degree \(r\)

For this, \(h(x)\) is just the minimal polynomial of \(\theta\) 

So \(\mathbb{F}_{p^r}\simeq \frac{\mathbb{F}_p[x]}{(h(x))}\) 

Review of Galois Theory:

Suppose \(L/K\) is a finite extension of field.

Then \(\operatorname{Aut}(L/K) = \left\{ \sigma : L \to L \text{ isomorphisms so that }\sigma|_K = Id_K  \right\}  \) 

\begin{definition}
    \(L/K\) is Galois if \(\vert \operatorname{Aut}(L/K)=\vert L : K \vert   \vert \) 
\end{definition}

For example, \(\mathbb{Q}(\sqrt{2})/Q\) is Galois but for \(\sqrt[3]{2} \)  it's not.

\begin{definition}
    If \(L / K\) is Galois, define,

    \(\operatorname{Gal}(L / K) \coloneqq \operatorname{Aut}(L / K) \) 
\end{definition}

\begin{theorem}
    Fundamental Theorem of Galois Theory: Suppose \(L / K\) is Galois. Then the map of subgroups of \(\operatorname{Gal}(L / K)\) to fields \(E\) such that \(L-E-K\) given by \(H \mapsto L^H\) is a bijection.   
\end{theorem}

Back to finite fields: \(\sigma (x)=x^p\)
    
Then \(\sigma \in \operatorname{Aut}(\mathbb{F} _{p^r} / \mathbb{F} _p) \) 
   
Since \(\mathbb{F} _q ^\times \) is cyclic, \(\operatorname{ord} \sigma =r \) 

So, \(\sigma ^r = Id\)
    
This means \(r = \vert \mathbb{F} _{p^r} : \mathbb{F} _p \vert \geq  \operatorname{Aut}(\mathbb{F} _{p^r} / \mathbb{F}_p) \geq r \) 

So, \(\operatorname{Gal}(\mathbb{F} _{p^r} / \mathbb{F} _p) = \langle \sigma  \rangle \simeq \mathbb{Z} / r\) 

Corollary: For \(d \vert r, \exists !\) \textbf{su} ield \(F\) of \(\mathbb{F} _q\) of order \(p^d\)

For assignment, we were supposed to analyze the lattice.

Given \(\mathbb{F} _{p^{n_1}},\mathbb{F} _{p^{n_2}}\) we can embed in \(\mathbb{F} _{p^{n_1 n_2}}\)  

Note:

\(\operatorname{Gal}(\mathbb{F} _{p^n}) / \mathbb{F} _p \cong \mathbb{Z} _N \) and the geneeratoris the frobnius map \(\sigma : \mathbb{F} _{p^n} \to \mathbb{F} _{p^n}\) given by \(\sigma (a)=a^p\)  

This is true because \(a^{p^n}=a\) 

\hfil
\hrule

Class 06: 01/22

We start on Atiyah MacDonald on wednesday.

Finite Field Definition: 13.5+14.3

Cyclotomic FIeld Definition: 13.6+14.5

\begin{table}[H]
    \centering
    \begin{tabular}{c|c}
        \toprule
            Finite Fields &  Cyclotomic \\
        \midrule
            \(\mathbb{F}_{p^n}\)  &  \(\mathbb{Q}(\zeta _n)\)  \\
            Splitting Field of \(x^{p^n}-x\)  & Minimal Polynomial \(\Phi_n\)   \\
            \(\operatorname{Gal}(\mathbb{F} _{p^n} / \mathbb{F} _p) \cong \mathbb{Z} /n \) with \(a \mapsto a^{p^k}\)   & \(\operatorname{Gal}(\mathbb{Q} (\zeta _n) / \mathbb{Q}) \cong \mathbb{Z} / n^\times \) with \(\zeta_n \mapsto \zeta_n^k\)   \\
             &   \\
        \bottomrule
    \end{tabular}
    \caption{Finite vs Cyclotomic Fields}
    \label{tab:finvscycl}
\end{table}

Let \(\zeta _n = e^{2\pi i/n} \) 

Let \(\mu_n = \left\{ \zeta \in \mathbb{C} ^\times : \zeta ^n = 1 \right\} < \mathbb{C} ^\times \) 

Then \(\mu _n = \langle \zeta _n \rangle \) and is cyclic of order \(n\)

\begin{definition}
    \(n\) th cyclotomic polynomial:
    \[
        \Phi_n(x)=\prod_{\text{primitive }\zeta \in \mu_n } (x-\zeta)=\prod_{0<a<n, (a,n)=1} (x-\zeta_n^a)
    \]
\end{definition}

Note: \(\deg\Phi_n=\phi(n)\) by definition.

Also: \(\displaystyle X^n - 1 = \prod_{d \mid n}^{} \Phi_d(X)  \)

This gives us a recursive formula for cyclotomic polynomials:

\[
    \Phi_n(x) = \frac{x^{n-1}}{\prod_{d\mid n,d \neq n}\Phi_d(x)}
\]

\begin{table}[H]
    \centering
    \begin{tabular}{c|c}
        \toprule
            \(n\)  &  \(\Phi_n(x)\)  \\
        \midrule
            \(1\)  &  \(x-1\)  \\
            \(2\)  &  \(x+1\)  \\
            \(3\)  &  \(x^2 + x + 1\)  \\
            \(4\)  &  \(x^2 + 1\)  \\
            \(5\)  &  \(x^4 + x^3 + x^2 + x + 1\)  \\
            \(6\)  &  \(x^2 - x + 1\)  \\
        \bottomrule
    \end{tabular}
    \caption{Cyclotomic Polynomials}
    \label{tab:cyclo}
\end{table}

Corollary: \(\Phi_n(x)\in \mathbb{Q} [x]\) 

Corollary of Gauss' Lemma:

Suppose \(R\) is a UFD. Let \(k=\operatorname{Frac}(R) \). Then \(f(x)=g(x)h(x)\) in \(k[x]\) with \(f,g,h\) monic and \(f(x)\in R[x]\) then \(g,h\in R[x]\) 

We can use this to show that all the \(\Phi_n\) are monic.

\begin{proof}
    Since \(g,h\) are monic, we can find \(d,e\in R\) so that \(dg(x),eh(x)\in R[x]\) and they are primitive [\(d,e\) are lcm of numerators.]
    
    Then, \(def(x)=[dg(x)][dh(x)]\), but \(def(x)\) and \(f(x)\) are both primitive which can only happen if \(de\) is a unit aka \(d,e\) are both units. So \(g(x),h(x)\) are both primitive in \(R(x)\) 

\end{proof}

This proves that all the \(\Phi_n\) are integer polynomials.

Lemma: \(\Phi_n(x)\) are irreducible in \(\mathbb{Q}[x]\) 

This actually implies \(\vert \mathbb{Q} (\zeta _n) : \mathbb{Q}  \vert =\phi(n)\) 

\begin{proof}
    Due to Dedekind.

    Suppose \(\Phi_n(x)=f(x)g(x)\) with \(f,g\) both monic.

    With \(f\) irreducible, \(\deg f > 0\)
    
    Now, by corroloary of Gauss Lemma, \(f,g\in \mathbb{Z} [x]\) 

    Claim: Suppose \(p\) is a prime, \(p\nmid n\) and \(\zeta \) is a primitive \(n\) th root of unity and \(f(\zeta) = 0\). Then, \(f(\zeta^p)\) 
    
    Proof of Claim by contradiction:

    Note that, \(\Phi_n(\zeta ^p)=0\). Suppose \(g(\zeta ^p)=0\) 

    Since \(f\) is irreducible it is the minimal polynomial of \(\zeta \) so we have \(f(x)\mid g(x^p)\). Then, \(g(x^p)=f(x)h(x)\)  

    Consider \(\pmod p\) which gives us \(\overline{g}(x^p)=\overline{f}(x)\overline{h}(x)  \) 

    Now, \(\overline{g}(x)^p = \overline{g}(x^p)\mid \overline{f}(x)\mid x^n - 1  \) 

    But \(x^n - 1\) has no multiple factor in \(\mathbb{F} _p(x)\) [formal derivative trick.]

    

\end{proof}

\hfil
\hrule

Class 07: 01/24

Recall:

\begin{table}[H]
    \centering
    \begin{tabular}{c|c}
        \toprule
            Finite Fields &  Cyclotomic \\
        \midrule
            \(\mathbb{F}_{p^n}\)  &  \(\mathbb{Q}(\zeta _n)\)  \\
            Splitting Field of \(x^{p^n}-x\)  & Minimal Polynomial \(\Phi_n\)   \\
            \(\operatorname{Gal}(\mathbb{F} _{p^n} / \mathbb{F} _p) \cong \mathbb{Z} /n \), \(k \mapsto (a \mapsto a^{p^k})\)   & \(\operatorname{Gal}(\mathbb{Q} (\zeta _n) / \mathbb{Q}) \cong \mathbb{Z} / n^\times \), \(k \mapsto (\zeta_n \mapsto \zeta_n^k)\)   \\
             &   \\
        \bottomrule
    \end{tabular}
    \caption{Finite vs Cyclotomic Fields}
    \label{tab:finvscycl2}
\end{table}

Yesterday:

\[
    \Phi _n(x)=\prod _{\text{primitive \(\zeta \in \mu_n\) } } (x-\zeta) = \prod _{(a,n)=1,0\leq a<n} (x-\zeta _n^a)
\]

That gave us:

\(\deg\Phi_n(x)=\phi (n)\) 

\(x^n - 1 = \prod _{d\mid n}\Phi_d(x)\) 

\(\displaystyle \Phi_n(x)=\frac{x^n - 1}{\prod _{d\mid n,d\neq n} \Phi_d(x) } \in \mathbb{Q} [x]\) 

Gauss Lemma:

\(f\in \mathbb{Z} [x],f=gh\in \mathbb{Q} [x],g,h\) monic implies \(g,h\in \mathbb{Z} [x]\)  

Thus \(\Phi_d(x)\in \mathbb{Z} [x]\) 

Then we have the hard theorem we were in the middle of.

\(\Phi_n(x)\) is irreducible.

\begin{proof}
    \(\Phi_n(x)=fg\in\mathbb{Q} [x]\) 

    \(\zeta_n\) must be a root of either \(f\) or \(g\). WLOG \(f(\zeta_n)=0\) and \(f\) is irreducible. We want to show that \(g=1\)
    
    We had a tricky argument by Dedekind. The stuff on last class wasn't quite right.

    Claim:

    Suppose we have a primitive \(\zeta, f(\zeta)=0\). If \(p \nmid n\) then \(f(\zeta^p)=0\) 

    Proof by contradiction: Assume \(f(\zeta^p) \neq 0\). We know that \(\Phi_n(\zeta^p)=0=f(\zeta^p)g(\zeta^p)\)  

    \(f\) is irreducible so \(f\) is the minimal polynomial for \(\zeta \). But \(g(x^p)\) is a polynomial with \(\zeta \) as a \(0\). So, \(f(x)\mid g(x^p)\). Hence, \(g(x^p)=f(x)g(x)\)  

    Reducing mod \(p\) we see,

    \((\overline{g}(x))^p = \overline{g}(x^p)=\overline{f}(x)\overline{h}(x)   \) 

    So, \(\overline{f}\mid \overline{g}^p  \)
    
    So, \(\gcd(\overline{f} ,\overline{g} )\neq 1\)

    \((\gcd(\overline{f},\overline{g}  ))^2 \mid \overline{f} \overline{g} =\overline{\Phi_n(x)} \mid \overline{x^n - 1 } \) 

    But \(\gcd(\overline{x^n - 1},\frac{\mathrm{d}}{\mathrm{d}x} \overline{x^n - 1} ) = \gcd(\overline{x^n - 1 } , \overline{n x^{n-1} } )=1\) so we have no multiple factors which is a contradiction. 

    So, \(f(\zeta^p) = 0\)
    
    Claim 2: \((a,n)=1\) implies \(f(\zeta_n^a)=0\) 

    Proof: \(a\) is product of primes not dividing \(n\), induct.

    So, \(f\) has \(\phi(n)\) roots \(\zeta_n^a\), \(\Phi_n\) also has \(\phi(n)\) roots. 

\end{proof}

Corollary: \(\vert \mathbb{Q} (\zeta _n) : \mathbb{Q} \vert = \deg \Phi_n(x)=\phi(n) \) 

Also, \(\mathbb{Q} [\zeta_n]=\frac{\mathbb{Q}[x]}{(\Phi_n(x))}\) 

ALso, \(K(\alpha)\) is the smallest field and \(K[\alpha]\) is the smallest ring containing \(\alpha \) but it's the same if the degree of extension is finite. Not same for \(\pi\) tho.

Claim: Suppose \(L/K\) finite. Then,

\(\vert \operatorname{Aut}(L / K) \vert \leq \vert L:K \vert  \) 

\begin{definition}
    \(L/K\) is Galois if \(\vert \operatorname{Aut}(L / K) \vert = \vert L:K  \vert \) and we call \(\operatorname{Gal}(L / K)\coloneqq \operatorname{Aut}(L / K)  \) 
\end{definition}

\begin{theorem}
    \(\mathbb{Q}(\zeta_n)/\mathbb{Q}\) is Galois and

    \[
        (\mathbb{Z} / n)^\times \simeq \operatorname{Gal}(\mathbb{Q} (\zeta_n) / \mathbb{Q}) 
    \]

    Where \(k\) gets mapped to the map \((\zeta_n \mapsto \zeta_n^k)\) 

\end{theorem}

\begin{proof}
    \(\Phi_n\) is minimal polynomial. For \(\zeta _n\) and \(\zeta _n^k\) we have the following diagram [draw commutative digram] 
\end{proof}

\begin{definition}
    An extension \(L / K\)  is \underline{Abelian} if it is Galois and it's Galois Group \(G = \operatorname{Gal}(L / K) \) is abelian.
\end{definition}

\begin{definition}
    An extension \(L / K\) is simple if \(L=K[\alpha]\) for \(\alpha \in L\). Then, \(\alpha \) is called a primitive element. 
\end{definition}

We review the Fundamental Theorem of Galois Theory:

\begin{theorem}[Fundamental Theorem of Galois Theory]
    Let \(K / F\) be Galois. Then there exists a bijection:

    Intermediate extension \(K / E / F\) \(\leftrightarrow\) subgroups \(I - H - G\)  

    Given by:

    \(\text{Subgroup } H \mapsto \text{Extension }  K^H = \{ k : h(k) = h \forall h\in H \} \) 

    \(\text{Extension } E \mapsto \text{Subgroup } G_E = \{ g\in G : \forall e\in E, g(e)=e \}  \) 

\end{theorem}

We have a Corollary: \(K / E\) is also Galois and \(G_E = \operatorname{Gal}(K / E) \) 

This is useful in HW P3.

Another Corollary: If \(\operatorname{Gal}(K / E) \) is normal in \(\operatorname{Gal}(K / F) \) then \(E / F\) is Galois and we have \(\operatorname{Gal}(E / F) = \frac{\operatorname{Gal}(K / F)}{\operatorname{Gal}(K / E) }\) 

This is useful in HW P5.

Another Corollary: If \(L / K\) is Galois and simple [\(L = K[\alpha]\) ]

Then the minimal polynomial of \(\alpha\) is given by: \(\displaystyle \prod_{g\in \operatorname{Gal}(L / K) } (x - g(\alpha)) \)

Done.

\hfil
\hrule

Class 08: 01/26

\(\boxed{\text{Read Chapter 1 of Atiyah MacDonald (AM)}}\) 

\underline{Chapter 1: Rings and Ideals}

\begin{definition}

    Ring [Commutative Ring]
    
    Subring
    
    Ideal \(I \triangleleft A\), \((I,+)\) group, \(AI=I\) 
    
    Quotient ring \(A \to A/I\) 
    
    Homomorphism \(f:A\to B\) 
    
    \(\ker f \triangleleft A\) 
    
    \(\overline{f} : A / \ker f \overset{\simeq}{\longrightarrow} f(A)\)
    
    zero divisor \(\exists y \neq 0\) such that \(xy=0\) 

    domain

    PID

    units

    \(A^\times\) 
    
\end{definition}


\begin{proposition}
    [AM 1.1]

    If \(I \triangleleft A\) then there exists a bijection

    \[
        \{ J \triangleleft A \vert I \subset J \} \leftrightarrow \{ \overline{J} \triangleleft A / I \} 
    \]

    Map is \(J \mapsto q(J)\) the quotient map, and \(q^{-1}(J)\) in the other direction.

    Proof is in AM.

    Look at \(\mathbb{Z} \to \mathbb{Z} / 6 \mathbb{Z}\) 

    \(\mathbb{Z} \) has ideals \((2),(3),(6),(1)\) that contain \((6)\) 

    \(\mathbb{Z} / 6 \mathbb{Z} \) has ideals \((\overline{3} ), (\overline{2} ),(\overline{0}=\overline{6} ), (\overline{1} )\) 

\end{proposition}

\begin{definition}
    \underline{Field} is a ring such that \(1\neq 0\) and \(A^\times = A \setminus \{ 0 \} \) 
\end{definition}

\begin{proposition}
    [AM 1.2]

    The Following Are Equivalent (TFAE):

    \begin{enumerate}
        \item \(A\) is a field
        \item \(I\triangleleft A \implies I=0\) or \(A\) 
        \item \(f: A\to B \neq 0\) implies \(f\) is injective
    \end{enumerate}

\end{proposition}

\begin{proof}
    \(1 \implies 2\) : We use contradiction. Suppose \(0 \neq I \triangleleft A\) and let \(0 \neq x \in I\).

    So, \(A = (1) = (x) \subset I \subset A\) meaning \(I = A\) 

    \(2 \implies 3\) : \(f(1) = 1 \neq 0\) so \(1 \neq \ker f\) thus \(\ker f \neq A \implies \ker f=0\) so \(A \) is injective.

    \(3 \implies 1\) : Suppose \(x \in A - A^\times\) then the quotient map \(q : A \to A/(x)\) is injective which implies \(x = 0\) 
    
    Note that \(A / (x) = 0 \iff A = (x) \iff x\in A^\times\) 

\end{proof}

\section*{Prime and Maximal Ideals}

Atiyah MacDonald Subsection

\begin{definition}
    \(I \triangleleft A\) is a prime ideal if \(P \neq A\) and if \(xy\in P\) then \(x\in P\) or \(y\in P\)  
\end{definition}

A principal ideal \(I=(a)\) is prime means \(xy\in I \implies a\mid xy \implies a\mid x\) or \(a\mid y\) [Euclid's Lemma] which means \(a\) is prime.

\begin{definition}
    \(I\triangleleft A\) is \underline{proper} if \(I \neq A\)  
\end{definition}

\begin{definition}
    [Maximal Ideal]

    \(I \triangleleft A\) is maximal if it is a maximal proper ideal, i.e \(I \neq A\) , if \(I \subset J\) and \(J\) is a proper ideal of \(A\) then \(I=J\) 

    For example, \((2),(3)\) are both maximal ideals of \(\mathbb{Z}\) 

\end{definition}

Very Useful Trivialities:

\[
    I \text{ prime } \iff A/I \text{ domain} 
\]

\[
    I \text{ maximal } \iff A/I \text{ field}  
\]

Since all domains are fields, maximal ideal implies prime ideal.

But not the other way around. In \(\mathbb{R} [x,y]\) we have \((x-1)\) is prime but \((x-1,y-1)\) is maximal.

\(A\) is a domain \(\iff \) \(0 \triangleleft A\) is a prime

\begin{theorem}
    [AM 1.3]

    Every \(A \neq 0\) has a maximal ideal. [Uses special ase of Zorn's Lemma]
\end{theorem}

Suppose \(\Sigma \) is the collection of subsets of the set \(X\) 

Then Zorn's Lemma says that:

If every chain in \(\Sigma \) has an upper bound in \(\Sigma\) then \(\Sigma\) has a maximal element.

Note: \(\mathcal{C} \subset \Sigma \) is a \underline{chain} if \(A,B\in \mathcal{C} \implies  A \subset B\) or \(B \subset A\) 

\(\mathcal{C} \subset \Sigma\) has an upper bound if \(\exists B\in \Sigma \) such that \(A\in \mathcal{C} \implies A \subset B\) 

\(M \in \Sigma \) is maximal if \(M\subset N\in \Sigma \implies M = N\) 

\begin{proof}
    Let \(\Sigma \) be the set of proper ideals of \(A\). Consider chain \(\mathcal{C} \subset \Sigma\) then the ideal \(I = \bigcup_{J\in \mathcal{C} }^{} J\) is a proper ideal of \(A\) 

    Note that unions of proper ideals is not always proper (or even ideal) but if we have a chain it is true.

    It is proper because \(1 \notin I\) 

    Thus, there always exists an upper bound and therefore by Zorn's Lemma we have a maximal ideal

\end{proof}

Corollary: if \(A\neq 0\) ring then there exists a surjection \(A \hookrightarrow \) field.

\begin{proof}
    \(A \mapsto A / M\) 
\end{proof}

Corollary 1.4: Any proper ideal is contained in a maximal ideal.

\begin{proof}
    We have quotient \(A \to A / I\) then consider the maximal ideal \(\overline{M} \) and then just take \(M = q^{-1}(\overline{M} )\) 

    Or Zorn's lemma: Let \(\Sigma _I = \{ J \text{ proper ideal of } A \mid I \subset J \} \) 

\end{proof}

\begin{definition}
    \(A\) is a \underline{local ring} if there exists a unique maximal ideal.
\end{definition}

It has `something' to do with local in the sense of topology.

A field as unique maximal ideal \((0)\) 

\(\mathbb{Z} / p^2 \mathbb{Z} \) has local ideal \((p)\) 

For prime \(p\) we have the local ring \(\mathbb{Z}\) localized at \(p\) given by

\(\mathbb{Z} _{(p)} = \{ \frac{r}{s}\in \mathbb{Q} : r,s\in \mathbb{Z} , p\nmid s \} \subset \mathbb{Q} \) 

The unique maximal ideal is \(M = \mathbb{Z} _{(p)}p\) 

In fact \(\mathbb{Z}_{(p)}p / (p) \simeq \mathbb{F} _p\) 

If \(A\) is a local ring then \(A / M\) is the residue field.

\hfil
\hrule

Class 09: 01/29

\begin{definition}
    \(A\) is a \underline{local ring} if it has a unique maximal ideal.
\end{definition}

Another example: you can localize a polynomial ring at a point: \(\mathbb{C} [x_1, \dots, x_n ]_{(p)}= \left\{ \frac{f}{g} : f,g\in\mathbb{C} [x_1, \dots, x_n ],g(p) \neq 0 \right\} \) where \(p=(p_1, \dots, p_n )\) a point. 

\begin{proposition}
    Suppose \(M \triangleleft A\) and proper and \(A - M \subset A^\times \) then \(A\) is local and \(M\) is maximal. 
\end{proposition}

\begin{proof}
    We want to show that \(M\) is the unique maximal ideal. Let \(I\) be a maximal ideal. We want to show that \(I = M\) 

    By hypothesis, \(A - M \subset A^\times \subset A - I\) 

    Note that \(A^\times \subset A - I\) is true for any proper ideal \(I\) 

    This implies \(I \subset M\) which implies \(I = M\) 

\end{proof}

\(\mathbb{Z} / p^k \mathbb{Z} \) is maximal with \((p)\) as the maximal ideal by the above proposition [\(p\) prime].

\begin{proposition}
    maximal \(M \triangleleft A\) such that \(1 + M \subset A^\times \) then \(A\) is local.
\end{proposition}

\begin{proof}
    Let \(x \in A - M\), we will show \(x\) is a unit.

    Then \((x,M) = A\) 

    \(\implies ax+m = 1\) for some \(a \in A, m \in M\) 

    \(\implies ax = 1 - m \in 1 + M \subset A^\times \implies x\in A^\times\) 

    Therefore, \(A - M \subset A^\times \) and thus by previous proposition \(A\) is local.

\end{proof}

Exercise: \(A\) local ring \(\iff A - A^\times \triangleleft A\) 

\begin{definition}
    If \(A\) is a local ring with maximal ideal \(M\) then we have an associated canoncial field \(A / M\) called the \underline{residue field}.
\end{definition}

\section*{Nilradical and Jacobson Radical}

\begin{definition}
    Let \(A\) be a ring, \(x \in A\) is nilpotent if \(x ^ n = 0\) for some \(n \in \mathbb{Z} _{>0}\) 
\end{definition}

Examples: \(\overline{p} \in \mathbb{Z} / p^k \mathbb{Z} , \overline{x} \in \mathbb{R} [x] / (x^3), \begin{pmatrix}
    0 & \ast &  \ast \\
    0 & 0 &  \ast \\
    0 & 0 &  0 \\
\end{pmatrix}\) 

Note that the last one is not commutative.

\underline{Trick:} \(x\) is nilpotent \(\implies 1\pm x \in A^\times\)

This is because \(\frac{1}{1-x}=1+x+x^2 + \cdots\)

Which is a finite sum if \(x\) is nilpotent. Similar for \(\frac{1}{1+x}\) 

\begin{definition}
    The \underline{nilradical} of a ring \(A\), \(\operatorname{Nil} A = \mathfrak{N}(A) = \{ x \in A : x \text{ is nilpotent}  \} \) 
\end{definition}

This looks like a set [because it is] but it is actually an ideal!

\begin{proposition}
    If \(A\) is a ring then \(\mathfrak{N}(A) \triangleleft A \) and \(\mathfrak{N}(A / \mathfrak{N}(A) ) = 0 \)  
\end{proposition}

\begin{proof}
    \(x^n = 0 \implies (ax)^n = 0\)
    
    \(x^m = 0, y^n=0\)
    
    \(\implies (x+y)^{m+n} = \displaystyle \sum_{j=0}^{m + n} \binom{m + n}{j} x^j y^{m+n-j} = 0\)
    
    Thus \(\mathfrak{N}(A) \) is an ideal.
    
    Now, suppose \(\overline{X}\in A / \mathfrak{N}(A)\). We prove that it is multipotent by contradiction.
    
    Suppose \(\overline{x}^n = 0 \)
    
    Then \(x^n \in \mathfrak{N}(A) \)
    
    Thus \((x^n)^k = 0\) for some \(k\) 

    Thus \(x^{nk} = 0 \)
    
    Thus \(x\) is nilpotent. So we have a contradiction.
    
\end{proof}

\begin{definition}
    Ring \(A\) is reduced \(\iff \mathfrak{N}(A)=0 \) 
\end{definition}

We can reduce any ring by quotienting out the nilradical.

Motivation: In Algebraic Geometry, if \(V\) is a variety, then \(k[x_1, \dots, x_n ] / I(V)\) is always reduced. Having nilpotents is problematic if we want to do algebraic geometry. 

\begin{proposition}
    [1.8] \(\displaystyle \mathfrak{N}(A) = \bigcap_{\text{prime ideal } P\triangleleft A}^{} P \) 
\end{proposition}

Useful in homework.

\begin{proof}
    We want to show that the sets contain each other.

    Suppose \(x\in \mathfrak{N}(A) \implies x^n =0\)
    
    Let \(P \triangleleft A\) be prime.

    Thus \(0\in P\implies x^n\in P\implies x\in P\) 

    For other direction, we do contrapositive.

    Suppose \(x\notin \mathfrak{N}(A) \)
    
    Zorn's Lemma time! We construct a prime ideal \(x\) is not inside of.

    Let \(\Sigma = \{ I \triangleleft A : \overline{x} \notin \mathfrak{N}(A / I)  \} = \{ I \triangleleft A : x^n \notin I \forall n \}  \) 

    Note: \((0) \in \Sigma\) 

    Any chain in \(\Sigma \) has an upper bound in \(\Sigma\) and thus there exists a maximal element \(P \in \Sigma\) 

    Claim: \(x\notin P\) and \(P\) is prime.

    \(x\notin P\) since \(P\in \Sigma\) 

    To prove \(P\) is prime, we prove \(a,b\notin P\implies ab\notin P\) 

    Assume \(a,b\notin P\) 

    Then \(P+(a),P+(b)\notin \Sigma \) 

    \(\implies x^m \in P + (a), x^n \in P + (b)\) 

    \(\implies x^{m+n} \in P + (ab) \) 

    \(\implies P + (ab) \notin \Sigma\) and \(P \in \Sigma \) 

    \(\implies ab\notin P\) 

\end{proof}

\begin{definition}
    [Jacobson Radical]

    Since nilradical is the intersection of prime ideals, Jacobson Radical is the intersection of maximal ideals.

    \[
        \mathfrak{J}(A) = \bigcap_{M\text{ maximal ideal of }A}^{} M  
    \]

\end{definition}

Note that \(\mathfrak{N}(A) \subset \mathfrak{J}(A)  \)  since maximal ideals are prime.

Example where nilradical and jacobson radical are different:

Let \(\mathbb{Z} _{(2)} = \frac{p}{q}\) where \(q\) odd.

Then \(\mathfrak{N}(\mathbb{Z} _{(2)}) = 0\) but \(\mathfrak{J}(\mathbb{Z} _{(2)}) =2(\mathbb{Z})\) 

\hfil
\hrule

Class 10: 01/31

\begin{proposition}
    [AM 1.9] The Jacobson Radical \(\mathfrak{J} (A)=\{ x\in A : 1 - x A \subset A^\times \} \) 
\end{proposition}

\begin{proof}
    We show it by the two sided inclusion, and we show that by the contrapositive.

    Suppose \(1-xa \notin A^\times\) 

    Then there exists maximal ideal \(M\) so that \(1 - x a \in M\) 

    \(\implies 1\in xa + M\) 

    \(\implies xa \notin M\)
    
    \(\implies x\notin M\) 

    \(\implies x\notin \mathfrak{J} (A)\) 

    Now suppose \(x\notin \mathfrak{J} (A)\) 

    \(\implies x\notin M\) for some maximal ideal \(M\) 

    \(\implies (x,M) =A\)
    
    \(\implies 1 = xa+m\) for some \(a\in A,m\in M\) 

    \(\implies 1-xa\in M\)

    Since \(M\) is a proper ideal, \(1-xa\notin A^\times\)

\end{proof}

\section*{Operations on Ideals}

Suppose we have ideals \(I,J\) in \(A\) 

Then \(IJ \subset I \cap J \subset I + J = (I,J)\) are all ideals but \(I \cup J\) is not an ideal.

\(I+J = \{ i + j : i\in I, j\in J \} \) it is an abelian group, and absorbs multiplication so it is an ideal. It is \((I,J)\) so it is an ideal.

Note that \(IJ \neq \{ ij : i\in I, j\in J \} \) 

Instead, \(IJ = \{ \sum_{k=1}^n i_k j_k : i_k j_k : i_k \in I, j_k\in J \} \) [finite sum]

Canonical Example: Suppose \(A = \mathbb{Z}\) 

Then \((a)(b)=(ab)\) 

But \((a)\cap (b)=(\operatorname{lcm}(a,b) )\) 

And \((a)+(b)=(\gcd(a,b))\) 

This is true for PIDs, potentially UFDs?

\section*{Direct Product of Rings}

If we have rings \(A_1, A_2,\cdots, A_n\) then we can define \(A_1 \times A_2 \times \cdots \times A_n = \prod A_i\) 

Given two rings \(A_1\) and \(A_2\) we can define a ring structure on their cartesian product \(A_1 \times A_2 = \{ (a_1,a_2) : a_1\in A_1, a_2\in A_2 \} \) where sum and product are done componentwise. Then \(A_1 \times 0\) and \(0\times A_2\) are zero divisors and \(1_{A_1\times A_2}=(1,1)\).

An element \(e\in A\) is an idempotent if \(e^2 = e\). Examples: \(0,1\) are trivial idempotents.

\(A \cong A_1 \times A_2\) with \(A_j \neq 0\) \(\iff \) there exists a non-trivial idempotent in \(A\).

Suppose \(e\) is a non-trivial idempotent. We want \(e=(1,0)\) then \(1 = e + (1-e)\) so we have \(A_1 = Ae\) and \(A_2 = A(1-e)\).

\(A_1 \times A_2\) is product of \(A_1\) and \(A_2\) in the category Ring and CRing

[insert \(R\) to \(A_1\times A_2\), \(A_1\) and \(A_2\) category picture]

(maps into products are easy)

\begin{definition}
    Ideals \(I,J\) are co-prime if \(I+J=A\) i.e. \(1 = i + j\) for some \(i\in I\) and \(j\in J\) 
\end{definition}

\begin{proposition}
    [AM 1.10] Let \(I_1, \dots, I_n \) be ideals of \(A\) and let \(\phi: A \to \prod_{i=1}^{n} (A / I_i)\) then \(\phi(a)=(a+I_1,\cdots,a+I_n)\) 
    
    \begin{enumerate}
        \item If \(\{ I_j \} \) are pairwise co-prime then \(\prod I_j = \cap I_j\)
        \item \(\phi\) is surjective if and only if the ideals are pairwise co-prime
        \item \(\phi\) is injective if and only if \(\cap_j I_j = 0\)  
    \end{enumerate}

\end{proposition}

We only prove for \(n=2\), induction is automatic.

\begin{proof}
    1: \(I_1 I_2 \subset I_1 \cap I_2\) always true

    Conversely, if \(1 = i_1 + i_2\), for \(a\in I_1\cap I_2\) we see that \(a = a\cdot 1 = a\cdot i_1 + a\cdot i_2\in I_1 I_2\) 

    2: \(\implies \) suppose \(\phi\) is surjective. Then, there exists \(x\) such that \(\phi(x)=(1,0)\) and thus \(1 = 1 - x + x\) where \(1-x\in I_1,x\in I_2\), since \([x]=0\) in \(A / I_2\) and \([x]=1\) in \(A / I_1\) implies \([1 - x]=0\) in \(A / I_1\) so \(I_1,I_2\) are co-prime

    \(\impliedby\) Since co-prime we can find \(1 = i_1 + i_2 \) so \(\phi(1-i_1 )=(1,0),\phi(1-i_2)=(0,1)\) so we have that \(\phi\) is onto.

    3. Suppose \(\phi\) is injective. Then \(\ker\phi = 0\). But for any \(I_1,I_2\) we have \(\ker\phi = I_1\cap I_2\). So injectivity is equivalent to \(I_1\cap I_2 = 0\) 

\end{proof}

\begin{proposition}
    [AM 1.11] Let \(P_1, \dots, P_n \) be prime ideals and let \(I \subset \cup P_j\) then \(I \subset P_j\) for some \(j\).

    If prime ideal \(P \supset \cap P_j\) then \(P \supset I_j\) for some \(j\)  
\end{proposition}

\begin{proof}
    We just do the \(n=2\) case. We do contrapositive.

    Suppose \(I\not\subset P_1,P_2\). We want to prove that \(I\not\subset P_1\cup P_2\) 

    Now, \(P_1\) and \(P_2\) are prime.

    Then \(\exists x_1,x_2\in I\) so that \(x_1\notin P_2,x_2\notin P_1\). 

    Case 1: \(x_1\notin P_1\) or \(x_2\notin P_2\) in this case we're done since either \(x_1\notin P_1\cup P_2\) or \(x_2\notin P_1\cup P_2\) 

    Case 2: \(x_1\in P_1,x_2\in P_2\). In this case, \(x_1 + x_2 \in I\) but \(x_1 + x_2 \notin P_1\cup P_2\) which provides a contradiction.

    Thus we're done with part 1.

\end{proof}

\hfil
\hrule

Class 11: 02/02

Ideal \(P\) is prime \(\iff \) compelement is multiplicatively closed

\(P\) is prime \(\iff \) (\(ab\in P \implies a\in P\) or \(b\in P\)) \(\iff \) \((a \notin P,b\notin P \implies ab \notin P)\) 

\begin{proposition}
    [AM 1.11:]

    1. Suppose we have prime ideals \(P_1, \dots, P_n \) and an ideal \(I\). Then, \(I \subset \cup P_i \implies I \subset P_i\) for some \(i\) 

    2. Suppose we have ideals \(I_1, \dots, I_n \) and prime ideal \(P\).
    
    2a. If \(\cap I_j \subset P\) then \(I_j \subset P\) for some \(j\) 

    2b. If \(\cap I_j = P\) then \(I_j = P\) for some \(j\)  

\end{proposition}

\begin{proof}
    1: We prove the contrapositive.

    \(I \not \subset P_i\) for all \(i\) \(\implies I \not \subset \cup P_i\)
    
    We induct. For \(n=1\) it's trivial.

    Assume true for \(n-1\) 

    Then we have \(x_i\in I, x_i\notin \cup_{j \neq i}P_j\) 

    Suppose we have some \(i\) so that \(x_i \notin P_i\) 

    Then \(x_i\notin \cup P_i\)  so \(I\not \subset \cup P_i\)
    
    For the other case, for all \(i\) we have \(x_i\in P_i\) 

    Then, product of \(x_j\) without \(i\) is in \(P_j\) for all \(j \neq i\) and it's not in \(P_i\) since \(P_i\) is prime.

    Let \(y\) be the sum of these. Then \(y\) is not in any of the \(P_i\) so we're done.

    2a: Contrapositive.

    \(\forall i\) , \(I_i \not \subset P \implies \cup I_i \not \subset P\) 

    \(\forall i, \exists x_i\in I_i - P\)
    
    \(y = \prod x_i \in \prod I_i \subset \cap I_i\) 

    \(y\notin P\) since \(P\) is prime.

    2b. \(P = \cap I_i\) 

    So there exists \(i\) so that \(I_i \subset P \subset \cap I_i \subset I_i\)     

\end{proof}

\section*{Radicals of Ideals}

If \(I \triangleleft A\) we define radical \(\sqrt{I} = \{ x : x^n \in I \} \triangleleft A\) 

In integers, \(\sqrt{(p_1 ^{e_1}\dots P_r^{e_r})} = (p_1 \dots p_r)\) 

And of course \(I \subset \sqrt{I} \) 

So the nilradical \(\mathfrak{N} (A)= \sqrt{0} \) 

So, \(\sqrt{I} = q^{-1}(\mathfrak{N}(A) / I )\) whee \(q: A \to A / I\) 

\begin{proposition}
    [AM 1.14] \(\sqrt{I} = \cap_{I \subset P\text{ prime} }P\) 
\end{proposition}

\begin{proof}
    \(\sqrt{I} = q^{-1}(\mathfrak{N}(A) / I ) = q^{-1}(\cap_{\text{prime} \overline{P} \triangleleft A / I }) = \cap q^{-1} \overline{P} = \cap_{I \subset P}P \) 
\end{proof}

Exercise 1.15:

\begin{enumerate}
    \item \(I \subset \sqrt{I} \) 
    \item \(\sqrt{\sqrt{I} } =\sqrt{I} \) 
    \item \(\sqrt{IJ} = \sqrt{I\cap J} = \sqrt{I} \cap \sqrt{J} \) 
    \item \(\sqrt{I}= A \iff I = A \)
    \item \(\sqrt{I + J} = \sqrt{\sqrt{I} + \sqrt{J} } \) 
    \item \(\sqrt{P^n} = P \) when \(P\) is prime. 
\end{enumerate}

An ideal \(I\) is radical if \(\sqrt{I} = I\) 

So \(\sqrt : \) ideals \(\to \) radical ideals

\(\sqrt{I} \) is the smallest radical ideal containing \(I\) 

The motivation for studying radicals come from algebraic geometry.

Recall the correspondence between varieties in \(k^n\) and ideals in \(k[x_1, \dots, x_n ]\).

So we have \(V(J)=V(\sqrt{J})\) 

Because \(x^2 = 0 \iff x=0 \) 

Another motivation is the \underline{Nullstellensatz}.

Suppose \(k\) is algebraically closed. If we take an ideal \(J\) and take its variety \(V(J)\) then \(I(V(J))=\sqrt{J} \)  where \(I\) is the map from varities to ideals.

\underline{Ideal Quotient}

Suppose we have ideals \(I,J \triangleleft A\) 

Then \((I : J)\) is supposed to be like \( I / J\) 

\((I : J) = \{ x \in A : xJ \subset I \} \) This is an ideal because it is closed under addition and multiplication by members of \(A\) 

From wikipedia: \(KJ \subset I \iff K \subset (I : J)\) 

Also from wikipedia: \(I ( V - W) = (I(V):I(W))\) when varities \(V,W\subset k^n\) 

\((0:J)=\operatorname{Ann }(J) \) is the annihilatoz of \(J\) 

\(\operatorname{Ann}(\mathbb{R} \times 0) = (0\times \mathbb{R} ) \) 

In AM, \(\{ \text{zero divisors} \} = \cup_{x \neq 0} \operatorname{Ann}(x)  \) 

Useful in HW.

For a set \(E \subset A\) we can talk about its radical \(\sqrt{E} = P\{ x : x^n \in E \} \) for some \(n\)

Then \(\sqrt{\cup E_\alpha} = \cup \sqrt{E_{\alpha}}  \) 

\begin{proposition}
    [AM 1.15]

    \(\{ \text{zero divisors}  \} = \cap_{x \neq 0}\sqrt{\text{Ann}(x)} \) 

    \(\{ \text{zero divisors}  \} = \sqrt{\{\text{ zero divisor }\} } = \cup_{x \neq 0} \sqrt{\text{Ann} (x)} \) 

\end{proposition}

\begin{proposition}
    [AM 1.16] If \(\sqrt{I} +\sqrt{J} = A\) then \(I + J = A\)
    
    [radicals co-prime means ideals co-prime]

    We use 1.15(v).

    \(\sqrt{I + J} = \sqrt{\sqrt{I} + \sqrt{J}} = \sqrt{A} = A  \)
    
    1.15(iv) tells us \(I + J = A\) 

    Next, extension and contraction.

    Let \(f : A \to B\) 

    If \(J \triangleleft B, J^c  f^{-1}J \triangleleft A \) contraction
    
    If \(I \triangleleft A, I^e = (f(I))\) extension

\end{proposition}

\hfil
\hrule

Class 12: 02/05

\section*{Extension and Contraction}

Consider a ring homomorphism \(f:A\to B\)

Then, ideals \(I\triangleleft A\) and ideals \(J\triangleleft B\) has a correspondence:

\(I \overset{e}{\to} J\) 

\(J \overset{c}{\to} I\) 

These are extension and contraction.

\(C =\) image \(c=\{ J^c : J\triangleleft B \} \) , \(E=\) image \(e\) 

\(J^c = f^{-1}(J)\) 

\(I^e = (f(I)) = Bf(I) = \{ \sum_{i} b_i f(a_i) | a_i\in I \} \) 

\(J\) prime implies \(J^c\) prime

Thus we have \(\operatorname{Spec}(f) : \operatorname{Spec}(B) \to \operatorname{Spec}(A) \) 

Note that \(I^e \neq f(I)\) in general.

Consider \(f:\mathbb{Z} \curvearrowleft \mathbb{Q}\) 

Then \((2\mathbb{Z})^e = \mathbb{Q}\mathbb{Z}=\mathbb{Q}\) which is not \(f(2\mathbb{Z})\)  

If \(f\) is onto / surjective,

\(I^e = f(I)\) 

[insert category theory image of \(A \overset{f}{\to} B\) and \(f(A)\) here]

Extension and Contraction are also called going up and going down.

Consider \(f:\mathbb{Z} \to\mathbb{Z} [i]\) 

Consider odd prime \(p\in\mathbb{Z}\) 

Then \(\frac{\mathbb{Z}[i]}{(p)}=\frac{\mathbb{Z}[x]}{(p,x^2 + 1)}=\frac{\mathbb{F}_p[x]}{(x^2 + 1)}=\begin{dcases}
    \mathbb{F} _{p^2}, &\text{ if } p\equiv 3\pmod 4 ;\\
    \mathbb{F} _p\times\mathbb{F} _p, &\text{ if } p\equiv 1\pmod 4 ;\\
\end{dcases}\) 

\((2)^e = ((1+i)^2)\) in this case it is called \underline{ramified}

\(p\equiv 3: (p)^e\) is prime and in this case it is called \underline{inert}

\(p\equiv 1: (p)^e = Q_1 Q_2\) product of distinct primes, here it is called \underline{split}

There are about 15 properties of extension and contraction.

\begin{proposition}
    [AM 1.17]

    \begin{enumerate}
        \item \(I \subset I^{ec} , J^{ce} \subset J\) 
        \item \(I^e = I^{ece} , J^{cec} = J^c\) 
        \item \(C \overset{e,c}{\longleftrightarrow} E\) is a bijection, \(C=\{ I:I=I^{ec} \}, E=\{ J:J=J^{ce} \}  \)  
    \end{enumerate}

\end{proposition}

\begin{proof}
    Exercise. Use map of sets: \(g:C\to D\). If \(Y\) is a subset, \(g(g^{-1}(Y))\subset Y\) with equality iff \(Y\subset g(C)\) but \(X\subset g^{-1} (g(X))\). Also, \(I^e = Bf(I)\) and \(J^c = f^{-1}(J)\) 
\end{proof}

\begin{proposition}
    [AM 1.18]

    \begin{enumerate}
        \item \((I_1 + I_2)^e = I_1^e + I_2^e,(I_1 I_2)^e = I_1^e I_2^e\)
        \item \((J_1 + J_2)^c \supset J_1 ^ c + J_2 ^ c,(J_1 J_2)^c \supset J_1^c J_2^c\)
        \item \((I_1 \cap I_2)^e \subset I_1 ^ e \cap I_2^e\) 
        \item \((J_1 \cap J_2)^c = J_1^c \cap J_2^c\) 
        \item \((\sqrt{I} )^e \subset \sqrt{I_e^e}\)
        \item \((\sqrt{J})^c = \sqrt{J^c} \)
        \item \((I_1 : I_2)^e \subset (I_1^e : I_2^e)\)
        \item \((J_1 : J_2)^c \subset (J_1^c : J_2^c)\) 
        \item \(e\) is closed under sum and product, and \(c\) is closed under the other three operations (??? not sure what this means) 
    \end{enumerate}

\end{proposition}

\underline{We're Moving on to Chapter 2 of Atiyah MacDonald!!!}

\section*{Modules}

[The M word]

\begin{definition}
    Let \(A\) be a ring. Then a \(A\)-module is a abelian group \(M\) and a function \(A\times M\to M\) given by \((a,m)\mapsto am\) such that \(a(x+y)=ax+ay,(a+a^{\prime} )x=ax+a^{\prime} x, (a a^{\prime})x = a(a^{\prime} x),1x=x\) for all \(a,a^{\prime} \in A,x,y\in M\) 
\end{definition}

The concept of modules generalize both vector spaces and ideals.

Atiyah Macdonald makes an observation that is obvious but not really:

\(M\) is a \(A\)-Module if and only if \(M\) is a representation of \(A\) 

\begin{proof}
    We need a ring homomorphism \(A\to End(M)\) where \(End(M)=Hom(M,M)\), the endomorphism ring of \(M\)

    Here we have addition, multiplication(composition) and multiplication by \(a\) is a homomorphism of \(M\) 

\end{proof}

Examples:

\(I\triangleleft A\) is an \(A\)-module

\(\mathbb{Z}-\)module is an abelian group

\(k\)-module \(\iff\)  vector space over \(k\) 

\(k[x]\)-module \(\iff\) vector space over \(k\) with \(T:V\to V\)

Here \(M=V,xv\coloneqq T(v)\) 

\(k[x,x ^{-1}]\)-module \(\iff\) vector space over \(k\), \(T:V\overset{\approx}{ \to } V\) 

\(kG\)-module \(\iff k-\)represetation of \(G\)

Modules form a Category.

There is a Category. \(Mod A\) is the category of modules of \(A\). The objects are \(A\)-modules. The morphisms [the ones we care about] is an \(A\)-module map, or an \(A\)-map

\(f:M\to N\) is an \(A\)-module map if it respects the structure: \(f(x+y)=f(x)+f(y),f(ax)=af(x)\) 

We need a bit more for categories: composition \((f\circ g)\)  and identity \(Id_M\) 

Compositions of \(A\)-module maps are \(A\)-module maps.

\(\operatorname{Mod} A \) is a category enriched in \(\operatorname{Mod} A \) [wtf???]

\(Hom_A(M,N)\) [often just called \(Hom(M,N)\)] is the set of maps from \(M\) to \(N\)

This is the set of morphisms \(\operatorname{Mor}_{\operatorname{MOd}(A)}(M,N)  \) 

If \(f,g:M\to N\) then \(f+g:M\to N\) and \((af)(m)=af(m)\) 

So \(Hom(M,N)\) is an \(A\)-module.

\(Hom(M,N)\times Hom(N,P)\to Hom(M,P)\) is \(A\)-bilinear, given by \((f,g)\mapsto g\circ f\) 

Related fact: Given \(f:M\to M^{\prime}\)  We can define an \(A\)-module map \(Hom(f,Id):Hom(M^{\prime} ,N)\to Hom(M,N)\) given by \(h \mapsto h\circ f\) 

Similarly, given \(N\to N^{\prime}\) we have a map \(Hom(M,N)\to Hom(M,N^{\prime})\) given by \(h\to g\circ h\) 

We have \(Hom(A,M)=M\) given by \(\phi \mapsto \phi(1)\) 

\(Hom(M,A)=M^{\star} \) the dual of \(M\)

\hfil
\hrule

Class 13: 02/07

\section*{Submodules and Quotient Modules}

\begin{definition}
    [Submodule]

    A submodule \(M^{\prime} \) of an  \(A\)-module \(M\) if it is a subgroup of \(M\) such that \(am^{\prime} \in M^{\prime} \) for all \(a\in A,m^{\prime} \in M^{\prime} \) 
\end{definition}

We'll use notation \(M^{\prime} \triangleleft M\) 

\begin{definition}
    [Quotient Module]

    If \(M^{\prime} \) is a submodule of \(M\) quotient module is the module of cosets/equivalence classes \(m+M^{\prime}=[m]\) where \(m \sim \hat{m} \) if \(m - \hat{m} \in M^{\prime} \) 
\end{definition}

We have the quotient map \(q : M \to M / M^{\prime} \) 

Now we talk about Kernels and Cokernels.

Let \(f : M \to N\) be a module map.

Then \(\ker f \triangleleft M\)

\(\operatorname{im}f \triangleleft N \)

\(\operatorname{cok}f = N / \operatorname{im}f \leftarrow N\) 

\(\ker f = 0 \iff f\) 1-1 or injective

\(\operatorname{cok} f = 0 \iff f \) onto or surjective

\begin{theorem}
    [1st Isomorphism Theorem]

    If \(f:M \to N\) onto, then \(\overline{f}:M / \ker f \overset{\approx}{\longrightarrow} N \) is a well-defined isomorphism.
    
    \(\overline{f}([m])=f(m),\overline{f}(m+\ker f)=f(m) \) 
\end{theorem}

[draw commutative diagram of first isomorphism theorem]

\section*{Operations On Submodules}

Submodules are generalizations of ideals.

Suppose we have a module \(M\) and submodules \(\{ M_i \} _{i\in I}\) 

Then \(\sum_{i} M_i \triangleleft M\)

\(\bigcap_i M_i \triangleleft M \) 

\begin{proposition}
    [AM 2.1]

    Out of order

    \begin{enumerate}
        \item (third isomorphism theorem) \(N \triangleleft M \triangleleft L\), \(A\)-modules imply \((L / N) / (M / N) \cong L / M\)  
        \item (second isomorphism theorem) If \(M_1, M_2 \triangleleft M\) then \(M_2 / (M_1 \cap M_2) \cong (M_1 + M_2) / M_1\) 
    \end{enumerate}
\end{proposition}

\begin{proof}
    [1]

    Consider \(\theta : L / N \to L / M\) given by \(\theta (l + N) = l + M\). Since \(\theta \) is onto and kernel is \(M / N\) we have the theorem.
    
    [2]

    \(M_2 \to \frac{M_1 + M_2}{M_1}\) is onto, kernel is \(M_1 \cap M_2\) so we have the theorem.
\end{proof}

\begin{definition}
    Let ideal \(I \triangleleft A\) and subset \(\Sigma \subset M\) where \(M\) is a module.

    Define \(I \Sigma = \{ \sum_{i}^n a_i \sigma_i : a_i\in I, \sigma_i \in \Sigma \} \triangleleft M \) 
\end{definition}

\begin{definition}
    Let \(\Sigma \subset M\) where \(M\) is an \(A\)-module.

    Then \((\Sigma)= A \Sigma\) which is the submodule of \(M\) generated by \(\Sigma\) 

    Think of span.

    If \(M = (\Sigma)\) then \(M\) is \underline{generated} by \(\Sigma\) 

    A module \(M\) is finitely generated if we can write \(M = (\Sigma)\) for some finite subset \(\Sigma\) 
\end{definition}

\begin{definition}
    [module quotient]

    Let \(N,P \triangleleft M\).

    Then module quotient: \((N : P) = \{ a\in A : aP \subset N \} \triangleleft A\) 

\end{definition}

\begin{definition}
    [Annihilator]

    \(Ann(M)=(0:M)=\{ a\in A : aM = 0 \} \) is the Annihilator of \(M\).
\end{definition}

Example: Let \(A = \mathbb{Z}\) then \(Ann(\mathbb{Z} / 15 \times \mathbb{Z} /6) = \mathbb{Z} 30\) 

\(M\) is an \(A / Ann(M)\)-module

\textbf{Exercise 2.2:} \(Ann(M_1 + M_2)= Ann(M_1) \cap Ann(M_2)\) and \((N : P) = Ann\left( \frac{N+P}{N} \right) \)  

\section*{Category Theory!!!!}

\begin{definition}
    A category \(\mathcal{C}\) is:

    \begin{enumerate}
        \item A collection of objects \(Ob \mathcal{C}\) 
        \item \(\forall x,y\in Ob \mathcal{C}\), a collection of morphisms, \(\mathcal{C}(x,y)\) [Alternatie: \(Mpr_{\mathcal{C}} (x,y)\) ]
        \item \(\forall x,y,z\in Ob \mathcal{C}\) we have a map \(\mathcal{C} (x,y) \times \mathcal{C}(y,z) \to \mathcal{C} (x,z)\) given by \((g,f) \mapsto f\circ g\) [composition law]
        \item \(\forall x\in Ob x\) we have \(Id_X \in \mathcal{C}(x,x)\)
    \end{enumerate}

    So that \((f\circ g)\circ h = f \circ (g\circ h)\) 

    And \(Id_y \circ f = f = f\circ Id_X\) 

\end{definition}

We often write \(f : X \to Y\) or \(X \overset{f}{\to } Y\) for \(f\in C(X,Y)\) and it might not be a function because everything is abstract.

We write \(X\in \mathcal{C}\) for \(X\in Ob \mathcal{C}\) 

\begin{definition}
    [isomorphism]

    \(f: X \to Y\) is an \underline{isomorphism} if \(\exists g: Y \to X\) so that \(f\circ g = Id_Y\) and \(g\circ f = Id_X\) and say \(X\) and \(Y\) are \underline{isomorphic}, written \(X\cong Y\) 
\end{definition}

Example: Set, CRing (in 502, Ring), Rng [Rings possibly without identity], Top

If we have a group \(G\) we have a category \(BG\) so that \(Ob BG = \{ * \} \) and \(BG(*,*) = G\) so morphisms need not be maps.

\hfil
\hrule

Class 15: 02/09

\section*{Direct Sums and Products}

\(\{ M_i \}_{i\in I}\) is a family of \(A\)-modules.

We can define the direct product and direct sum.

\begin{definition}
    [Direct Produt]

    \(\prod_{i\in I} M_i\) is direct product, elements are \(i\)-tuples \((x_i)_{i\in I}\), and operations are done componentwise.
\end{definition}

\begin{definition}
    [Direct Sum]

    The direct sum is a submodule of direct product. \(bigoplus_{i\in I} \subset \prod_{i\in I}M_i\) containing \((X_i)_{i\in I}\) which vanish a.e.

    \(\#\{ i | X_i \neq 0 \}\) is finite.

\end{definition}

\(\prod_{i\in I}M_i, p_i : \prod M_i \to M_i\) is a product in the category of \(A\)-modules.

\(\forall ( M, \{ \phi_i : M \to M_i \}_{i\in I} ) \) there exists a unique map \(\phi : M \to \prod M_i\) so that \(p_i \circ \phi = \phi_i\)

[Insert category theory picture of \(M, \prod M_i\) and \(M_i\)]

Maps into products are easy. We can write \(\phi = \prod \phi_i\) and we can call it `universal property'.

Direct sum is a co-product and we reverse all the arrows in this case.

Suppose we have \((\bigoplus_{j\in I} M_j, i_j \to \bigoplus M_j)\) is a coproduct in the category of \(A\)-modules.

\(\forall (M, \{ \phi: M_j \to M \}_{j\in I})\) there exists a unique map \(\phi: \bigoplus M_j \to N\) so that \(\phi\circ i_j = \phi_i\) 

[insert reverse commutative diagram here]

Notation: \(\phi = \bigoplus \phi_j \) 

There is a map \(\bigoplus M_j \to \prod M_j \). It is an isomorphism if \(\vert J \vert < \infty\) 

Remark: \(A \cong I_i \oplus I_2\) if and only if \(A \cong A_1\times A_2\) 

\section*{Finitely Generated Modules}

\begin{definition}
    [Finitely Generated Modules]

    \(A\) is a ring and \(M\) is a \(A\)-module. If there is a finite subset \(\Sigma \subset M\) so that \(M = A \Sigma\) then \(M\) is finitely generated.

\end{definition}

\begin{definition}
    \(B \subset M\) is a basis for \(M\) if any \(m\in M\) can be expressed uniquely as a linear combination of elements of \(B\): \(m = a_1 b_1 + \dots + a_n b_n\) 
\end{definition}

\underline{Lemma}:

\(M\) has a basis \(iff M \cong \bigoplus_{i\in I}^{} A\) 

\begin{proof}
    Exercise
\end{proof}

\begin{definition}
    \(M\) is free if either side of previous lemma holds.
\end{definition}

Example: \(\prod_\infty \mathbb{R}\) is a free \(\mathbb{R}\)-module, but \(\prod_\infty \mathbb{Z}\) is not a free \(\mathbb{Z}\)-module 

\begin{proposition}
    [AM 2.3]

    \(M\) is finitely generated \(\iff\) \(\exists A^n \to M\) 
\end{proposition}

\begin{proof}
    Atiyah Macdonald
\end{proof}

\begin{proposition}
    [Nakayama's Lemma]

    Suppose \(M\) is finitely generated.
    
    Then \(J(R)M = M \iff M = 0\) 

    textbook version: Let \(M\) be finitely generated and \(I \triangleleft J(R)\) then \(IM = M \implies M = 0\) 
\end{proposition}

\underline{Applications:}

Assume Nakayama's Lemma.

\underline{Corollary AM 2.7:}

Suppose \(M\) is a finitely generated \(A\)-module and \(N\) is a submodule (\(N \triangleleft M\)).

\(M = J(A)M + N\) implies \(M = N\) 

\begin{proof}
    We apply Nakayama's lemma to the quotient.

    \(J(A)(M / N) = (J(A)M + N) / N = M / N \implies M / N = 0 \implies M = N\) 
\end{proof}

\begin{proposition}
    [AM 2.8]

    Let \(A\) be a local ring. Let \(J\) be the uniuqe maximal ideal of \(A\). Let \(k = A / J\). Suppose \(M\) is a finitely generated \(A\)-module. Let \(\{ x_1, \dots , x_n \} \subset M\) so that \(\overline{x}_1, \dots , \overline{x}_n\) is a \(k\)-basis for \(M / JM\) [which is a \(A / J\) module or a \(k\)-module]. [In other words, \(\overline{x}_1,\cdots,\overline{x}_n\) generate \(M / JM\) as \(A\)-module ] 
\end{proposition}

\begin{proof}
    Let \(N = A \{ x_1, \dots ,x_n \} \triangleleft M\)
    
    Then \(N \hookleftarrow M \to M / JM\) [hooked arrow means onto] composition is into.

    Thus \(N + JM = M \overset{27}{\implies} M = N \)
    
    
\end{proof}

\begin{proposition}
    [AM 2.4]

    Suppose \(M = A \{ x_1, \dots ,x_n \} \) is a finitely generated module. Let \(\phi: M \to M\) be an endomorphism and \(I \triangleleft A\) and \(\phi(M) \subset IM\).

    Then \(\phi^n + a_1 \phi^{n-1} + \dots +a_0 = 0\) with \(a_i \in I\). [Cayley Hamilton].
\end{proposition}

Recall that, for square matrix \(P\), there is adjugate matrix \(adj(P)\) so that \(P \cdot adj(P) = (\det P) I\) which is the transpose of matrix of cofactors.

This generalizes to commutative rings.

We want to come up with matrices \(\phi(x_i) = \sum a_{ij}x_i, a_{ij}\in I \) 

Then \(\sum_{j} (\delta_{ij}\phi - a_{ij})X_j = 0\). Let \(P\) be the matrix \((\delta_{ij}\phi - a_{ij})X_j = 0\) 

Then, \(P \underline{x} = 0 \implies (\det P)\underline{x} = adj(P)P(\underline{x}) = 0\) 

This means, for all \(X_i\) we have \((\det P)(x_i)=0\) so \(\det P\) letting it be an endomorphism on \(M\) must be \(0\).

Let \(P \in M_n A[t]\). Consier \(\det P\) and plug in \(t = \phi\).

\hfil
\hrule

Class 15: 02/12

\begin{proposition}
    [AM 2.4] Suppose \(M\) is a finitely generated \(A\)-module generated by \(u_1, \dots, u_n \) [we write it \(M = A \{ u_1, \dots, u_n  \} \) ]. Suppose \(I \triangleleft A\), there is \(\phi:M\to M\) so that \(\phi(M) \subset IM\). Then \(\exists a_1, \dots, a_n \in I\) sp that there exists equation \(\phi^n + a_1\phi^{n-1}+\cdots + a_{0I = 0}\).
\end{proposition}

\underline{Remark:} If \(I = A = k\) [a field] then this is Cayley-Hamilton Theorem. We let \(p(x)=\det(xI-\phi)\). Then \(p(\phi)=0\).

\begin{proposition}
    [AM Corollary 2.5] If \(M\) is a finitely generated \(A\)-module and \(IM = M\),

    \begin{enumerate}
        \item Then \(\exists x\equiv I \pmod I\) so that \(x\in Ann(M)\) 
        \item \(\exists i\in I\) so that \(\forall m\in M, im=m\).
    \end{enumerate}
\end{proposition}

\begin{proof}
    1: use 2.4. Let \(\phi = Id\) and let \(x = 1 + a_1 + \dots + a_n\). Then, \(xm = (Id^n + a_1 Id^{n-1}+ \dots + a_0)m = 0m = m\).

    2: Let \(i = x + 1\), then \(im = xm + m = m\)
\end{proof}

Recall the Jacobson Radical:

\[
    J(A) = \bigcap_{\text{maximal ideal}} I \overset{AM 1.9}{=} \{ j \in A : 1+jA \subset A^\times \} 
\]

We also have Nakayama's Lemma:

\begin{proposition}
    [Nakayama's Lemma]

    Let \(M\) be finitely generated \(A\)-module. Then,

    \[
        J(A)M = M \implies M = 0
    \]
\end{proposition}

Atiyah Macdonald gives two proofs, second proof is cooler. - Davis

\underline{First Proof:}

\(J(A)M = M\), by AM2.5 we have \(\exists x\equiv 1\pmod{J(A)}\) such that \(xM = 0\). Using 1.9, \(x\in A^\times, xM=0 \implies M = 0\).

\underline{Second Proof:}

Let \(M = A \{ u_1, \dots, u_n  \} \), \(n\) is minimal, \(J(A)M = M\), and \(M \neq 0\). We try to find a contradiction.

\(u_n\in M = J(A)M\)

Thus, \(u_n = j_1 u_1 + \dots + j_n u_n \) where \(j_i \in J(A)\).

Thus, \((1-j_n)u_n = j_1 u_1 + \dots + j_{n-1}u_{n-1}\) 

\(1 - j_n\) is a unit by 1.9, which means \(u_n\) can be written in terms of \(u_1, \dots , u_{n-1}\). This contradicts minimality of \(n\).

\section*{Exact Sequences}

\begin{definition}
    Let \(M,M^{\prime} ,M^{\prime\prime} \) be \(A\)-modules.

    \[
        M^{\prime} \overset{\alpha }{\to} M \overset{\beta}{\to} M^{\prime\prime} 
    \]

    is exact at \(M\) if \(im \alpha = \ker \beta\) 
\end{definition}

Note that \(\im \alpha \subset \ker \beta \iff \beta \circ \alpha = 0\) 

\begin{definition}
    Sequence of homomorphisms

    \[
        M_n \to M_{n-1} \to \dots \to M_0 
    \]

    is \underline{exact} if it is exact at \(M_{n-1}, \dots , M_2,M_1\) 
\end{definition}

Note, \(0 \to M \overset{\alpha }{\to} N\) is exact if and only if \(\alpha \) is injective. This is equivalent to saying \(\ker \alpha = 0 \iff  \alpha \) is 1-1 or injective.

\(M \overset{\beta}{\to} N \to 0\)  is exact if and only if \(\im \beta = N \iff  \beta\) is onto.

If \(M \hookrightarrow N\) then \(0 \to M \to N \to N / M \to 0\) is exact.

Memorize these.

\underline{Most important special case:}

\[
    0 \to M^{\prime} \overset{\alpha}{\to} M \overset{\beta}{\to}  M^{\prime\prime} \to 0
\]

is called a short exact seuqence.

This means \(\alpha \) is injective, \(\beta\) is surjective.

We have \(\overline{\beta} : M / \im M^{\prime} \overset{\cong}{\to} M^{\prime\prime}\).

So, \(M^{\prime} \cong \alpha (M^{\prime} ), M\cong M, M^{\prime\prime} \cong M / \alpha (M^{\prime})\) 

[insert commutative diagram about this here]

Example of short exact sequence:

\[
    0 \to M^{\prime} \to M^{\prime} \oplus M^{\prime\prime} \to M^{\prime\prime} \to 0
\]

Question: are all short exact sequences created this way?

Answer: No! example:

\[
    0 \to \mathbb{Z} / 2 \overset{\cdot 2}{\to} \mathbb{Z} / 4 \overset{q}{\to} \mathbb{Z} / 2 \to 0
\]

Multiplication by \(2\) sends \([m]\) to \([2m]\).

When is it created this way? If we have pseudo-inverse: If there exists \(M \overset{s}{\leftarrow} M^{\prime\prime}\) such that \(\beta \circ s = Id_M\).

\begin{proposition}
    Suppose \(A = k\), a field. Then,

    \[
        0 \to U \to V \to W \to 0
    \]

    implies \(\dim V = \dim U + \dim W\).
\end{proposition}

\begin{theorem}
    [Consequence of Third Isomorphism Theorem:] If we have \(L \triangleleft M \triangleleft N\) then,
    
    \(0 \to M / L \to N / L \to N / M \to 0\) is a short exact sequence.
\end{theorem}

\section*{Hom}

\(Hom(-,-)\) \(A\)-module \(\times A\)-module \(\to\) \(A\)-module

This is a bifunctior, and contravariant in the first variable. It is also left exact.

Suppose we have a short exact sequence \(0 \to M^{\prime}  \to M \to M^{\prime\prime} \to 0\). Then, for all \(N\),

\[
    0 \to Hom(N,M^{\prime}) \to Hom(N,M) \to Hom(N,M^{\prime\prime})
\]

is exact,

\[
    0 \to  Hom(M^{\prime\prime},N) \to Hom(M,N) \to Hom(M^{\prime} , N)
\]

Atiyah Macdonald has a fancier way to show this.

\begin{proposition}
    [AM 2.9:]

    \begin{enumerate}
        \item \(M^{\prime}  \to  M \to M^{\prime\prime} \to 0\) is exact if and only if \(\forall N\), we have \(0 \to Hom(M^{\prime\prime},N)\to Hom(M,N) \to Hom (M^{\prime} ,N)\) is exact.
    \end{enumerate}
\end{proposition}

Sample proof: i: \(M^{\prime} \overset{u}{\to} M \overset{v}{\to} M^{\prime\prime} \to 0\) is exact.

We want to show \(0 \to Hom (M^{\prime\prime}, N)\overset{v^{\star}}{\to}Hom(M,N)\overset{u^{\star}}{\to} Hom(M^{\prime} , N) \) 

Id \(v^{\star}\) injective? \(0 = v^{\star}(\phi) = \phi\circ v\), moreover \(\phi(M^{\prime\prime} )=\phi(b,V)\)  

\(\phi(M^{\prime\prime})=\phi(u(M))=0 \) so \(\phi= 0\) 

[insert commutative diagram here].

Since \(u^{\star} \phi = 0, \phi = (\overline{\phi} \circ \overline{v} ^{-1} )\circ v \in \im v^{\star}\) 

\hrulefill

Class 16: 02/14

Question: Are all short exact sequences same?

Answer: Yes and No.

Yes: all are \(0 \to M^{\prime} \hookrightarrow M \to M^{\prime} / M \to 0\) 

No: Not all are \(0 \to M \overset{i}{\to} M \oplus M^{\prime} \overset{p}{\to} M^{\prime} \to 0\) 

\begin{proposition}
    [AM 2.10, Snake Lemma]

    If white stuff is in short exact, then yellow is exact [picture]

    [just see Ivan notes]

    \(\partial m^{\prime\prime} = \left[ (v^{\prime} )^{-1} f u ^{-1} m ^{\prime\prime}  \right] \in cok f^{\prime} \) 

\end{proposition}

\section*{`Euler Characteristic'}

Let \(\mathcal{C} \) = a collection of \(A\)-modules

\(G\) = abelian group

Then \(\lambda : \mathcal{C} \to G\) is \underline{additive} if every short exct sequence

\[
    0 \to M^{\prime} \to M \to M^{\prime\prime} \to 0
\]

with modules in \(\mathcal{C}\) has \(\lambda (M) = \lambda (M^{\prime} )+\lambda (M^{\prime\prime} )\) 

eg \(A = \mathbb{Q}\) and \(\mathcal{C} =\) finite dimensional vector spaces over \(\mathbb{Q}\) and \(\lambda(M)=\dim_\mathbb{Q} (M)\) 

If \(\mathcal{C}\) = finitely generated abelian groups, \(A = \mathbb{Z}\) and \(\lambda (M) = rank(M)\) [rank is the dimension of the free part. In other words, it is \(\max{n : \exists \mathbb{Z}^n \hookrightarrow M}\). For all abelian groups, rank is a non-negative integer or infinity. for example, \(rank \mathbb{Q} = 1\)]. \(\mathcal{C} ^{\prime} =\) all abelian groups of finite rank.

Suppose \(\mathcal{C} \) is the collection of finite (abelian) groups and \(G = \mathbb{Q} ^\times \) and \(\lambda (M) = \vert M \vert \). This works, but \(+\) is actually the group operation of \(\mathbb{Z}\) 

\begin{proposition}
    [AM 2.11] If we have an exact sequence

    \[
        0 \to M_n \to M_{n-1} \to \cdots \to M_1 \to M_0 \to 0
    \]

    With \(M_i\in \mathcal{C}\) and \(K_i = \ker (M_i \to M_{i-1})\in \mathcal{C}\) 

    And \(\lambda \) additive,

    Then \(\sum_{i=1}^{n} (-1)^i \lambda (M_i) = 0\) 

\end{proposition}

Consequene: Consider finite abelian groups \(F_j\). Suppose:

\[
    0 \to F_n \to \dots \to F_1 \to F_0 \to 0
\]

Then, \(\displaystyle \prod_{i=0}^{n} \vert F_i \vert ^{(-1)^i} = 1 \text{ or } \prod_{i\text{ even}}\vert F_i \vert = \prod_{i\text{vodd}}\vert F_i \vert \) 

\begin{proof}
    For all homomorphism \(f: M \to N\) we have a short exact sequence \(0\to \ker f \to M \to im f \to 0\) 

    Thus, \(0 \to K_i \to M_i \to K_{i-1} \to 0 \) Thus,

    \[
        \sum_{i=0}^{n} (-1)^i \lambda (M_i) = \sum (-1)^i \lambda (K_i) + \sum (-1)^i \lambda(K_{i-1})
    \]

\end{proof}

We're not actually going to talk about Euler Characteristics.

\section*{Tensor Products}

It is a functor:

\(- \otimes_A - : A-mod \times A-mod \to A-mod\) 

First we want some notation for free module.

If we have a ring \(A\) and a set \(S\) then we have:

\(A[x]\) (or \(A^{(s)}\)): the free module with basis \(S\).

Then \(a_1 s_1 + \dots + a_n s_n \in A[S]\) where \(s_i\) distrinct.

If \(S\) is finite we have no problem, if \(S\) is infinite just consider it to be the formual sum.

\(A[S] \cong \bigoplus_{S}^{} A\) 

This is the set of (set-theoretic) functions \(S \to A\) that vanish almost everywhere.

\[
    A[S] \cong \{ f : S \to A : f(s)=0 \,a.e. \} 
\]

If \(M,N,P\) are \(A\)-modules and we have bilinear \(f: M\times N \to P\) so that \(f(-,n):M\to P\) and \(f(m,-):N\to P\) are both linear for all \(m,n\),

We sometimes write \((P,f:M\times N \to P)\) as \((P,f)\) 

\underline{Goal}: Associate bilienar maps \(f:M\to N\to P\) with a \underline{linear map} \(f^{\prime} : M \otimes _A N \to P\) 

\begin{proposition}
    [AM 2.12] Suppose we have \(A\)-modules \(M\) and \(N\).

    i: \underline{Existence:} \(\exists (T, g: M\times N \to T)\) a bilinear map which is `initial/universal' in the following sense: Any bilinear map factors through this. That is, for any bilienar map \(f: M\times N \to P\), there exists a unique map \(f^{\prime} : T\to P\) so that \(f^{\prime} \circ g = f\) [insert commutative diagram here].

    ii: \underline{Uniqueness:} Given \((T,g),(T^{\prime} ,G^{\prime} )\) there exists a unique \(j:T \to T^{\prime} \) such that it is an isomorphism and \(g^{\prime} \circ j = g\) 

\end{proposition}

We're going to construct \(g\) and show that it has this property.

\begin{proof}
    \underline{Existence} \(\implies \) \underline{Uniqueness}
    
    Suppose \(M\times n \overset{g}{\to} T\) and \(M\times N \overset{g^{\prime} }{\to} T^{\prime} \) [insert commutative diagram]. We want to say \(T\) and \(T^{\prime} \) are the same. By \(i\) there exists unique \(j\) such that \(T \overset{j}{\to } T^{\prime} \) and we also have a unique \(j^{\prime} : T^{\prime} \overset{j^{\prime} }{\to} T\).

    Now we have \(M\times N \overset{g}{\to} M\times N\) and \(M\times N \overset{g}{\to} T\) so there is unique \(T \overset{Id}{\to} T\) 
    
    So we have \(j^{\prime} \circ j = Id\) 

    \underline{Proof of Existence:}

    Let \(T\coloneqq \frac{A[M\times N]}{R}\) [`generators \(M\times N\) relations']

    Where \(R\) is the submodule generated by \((m+m^{\prime} ,n)-(m,n)-(m^{\prime} ,n)\) and \((m,n + n^{\prime} ) - (m,n) - (m,n^{\prime})\) and \((am,n) - a(m,n)\) and \((m,an) - a(m,n)\).
    
    We call \(M \otimes _A N \coloneqq T\) and call \(m \otimes n \coloneqq [(m,n)]\) 

    We have \(g:M\times N \to M \otimes _A N\) so that \(g(m,n) \to [(m,n)]\). This is bilinear by definition, and given \(f: M\times N \to P,\) we have \(F: A[M\times N]\to P\) [by exercise 3 in next assignment], thus, since \(F\) is bilinear, we have \(f^{\prime} :\frac{A[M\times N]}{R}\to P\) 

    This is true since \(R \subset \ker F\).

\end{proof}

\hrulefill

Class 17: 02/16

Recall: Tensor Products are given by \(M \otimes_A N = \frac{A[M \times N]}{R}\). We write \(\otimes_A = \otimes\) 

We have universal property: any bilinear map factors through the tensor product [insert commutative diagram here]

We have a map \(M\times N \overset{g}{\to}  M \otimes N\) 

Let \(m\otimes n \coloneqq g(m,n)\) 

Then \(m_1 \otimes n_1 + \dots + m_k \otimes n_k\in M \otimes N\) 

Due to the relations we modded out, we have these relationships:

\begin{itemize}
    \item \((m + m^{\prime}) \otimes n = m\otimes n + m^{\prime} \otimes n\) 
    \item \(m \otimes (n + n^{\prime} ) = m \otimes n + m \otimes n^{\prime} \) 
    \item \((am) \otimes n = m\otimes (an) = a(m \otimes n)\) 
\end{itemize}

Note, \(0_M p\times n = 0_A (0_M \otimes n) = 0_{M \otimes N}\)

Basically \(0 \times n = 0\) 

\begin{proposition}
    [AM 2.14]
    We have the following:

    i. \(M \otimes N \cong N \otimes M\) with isomorphism \(m \otimes n \leftrightarrow n \otimes m\) 

    ii. \((M \otimes N) \otimes P \cong M \otimes (N \otimes P)\) with isomorphism \((m \otimes n) \otimes p \leftrightarrow m \otimes (n \otimes p)\) 

    iii. \((M \oplus N) \otimes P \cong (M \otimes P) \oplus (N \otimes P)\) with isomorphism \((m,n)\otimes p \leftrightarrow (m \otimes p, n \otimes p)\) 

    iv. \(A \otimes M \cong M\) with \(a \otimes m \mapsto am, 1 \otimes m \mathrel{\reflectbox{\ensuremath{\mapsto}}} m\) 
\end{proposition}

\begin{proof}
    i: Consider maps from \(M \times N\) to \(M \otimes N, N \otimes M\). By universal property there is an invertible map between them.

    iv: Consider the maps from \(A \times M\) to \(M \) and \(A \otimes M\). By universal property, there exists a unique map from \(A \otimes M\) to \(M\). \(a \otimes m \mapsto am\) gives that to us.
\end{proof}

Note that \(2.14\) immedialtely givs us that:

\(A^m \otimes _A A^n \cong \left( \bigoplus_{m}^{} A \right) \otimes A^n \cong \bigoplus_{m}^{} (A \otimes A^n)\cong\bigoplus_{m}(A \otimes \bigoplus_{n}A)\cong \bigoplus_{m,n}A \otimes _A A \cong \bigoplus_{m,n} A \cong A^{mn}\) 

A \(\mathbb{Z}\)-module \(T\) is \underline{torsion} if \(\forall t\) there exists non-zero \(n\) so that \(nt = 0\). Basically, every element has an [additive] order.

Exercise: \(T\) is torsion if and only if \(T \otimes_\mathbb{Z} \mathbb{Q} = 0\) 

for example \(T = \mathbb{Q} / \mathbb{Z}\) is torsion.

\begin{proposition}
    [AM Corollary 2.13]

    If \(\sum_{i} m_i \otimes n_i - 0 \in M \otimes N\) then \(\exists\) finitely generated \(M_0 \triangleleft M, N_0 \triangleleft N\) so that:

    \(m_i \in M_0\) 

    \(n_i \in N_0\) 

    And \(\sum_{i} m_i \otimes n_i = 0 \in M_0 \otimes _A n_0\) 
\end{proposition}

\begin{proof}
    This is a corollary of the construction

    Use the fact that \(M \otimes N = \frac{A[M \times N]}{R}\).

    \(\sum_{i} m_i \otimes n_i = 0 \implies \sum_{i} [m_i, n_i] = \sum_{j} r_j \in R\) 

    Thus, \(r_j = \sum_{jk} a_{jk}(m_{jk},n_{jk})\) 

    Let \(M_0 = (m_i, m_{jk}), N_0 = (n_i, n_{jk})\) 

    That gives us the answer.

\end{proof}

Now we prove the previous fact.

\begin{proposition}
    Let \(T\) be a \(\mathbb{Z}\) module. Then \(T \otimes \mathbb{Q} = 0 \iff T\) torsion
\end{proposition}

\begin{proof}
    \(\impliedby \) (easy)

    For any \(t\in T\) there exists \(n \neq 0\) so that \(nt = 0\).

    Then \(t \otimes q = t \otimes \frac{n}{n}q = nt \otimes \frac{q}{n} = 0\times \frac{q}{n} = 0\) 

    \(\implies \) (uses the corollary)

    Assume \(T \otimes \mathbb{Q} = 0\). We want to prove that \(T\) is torsion.

    For any \(t\in T\) we have \(t \otimes 1 = 0\) 

    \(2.13\) says: \(t \otimes 1 = 0\) in \(T_0 \otimes \mathbb{Q}_0\) where \(T_0,\mathbb{Q}_0\) are finitely generated.

    Since \(Q_0\) is a finitely generated submodule of \(\mathbb{Q}\) it has to be \(\frac{1}{q}\mathbb{Z}\cong \mathbb{Z}\)

    Then \(T_0 \otimes \frac{1}{n}\mathbb{Z} \cong T_0\) with map \(x \otimes \frac{k}{n} \mathrel{\reflectbox{\ensuremath{\mapsto}}} kx\)
    
    \(t \otimes 1\) must go to \(nt\), since \(t \otimes 1 = 0\) we have \(nt = 0\).

    [???]

\end{proof}

\begin{definition}
    [Functor] A functor \(F: \mathcal{C} \to \mathcal{D}\) is \(F: Ob \mathcal{C} \to Ob \mathcal{D}\) so that if we have a morphism, then \(F: \mathcal{C} (x,y) \to \mathcal{D} (F(x),F(y))\) such that \(F(f\circ g) = F(f)\circ F(g)\) and \(F(1_X)=1_{F(X)}\). So \(X \mapsto F(X)\), \(X \overset{f}{\to } Y \mapsto F(X) \overset{F(f)}{\to } F(Y)\) 
\end{definition}

Now, \(- \otimes_A -\) is a functor.

If we have \(f: M \to M^{\prime} \) and \(g: N \to N^{\prime} \) we can define the corresponding thing on morphism: we have a map \(f \otimes g: M \otimes N \to M^{\prime} \otimes N^{\prime} \) given by \(f(\otimes g)(m \otimes n) = f(m) \otimes g(n)\) 

Since tensor product is a functor it respects maps.

\begin{proposition}[AM 2.18]
    Tensor Product is \underline{Right Exact}
\end{proposition}


This is useful for computation. See Exercise 7. 7 helps in 8.

Suppose \(M^{\prime} \to M \to M^{\prime\prime} \to 0\) is exact.

Then, for all \(N\),

\[
    M^{\prime} \otimes N \to M \otimes N \to M^{\prime\prime} \otimes N \to 0
\]

is exact. Even the maps are given by \(f \otimes 1\) and \(g \otimes 1\).

\begin{proof}
    \((g \otimes 1) \circ (f \otimes 1) = (g\circ f) \otimes 1 = 0\otimes 1 = 0\) 

    Define: \(\overline{g \otimes 1} : \frac{M \otimes N}{im(f \otimes 1)} \to M^{\prime\prime} \otimes N \)
    
    Take \([m \otimes n]\in \frac{M \otimes N}{im(f \otimes 1)}\). Then \([m \otimes n] \mapsto  g(m) \otimes n\) 

    Take \(m^{\prime\prime} \otimes n\). Since onto, we can lift it: \([\hat{m}^{\prime\prime} \otimes n] \mathrel{\reflectbox{\ensuremath{\mapsto}}} m^{\prime\prime} \otimes n\) whee \(g(\hat{m}^{\prime\prime} )= m^{\prime\prime} \) 
    
\end{proof}

AM adjoint proposition:
\(Hom(M \otimes N,P)\cong Hom(M,Hom(N,P))\) so that \(\varnothing \mapsto (m \mapsto (n \mapsto \phi(m \otimes n)))\)  

\hrulefill

Class 18: 02/19

\section*{Flat Modules}

\(-\otimes _A N\) is right exact, but not exact.

For example, \(-\otimes_\mathbb{Z} \mathbb{Z} / 2\) is not exact. Consider the following exact sequence:

\[
    0 \to \mathbb{Z} \overset{2}{\to} \mathbb{Z} \to \mathbb{Z} / 2 \to 0
\]

But from homework, tensoring doesn't make it exact.

\begin{definition}
    \(N\) is a \underline{flat} \(A\)-module if \(- \otimes_A N\) is an exact functor, i.e. any short exact sequence \(0 \to M^{\prime} \to M \to M^{\prime\prime} \) we have \(0 \to M^{\prime} \otimes N \to M \otimes N \to M^{\prime\prime} \otimes N \to 0\) is also short exact sequence.
\end{definition}

For example \(A\) is flat since \(M \otimes_A A = M\) 

\(A^n\) is flat since \(M \otimes_A A^n = M^n\) 

Any free module is flat. Recall free module is given by \(\bigoplus_{S}^{} A = A[S],(\bigoplus_{S}^{} A)\otimes M = \bigoplus_{S}^{} M\) 

Note that \(\mathbb{Z} / 2\) is not a flat \(\mathbb{Z}\)-module.

Atiyah Macdonald says: \(\mathbb{Q}\) is a flast \(\mathbb{Z}\) module.

\(N\) is a flat \(\mathbb{Z}\)-module if and only if \(N\) is torsion-free, which means it has no element of finite order.

\begin{proposition}
    [AM 2.19]

    Let \(N\) be an \(A\)-module. TFAE:

    \begin{enumerate}
        \item \(\forall \dots \to M_{i+1} \to M_i \to M_{i-1} \to \dots   \) exact \(\implies \) \(\dots \to M_{i+1} \otimes N \to M_i \otimes N \to M_{i-1} \otimes N \to \dots \) is exact
        \item  \(\forall 0 \to M^{\prime} \to M \to M^{\prime\prime} \to \) shot exact sequence implies \(0 \to M^{\prime} \otimes N \to M \otimes N \to M^{\prime\prime} \otimes N \to 0\) short exact sequence
        \item For any injection \(M^{\prime}  \to M\) we have an injection \(M^{\prime} \otimes N \to M \otimes N\) 
        \item For any injection \(M^{\prime}  \to M\) so that \(M^{\prime} ,M\) are finitely generated, we have an injection \(M^{\prime} \otimes N \to M \otimes N\) 
    \end{enumerate}
\end{proposition}

\begin{proof}
    \(1 \iff 2\):

    Suppose we have \(\dots \to C_{i+1}\overset{f_{i+1} }{\to } C_i \overset{f_i}{\to } C_{i-1} \to \dots \)
    
    Let \(B_i = im f_{i+1}, B_{i-1}= im f_i \)  

    Then \(\to C_{i+1}\overset{f_{i+1} }{\to }C_i \overset{f_i}{\to } C_{i-1} \) is exact if and only if:
    
    \(0 \to B_i \to C_i \to B_{i-1}\) is exact for all \(i\).

    So \(1 \iff 2\) 

    \(2\iff 3\) by AM 1.18 [right exactness of \(- \otimes_A N\) ]

    \(3 \implies 4\) 4 is just a special case of 3.
    
    \(4 \implies 3\): We use AM 2.13. Consider an injection \(f:M^{\prime} \to M\). Consider \(u\in\ker(f \otimes 1: M^{\prime} \otimes N \to M \otimes N)\) 

    Let \(u = \sum_{i} x_i^{\prime} \otimes y_i\)
    
    \(0 = (f \otimes 1)i = \sum_{i} f(x_i^{\prime} )\otimes y_i\) Let \(M_0^{\prime} = A(x_i^{\prime} )\triangleleft M^{\prime} \) 

    2.13 implies there exists finitely generated \(M_0 \triangleleft M, N_0 \triangleleft N\) so that \(\sum_{i} f(x_i^{\prime} \otimes y_i)=0\in M_0 \otimes N_0\) 

    WLOG \(\overline{M_0} \supset f(M_0^{\prime} )\). Let \(\overline{M_0}=(M_0,f(x_i^{\prime} ))\)  

    Thus \(u\in\ker(f \otimes 1) [M_0^{\prime} \otimes N \to \overline{M_0} \otimes N]\). By 4, we have \(u = 0\) so we're done.

    Corollary: \(\mathbb{Q}\) is a flat \(\mathbb{Z}\) module.

    Idea: \(M\) finitely generated means \(M =\) torsion \(\oplus\) free.

\end{proof}

\section*{Restriction and Extension of Sclalars}

Extension of Scalars is also called \underline{induction}.

Suppose we have a homomorphism \(f: A \to B\).

Then we have the module homomorphisms:

\(f^{\ast}:B-mod \to A-mod\) 

\(f_{\ast}:A-mod \to B-mod\)

\underline{Restriction}: \(f^{\ast} N\) is \(N\) as an abelian group like \(a\cdot n \coloneqq f(a)n\).

AM writes \(N\) instead of \(f^{\ast} N\) 

\underline{Induction / Extension of Scalars}:

\(f_{\ast} M = B \otimes_A M\) [or \(f^{\ast} B \otimes _A M\) ]

Note that \(B\) is \(A\)-module with \(a\cdot b = f(a)b\) 

\(f_{\ast} M\) is a \(B\)-module by \(b^{\prime} (b \otimes m) = b^{\prime} b \otimes m\) 

\begin{proposition}
    [AM 2.16]

    If \(N\) is finitely generated and \(B\) is finitely generated as \(A\)-module,

    \(f^{\ast} N\) is finitely generated as \(A\)-module.
\end{proposition}

\begin{proposition}
    [AM 2.17]

    If \(M\) is finitely generated \(A\)-module then \(f_{\ast} \) is finitely generated \(B\)-module.
\end{proposition}

\(M = A(x_i), f_{\ast} M = B(1 \otimes x_i)\) 

\underline{Observaton:}

\(f^{\ast} \) preserves exactness.

\(f_{\ast}\) preserves freeness.

If \(N^{\prime} \to N \to N^{\prime\prime}\) is exact then \(f^{\ast} N^{\prime} \to f^{\ast} N \to f^{\ast} N^{\prime\prime}\) is exact.

\(f_{\ast} A = B \otimes_A A = B\) 

\(f_{\ast} A^n = B^n\) 

\section*{Algebra}:

\begin{definition}[Algebra]
    Consider rings \(A,B\).

    If we have a homomorphism \(f:A \to B\) we call \(B\) an \underline{\(A\)-algebra} 
\end{definition}

homomorphism of \(A\)-algebra: a ring map \(B \overset{h}{\to} C\) so that \(h\circ f = g\) where \(A\overset{f}{\to } B\) and \(A \overset{g}{\to } C\) 

Category: \(A\)-algebra = \(A\downarrow\) Ring

So, \(\mathbb{Z}\)-algebra = Ring

Suppose \(k\) is a field. If we have \(f:k\to B\) then \(f\) is injective. So, \(k\)-algebra is the same thing as a ring \(B\) so that \(k \subset B\).

Note that \(f\) need not be injective so this is not necessarily true.

If \(B\) is an \(A\)-algebra \(\implies \) \(B\) is an \(A\)-module with \(a\cdot b = f(a)b\).

Consider the following `competing' definition:

\begin{definition}
    \(B\) is a finite \(A\)-algebra if \(B\) is a finitely generated \(A\)-module
\end{definition}

\begin{definition}
    \(B\) is a finitely generated \(A\)-algebra if \(\exists b_1,\cdots,b_n\) such that \(B = f(A)[b_1,\cdots b_n]\) polynomials with \(b_i\) coefficients in \(f(A)\)
\end{definition}

Note that finite \(\implies \) finitely generated.

\(B\) is a finitely generated \(A\)-algebra \(\iff\) \(\exists A[t_1,\cdots,t_n]\rightarrow B\) where \(a \mapsto f(a),t_i \mapsto b_i\) 

eg \(\mathbb{C}\) is a finite \(\mathbb{R}\)-algebra:

\(\mathbb{R}[\overline{x} ,\overline{y} ]=\frac{\mathbb{R}[x,y]}{(y-x^2)}\) is a finitely generated \(\mathbb{R}\)-algebra but it is not finite. 

\hrulefill

Class 19:02/21

\begin{proposition}
    If \(B,C\) are \(A\)-algebras then \(B \otimes _A C\) is also an \(A\)-algebra
\end{proposition}

\begin{proof}
    To give a map out of tensor product, we need to check what it does on pure tensors and then check if it is bilinear. We check the multiplication map:

    \[
        (b_1 \otimes c_1)\cdot (b_2 \otimes c_2) \coloneqq (b_1 b_2) \otimes (c_1 c_2)
    \]

    We want to check if it is well defined. For that, we need to check bilinear. That is trivial.

\end{proof}

We have \underline{Universal Property}: An algebra map out of \(B \otimes _A C\) is the same thing as giving algebra maps out of \(B\) and \(C\).

In other words, if we have algebra maps:

\[
    f: B \to D
\]

\[
    g: C \to D
\]

Then there exists a unique algebra map

\[
    f \otimes g: B \otimes _A C \to D
\]

Such that \(f \otimes g(b \otimes 1)=f(b)\) and \(f \otimes g(1 \otimes c) = g(c)\) 

\underline{Warning}: This is a class on commutative algebra, so we can assume everything is commutative, but in this case commutativity is essential. \((b \otimes 1)\) and \((1 \otimes c)\) \underline{commute}! This is \underline{NOT} fine in non-commutativity case. We need a map such that the images of \(B\) and \(D\) commute.

\section*{Rings and Modules of Fractions}

If \(D\) is a domain [example: \(\mathbb{Z}\)] we can construct a field of fractions \(\operatorname{Frac}(D)\) [example: \(\mathbb{Q}\)] which is constructed as \(D \times D_{\neq 0} / \equiv\) an equivalence class so that \((a,s)\equiv (b,t)\iff at - bs = 0\) which we write \(a / s\). We can give it a ring structure by \((a / s) \cdot (b / t) = (ab / st)\) and \((a / s) + (b / t) = (at + bs) / st\).

This is a ring, and \(D \hookrightarrow \operatorname{Frac}(D) \) by \(x \mapsto x / 1\). Moreover, \(\operatorname{Frac}(D) \) is a field with \((a / s) \cdot (s / a) = 1\). So we have a multiplicative inverse as long as \(a \neq 0\) but \(a = 0\) implies \(a / s = 0\) so all non-zero elements have inverses.

We can \underline{generalize} this construction in two different ways.

\underline{First way:} General Commutative Rings [not just domains]

\underline{Second way:} Only inverst some elements [instead of all nonzero elements]

Note that if we make generalization one then we must make generalization two, but not the other way around. This is because we can't invert all elements of general rings.

Example: Dyadic Rationals: Fractions whose denominator is a power of \(2\). This is a ring.

Another example: fractions with odd denominator. Multiplication of odd denominators is odd, and lcm of odd numbers is odd so this is also rings.

\underline{Questions}: What kind of subsets of \(A\) should we be allowed to invert?

If two things are allowed to be a denominator then their product must also be allowed to be a denominator.

We want \(1\) to be an allowed denominator since we wand \(a \mapsto a / 1\) to be a valid map.

\begin{definition}
    \(S \subset A\) is a \underline{multiplicatively closed subset} [not subring or ideal] if \(1\in S\) and \(S\) is closed under multiplication.
\end{definition}

\begin{definition}
    \(S ^{-1} A\) is defined as \(A \times S / \equiv \) so that \((a,s) \equiv (b,t)\) if \((at - bs)\cdot u = 0\) for some \(u\in S\) [we can't use the previous one because in order for it to be an equivalence relationship, we need transitivity which we don't have. So zero divisors play the role of zero].

    We still have addition and multiplication like before.
\end{definition}

\underline{Lemma}: \(\equiv\) is an equivalence relationa and \(S ^{-1} A\) is a ring.

\underline{Warning}: There is always a map \(A\to S ^{-1} A\) with \(x \mapsto x / 1\). This will be a ring homomorphism, but if might not be \underline{injective} anymore.

Example: if \(0\in S\) then \(S ^{-1} A\) is isomorphic to the zero ring, since everything will be equivalent.

Example: if \(S = \{ 1, 2, 4 \} \) in \(\mathbb{Z} / 6\) then \(S ^{-1} \mathbb{Z} / 6 \cong \mathbb{Z} / 3\)

So, if we try to invert a zero divisor, it only kills the things that it multiplies with to make zero. If we try to invert zero then that kills everything so we can only have the zero ring.

\underline{Univeral Property} If \(g: A \to B\) is a ring homomorphism such that \(g(s)\) is invertible in \(B\) for all \(s\in S\), then there exists a unique map \(h:S ^{-1} A \to B\) 

[draw commutative diagram of \(A, B, S ^{-1} A\)]

Example, we have a map from \(\mathbb{Z} / 6\) to \(\mathbb{Z} / 3\) so it must factor through some \(S ^{-1} A\) where \(S = \{ 1,2,4 \} \) 

\begin{proof}
    \underline{Uniqueness}: \(h(a / s)\) must be equal to \(h(a / 1)\cdot h((s / 1)^{-1}) = g(a)\cdot g(s)^{-1}\) so the only possible way to define this map is \(h(a / s)=g(a)\cdot g(s)^{-1}\) 

    \underline{Existence}: Define \(h(a / s) \coloneqq g(a)\cdot g(s)^{-1}\) and check that it is well-defined [does it respect equivalence?]

    Suppose \((a,s)\equiv (a^{\prime} ,s^{\prime} )\). Then \((as^{\prime} -a^{\prime} s)t = 0\) for some \(t\in S\).
    
    Then \((g(a)g(s^{\prime} )-g(a^{\prime})g(s))g(t) = 0\). Since \(g(t)\) is invertible, we can cancel it.
    
    \(\implies g(a)g(s^{\prime} ) = g(a^{\prime} )g(s) \implies g(a)g(s)^{-1} = g(a^{\prime} )g(s^{\prime} )^{-1}\) 
    
    So the map indeed exists.

\end{proof}

\underline{Corollary}: If \(g: A \to B\) is a ring homomorphism such that:

\begin{enumerate}
    \item \(s\in S \implies g(s)\) is invertible
    \item \(g(a) = 0 \implies as = 0\) for some \(s\in S\) 
    \item Every element of \(B\) is of the form \(g(a)g(s)^{-1}\) 
\end{enumerate}

Then \(\exists !h:S ^{-1} A \to B\) that is an isomorphism such that \(g = h \circ f\) 

\(1\) tells us that a map exists, \(2\) tells us that the map is injective [kernel is 0] and \(3\) gives us surjectivity.

Example:

In the map \(\mathbb{Z} / 6 \to \mathbb{Z} / 3\) and we have \(1 \mapsto 1, 2 \mapsto 2, 4 \mapsto 1\) all of whom are invertible, and \(3,6\) gets mapped to \(0\) for whom \(2\cdot 3, 1\cdot 6 = 0\).

Example: If \(f\in A\), we can set \(S = \{ f^n \} _{n\in \mathbb{Z} _{>0}}\). Then we can look at \(A_f \coloneqq S ^{-1} A\) for this particular \(S\).

This includes the dyadic rationals, \(\mathbb{Z}_2\).

Other example: Consider ideal \(\mathfrak{p}\). Then \(A - \mathfrak{p}\) is multiplicatively closed if and only if \(\mathfrak{p}\) is a prime ideal. Then we can take \(S = A - \mathfrak{p}\).

\(A_{\mathfrak{p}} \coloneqq S ^{-1} A\) for this \(A\). We can consider the fractions with odd denominators \(\mathbb{Z}_{(2)}\).

\hrulefill

Class 20: 02/23

We finish with one more example.

Take \(\mathbb{C} [x]\), and localize at the ideal generated by \(x\).

The fraction field of \(\mathbb{C} [x]\) are rational functions

\(\mathbb{C} [x]_{(x)}\) then are the rational functions defined at \(0\).

\(\mathbb{C} [x]_{(x-a)}\) are the rational functions defined at \(a\)

\begin{definition}
    If \(S\) is a multiplicatively closed subset of \(A\) and \(M\) is an \(A\)-module then we define \(S ^{-1} M\) is defined to be \(M \times S / \equiv\) so that \((m,s) \equiv (m^{\prime} , s^{\prime} ) \iff \exists t\in S, t\) so that \(t(sm^{\prime} - s^{\prime} m) = 0\) 
\end{definition}

We need to check:

1. \(\equiv\) is a equivalence relation

2. \(S ^{-1} M\) is an \(S ^{-1} A\)-module

Again, we write \(M_f\) where \(f\) is an element and \(M_{\mathfrak{p}} \) where \(\mathfrak{p}\) is a prime ideal of \(A\).

Category theory: ``\(S^{-1}\) is functorial''. That is to say, not only does it make sense to apply \(S ^{-1} \) to a module, but it also makes sense to apply \(S ^{-1} \) to maps.

If \(f:M \to N\) is a map then we have the map \(S ^{-1} f: S ^{-1} M \to S ^{-1} N\) that is defined by: \(\frac{m}{s} \mapsto \frac{f(m)}{s}\). 

Check that it is a module.

We also need: \(S ^{-1} (id_M) = id_{S ^{-1} M}\) 

\(S ^{-1} (f \circ g) = S ^{-1} f \circ S ^{-1} g\) 

\begin{proposition}
    \(S ^{-1} \) is \underline{exact}.
\end{proposition}

What does this mean?

If we have \(M^{\prime} \overset{f}{\to } M \overset{g}{\to } M^{\prime\prime} \) is exact, then:

\[
    S ^{-1} M^{\prime} \overset{S ^{-1} f}{\to } S ^{-1} M \overset{S ^{-1} g}{\to} S ^{-1} M^{\prime\prime} 
\]

Is exact.

\underline{Corollary}: If \(f\) is injective then \(S ^{-1} f\) is injective.

If \(f\) is surjective then \(S ^{-1} f\) is surjective.

Proof is by putting \(0\) in the exact sequence.

\begin{proof}
    We want to show that:

    \(\ker (S ^{-1} g) = \im (S ^{-1} f)\).

    We are going to show inclusion in both direction.

    \(\ker (S ^{-1} g) \supseteq \im (S ^{-1} f)\) by functoriality.

    For \(\subseteq \):

    Suppose \(\frac{m}{s}\in S ^{-1} g\).

    Then, \(S ^{-1} g (\frac{m}{s}) = 0\) 

    \(S ^{-1} g(\frac{m}{s}) = \frac{g(m)}{s}\)
    
    Thus, there exists \(t\in S\) such that \(t\cdot g(m) = 0 \implies g(tm) = 0 \implies tm \in\ker g = \im f\)
    
    So, \(f(x) = tm \implies S ^{-1} f(\frac{x}{ts}) = \frac{mt}{ts}=\frac{m}{s}\) 

\end{proof}

\underline{Corollary}:

If \(N,P\) are submodules of \(M\),

\begin{enumerate}
    \item \(S ^{-1} (N + P) = S ^{-1} N + S ^{-1} P\) 
    \item \(S ^{-1} (N \cap P) = S ^{-1} M \cap S ^{-1} P\) 
    \item \(S ^{-1} (M) / S ^{-1} (N) \overset{\sim}{\to } S ^{-1} (M / N)\) 
\end{enumerate}

\begin{proof}
    1. We use the definition and how we add fraction.

    2. \(S ^{-1} (N \cap P) \subseteq S ^{-1} M \cap S^{-1} P\) is easy.

    For \(\supseteq\), suppose \(\frac{y}{s} = \frac{z}{t}\) where \(y\in N, z\in P\).

    So, there exits \(u\in S\) so that \(u(ty-sz) = 0 \implies \exists w = uty = usz\).

    Thus, \(w\in N \cap P\).

    Finally, \(\frac{y}{s} = \frac{w}{uts}\) 

    3. [this is actually the corollary].

    Take the exact sequence:

    \[
        0 \to N \to M \to M / N \to 0
    \]

    Then,

    \[
        0 \to S ^{-1} N \to S ^{-1} M \to S ^{-1} (M / N) \to 0
    \]

    Thus, first map is injective, last map is surjective, kernel is image.
    
    Thus \(S ^{-1} (M / N) \cong S ^{-1} (M) / S ^{-1} (N)\) 

\end{proof}

\begin{proposition}
    \(S ^{-1} A \otimes _A M \overset{\sim}{\to} S ^{-1} M\) 
\end{proposition}

\begin{proof}
    We prove it works on pure tensors and check it's bilinear.

    \[
        \frac{a}{s} \otimes m \mapsto \frac{am}{s}
    \]

    This is a map of \(S ^{-1} A\)-modules. It exists by the universal property of tensor products since it is bilinear.

    It is also an \(S ^{-1} A\) module map.

    It is also surjective: everything of \(S ^{-1} M\) is of the form \(\frac{m}{s}\). We have: \(\frac{1}{s} \otimes m \mapsto \frac{m}{s}\) 

    For injectivity:

    \underline{Lemma}: Every element of \(S ^{-1} A \otimes M\) is of the form \(\frac{1}{s} \otimes m\).

    Note that usually tensor products are usually huge sum. The way we write it as one sum is: \underline{common denominators!}

    \[
        \sum_{i=1}^{n} \frac{a_i}{s_i} \otimes m_i = \frac{1}{\prod s_i} \sum_{i}\left(\prod_{j \neq i} s_j\right)a_i \otimes m_i = \sum_{i}^{} \frac{1}{s} t_i a_i \otimes m_i
    \]

    \[
        = \sum_{i}^{} \frac{1}{s}\otimes a_i t_i m_i = \frac{1}{s} \otimes \sum_{i} a_i t_i m_i = \frac{1}{s} \otimes m
    \]

    Thus we have proved the lemma.

    Suppose \(0 = f(\frac{1}{s} \otimes m) = \frac{m}{s}\), then \(\exists t\in S\) so that \(mt = 0\) thus \(\frac{1}{s} \otimes m = \frac{t}{st} \otimes m = \frac{1}{st} \otimes tm = \frac{1}{st} \otimes 0 = 0\).

    So, the kernel is \(0\) and thus it is injective.

\end{proof}

\hrulefill

Class 21: 02/26

Localization review.

Suppose we have a ring \(A\) and a multiplicatively closed set \(S\).

\begin{definition}
    \(S \subset A\) is \underline{multiplicatively closed} [mc] if \(S\) is a submonoid of \((A,\times)\). By monoid, we mean \(1\in S, a,b\in S \implies ab\in S\).
\end{definition}

In this situation, we can \underline{localize}, or invert in \(S\).

If \(M\) is an \(A\)-module, we define an equivalence relation on \(M \times S\) so that \((m,s)\sim (m^{\prime} ,s^{\prime} )\) if there exists \(s^{\prime\prime} \in A\) such that \((ms^{\prime} - m^{\prime} s)s^{\prime\prime} = 0\) 

We write \(\frac{m}{s} = [(m,s)]\) the equivalence class.

We write \(S ^{-1} M = M \times S / \sim\).

\(S ^{-1} M\) is an abelian group.

\(\left( \frac{m}{s} + \frac{m^{\prime}}{s^{\prime}}  \right) = \frac{ms^{\prime} +sm^{\prime} }{ss^{\prime} }\) 

If \(M = A\) then \(S ^{-1} A\) is a ring, \(\left( \frac{a}{s} \right) \left( \frac{a^{\prime} }{s^{\prime} } \right) = \frac{a a ^{\prime} }{s s^{\prime} }\) 

We have a map \(f: A \to S ^{-1} A\) given by \(a \mapsto a / 1\).

\(f\) is \(S\)-inverting \((f(S) \subset (S ^{-1} A)^\times )\) and initial.

\underline{Universal Property (3.1):}

If \(g: A \to B\) is \(S\)-inverting, that is, \(g(S) \subset B^\times \) then \(g\) factors through the map \(f: A \to S ^{-1} A\) so we have a unique map \(h\) so that \(h\circ f=g\).

Examples: If \(0\in S\), that's bad because we shouldn't be able to invert \(S\). Then \(S ^{-1} A = 0\).

If \(A\) is a domain and \(S = A - 0 \), then \(S ^{-1} A = Frac(A)\).

An ideal \(I\) of \(A\)  is prime if and only if \(A - I\) is multiplicatively closed. This is just the contrapositive of the definition of the prime ideal.

If we have a prime ideal \(P\) in \(A\) then \(A - P\) is multiplicatively closed, then \(A_P\) is defined to be \((A - P)^{-1} A\), and it is called the \underline{localization} of \(A\) at \(P\).

\(A_P\) is a local ring with maximal ideal \(PA_P\).

We have a correspondence: Ideals of \(A_p\) \(\leftrightarrow\) \(\{ I \triangleleft A | I \subset P \} \) 

Note that \(A \to S ^{-1} A\) is injective if and only if \(S \subset \) non-zero divisors.

If we have \(f\in A\) then we can look at the multiplicatively closed subset \(\{ 1, f, f^2, \cdots \} \) and then we can look at \(\{ 1, f, f^2,\cdots \} ^{-1} A\). AM notation for this is \(A_f\), better notation is \(A[\frac{1}{f}]\).

Example: \(\mathbb{Z}_{(2)}\), where we invert everything outide \((2)\) so it is \(\{ \frac{a}{b}\in\mathbb{Q} | a,b\in\mathbb{Z} ,2 \not\mid b \} \). This contains \(\mathbb{Z}\).

Also, \(\mathbb{Z}_2 = \mathbb{Z}[\frac{1}{2}] = \{ \frac{a}{2^k}\in\mathbb{Q} \} \) 

In 3.3, we proved \(S ^{-1} : A-mod\to S^{-1} A-mod\) is flat.

A functor \(F\) is flat if it takes exact sequences to exact sequences, and \(-\otimes_A M\) is flat then we say the module \(M\) is flat.

In 3.5, we proved \(S ^{-1} A \otimes_A M \overset{\approx}{\to } S ^{-1} M\).

3.3 and 3.5 implies corollary 3.6:

\(S ^{-1} A\) is a flat \(A\)-module, since \(S ^{-1} A \otimes -\) is flat.

For example, \(\mathbb{Q}\) is a flat \(\mathbb{Z}\)-module.

Also, 3.5 and chapter 2 implies Proposition 3.7, which says \(S ^{-1} M \otimes_{S ^{-1} A} S ^{-1} N \overset{\approx }{\to } S ^{-1} (M \otimes _A N)\) 

\section*{Local Properties}

\begin{proposition}
    [AM 3.8]

    ``Being zero is a local property''.

    Let \(M\) be an \(A\)-module. Then TFAE:

    \begin{enumerate}
        \item \(M = 0\) 
        \item \(M_P = 0 \forall\) prime \(P \triangleleft A\) 
        \item \(M_I = 0 \forall \) maximal ideal \(I \triangleleft A\) 
    \end{enumerate}
\end{proposition}

\begin{proof}
    1 implies 2 is obvious, 2 implies 3 since all maximal ideals are prime.

    Instead of 3 implies 1 we prove the contrapositive.

    \(M \neq 0 \implies \exists x\neq 0\in M\).

    Thus, \(Ann(x)\) is a proper ideal of \(A\) since it doesn't contain \(1\).

    By Zorn's lemma, \(Ann(x)\) is contained in a maximal ideal \(I\).

    For all \(s\in A - I, s\notin Ann(x)\) thus \(sx\neq 0\).

    \(\frac{x}{1}\neq M_i\) is nonzero if and only if \(\forall s\in A - I, sx\neq 0\), which we proved before.

    Thus, \(M_I\neq 0\).

\end{proof}

\begin{proposition}
    [AM 3.9]

    Let \(\phi:M\to N\) be a module map. Then TFAE:

    \begin{enumerate}
        \item \(\Phi\) is 1-1.
        \item \(\Phi_P\) is 1-1 for all prime \(P\).
        \item \(\Phi_I\) is 1-1 for all maximal \(I\). 
    \end{enumerate}

    Same hold for onto.

\end{proposition}

\begin{proof}
    1 implies 2 is true since localization is exact [3.3]. Think of the morphism \(\Phi_P:M_P\to N_P\)
    
    2 implies 3 since maximal are prime.

    3 implies 1 since if \(\phi:M\to N\) and let \(M^{\prime} \) be the kernel of \(\Phi\) then we have the exact sequence \(0 \to M^{\prime} \to M\to N\) is exact, and by 3.3 we have \(0 \to M_I^{\prime} \to M_I \to N_i\) is exact, and by the hypothesis we must have \(M_I^{\prime} = 0\) for all \(I\) , and by 3.8 we have \(M^{\prime} =0\) therefore \(\Phi\) is injective.

    For the onto part, reverse the arrows.
\end{proof}

\begin{proposition}
    [AM 3.10]

    `Flatness is a local property'.

    A module is flat if and only if its localization and prime ideals are flat [or maximal].
\end{proposition}

\hrulefill

Class 22: 02/28

\begin{proposition}
    [AM 3.10]

    ``Flatness is a Local Property''

    Let \(M\) be an \(A\)-module. Then the following are equivalent:

    \begin{enumerate}
        \item \(M\) is flat, that is tensor with \(M\) is an exact functor
        \item For all prime \(P \triangleleft A\), \(M_P\) is flat \(A_P\) module
        \item For all maximal \(I \triangleleft A\), \(M_I\) is flat \(A_I\) module. 
    \end{enumerate}

\end{proposition}

\begin{proof}
    \(i \implies ii\):

    Note, flatness means it takes injective maps to injective maps.

    Suppose \(M\) is flat, and we have a injective map \(N \to Q\) of \(A_P\) modules.

    Then \(M \otimes_A N \to M \otimes_A Q\) is injective since \(M\) is flat

    Then \(A_P \otimes_A M \otimes_A N \to A_P \otimes_A M \otimes_A Q\) is injective since \(S ^{-1} A\) is a flat \(A\)-module

    Note that \(A_P \otimes_A M \otimes_A N = M_P \otimes_{A_P} N\) and \(A_P \otimes_A M \otimes_A Q = M_P \otimes_{A_P} Q\) 

    So, \(M_P \otimes_{A_P} N \to M_P \otimes_{A_P} Q\) is injective

    Thus \(M_P\) is flat \(A_P\) module.

    \(ii\implies iii\) follows from the fact that maximal ideals are prime.

    \(iii \implies i:\) 

    Suppose \(N \to Q\) is injective \(A\)-module map. Since injectivity is a local property [3.9] we have for all maximal \(I\), \(N_I \to Q_I\) is injective.
    
    Therefore, for all \(I\), \(M_I \otimes N_I \to M_I \otimes Q_I\) is injective [since \(M_I\) is flat by hypothesis]

    Note that \(M_I \otimes N_I = (M \otimes N)_I\) and \(M_I \otimes Q_I =(M \otimes Q)_I\) by 3.7

    Therefore, \(M \otimes N \to M \otimes Q\) is injective by 3.9, injectivity is local property.

\end{proof}

\subsection*{Extension, Contraction, Localization}

Preview: Corollary 3.13:

We can look at ideals in \(A_P\) where \(P\) is prime, which is equal to \((A - P)^{-1} A\) by definition.

We want to show that ideals in \(A_P\) are in a bijective correspondence with ideals \(I \subset P \triangleleft A\).

\(I \subset P \triangleleft A, I\cap S = \varnothing \overset{c,\approx}{\longleftarrow} \) ideals in \(A_P\)  

So, we have \(f ^{-1} J \longleftarrow J\) where \(f: A \to S ^{-1} A\).

Example: consider \(\mathbb{Z}_{(2)}\). This has ideals \(\frac{a}{odd}\in\mathbb{Z}_{(2)}\).

Then contraction gives us ideals \(\mathbb{Z} 2^k\) of \(\mathbb{Z}\) 

In general, if we have \(f: A \to B\), we have a correspondence between \(I \triangleleft A\) and \(J \triangleleft B\) by extension and contraction: \(J^c = f ^{-1} (I)\) and \(I^e = Bf(I)\) 

If \(C = \im c\) and \(E = \im e\) then we have a bijection from \(C\) to \(E\).

Now we are going to apply this to \(A \to S ^{-1} A\).

\(I^e = S ^{-1} I = \{ \frac{f(i)}{s}|s\in S, i\in I \} \) 

\begin{proposition}
    [AM 3.11]

    i: Every ideal in \(S ^{-1} A\) is extended

    ii: \(I \triangleleft A \implies I^{ec} = \bigcup_{s\in S}^{} (I:s) \)  

    ii': \(I^e = (1) \iff I \cap S \neq \varnothing\) 

    iii:  \(I\in C \iff\) no element of \(S\) is zero divisor in \(A / I\)  

    iv:
\end{proposition}

\begin{proof}

    i: 
    Suppose \(J \triangleleft S ^{-1} A\). We have \(J \supset J^{ce} \) by 1.17

    Consider \(\frac{x}{s}\in J\) this implies \(\frac{x}{1}\in J\) this implies \(x\in J^c\) this implies \(\frac{x}{s}\in J^{ce}\)
    
    So \(J \subset J^{ce}\). Therefore \(J = J^{ce} \) so \(J\) must be an extended ideal

    ii:

    Recall \((I:s)=(I:(s))=\{ x | xs \in I \} \) `ideal quotient'

    \(x\in I^{ec} = (S ^{-1} I)^c\)
    
    \(\iff \frac{x}{1} = \frac{i}{s} \in S ^{-1} A\) for \(i\in I, s \in S\) 

    \(\iff (xs-i)s^{\prime} = 0\in A\) where \(i\in I, s,s^{\prime} \in S\) 

    \(\iff xss^{\prime} \in I\)
    
    \(\iff xs^{\prime\prime} \in I\) 

    \(\iff x\in (I:s^{\prime\prime} )\) 

    iii: \(I\in C \iff I^{ec} = I\) 

    \(\iff (sx\in I \implies x\in I)\)
    
    \(\iff \overline{s}\) is not zero divisor in \(A / I\) 

\end{proof}

\hrulefill

Class 23: 03/01

\[
    \{ \text{ideals } (2^k) \triangleleft \mathbb{Z}  \} \longleftrightarrow \{ \text{ideals of } \mathbb{Z} _{(2)} \} 
\]

\[
    \mathbb{Z} / 2^k \mathbb{Z} \cong \mathbb{Z}_{(2)} / 2^k\mathbb{Z} _{(2)}
\]

\begin{proposition}[AM 3.11]
    Consider \(A \mapsto S ^{-1} A\).

    \begin{enumerate}
        [label=\roman*]

        \item Every ideal in \(S ^{-1} A\) is extended
        \item \(I^{ec} = \bigcup_{s\in S} (I:s)\), \((I:s) = \{ x\in A | xs\in I \} \) 
        \item \(I\) contracted if and only if \(\im (S \to A / I) \subset nzd(A / I)\) 
        \item \(\{ \text{prime } P \triangleleft A, P\cap S\neq \varnothing \} \overset{\text{bijection} }{\longleftrightarrow} \{ \text{prime ideals in } S ^{-1} A \} \) 
        \item \(S ^{-1} \) commutes with finite sums, finite products, finite intersections and radicals.

    \end{enumerate}

\end{proposition}

For v, we use 1.18 and 3.4.

\begin{proof}
    (iv):

    Map from left to right: \(P \mapsto P^e\) 

    Map from right to left: \(Q \mapsto Q^c\) 

    Pick prime \(Q \triangleleft S ^{-1} A\). Note that \(Q^c\) is a prime ideal.

    [\(Q^c\) is prime if and only if \(A / Q^c\) is a domain, which embeds \(A / Q^c \hookrightarrow S ^{-1} A / \mathbb{Q}\) which is a domain].

    \(Q = Q^{ce}\) by (i). So we get an identity. So contraction has a one sided inverse.

    Suppose we have prime \(P \triangleleft A\). Then \(A / P\) is a domain. Let \(\overline{S} \) be the image of \(S\) in \(A / P\). Then \(\overline{S} ^{-1} (A / P) = S ^{-1} A / S ^{-1} P\) [which is 3.4].

    We have two cases here. Case 1: when this ring is zero, case 2: where it is nonzeo.

    Case 1: \(\overline{S} ^{-1}  (A / P) = 0\): This means \(0 \in \overline{S} ^{-1} \) which means \(S \cap P \neq \varnothing \) so it doesn't satisfy the condition.
    
    Case 2: \(\overline{S} ^{-1} (A / P) \neq 0\). This implies \(S \cap P = \varnothing\), thus \(\overline{S} ^{-1} (A / P) \subset Frac(A / P) \implies S ^{-1} A / S ^{-1} P\) is a domain, thus \(S ^{-1} P=P^e\) is prime.
    
    \underline{Corollary 3.12}: \(Nil(S ^{-1} A) = S ^{-1} Nil(A)\)
    
\end{proof}

\begin{proof}
    Follows from 3.11
\end{proof}


\underline{Corollary 3.13} Suppose \(P \triangleleft A\) is prime. Then,

\[
    \{ \text{prime ideal of } A \subset P \} \longleftrightarrow \{ \text{prime ideals of } A_{\mathbb{P}} \} 
\]

\begin{proof}
    Let \(S = A - P\) and use 3.11(iv).
\end{proof}

\underline{Remark}: \(P,Q\) prime in \(A\). Then Spec is a contravariant function:

\[
    Spec(A_p) \hookrightarrow Spec(A)
\]

We also have:

\[
    Spec(A / P) \hookrightarrow Spec(A)
\]

Image in \(P^{\prime} \supset P\) 

Suppose we have \(P \subset Q \triangleleft A\) where \(P,Q\) are prime. consider:

\[
    \im Spec A_Q \cap \im Spec(A / P) = \{ P \subset s P^{\prime} \subset  \}, A / Q / P \cong A / P_{\overline{Q} }
\]

And also \(A_Q / S ^{-1} P \cong A / P_{\overline{Q} }\) 

When \(P=Q\) \(A_P / P = Frac(A/)\)  

residue field.

\begin{proposition}
    [AM 3.14]

    Suppose \(M\) is a finitely generated \(A\)-module. Then,

    \[
        S ^{-1} Ann(M) = Ann(S ^{-1} M) \triangleleft S ^{-1} A
    \]
\end{proposition}

\begin{proof}
    Case 1: \(M\) is cyclic

    \[
        0 \to Ann(M) \to A \to M \to 0
    \]

    \[
        0 \to S ^{-1} An n M \to S ^{-1} A \to S ^{-1} M \to 0
    \]

    \[
        S ^{-1} M = \frac{S ^{-1} A}{S ^{-1} An n M}
    \]

    \[
        0 \to S ^{-1} An n(M) \to S ^{-1} A \overset{\pi }{\to } S ^{-1} M \to 0
    \]

    \(An n(S ^{-1} M) = \ker \pi = S ^{-1} An n(M)\) 

    Case 2: Assume we have \(S ^{-1} A n n(M) = A n n (S ^{-1} M)\)
    
    \(S ^{-1} A n n(N) = S ^{-1} A n n (S ^{-1} N)\)
    
    Claim: \(S ^{-1} A n n(M + N) = A n n(S ^{-1} (M + N))\)
    
    Note tha the claim implies the proposition.

    proof of claim:

    \(S ^{-1} A n n(M + N) = S ^{-1} (A n n(M) \cap A n n(M))\) [2.2]

    \(= S ^{-1} A n n M \cap S ^{-1} A n n N\) [3.11(v)]

    \(= An n (S ^{-1} M) \cap A n n(S ^{-1} N)\) [hypothesis]

    \(= A n n (S ^{-1} M + S ^{-1} N)\) [2.2]

    \(= A n n(S ^{-1} (M + N))\) so we're done.

\end{proof}

\underline{Corollary 3.15}: Suppose we have \(N,P \triangleleft A\) , \(P\) finitely generated. Maybe \(P\) is not prime. Then,

\[
    S ^{-1} (N : P) = (S ^{-1} N : S ^{-1} P)
\]

\begin{proposition}
    [AM 3.16]

    For a general homomorphism \(f:A \to B\) and prime \(P \triangleleft A\), \(P\in C \iff P = P^{ec}\)
\end{proposition}

\hrulefill

Class 24: 03/04

\section*{Some Algebraic Geo}

Let \(k\) be a field, and \(I \triangleleft k[x_1, \dots , x_n]\) 

\(f\in k[x_1,\cdots, x_n]\) gives function \(f: k^n \to k\)

By the evalutation map \(f(a_1,\cdots, f(a_n))\).

Given a polynomial \(f\) we can look at its variety, \(V(f)= f^{-1}(0) = \{\mathbf a\in k^n : f(a_1,\cdots,a_n) = 0 \} \) 

If \(f\neq 0\) then \(V(f) \subset k^n\) is called a hypersurface.

For example draw \(V(x_2^2 - x_1(x_1^2 - 1))\) or \(V(x_2^2 - x_1^2(x_1 + 1) )\)  or \(V(x_3^{2-(x_1^2 +x_2^2 )} )\) or \(V(x_2^2 - x_1 x_2 - x_1^2 x_2 + x_1^3)\) which is \(V((x_2 - x_1)(x_2 - x_1^2))\). The last case is reducible: a variety is reducible if it is union of two varieties.

Suppose \(S \subset k[x_1,\cdots,x_n]\) 

\begin{definition}
    \(\mathcal{V} (S) = \bigcup_{f\in S}^{} \mathcal{V} (f) \subset k^n\) 
\end{definition}

Remark: \(V(S) = V(I(S))\) where \(I(S)\) is the ideal generated by \(S\).

\begin{definition}
    A variety is a subset of \(k^n\) of the form \(\mathcal{V}(I)\) for some ideal \(I \triangleleft k[x_1,\cdots, x_n]\).
\end{definition}

Suppose \(X \subset k^n\) ideal of \(X\) 

\(I = \{ f\in k[x_1,\cdots, x_n] : f(X) = 0 \} \triangleleft [x_1,\cdots,x_n] \) 

If \(X=k^n\) define the coordinate ring of \(X\).

\(\Gamma[X] = \frac{k[x_1,\cdots, x_n]}{I(X)}\) 

Then each \([f]\in\Gamma[X]\) defines \(f:X\to k\) by \([f](X)=f(x)\) 

\underline{Lemma}:

\begin{itemize}
    \item \(S \subset I(V(S))\) for any \(S \subset k[x_1,\cdots, x_n]\) 
    \item \(X \subset V(I(X))\) for any \(X \subset k^n\) 
    \item \(I(X)=I(V(I(X)))\) for ideal \(I \subset k[x_1,\cdots,x_n]\) 
    \item \(V(S)=V(I(V(S)))\) for any \(X \subset k^n\) 
\end{itemize}

Now, in 0 characeristics, \(V(f)=V(f^n)\).

In a general ring, \(\sqrt{I} = \{ a\in A:a^n\in I \} \) and \(Nil(A) = \sqrt{0}\). \(A\) is a reduced ring \(Nil A = 0\) aka \(\sqrt{0} = 0\) 

\underline{Lemma}:

\(I(X)\) is a radical ideal and \(\Gamma (X)\) is a reduced ring.

\begin{definition}
    A variety \(V\) is reducible if \(V = V_1 \cup V_2\) where \(V_i\) are varieties with \(V_i \neq V\) 
\end{definition}

eg \(V(x_1 x_2 )\) is reducible.

We also have the following lemama:

The following are equivalent:

\begin{itemize}
    \item \(V\) is irreducible
    \item \(\Gamma (V)\) is a domain
    \item \(I(V)\) is a prime ideal
\end{itemize}   

from definiton we already have 2nd iff 3rd.

In AM: variety, in others algebraic set.

AM: irreducible variety, in others variety

Important theorems: Hilbert Basis ts,heorem and Nullstellensatz.

\begin{theorem}
    [Hilbert Basis Theorem]

    Any ideal \(I \triangleleft k[x_1,\cdots,x_n]\) is finitely generated.

    Equivalently, any variety is intersection of finite number of hypersurfaces.
\end{theorem}

Questin: is a variety the intersection of \(n\) hypersurfaces?

\begin{theorem}
    [Nullstellensatz]

    Explanation: zero place theorem. We need \(k\) be algebraically closed.

    \begin{enumerate}
        \item (Weak Nullstellensatz) Let \(I \triangleleft k[x_1,\cdots,x_n]\) be a proper idal. Then \(V(I)\) is non-empty.
        \item (Strong Nullstellensatz) \(I(V(I)) = \sqrt{I} \)
    \end{enumerate}
\end{theorem}

\hrulefill

Exam review recitation

6 question.

We have \(ED \implies PID \implies UFD\) 

PID is nicer than UFDs. If all primes are maximal ideals, then UFD is a PID.

Example: \(\mathbb{Z}[x]\) is not a PID, since \(\mathbb{Z} [x]/(x)=\mathbb{Z}\) which is not a field but an integral domain, which means \((x)\) is prime but not maximal. \((2,x)\) contains it, for example.

Galois theory:

Cyclotomic polynomial: \(\Phi_n(x)\) product of (x - primitive roots)

eg find \(\Phi_{10}(x)\)

We have \(\Phi_{10}(x)|x^{10} - 1 = x^{10} - 1 = (x^5 - 1)(x^5 + 1) = (x^5 - 1)(x + 1)(x^4 - x^3 + x^2 - x + 1) \) 

So we have \(x^4 - x^3 + x^2 - x + 1\) 

order of galois group is also \(4\). So it is \(\mathbb{Z}_4\) or \(\mathbb{Z}_2\times \mathbb{Z}_2\) 

\(Nil(A) =\) intersection of primes

\(Jac(A)\) intersection of max

\(A\cong A_1 \times A_2\) if and only if there exists nontrivial idempotent

\(Spec(A)\) is all prime ideals of \(A\).

\(f:A \to B\) gives us \(Spec(f):Spec(B) \to Spec(A)\)

This means Spec is a contravariant functor

Nakayama's Lemma.

If \(M\) if finitely generated and \(I \subset Jac(A)\) and \(IM=M\) then \(M = 0\) 

If \(A\) is a local ring with max ideal \(I\) and \(M\) finitely generated and \(\frac{A}{I} \otimes _A M = 0 \) then \(M = 0\) 

If \(k\) is a field and \(M,N\) are \(k\) algebra then \(M \otimes _k N\) is a \(k\) algebra, and so \((a \otimes b) (c \otimes d) = ac \otimes bd\) 

If \(M,N\) are finitely generated then \(M,N\) are finite dimensional with basis \(e_i\) and \(f_j\) respectively. \(M \otimes _k N\) is also a finite dimensional vector space with basis \(\{ e_i \otimes f_j \} \) 

\hrulefill

Class 25: 03/08

We skip chapter 4.

\section*{Chapter 5: Integrality/Valuations}

It's analogous to what we learned from fields.

Suppose we have a field extenstion \(K\) over \(k\).

\begin{definition}
    \(x\in K\) is \underline{algebraic} over \(k\) \(\iff \) it satisfies some polynomial, aka

    \[
        a_0 x^n + a_1 x^{n-1} + \cdots + a_n = 0 
    \]

    where all \(a_i\in k\) not all zero

    \(\iff \vert k[x] : k \vert < \infty\) 
\end{definition}

What about rings? Suppose we have \(x\in B\) where \(B\) is a ring over \(A\).

Throughout today, \(A\) is a subring of \(B\). It does contain the identity. We will also have \(x\in B\).

\begin{definition}
    \(x\in B\) is \underline{integral} over \(A\) if there exists an equation:

    \[
        x^n + a_1 x^{n-1} + \cdots + a_n = 0 
    \]

    with \(a_i\in A\).

    Notice that the polynomial is monic.

    i.e. \(x\) is a root of a monic polynomial in \(A[t]\).
\end{definition}

We can think of \(k\in K\), or we can look at \(\mathbb{Z} \subset \mathbb{Q}\), we can think of \(\mathbb{Z} \subset \overline{\mathbb{Q}} \) or \(\mathbb{Z} \subset \mathbb{Q} [i]\) or \(\mathbb{Q} [t] \subset \mathbb{Q} (t^\frac{1}{2})\) 

If we have \(k \subset K\) extension of fields, then \(x\) integral \(/ k\) is the same thing as saying \(x\) is algebraic \(/ k\), since we can divide by the leading term.

5.0 claim: \(x\in \mathbb{Q}\) integral over \(\mathbb{Z} \implies x\in\mathbb{Z}\) 

\begin{proof}
    Suppose \(x = \frac{r}{s}\) with \(r,s\) integers, \((r,s)=1\). So,

    \[
        (\frac{r}{s})^n + a_1 (\frac{r}{s})^{n-1} + \cdots + a_0 = 0 
    \]

    \[
        \implies r^n = - a_1 r^{n-1} s - \cdots - a_0 s^n 
    \]

    Thus, \(s\mid r^n\) but since \((r,s)=1\) this implies \(s\mid\pm 1\) which means \(x\in\mathbb{Z}\) 

\end{proof}

Now, consider \(\mathbb{Z} \subset \overline{\mathbb{Q}} \subset \mathbb{C}\).

Recall that \(\overline{\mathbb{Q}}\) is \(\{ x\in\mathbb{C} : x \text{ algebraic over } \mathbb{Q} \} \), these are called `algebraic numbers'

It is due to Gauss that \(\mathbb{C}\) is algebraically closed.

Define \(\mathbb{A} = \{ x\in\mathbb{C} : x \text{ integral over } \mathbb{Z} \} \). These are called `algebraic integers'. For example, \(\sqrt{2} \in \mathbb{A}\)

Note that \(\frac{1}{5}\notin \mathbb{A}\) 

Is \(\mathbb{A}\) a ring?

\begin{theorem}
    [Proposition 5.1]: we have \(A\) a subring of \(B\) and \(x\in B\). Then TFAE:

    \begin{enumerate}
        \item \(x\) integral / A
        \item \(A[x]\) is a finitely generated \(A\)-module
        \item There exists subring \(C\) such that \(A[x] \subset C \subset B\) and \(C\) is finitely generated as \(A\)-module.
        \item \(\exists A[x]\) module \(M\) which is faithfull (\(Ann(M)=0\)) and \(M\) is finitely generated.
    \end{enumerate}
\end{theorem}

\begin{proof}
    We prove \(1 \implies 2 \implies 3 \implies 1\) we will not use \(iv\). AM uses \(2.4\) [the determinant trick]

    Assume we have \(x^n + a_1 x^{n-1} + \cdots + a+0 = 0\).
    
    Claim: \(A[x]=(1,x,\cdots,x^{n-1})=I\). If this is true we have \(A[x]\) is finitely generated.

    Note that, \(x^n = -a_1 x^{n-1} - \cdots - \cdots - a_0\in I\).

    Multiplying by \(x^{n+r}\) we see that this is also in \(I\) by induction. Thus we are done.

    \(2 \implies 3\) just take \(C = A[x]\) 
    
    \(3 \implies 1\): Suppose \(C\) is generated by \(c_1,\cdots,c_n\). We have, \(xC \subset C\). This implies,

    \[
        x \begin{bmatrix}
             c_1 \\
             \vdots \\
             c_n \\
        \end{bmatrix} = [a_{ij}] \begin{bmatrix}
             c_1 \\
             \cdots \\
             c_n \\
        \end{bmatrix}
    \]

    Since \(xc_1 = a_{1 1} c_1 +\cdots + x_{1 n}c_n\) 

    Thus,

    \[
        \implies  (xI - (a_{i j})) \begin{bmatrix}
             c_1 \\
             \vdots \\
             c_n \\
        \end{bmatrix} = \begin{bmatrix}
             0 \\
             \vdots \\
             0 \\
        \end{bmatrix}
    \]

    But we need to prove the last implication. So we multiply by the \underline{adjugate} on the left. Just see proof of 2.4.

    Thus,

    \[
        dI \begin{bmatrix}
             c_1 \\
             \vdots \\
             c_n \\
        \end{bmatrix} = \begin{bmatrix}
             0 \\
             \vdots \\
             0 \\
        \end{bmatrix}
    \]

    This means \(dc_i = 0\) for all \(i\). Since \(1\in C, 1 = \sum_{i} f_i c_i\) so all can't be \(0\) Which means \(d = 0\) whih gives us a monic polynomial.

\end{proof}

Now we do a sequence of definitions which we'll use to answer whether the set of algebraic integers is a ring.

\begin{definition}
    \(B\) is \underline{integral} over \(A\) if every \(x\in B\) is integral. \(B\) can also be called an integral extension.
\end{definition}

\begin{definition}
    Integral closure of \(A\) in \(B\) 

    notation: \(I \subset (A \subset B) = \{ x\in B : x \text{ is integral / } A \} \) 

\end{definition}

\begin{definition}
    \(A\) is integral closed in \(B\) if \(A = I \subset (A \subset B)\). Exxample: \(\mathbb{Z}\) is integral closed in \(\mathbb{Q}\) 
\end{definition}

\begin{definition}
    A domain \(A\) is \underline{integrally closed} if it is integrally closed in \(Frac(A)\) eg \(\mathbb{Z}\) is integrally closed.
\end{definition}

Two obvious question about integrably closed \(I \subset (A \subset B)\):

One: Is this a ring? we are going to say yes, it's a consequence of the theorem.

Two: Is \(I \subset (A \subset B)\) integrally closed in \(B\)?

Corollary 5.2: Let \(x_1,\cdots x_n\in B\) be integral / \(A\). Then [this statement is stronger than AM]

\begin{enumerate}
    \item \(A[x_1,\cdots, x_n]\) is finitely generated \(A\)-module [this is in AM]
    \item \(A[x_1,\cdots x_n]\) is intgegral / \(A\)
\end{enumerate}

\begin{proof}
    1: by induction on \(n\). For \(n = 1\) we want to know if \(A[x_1]\) is a finitely generated \(A\)-module which is implied by \(5.1\). Assume \(A_{n-1} \coloneqq A[x_1,\cdots, x_{n-1}]\) is a finitely generated \(A\)-module. Then, \(A_{n-1}[x_n]\) is a finitely generated \(A_{n-1}\) module, and the previous things imply \(A_n\) is finitely generated \(A\)-module.
    
    2: Let \(x\in A[x_1,\cdots,x_n]\). Then, \(A \subset A[x] \subset A[x_1,\cdots, x_n]\). Since \(3\) implies \(1\) in 5.1 this implies \(x\) is integral over \(A\).
\end{proof}

Thus, \(x,y\) integral over \(A\) implies \(x+y,xy\) integral over \(A\).

Also, corollary: \(I \subset (A \subset B)\) is a ring.

corollary: \(\mathbb{A}\) is a ring.

\hrulefill

Class 26: 03/18

Subring \(A \subset B\) 

\(x \in B\) is \underline{integral} over \(A\) \(\exists\) monic \(f\in A[t]\) such that \(f(x)=0\) 

eg \(x\in A\) is a root of \(t - x\) 

\begin{definition}
    \(B\) is integral over \(A\) if \(\forall x\in B, x\) is integral over \(A\). For example, \(\mathbb{Z} [i]\) is integral over \(\mathbb{Z}\) since if \(x=a+ib\) then \((t-(a+ib))(t-(a-ib))\in\mathbb{Z} [t]\)     
\end{definition}

\begin{definition}
    Integral closure of \(A\) in \(B\) 

    \[
        I \subset (A \subset B) = \{ x\in B\mid x \text{ integral over } A \} 
    \]
\end{definition}

Corollary 5.3: \(I \subset (A \subset B)\)  is a ring

Corollary 5.4: `transitivity'

If we have \(B\) is integral over \(A\) and \(C\) is integral over \(B\) then \(C\) is integral over \(A\) 

\begin{proof}
    If \(x\in C\) then there exists:

    \[
        x^n + b_1 x^{n-1} + \cdots + b_n = 0 
    \]

    \(b_i \in B\). We had \(B^{\prime} = A[b_1,\cdots,b_n]\) a finitely generated \(A\)-module.

    \(B^{\prime} [x]\) is a finitely generated \(B^{\prime}\)-module, so \(B^{\prime} [x]\) is a finitely generated \(A\)-module.
    
    By 5.1(iii) we have \(x\) is integral over \(A\).
\end{proof}

We have

\[
    \begin{tikzcd}
        & & B \\
        & I \subset (A \subset B) \ar[ur, no head] & \\
        A \ar[ur, no head] & &
    \end{tikzcd}
\]

\begin{definition}
    \(A\) is \underline{integrally closed} in \(B\) if \(A = I \subset (A \subset B)\) 
\end{definition}

eg \(\mathbb{Z} \subset \mathbb{Q}\) is I.C.

eg \(\mathbb{Q} \subset \mathbb{R}\) not I.C.

\(\sqrt{2}\in I \subset (\mathbb{Q} \subset \mathbb{R}) \), \(\sqrt{2} \notin \mathbb{Q}\)

\(\mathbb{Z} \subset \mathbb{Z} [i]\) not I.C.

Corollary 5.5: \(I \subset (A \subset B)\) is I.C. in \(B\) 

\begin{proof}
    Let \(x\in B\) be integral over \(I \subset (A \subset B)\). We have:
    
    \[
        \begin{tikzcd}
            & & & B\\
            & & I \subset (A \subset B)[x] \ar[ur, no head, "int"] & \\
            & I \subset (A \subset B) \ar[ur, no head, "int\quad 5.2"] & & \\
            A \ar[ur, no head, "int"] & & &
        \end{tikzcd}
    \]

    5.4 implies \(I \subset (A \subset B)[x]\) is integral over \(A\) which means \(x\) is integral over \(A\) 
\end{proof}

Proposition 5.6: Suppose \(B\) is integral over \(A\). Then,

i: \(J \triangleleft B, I = A\cap J\) implies \(B / J\) is integral over \(A / I\) 

ii: If \(S\) is multiplicatively closed subset of \(A\) then \(S ^{-1} B\) is integral over \(S ^{-1} A\).

\begin{proof}
    \(\forall x\in B\)  we have the equation *:

    \[
        x^n  + a_1 x^{n-1} + \cdots + a_n = 0 
    \]

    for i: reduce * mod \(J\) 

    ii: Let \(\frac{x}{s}\in S ^{-1} B\) where \(x\in B\), then multiply * by \(s ^{-n} \)   

\end{proof}

Remark/Definition: \(B\) is an integral \(A\)-algebra if:

We have a ring homomorphism \(f:A \to B\) and \(B\) is integral over \(f(A)\)

We can define more generally over subrings.

Note: finite type + integral = finite

finite type means finitely generated as an \(A\)-module and finite means finitely generated as \(A\)-algebra.

This follows from 5.1 and 5.2.

Example: \(\mathbb{Z} / 6 [t]\) is not integral \(\mathbb{Z}\)-algebra. But \(\mathbb{Z} / 6\) is integral as \(\mathbb{Z}\)-algebra.

Recall, if we have a field \(k\) then \(x\) is algebraic over \(k\) if and only if \(k[x]\) is a field.

\section*{Going Up Theorem}

Suppose we have rings \(A \subset B\) and prime ideals \(P \triangleleft A\) and \(Q \triangleleft B\),

If we have \(P = Q^c = A \cap Q\) then:

\[
    \begin{tikzcd}
                           & B \ar[d, no head] \\
        Q \ar[d, no head] \ar[ru, no head] & A\\
        P \ar[ru, no head] & 
    \end{tikzcd}
\]

We say \(Q\) lies above \(P\)

We also say \(P\) lies below \(Q\).

We have:

\[
    \begin{tikzcd}
        & & \mathbb{Q} [i]\ar[d, no head]\\
        & \mathbb{Z} \ar[ur, no head] [i] & \mathbb{Q} \ar[u, no head] \\
        3\mathbb{Z}[i] \ar[ur, no head] & \mathbb{Z} \ar[ur, no head] \ar[u, no head] & \\
        3\mathbb{Z} \ar[u, no head] \ar[ur, no head] & &
    \end{tikzcd}
\]

Claim: \(3\mathbb{Z} [i]\) is prime in \(\mathbb{Z} [i]\)  

Proof:

\[
    \frac{\mathbb{Z} [i]}{(3)} = \frac{\mathbb{Z} [t] / (t^2 + 1)}{(3)} = \frac{\mathbb{Z} [t]}{(3,t^2 + 1)} = \frac{\mathbb{F}_3 [t]}{(t^2 + 1)} = \mathbb{F}_q
\]

Observation: If we have \(A \subset B\) and

\[
    \begin{tikzcd}
        & B \ar[d,no head]\\
        Q \ar[d, no head] \ar[ur, no head] & A\\
        P = Q\cap A \ar[ur, no head] & &
    \end{tikzcd}
\]

For all prime \(Q \triangleleft B\) there exists prime \(P\) lying under \(Q\).

\(A / P \hookrightarrow B / Q\) domain. 

Proposition 5.7: If \(A \subset B\) domains, \(B\) is integral over \(A\) then \(A\) is a field if and only if \(B\) is a field.

Proof: \(\implies\): \(0\neq y\in B\), choose smallest degree polynomial \(y^n + a_1 y^{n-1} + \cdots + a_n = 0\). Since domain, \(a_n\neq 0\) Solve for \(a_n\) and factor for \(y\). We have:

\[
    y \left[ \frac{-y^{n-1}-\cdots-a_{n-1}}{a_n} \right] = 1
\]

So \(y ^{-1}  \in B\) so we have field. 

\(\impliedby :\) supposse \(0\neq x\in A\). THen \(x ^{-1} \in B\) which is integral over \(A\) so we have polynomial:

\(x ^{-m} + a_1^{\prime} x^{-m-1} + \cdots + a_m^{\prime} = 0 \) 

Solve fore \(x^{-m} \) and multiply by \(x^{m-1} \) 

\[
    x ^{-1} = - (a_1^{\prime} + \cdots + a_m x^{m-1} ) \in A
\]

So a field.

\hrulefill

Class 27: 03/20

Recall

\begin{proposition}
    [AM 5.7]

    If \(A \subset B\) and \(B\) is a domain, \(B\) is integral over \(A\), then \(A\) is a field if and only if \(B\) is a field.
\end{proposition}

In the same spirit, if \(x\) is algebraic over \(k\) then \(k[x]\) is a field.

\underline{Corollary 5.8:} If we have

\[
    \begin{tikzcd}
        & B \\
        Q \ar[ur, no head] & A \ar[u, no head, "int", swap]\\
        P \ar[u, no head] \ar[ur, no head] &
    \end{tikzcd}
\]

Then \(Q\) maximal if and only if \(P\) maximal.

\begin{proof}
    \(B / Q\) is integral over \(A / P\) by 5.6.

    \(Q\) maximal iff \(B / Q\) field iff \(A / P\) field iff \(P\) maximal.
\end{proof}

\underline{Corollary 5.9:} 

\[
    \begin{tikzcd}
        & B\\
        Q \subset Q^{\prime} \ar[ur, no head] & A \ar[u, no head]\\
        P \ar[u, no head] \ar[ur, no head] &
    \end{tikzcd}
\]

\(Q\cap A = P, Q^{\prime} \cap A = P\) 

then \(Q = Q^{\prime}\) 

\begin{proof}
    [idea] Replace \(A\) by local \(A_P\) and use 5.8.

    \(S = A - P \subset A\). Then,
    
    \[
        \begin{tikzcd}
            & S ^{-1} B = B_P \\
            S ^{-1} Q \subset S ^{-1} Q^{\prime} \ar[ur, no head] & S ^{-1} A = A_P \ar[u, no head] \\
            S ^{-1} P \ar[u, no head] \ar[ur, no head] &
        \end{tikzcd}
    \]

    \(S ^{-1} P\) is maximal since \(A_P\) is local, which gives us \(S ^{-1} Q \subset S ^{-1} Q^{\prime} \) maximal by 5.8 which tells us \(S ^{-1} Q = S ^{-1} Q^{\prime} \). Therefore, \((S ^{-1} Q)^c = (S ^{-1} Q^{\prime} )^c \implies Q = Q^{\prime} \)     
\end{proof}

\begin{theorem}
    [AM 5.10]

    Suppose \(B\) is integral over \(A\) and \(P\) is a prime ideal of \(A\). Then there exists a prime \(Q\) lying over \(P\).

    Basically, we can complete the following:

    \[
        \begin{tikzcd}
            & B \\
            & A \ar[u, no head, "int", swap]\\
            P \ar[ur, no head] & 
        \end{tikzcd}
    \]
\end{theorem}

\begin{proof}
    Consider the following commutative diagram:

    \[
        \begin{tikzcd}
            B \ar[r, "\beta"] & S ^{-1} B \\
            A \ar[u, "i", tail] \ar[r, "\alpha"] & A_P = S ^{-1} A \ar[u, "j \text{ (int by 5.9)} ", tail, swap]
        \end{tikzcd}
    \]

    Let \(N \triangleleft S ^{-1} B\) be maximal. Then,
    
    \[
        \begin{tikzcd}
            Q = B ^{-1} N \ar[r, no head] & N \\
            Q \cap A \ar[u, no head] \ar[r, no head] & M \coloneqq N \cap A_P \ar[u, no head]
        \end{tikzcd}
    \]

    \(N\) exists by Zorn's lemma

    By 5.8 \(M\) is maximal.

    Now, \(Q\cap A = i ^{-1} \beta ^{-1} N = \alpha ^{-1} j ^{-1} N = \alpha ^{-1} M = \alpha ^{-1} (P A_P) = P\) since \(A_P\) local and \(PA_P\) maximal.

\end{proof}

Remark: \(Q\) may not be unique. See:

\[
    \begin{tikzcd}
        (2+i)\mathbb{Z}[i], (2-i)\mathbb{Z}[i] \ar[r, no head] & \mathbb{Z} [i]\\
        5\mathbb{Z} \ar[u, no head] \ar[r, no head] & \mathbb{Z} \ar[u, no head]
    \end{tikzcd}
\]

Note: \(5\mathbb{Z}[i] = ((2+i)\mathbb{Z}[i])((2-i)\mathbb{Z}[i])\) 

Now we have the \underline{Going-up theorem}.

\begin{theorem}
    [AM 5.11]

    Suppose we have integral extension \(B\) over \(A\) and we have chain of prime ideals \(P_1 \subset P_2 \subset \cdots P_n\) of \(A\) and chain of prime ideals \(Q_1 \subset \cdots \subset Q_m\) of \(B\) and \(\forall i \leq m, Q_i\cap A = P_i\) then we can extend the chain of \(Q\)'s to \(Q_1 \subset \cdots \subset Q_n\) such that for all \(i\), \(Q_i\cap A = P_i\).
    
    We basically have,

    \[
        \begin{tikzcd}
            Q_1 \subset \cdots \subset Q_n \subset \cdots \subset Q_m \ar[r, no head] & B \\
            P_1 \subset \cdots \subset P_n \ar[u, no head]\ar[r, no head] & A\ar[u, no head]
        \end{tikzcd}
    \]
\end{theorem}

\begin{proof}
    We use 5.10. Base case is 5.10.

    We want to define \(Q_{m+1}\). If \(n > m > 0\),

    Recall: quotient of integral extensions are integral by 5.6. Let \(\overline{B} = B / Q_m, \overline{A} = A / P_m\). So we have by 5.10
    
    \[
        \begin{tikzcd}
            \exists \overline{Q_{m+1}} \ar[r, no head] & \overline{B} = B / Q_m \\
            \overline{P_{m+1}}  \ar[u, no head] \ar[r, no head] & \overline{A} = A / P_m \ar[u, no head, "\text{int  by 5.6}", swap] 
        \end{tikzcd}
    \]

    Let \(Q_{m+1} = \overline{Q_{m+1}}^c\) and we are done. 

\end{proof}

\underline{Context}:

Krull dim \(A = \max \{ n \vert P_0 \subsetneq \cdots \subsetneq P_n \text{ primes in } A \} \) 

A corollary of 5.11 tells us, if \(B\) is integral over \(A\) then \(\dim B \geq \dim  A\).

\section*{Going down theorem}

\begin{proposition}
    [AM 5.12]

    Localization respects integral closure.

    Suppose \(B\) is integral over \(A\) and \(S \subset A\) is multiplicatively closed. Then, we have \(C = I \subset (A \subset B)\) the integral closure. Then, \(S ^{-1} C = I \subset (S ^{-1} A \subset S ^{-1} B)\) 
\end{proposition}

\begin{proof}
    5.6 implies \(S ^{-1} C \subset I \subset (S ^{-1} A \subset S ^{-1} B)\). We have one direction. For the other direction,
    
    Let \(\frac{b}{s} \in I \subset (S ^{-1} A \subset S ^{-1}  B)\). So there is some integral dependense relation:
    
    \[
        \left( \frac{b}{s} \right)^n + \left( \frac{a_1}{s_1} \right) \left( \frac{b}{s} \right)^{n - 1} + \cdots + \frac{a_n}{s_n} = 0   
    \]

    Let \(t = s_1 \cdots s_n\). Multiply the polynomial by \((st)^n\) so we have
    
    \[
        (bt)^n + \cdots + (st)^n \frac{a_n}{s_n} = 0
    \]

    Thus, \(bt\in C = IC(A \subset B)\)
    
    Thus, \(\frac{b}{s} = \frac{bt}{st} \in S^{-1} C\) 

    So we're done.
\end{proof}

\hrulefill

Class 28: 03/22

Recall the going up theorem.

(krull) \(\dim A = \max_n \{ \exists P_o \subsetneq P_1 \subsetneq \cdots \subsetneq P_n \triangleleft A \} \) 

dim field \(=0\) 

dim \(\mathbb{Z} = 1\)

dim \(k[x_1,\cdots, x_n] = n\) 

variety \(V, \dim V = \dim \frac{k[x_1,\cdots,x_n]}{I(V)}\) 

\underline{Corollary}:

If \(B\) is integral over \(A\) then \(\dim A = \dim B\)

Another corollary

\(A\) domain, \(\dim A = 0\) if and only if \(A\) field.

Simply:

Let \(A \subseteq B\).

Going up:

If \(B\) is integral over \(A\) 

\(P_1 \subseteq P_2 \subseteq \cdots \subseteq P_n\) in \(A\) 

\(Q_1 \subseteq Q_2 \subseteq \cdots \subseteq Q_m\) in \(B\) 

implies \(Q_1 \subseteq Q_2 \subseteq \cdots \subseteq Q_n\) 

Going down:

If \(B\) is integral over \(A\) and is a \underline{domain} 

\(P_1 \supseteq P_2 \supseteq \cdots \supseteq P_n\) in \(A\) 

\(Q_1 \supseteq Q_2 \supseteq \cdots \supseteq Q_m\) in \(B\)

implies \(Q_1 \supseteq Q_2 \supseteq \cdots \supseteq Q_n\) 

\begin{proposition}
    [AM 5.13]

    Let \(A\) be a domain. Integral closure is a local property. Meaning TFAE:

    \begin{enumerate}
        [label=\roman*]

        \item \(A\) is IC
        \item \(\forall\) prime \(P \triangleleft A, A_P\) is IC
        \item \(\forall\) maximal \(M \triangleleft A, A_M\) is IC.    
    \end{enumerate}
\end{proposition}

\begin{proof}
    Let \(k = Frac(A)\) and let \(C = I \subset (A \subset k)\)
    
    Inclusion: \(f:A\hookrightarrow C\)
    
    \(f_P : A_P \hookrightarrow C_p \coloneqq (A-P)^{-1}  \overset{5.12}{-} I \subset (A_P \subset k) \subset k\) 

    \(f_M : A_M \hookrightarrow C_M\) 

    Now, \(\begin{Bmatrix}
         A \\
         A_P \\
         A_M \\
    \end{Bmatrix}\) ic IC \(\iff \begin{Bmatrix}
         f \\
         f_P \\
         f_M \\
    \end{Bmatrix}\) is surjective  

    3.9 implies surjectivity is a local property so we are done.
\end{proof}

\newcommand{\Spec}{\operatorname{Spec}}

Going up iff \(\Spec B \to \Spec A\) is closed map

Going down iff \(\Spec B \to \Spec A\) is open map

\begin{definition}
    Suppose \(A \subset B\) rings and \(I \triangleleft A\) and \(x\in B\).
    
    \(x\) is \underline{integral} over \(I\), if there exists equation

    \[
        x^n + a_1 x^{n-1} + \cdots + a_n = 0
    \]

    where \(a_i\in I\).
\end{definition}

eg \(2\) is integral over \(4\mathbb{Z} \triangleleft \mathbb{Z}\) since \(2\) satisfies \(x^2 - 4\)

\begin{definition}
    Suppose \(B\) is integral over \(A\) and \(I \triangleleft A\). We can define the integral closure of this ideal.

    \[
        IC(I \subset B) \coloneqq \{ x\in B : x \text{ is integral over } I \} 
    \]
\end{definition}

\underline{Lemma 5.14:} Suppose \(B\) is integral over \(A\)  and \(I \triangleleft A \). Then

\[
    \begin{tikzcd}
        & & & B\\
        IC(I \subset B) & \triangleleft & IC (A \subset B) \ar[ur,no head] &\\
        & A \ar[ur, no head] & &
    \end{tikzcd}
\]

In fact \(IC(I \subset B) = \sqrt{I^e} = \sqrt{IC(A \subset B)I}  \) 

\begin{proof}
    \(C\coloneqq IC(A \subset B)\) ring

    \(IC(I \subset B)=\sqrt{I^e} \overset{def}{=} \sqrt{CI} \)  
    
    Want to show \(IC(I \subset B)=\sqrt{CI}\)
    
    \(\subseteq :\) 

    \(x\in IC(I \subset B)\)
    
    ie \(\exists x^{n+a_1 x^{n-1} + \cdots + a_0 = 0} \) where \(a_i\in I\) 
    
    Thus \(x^n\in CI\) 

    So \(x\in \sqrt{CI}\) 

    \(\supseteq :\)
    
    \(x\in \sqrt{CI}\)
    
    So \(x^n = \sum_{i} a_i c_i\) where \(a_i\in I, c_i\in C\)
    
    5.2 implies,

    \(M\coloneqq A[c_1,\cdots,c_p]\) finitely generated \(A\)-module.
    
    and \(x^n M \subset I M\)
    
    by 2.4 \(x^n\) integral over \(I\) 

    So \(x\) integral over \(I\)

    So we're done.
\end{proof}

\hrulefill

Class 29: 03/25

\underline{Special Case of 5.14} 

\[
    \begin{tikzcd}
        & & B \ar[ld, no head] & \\
        I \triangleleft & A & & = IC(A \subset B)
    \end{tikzcd}
\]

Then \(IC(A \subset B) = \sqrt{I}\) 

\begin{proposition}
    [AM 5.15]

    Suppose \(B\) is a domain, \(B\) is integral over \(A\) and \(A=IC(A \subset B)\). Let \(I \triangleleft A\)
    
    Then \(x\in B\) is integral over \(A\).

    Suppose \(K = \Frac(A)\). Then \(\operatorname{Irr}_K(x)(t) = t^n + a_1 t^{n-1} + \cdots + a_n \in K[t]\) [minimal polynomial] satisfies \(a_i\in \sqrt{I}\).  
\end{proposition}

\begin{proof}
    \(x\) integral over over \(I\) implies \(x\) is algebraic over \(K\).

    Let \(L\) be a splitting field for \(\operatorname{Irr}(x) \).
    
    \(\operatorname{Ir r}(x) = \prod (t - x_i)\)
    
    \(x_i\)'s are the conjugates of \(x\) 

    \underline{Claim}: \(x_i\) are integral over \(I\).

    This is becaue they are conjugates of \(x\).

    \(x\) is integral over \(I\) means there exists monic polynomial \(g(t)=t^k + b_1 t^{k-1} + \cdots + b_k \) where \(b_k \in I, g(x)=0\) where \(\operatorname{Ir r}(x) \mid g\)
    
    So \(g = h \operatorname{Ir r}(x) \) 

    So \(g(x_i) = h(x_i)(\operatorname{Irr}(x)(x_i))=0\) so \(x_i\) is integral over \(I\).
    
    Now,

    \(\prod (t -x_i) = \operatorname{Ir r}(x) = t^n + a_1 t^{n-1} + \cdots + a_n \)
    
    \(a_j\) are polynomials in \(x_1,\cdots, x_n\in IC(I \subset B)= \sqrt{I}\)

    [elementary symmetric polynomials]

\end{proof}

This will be useful in hw.

Useful even when \(I = A\)

Note: integral means some monic polynomial with good coefficient exists, this proposition lets us take the minimal polynomial.

\begin{proposition}
    [AM 5.17]

    Let \(A\) be a domain.
    
    \[
        \begin{tikzcd}
            IC(A \subset L) = B \ar[r, no head] & L \\
            IC(A \subset K) = A \ar[u, no head] \ar[r, no head] & K = \Frac(A) \ar[u, no head]
        \end{tikzcd}
    \]

    Where \(L\) is a field with characteristic zero.

    Suppose \(n = \vert L : K \vert < \infty\) 

    Then there exists basis \(u_1,\cdots u_n\) and \(v_1,\cdots, v_n\) of \(L / K\) such that:

    \[
        \sum Au_i \subset B \subset \sum A v_i
    \]
\end{proposition}

If we have

\[
    \begin{tikzcd}
        K\cap\mathbb{A} = \mathcal{O}_k \ar[r, no head] & K \\
        \mathbb{Z} \ar[u, no head] \ar[r, no head] & \mathbb{Q} \ar[u, no head]
    \end{tikzcd}
\]

5.17 implies \(\mathcal{O}_K\) is free abelian of rank \(n\).

\(\mathcal{O}_K \cong \mathbb{Z}^n\) as abelian group. 

\underline{Trace} 

Suppose \(\overline{K}\) is the algebraic closure of \(K\).

Suppose \(x\in \overline{K}\) is algebraic over \(K\).

\newcommand{\Tr}{\operatorname{Tr}}

Define \(\Tr(x)=\sum_{i} x_i\) 

Now, suppose:

\[
    \begin{tikzcd}
        & \overline{K} \ar[d, no head]\\
        x\in & L\ar[d, no head,"\text{finite}"] \\
        & K
    \end{tikzcd}
\]

then \(\Tr_{L / K}(x) = \vert L:K(x) \vert \Tr (x) \overset{fact}{=} \Tr(L \overset{\cdot x}{\to} L)\) 

\(T=\Tr_{L / K}: L \to K\) is \(K\)-linear. 

\(\Tr(1)=n\), the dimension.

Also,

\[
    \prod_{i=1}^d (t - x_i) = \operatorname{Ir r}(x) = t^d + a_1 t^{d-1} + \cdots 
\]

So \(T(x) = \Tr_{L / K}(x) = (\frac{n}{d})(-a_1)\) 

\underline{Corollary of 5.15: [special case where \(I = A\)]} 

\(\Tr_{L / K}(B) \subset A\).

\underline{Claim 1}: \(\forall v\in L, \exists a \neq 0 \in A\) such that \(av\in B\).

\underline{Proof}: \(v\) is algebraic over \(K\) so we have some polynomial

\[
    v^n + \frac{a_1}{b_1} v^{n-1} + \cdots + \frac{a_r}{b_r}  0
\]

where \(a_i\in A\)

multiply by \(a = (\prod b_i)\)

multiply by \(a^r\)

\((av)^r + a \frac{a_1}{b_1} (av)^{r-1} + \cdots + (a)^r \frac{a_r}{b_r}=0\)   

So \(av\in B\) 

\underline{Claim 2:}

There exists basis \(u_1,\cdots, u_n\) of \(L / K\) such that \(u_i\in B\) 

\begin{proof}
    Let \(w_1,\cdots,w_n\) be basis of \(L / K\). By claim 1, choose \(a_i\) such that \(a_i w_i = B\) 

    Let \(u_i = aw_i\) 
\end{proof}

\hrulefill

Class 30: 03/27

Correcting previous class:

If \(B\) is a domain, \(I \triangleleft A\) and \(B\) is integral over \(A\) and \(K = \Frac(A)\) 

\(A\) is integrally closed (ie \(A = IC(A \subset K)\))

and \(x\in B\) is integral over \(I\)

Then,

If \(\operatorname{Irr}_K(x)=t^n + a_1 t^{n-1} + \cdots + a_n \)

Then \(a_i \in \sqrt{I}\).

We want to prove 5.17, which is important in algebraic number theory.

\begin{proposition}
    [AM 5.17]

    \[
        \begin{tikzcd}
            B \ar[r, no head] \ar[d, no head] & L\\
            A \ar[r, no head] & K \ar[u, no head]
        \end{tikzcd}
    \]

    Where \(L,K\) fields, \(n = \vert L : K \vert \) and \(char=0\) and \(K =\Frac(A), B=IC(A \subset L)\) 

    \underline{Then} \(\exists\) bases \(\begin{matrix}u_1,\cdots,u_n\\ v_1,\cdots, v_n\end{matrix}\)  of \(L / K\) such that \(\sum A u_i \subset B \subset \sum A v_i\) 
\end{proposition}

\underline{Lemma 5.14}: Suppose \(I \triangleleft A\) and \(B\) is integral over \(A\). Let \(C = IC(A \subset B)\). Then \(IC(I \subset B) = \sqrt{CA} \)   

5.14 implies,

1. \(IC(I \subset B)\) is closed under \(+\) and \(\times\)

2. Special case(A = B):

\(IC(I \subset A) = \sqrt{I}\)

eg \(IC(4\mathbb{Z} \triangleleft \mathbb{Z})= \sqrt{4\mathbb{Z}} \)

\(2\) is root of \(t^2 - 4\) 

\underline{Special case of 5.15}: \(A = I\) 

If \(x\in B\) is integral over \(I=A\) then coefficients lie on \(A\).

Similar to HW.

\begin{proof}
    \(\operatorname{Ir r}_K(x) = \prod (t-x_i)\in L[t]\)
    
    \(x_i\) are conjugates.

    \(\operatorname{Ir r}_K(x)=\operatorname{Ir r}_K(x_i) \) [we proved in last class].

    This implies \(x_i\) are integral \(/A\) 

    So \(x_i\in IC(I \subset B)\)
    
    So, 5.14 closed implies \(a_i = \sigma _i(x_1,\cdots,x_n)\in IC(I \subset B)\cap A =_{AIC} IC(I \subset A) =_{5.14} \sqrt{I} \) 
      
\end{proof}

trace \(T = \Tr_{L / K} : L \to K\)

\(K\)-linear

\(T(1)=n\)

\(T(B) \subset A\) by 5.15.

\underline{Claim}: \(\exists\) basis \(u_j\) for \(L / K\) such that \(\sum_{i} A u_j \subset B\)

\begin{proof}
    Clear denominator
\end{proof}

\underline{Claim}: there exists basis \(\{ v_i \} \) of \(L / K\) such that \(T(v_i u_j) = \delta _{i j}\)

\begin{proof}
    Define \(\beta : L \times L \to K\) by \(\beta(x,y) = T(xy)\). This is a \(K\)-bilinear form.
    
    So we have \(Ad \beta : L \to L^{\ast} = Hom_K(L,K)\) by \(x \mapsto (y \mapsto \beta(x,y))\)  

    \(Ad \beta \) is \(1-1\) ie \(B\) is non-degenerate:
    
    \(x\neq 0\) means \(Ad \beta (x)(x ^{-1}) = \Tr(x x ^{-1}) = \Tr (1) = n\neq 0\) so it is actually non-degenerate.
    
    So, \(Ad \beta : L \to L^{\ast}\) is actually an isomorphism.
    
    Let \(\hat{u} _i\) be the \(u_i\) dual basis of \(L\) 
    
    ie \(\hat{u}_i (u_j) = \delta_{i j}\) 

    Let \(v_i = (Ad \beta) ^{-1} \hat{u}_i\) 
\end{proof}

Now we prove \(B \subset \sum_{i} Av_i\) 

\begin{proof}
    Consider \(x\in B \implies x = \sum_{i} k_i v_i\), where \(k_i\in K\) 
    
    Note that \(x u_j \in B\) 

    5.15 means \(A\ni T(x u_j)=T(\sum_i k_i v_i u_j)=k_j\) so we're done. 
\end{proof}

\section*{Valuation Rings}

Let \(B\) be a domain and let \(K = \Frac(B)\) 

\begin{definition}
    \(B\) is a \underline{valuation ring} of \(K\) if \(x\in K^\times \implies x\in B\) or \(x ^{-1} \in B\)  
\end{definition}

Basically \(B \cup (B-0)^{-1} = K\) 

eg \(\mathbb{Z}_{(p)}\) is a valuation ring.

But \(\mathbb{Z}\) not a valuation ring.

Suppose \(A\) is a domain and \(K = \Frac(K)\) 

Then \(IC(A \subset K) = \cap_{A \subset B \subset K, B \text{ valuation ring}} B\) 

\begin{proposition}
    [AM 5.18]

    Let \(B\) be a valuation ring (over \(K\)). Then,

    i: \(B\) is a local ring

    ii: \(B \subset B^{\prime} \subset K\) implies \(B^{\prime}\) is a valuation ring.
    
    iii: \(B\) is integrably closed.

\end{proposition}

\begin{proof}
    i: Let \(M = B - B^\times\), non-units.
    
    We want to show \(M\) is an ideal, then we're done.

    \(M= \{ 0 \} \cup \{ x\in B : x ^{-1} \notin B \} \)

    First: \(M\) is closed under multiplication by \(B\).

    If \(a\in B,m\in M\), for contradiction assume \(am\notin M\). Then \((am)^{-1} \in B\) [since valuation ring], so \(m ^{-1} = a (am) ^{-1} \in B\) so we have contradiction.
   
    Second: \(M\) is closed under addition.

    \(x,y\in M - 0\) then \(x y ^{-1} \in B\) or \(x ^{-1} y\in B\) since valuation ring.
    
    WLOG \(x y ^{-1} \in B\)
    
    Then \(x+y = (1 + x y ^{-1}) y \in M\) 

    ii is clear

    iii: Suppose \(x\in K\) is integral over \(B\).

    Then \(x ^n + b_1 x^{n-1} + \cdots + b_n = 0 \)
    
    If \(x\in B\) then we're done.

    If \(x ^{-1} \in B\) then solve for \(x^n\) and multiply by \(x^{1-n}\) so \(x\in B\) 

\end{proof}

\hrulefill

Class 31: 03/29

\underline{One to One Correspondence Between Valuation and Valuation Ring} 

Ex 30 and 31

A \underline{valuation} on field is homomorphism \(v: K^{\ast} \to \Gamma\) where \(\Gamma\) is totally ordered abelian group such that \(v(x+y) \geq \min(v(x),v(y))\).

If \(\Gamma\) is discrete it is called a discrete valuation. 

\underline{Totally Ordered Abelian Group:} We have order, and \(a \geq b \implies a+c \geq b+c \forall c\). 

eg p-adic valuation on \(\mathbb{Q}\).

\(v_p:\mathbb{Q}^\times \to \mathbb{Z}\) given by:

\(v_p(p^k \frac{a}{b})=k\) where \(p\nmid a,b\)

Note: if we define for \(x\in\mathbb{Q}\),  \(\vert x \vert_p \coloneqq p^{- v_p(x)}\) it is like an absolute value. It is actually a non-archemedian absolute value.

We can define a metric: \(d_p(x,y) = \vert x - y \vert_p\)

Completion of this metric space is the \(p\)-adic numbers \(\mathbb{Q}_p\) 

Valuation \(\longleftarrow\) valuation ring

\(v: K^{\times} \to \Gamma \mapsto B = \{ x\in K : v(x) \geq 0\} \) 

\(\Gamma = K^\times / B^\times \leftarrow B\) 

\([x] \geq [y] \overset{def}{\iff} xy^{-1} \in B \)

\(B = v_p ^{-1} [0, \infty) = \mathbb{Z}_{(p)}\) 

Now, let \(K\) be a field and \(\Gamma\) be an algebraically closed field.

\(\Sigma = \{ (A,f) | A \subset K \text{ subring } f:A \to \Gamma \text{ homomorphism} \} \) 

Poset \((A,f) \leq (A^{\prime} ,f^{\prime} ) \overset{def}{\iff} A \subset A^{\prime} , f^{\prime} |_A = f \) 


Zorn's lemma implies: \(\exists \max (B,g)\in \Sigma\)

eg suppose \(\Omega = \overline{K}\). Then \(B=K\).

eg \(K=\mathbb{Q}\) implies \(\Omega = \overline{\mathbb{F}}_p = \cup \overline{\mathbb{F}}_{p^r}\)  

\begin{theorem}
    [AM 5.21]

    Let \(K\) be a field and \(\Omega\) be algebraically closed field. Let \((B,g)\) be maximal element of \(\Sigma\). Then \(B\) is a valuation ring of \(K\).  
\end{theorem}

\underline{Lemma 5.19}: \(B\) is a local ring with maximal ideal \(M = \ker g\)

\begin{proof}
    \(M = \ker g\)
    
    So \(B / M \cong g(B) \subset \Omega \implies M\) prime since \(g(B)\) is a domain
    
    \(g(B-M) \subset \Omega ^ \times \) 

    \(\implies \overline{g} \) such that \((B_M, \overline{g}) \geq (B,g)\) maximal.
    
    So \(B_M = B\)
    
    \(M\) maximal since \(B_M\) is local.
    
\end{proof}

\underline{Lemma 5.20}: Let \(x\in B^\times\).

Then \(M[x] \neq B[x]\) or \(M[x ^{-1} ] \neq B[x ^{-1}]\) 

Note: \(M[x] = M \{ 1,x,x^2,\cdots \} \triangleleft B[x] \) 

\begin{proof}
    Of 5.20

    By contradiction.

    Suppose \(M[x]=B[x]\) and \(M[x ^{-1}] = B[x ^{-1}]\) 

    ie \(1\in M[x], 1\in M[x ^{-1}]\)
    
    Then 1: \(u_0 + u_1 x + \cdots + u_m x^m = 1\) [\(u_i\in M\)]

    Also 2: \(v_0 + v_1 x ^{-1} + \cdots + v_n x^{-n}\) [\(v_\in M\)]
    
    WLOG \( m \geq n\) with \(m,n\) minimal.

    3: \(v_1 x^{n-1} + \cdots + v_n = (1-v_0)x^n \)
    
    \(1-v_0\in B - M\) [since \(M\) proper] \( \implies (1-v_0)^\times \in B\) since \(M\) maximal.

    4: \(w_0 + w_1 x + \cdots + w_m x^{m-1}=x^m\) 

    Plug 4 to 1, that contradits minimality. 
    
\end{proof}

Finally, we prove 5.21.

\begin{proof}
    \(x\in K ^\times\)
    
    WTS \(x\in B\) or \(x ^{-1} \in B\) 

    \(M = \ker g \triangleleft B\). Maximal.
    
    WMA (5.20) \(M^{\prime} = M[x] \trianglelefteq B[x]\eqqcolon B^{\prime} \)
    
    \(M = M^{\prime} \cap B\) since \(M\) maximal.
    
    \(B / M\) is a field.

    So we have map \(B / M \to \Omega\)
    
    If we go to \(B^{\prime} / M^{\prime}\) 
    
    By universal property, \(B^{\prime} / M^{\prime} = B / M [\overline{x}]\) 

    So \(\overline{x}\) algebraic over \(B / M\) 

    \(B^{\prime}  \to B^{\prime} / M^{\prime} \to \Omega\)
    
    \((B,g)\in \Sigma\) maximal implies \(B = B^{\prime} = B[x] \implies x\in B\).  

\end{proof}

\hrulefill

Class 32: 04/01

Let \(K\) be  field and \(\Omega\) be an algebraically closed field. 

Consider the poset \(\Sigma = \{ (B,g) | B \subset K, g:B \to \Omega \} \) 

Since the set has an upper bound, every chain has an upper bound.

Using Zorn's Lemma, there must exist a maximal element.

Most interesting case: \(K = \mathbb{Q}\) and \(\Omega\) is the algebraic closure of \(\mathbb{F}_p\)   

Theorem 5.21 states that if \((B,g)\) is a maximal element then \(B\) is a valuation ring.

Meaning \(x\in K^\times \implies x\) or \(x ^{-1}\in B\)  

\underline{Corollary 5.22}: Let \(A\) be a subring of the field \(K\). Then the integral closure of \(A\) in \(K\) [\(IC(A \subset K)\)] is the intersection of valuation ring containing \(K\) 

\[
    IC(A \subset K) = \bigcap_{A \subset B \subset K\text{, \(B\) valuation ring} } B
\]

Example: \(IC(\mathbb{Z} \subset \mathbb{Q})=\bigcap \mathbb{Z}_{(p)}\) 

\begin{proof}
    \(\subset :\) Let \(B\) be a valuation ring. 5.18(iii) implies \(B\) is IC \(\implies IC (A \subset K) \subset B\) so we're done with this direction.
    
    \(\supset :\) Suppose \(x\in IC(A \subset K)\). Note that \(x\) integral if and only if \(x\in A[x^{-1}] = A^{\prime}\).
    
    Thus, \(x ^{-1} \in A^{\prime}\) but is not a unit.
    
    Thus there exists a maximal ideal containing \(x ^{-1}\). So, \(x ^{-1} \in M^{\prime}\).
    
    We have a map \(A^{\prime} \to A^{\prime} / M^{\prime} = k^{\prime} \) since \(M^{\prime}\) is a maximal ideal. We can include \(k^{\prime} \hookrightarrow \overline{k^{\prime}}=\Omega\) the integral closure.
    
    Note that \(x ^{-1} \mapsto 0\)
    
    Extend to maximal \((B,g)\)
    
    \(g(x ^{-1}) = 0 \implies x ^{-1} \text{ not a unit in } B \implies x\notin B\)

    \(B\) is a valuation ring by theorem 5.21
    
\end{proof}

\begin{proposition}
    [AM 5.23]

    Suppose \(B\) is extension of ring \(A\) and \(B\) is finitely generated over \(A\), meaning \(B = A[x_1,\cdots, x_n]\) where \(x_j\) are elements of \(B\) and \(B\) is a domain.

    Suppose \(0\neq v\in B\).

    Then \(\exists u\in A\) such that:
    
    For any homomorphism \(f:A \to \Omega\) where \(\Omega\) is algebraically closed with \(f(u)\neq 0\)
    
    \(\exists g:B \to \Omega\) such that \(g|_A = f\) and \(g(v)\neq 0\)   
\end{proposition}

\begin{proof}
    By induction. Key case: \(n=1\) 

    Assume \(B = A[x]\) 

    Consider \(v\in B\) nonzero.

    Two cases: \(x\) is not algebraic over \(A\).

    Let \(v = a_0 x^n + \cdots + a_n\in B\) 

    Let \(u = a_0\) 

    \(\forall f:A \to \Omega\) such that \(f(u)\neq 0\) we can define:

    \(f(a_0)t^n + \cdots + a_0\in \Omega [t]\)
    
    has \(n\) roots.

    Choose nonroot \(\xi \in \Omega\)
    
    define \(g: B \to A[x] \to \Omega\) by \(g(x)=\xi\)
    
    Then \(g(v)\neq 0\) and we're done.

    Case ii: \(x\) is algebraic over \(A\) 

    Then \(\Frac(A)[x]\) is a field
    
    Then \(v ^{-1}\) is algebraic over \(A\)
    
    \(x\) algebraic.

    1: \(a_0 x^m + \cdots + a_m = 0\) 

    \(v ^{-1}\) algebraic
    
    2: \(a_0^{\prime} v^{-n}+\cdots + a_n^{\prime} \)
    
    Let \(u = a_0 a_0^{\prime} \) 

    Let \(f:A \to \Omega, f(a_0 a_0^{\prime}) \neq 0\)
    
    \(f\) extends to \(f_1:A[u ^{-1}] \to \Omega\) by \(f_1(u ^{-1}) = f(u)^{-1}\)  

    Now, \((A[u ^{-1}], f_1) \leq (C,h)\)
    
    Let \(g = h|_B\) 

    \(5.21 \implies C\) is valuation ring, \(5.18 \implies  C\) integrally closed.
    
    \(1 \implies x\) integral over \(A[u ^{-1}] \implies x\in C \implies B \subset C\)
    
    \(2 \implies v ^{-1}\) integral over \(A[u ^{-1}] \implies v ^{-1} \in C \implies v \text{ is a unit in } C \implies h(v) \neq 0 \implies g(v)\neq 0\)  

\end{proof}

We have a nice corollary:

\underline{Corollary 5.24 [Zariski's Lemma]}: Suppose we have a field \(k\) and a polynomial ring \(B = k[x_1,\cdots,x_n]\) field. Then \(\vert B : k \vert = \dim _k B < \infty\) 

\begin{proof}
    Apply 5.23 with \(v=1, \Omega = \overline{k} , f =\) inclusion.
    
    So, \(\exists g:B \to \overline{k}\), injective since \(B\) field 
    
    \(k \subset g(B) \subset \overline{k} \)
    
    \(g(B)=B\) 

    \(B / k\) algebraic and \(B\) finitely generated implies \(\vert B : k \vert < \infty\) 
\end{proof}

\underline{Corollary}: Weak Nullstellensatz (HWK)

Let \(k\) be algebraically closed. Let \(I\) be a proper ideal of \(k[t_1,\cdots,t_n]\)

Week: \(V(I)\neq \varnothing\)

Strong: \(I(V(I))=\sqrt{I}\) 

`counterexample': \(I = (x^2 + 1) \triangleleft \mathbb{R} [x]\) but then \(V(I)=\varnothing\). This is not really a contradiction since \(\mathbb{R}\) is not algebraically closed. 

\hrulefill

Class 33: 04/03

\section*{Chapter 6 Chain Conditions}

\begin{proposition}
    [AM 6.1]

    Let \((\Sigma , \leq )\) be a poset.

    TFAE:

    i: Ascending chain condition (acc): Every \(x_1 \leq x_2 \leq x_3 \leq \cdots\) is stationary: \(\exists n\) such that for all \(i,j\geq n\) we have \(x_i = x_j\)  

    ii: Maximal condition: every \(\phi \neq T \subset \Sigma\) has a maximal element.  
\end{proposition}

\begin{proof}
    \(\lnot ii \implies \lnot i\):
    
    \(0\neq T\) no max. \(x_1\in T\). \(x_1 < x_2 < x_3 \cdots\) 
    
    \(ii \implies i\): \(x_1 \leq x_2\leq \cdots\) has a maximal \(x_n\) 
\end{proof}

\begin{definition}
    Module \(M\) is Noetherian/Artinian

    if (\(\Sigma = \) submodules, \(\subseteq\)) satisfies acc (iff maximal condition) / dcc (iff minimal condition)
\end{definition}

\begin{definition}
    Ring \(A\) is Noetherian/Artinian if it is so as an \(A\)-module.
\end{definition}

Remark: Submodule of \(A \iff\) ideal of \(A\). So we are talking about chains of ideals.

eg \(A = k\) field, \(M = V\) vector space.

If \(V\) is finite dimensional, it satisfies this condition.

Call length of chain \(l(x_0 < x_1 < \cdots < x_n) = n\). Then length of chain of \(n\)-dimensional vector space is \(n\).

So, \(\dim V < \infty \iff V\) is N.

\(\mathbb{R}^{\infty}\) satisfies dcc,  but not acc. So it is Artinian.

\(k=\mathbb{Q} , V = \mathbb{R}\) is not Artinian.

eg \(A = \mathbb{Z}\) or any PID. It is Noetherian, but not Artinian.

eg \((6) \subset (2) \subset (1)\) and \((6) \subset (3) \subset (1)\)

eg \((2) \supset (4) \supset (8) \supset \cdots\) 

If \(A =\) field, then it is noetherian and artinian ring.

eg \(\mathbb{Z}\)-modules aka abelian groups.

\(\mathbb{Z}\) is noetherian not artinian.

\(\mathbb{Q} / \mathbb{Z}\) is artinian not noetherian.

\((1) \subset (\frac{1}{2}) \subset (\frac{1}{4}) \subset \cdots\) 

\(\mathbb{Q}\) is neither artinian nor noetherian [as a \(\mathbb{Z}\)-module]

eg \(k\)-algebras where \(k\) is a field.

If one variable \(k[t]\) then we are in a PID situation. So noetherian not artinian.

First: \(k[t_1,t_2,\cdots]\) infinite not Artinian, not Noetherian.

We have two other possibilities:

\(k[\alpha_1,\cdots,\alpha_n]\) finitely generated, and other possibility is finite type, \(\dim_k < \infty\) 

\(k[\alpha_1,\cdots,\alpha_n]\) finitely generated is Noetherian.

\(k[\alpha_1,\cdots,\alpha_n]\) finite type, \(\dim_k < \infty\) is Artinian and Noetherian.  

\begin{proposition}
    [AM 6.2]

    \(M\) is Noetherian \(A\)-module if and only if every submodule is finitely generated.
\end{proposition}

\underline{Corollary}: Ring \(A\) is Noetherian if and only if every ideal is finitely generated.

\begin{proof}
    \(\implies: P \subset M\) where \(M\) is noetherian.
    
    Let \(T = \{ \text{finitely generated submodules of } P \} \)
    
    \(6.1 \implies \exists\) maximal \(P_0 \in \Sigma\)
    
    So, \(\forall x\in P, (P_0,x) \subset P \implies (P_0,x)=P\).
    
    \(P_0\) maximal means \(x\in P\) 

    \(\impliedby\) suppose \(M_0 \subseteq M_1 \subseteq \cdots \subseteq M\).
    
    Take \(\bigcup M_i\). It is finitely generated.
    
    Choose \(n \gg 0\) such that \(M_n\) contains all generators. 
\end{proof}

\begin{proposition}
    [AM 6.3]

    Consider \(SES 0 \longrightarrow M^{\prime} \overset{\alpha}{\longrightarrow} M \overset{\beta}{\longrightarrow} M^{\prime\prime} \longrightarrow 0\).
    
    i: \(M\) is noetherian if and only if \(M^{\prime} , M^{\prime\prime} \) are noetherian.
    
    ii: \(M\) is artinian if and only if \(M^{\prime}\) and \(M^{\prime\prime}\) are artinian.  
\end{proposition}

\begin{proof}
    i: ascending chain in \(M^{\prime}\) (or \(M^{\prime\prime}\)) gives ascending chain in \(M\) via \(\alpha\) (or \(\beta ^{-1}\)). Hence stationary.
    
    \(\impliedby\) let \(M_0 \subset M_1 \subset \cdots\) ac in \(M\). Then \(\alpha ^{-1} M_i\) is ac in \(M^{\prime} \) and \(\beta(M_i)\) is ac in \(M^{\prime\prime}\).
    
    So, there exists \(n\) such that \(\alpha ^{-1} M_i\) and \(\beta(M_i)\) are stationary at \(n\). Then \(\forall j > n\)
    
    \[
        \begin{tikzcd}
            0 \ar[r] & \alpha ^{-1} M_n \ar[d,"="] \ar[r] & M_n \ar[r] \ar[d] & \beta M_n \ar[d,"="] \ar[r] & 0 \\
            0 \ar[r] & \alpha ^{-1} M_j \ar[r] & M_j \ar[r] & \beta M_j \ar[r] & 0
        \end{tikzcd}
    \]

    By the five lemma, \(M_n = M_j\) 

\end{proof}

\underline{Corollary 6.4}: If \(M_1,\cdots, M_n\) are N/A then \(\oplus M_i\) is N/A 

\begin{proposition}
    [AM 6.5]

    If \(A\) is N/A ring and \(M\) is finitely generated, then \(M\) is N/A.
\end{proposition}

\begin{proof}
    \(M\) finitely generated iff we have a surjection \(A^n \rightarrow M\). 6.4 implies \(A^n\) is N/A. 6.3 implies \(M\) is N/A. 
\end{proof}

\hrulefill

Class 34, 35 skipped

Due to Ben:

\begin{proposition}
    \(I \triangleleft A, A\) is noetherian (resp Atinian) implies \(A / I\) is noetherian (resp artinian)
\end{proposition}

\begin{proof}
    \(A/I\) noetherian \(A\)-module implies \(A / I\) noetherian \(A / I\) module.    
\end{proof}

\underline{Composition Series}:

\(M = M_0 \supsetneq M_1 \supsetneq \cdots \supsetneq M_n = 0\) maximal

i.e. \(M_{i-1} \supsetneq M_i\) maximal proper, i.e. \(M_i / M_{i-1}\) simple.

eg \(\mathbb{Z} / 12\)  as a \(\mathbb{Z}\)-module.

\(\mathbb{Z} / 12 \supset 2 \mathbb{Z} / 12 \supset 6 \mathbb{Z} / 12 \supset 0\). Simple factors \(\mathbb{Z} / 2, \mathbb{Z} / 3, \mathbb{Z} /2\)

Neg: \(12\mathbb{Z}\) has no composition series.

\begin{proposition}
    [Jordan-Holder] Suppose \(M\) has a c.s. of length \(n\). Then,

    i: Any strict chain in \(M\) can be extended to a c.s.

    ii: Any two c.s. for \(M\) has the same simple factors up to isomorphism, hence the same length.
\end{proposition}

\begin{proposition}
    \(0 \to M^{\prime} \to M \to M^{\prime\prime} \to 0\) SES implies \(l(M) = l(M^{\prime}) + l(M^{\prime\prime} )\)  
\end{proposition}

\begin{proposition}
    \(M\) has a c.s. \(\iff\) \(M\) is noetherian and artinian. 
\end{proposition}

\begin{proof}
    Suppose \(M\) has a c.s. Then all strict chains have length \(\leq l(M) < \infty\).
    
    Thus, \(M\) is noetheian AND artinian.

    Suppose \(M\) is both noetherian and artinian.

    \(M_0 = M\) has a maximal proper submodule \(M\), since \(M\) is noetherian. Continue \(M_0 \supsetneq M_1 \supsetneq M_2 \supsetneq \cdots\) will be stationary since \(M\) is artinian. 
\end{proof}

\begin{proposition}
    \(V\) is a vector space over \(k\). Then TFAE:

    i: \(\dim V < \infty\)
    
    ii: \(l(V) < \infty\) in which \(\dim V = l(V)\) 
    
    iii: \(V\) is noetherian.

    iv: \(V\) is artinian
\end{proposition}

\begin{proof}
    i \(\implies\) ii clear
    
    ii \(\implies\) iii, iv by previous proposion
    
    \(\lnot\) i \(\implies \lnot\) iii, \(\lnot\) iv: If \(\dim V = \infty\) then there exists linearly independent \(x_1,x_2, x_3,\cdots\). Let \(u_n = span(x_1,\cdots,x_n)\) and \(v_n=(x_{n+1},x_{n+2},\cdots)\)     
\end{proof}

\underline{Corollary}: Ring \(A\), suppose \(\exists\) maximal \(M_1,\cdots,M_n\) with \(M_1\cdots M_n = 0\) 

Then \(A\) is noetherian \(\iff A\) is artinian.

\begin{proof}
    Consider the chain: \(0 = M_1\cdots M_n \subset M_1 \cdots M_{n-1} \subset \cdots \subset M_1 M_2 \subset M_1 \subset A\).
    
    Each factor \(\frac{M_1\cdots M_i}{M_1\cdots M_{i+1}}\) is a vector space over \(\frac{A}{M_{i+1}}\). Then acc \(\iff\) dcc for each factor. 
    
    Then acc for \(A \iff\) dcc for \(A\) 
\end{proof}

\section*{Chapter 7: Noetherian Rings}

\begin{proposition}
    \(A \rightarrowtail B\). Then \(A\) noetherian \(\implies B\) noetherian   
\end{proposition}

\begin{proof}
    \(B \cong A / \ker\) noetherian
\end{proof}

\begin{proposition}
    \[
        \begin{tikzcd}
            & B \text{ f.g. \(A\)-module}\\
            A \text{ Noetherian}  \ar[ur, hook] 
        \end{tikzcd} 
        \implies B \text{ is noetherian}
    \] 
\end{proposition}

\underline{Class 35}:

\underline{Lemma}: Ring \(A\) noetherian \(\implies S ^{-1} A\) noetherian.

\begin{proof}
    \(I \triangleleft A \implies I = (x_1,\cdots,x_n) \implies S ^{-1} I = (\frac{x_1}{I},\cdots,\frac{x_n}{I})\) 
\end{proof}

\underline{Corollary}: \(P\) prime, \(A\) Noetherian \(\implies A_P\) Noetherian. 

\begin{theorem}
    [Hilbert Basis Theorem]

    If \(A\) is Noetherian then so is \(A[t]\) 
\end{theorem}

\underline{Corollary}: \(A\) noetherian implies \(A[t_1,\cdots,t_n]\) is noetherian.

\underline{Corollary} \(A\) noetherian implies so is any finitely generated \(A\)-algebra.

\begin{proof}
    (Proof of Hilbert Basis theorem skipped. Look it up)
\end{proof}

\begin{proposition}

    \[
        \begin{tikzcd}
            & & C = A[x_1,\cdots,x_n] \text{ ``f.g. \(A\)-algeba''} \ar[ld, no head] \\
            & B \ar[ld, no head] & \\
            A \text{ noetherian} 
        \end{tikzcd}
    \]

    Suppose either i: \(C\) is f.g. as a \(B\)-module or ii: \(C\) is integral \(/ B\). Then \(B\) is f.g. as an \(A\)-algebra.

\end{proposition}

\begin{proof}
    Note: 5.1 + ii \(\implies\) i so we may assume i. 

    So, \(C = \sum_{j=1}^{m} By_j\). Then \(x_i = \sum_{j} x_{ij}y_j. y_i y_j = \sum_{k} b_{ijk}y_k\). Let \(B_0 = A[b_{ij},b_{ijk}]\).
    
    Let \(B_0 = A[b_{ij},b_{ijk}]\). We know \(C\) is f.g. as a \(B_0\) module/

    \(B_0\) noetherian \(\implies B\) is f.g. as a \(B_0\)-module (since (\(A\) noetherian \(\implies B\) noetherian.)) \(\implies B_0\) is f.g. as an \(A\)-algebra.
    
    \(\implies B\) f.g. as an \(A\)-algebra. 
\end{proof}

\begin{proposition}
    [Zariski Lemma]

    \[
        \begin{tikzcd}
            & E = K[x_1,\cdots,x_n] \ar[ld, no head]\\
            K
        \end{tikzcd}
    \]

    \(E\) field \(\implies \vert E : K \vert < \infty\) hence \(E / K\) alg. 
\end{proposition}

\begin{proof}
    After reordering \(x_1,\cdots,x_n\),

    \[
        \begin{tikzcd}
            & & E=K[x_1,..,x_n] \ar[ld, no head, "alg"] \\
            & K(x_1,\cdots,x_r) = K[z_1,\cdots,z_s] = K \left[ \frac{f_1}{g_1},\cdots,\frac{f_s}{g_s} \right]\ar[ld, "transcendental", no head] \\
            K
        \end{tikzcd}
    \] 

    where \(f_i, g_i\in K[x_1,\cdots,x_n]\) 

    Choose irreducible \(h\in K(x_1,\cdots,x_r)\) such that \(h\) is relatively prime to \(g_1\).

    Claim: \(h ^{-1} \notin K[x_1,\cdots,x_r]=K\left[ \frac{f_1}{g_1},\cdots,\frac{f_s}{g_s} \right] \)
    
    \(\implies h^{-1} = \frac{l}{\prod g_i ^{d_i}}, (l,g_i)=1\)
    
    \(\implies h ^{-1} \in K[x_1,\cdots,x_r]\) contradicting \(h\) irreduible.
    
    claim \(\implies r = 0\) 
\end{proof}



\hrulefill

Class 36: 04/12

\section*{Primary Decomposition}

Chapter 4 and 7

\begin{definition}
    \(Q \triangleleft A\) proper is \underline{primary}  if every zero divisor in \(A / Q\) is nilpotent. 
\end{definition}

eg \(p^n \mathbb{Z} \triangleleft \mathbb{Z}\) is primary.

Zero divisors of \(\mathbb{Z} / p^n \mathbb{Z}\) is \(p(\mathbb{Z}  / p^n \mathbb{Z})\). This is both the set of zero divisors and the set of nilpotents.

Note that nilpotents are automatically zero divisors.

\underline{Note}: Powers of maximal ideals are primary.

\begin{proposition}
    Contraction of prime ideal is a prime ideal.

    If we have \(A \overset{f}{\to} B \triangleright Q\) then:
    
    \(A / f ^{-1} Q \rightarrowtail B / Q\)
    
    
\end{proposition}

\begin{proposition}[AM 4.1]
    If primary \(Q \triangleleft A\) then: \(\sqrt{Q}\) is smallest prime ideal containing \(Q\).

    Most interesting thing is: radical of a primary ideal is a prime ideal.
\end{proposition}

\newcommand{\Nil}{\operatorname{Nil}}

\begin{proof}
    (Special Case) \(Q=0\). This is primary.

    \(\sqrt{0} = \operatorname{Nil}(A)\). We want to show that this is prime.

    Suppose \(xy\in \Nil(A).\) Then \((xy)^n = 0\) 
    
    So, \(x^n y^n = 0\)
    
    Thus, either \(x^n = 0\) or \(y^n = 0\) or \(x,y\) are both zero divisors.

    Thus, \(x\in \sqrt{0}\) or \(y\in \sqrt{0}\) or \(x,y\) are both zero divisors.
    
    So, \(\Nil(A)\) is indeed prime.

    Since \(\Nil(A)=\cap_{prime}P\) 

    So \(\Nil(A)\) is indeed the smallest prime containing \(0\).

    General Case: this implies the special case.

    Note: \(A / \sqrt{Q} = (A / Q) / \sqrt{0}\) domain 

    If \(Q \triangleleft A\) is primary, then \(0 \triangleleft A / Q\) is primary.
    
    So, \(\sqrt{0}\) is the smallest prime of \(A / Q\)
    
    Thus \(\sqrt{Q}\) is smallest prime of \(A\) containing \(Q\).

\end{proof}

So, given a primary ideal we get an associated prime by proposition 4.1. We make this into a definition.

\begin{definition}
    If \(Q \triangleleft A\) is primary then \(P = \sqrt{Q}\) is prime.
    
    Then say: \(Q\) is \underline{\(P\)-primary}.
\end{definition}

Now we talk about primary decomposition.

\begin{definition}
    Primary decomposition of \(I \triangleleft A\) is \(I = Q_1\cap\cdots\cap Q_n\) where \(Q_j\) are distinct primary ideals.  
\end{definition}

Contrast it with: \(n = p_1^{e_1}\cdots p_r^{e_r}\) and \((n)=(p_1^{e_1})\cap\cdots\cap (p_r)^{e_r}\) where \(A=\mathbb{Z}\) 

Irreducible components are unique, but we are not going to prove that.

\begin{theorem}
    (AM 7.13)

    \(A\) is Noetherian \(\implies\) every ideal has a primary decomposition. 
\end{theorem}

Proof is not trivial, but uses usual Noetherian tricks.

\begin{definition}
    \(I \triangleleft A\) is irreducible if \(I = J \cap K \implies (I = J \text{ or } I = K)\). eg min prime implies irreducible.  
\end{definition}

\underline{Lemma 7.11}: \(A\) noetherian. Then every \(I \triangleleft A\) is a finite intersection of irreducible ideals.

\underline{Lemma 7.12}: \(A\) noetherian, \(Q\) irreducible means \(Q\) is primary.

\underline{Proof of 7.11}: By contradiction.

Let \(T = \) set of ideals which are not finite intersection of irreduible.

\(\varnothing \neq T \implies \exists\) maximal \(M\in T\) since \(A\) is noetherian.

\(M=J\cap K, J,K\notin T\) by maximality.

So, \(M = J \cap K =\) (intersection of irreducible) intersecting (intersection of irreducibles)

Which is a contradcition.

\underline{Proof of 7.12}: We may replace \(A\) by \(A / Q\) and assume \(0\) is irreducible.

We want to show that \(0\) is a primary ideal of \(A / Q\)

So, we want to show \(x\) zero divisor implies \(x\) is nilpotent.

Suppose \(x\) is a zero divisor.

We have \(xy=0,y\neq 0\).

Note that \(Ann(x) \subset Ann(x^2) \subset \cdots\) which is stationary from noetherian.

So, \(Ann(x^n)=Ann(x^{n+1})\) eventually.

Claim: \((x^n)\cap (y)=0\).

Claim along with the fact that \(0\) is irreducible means \((x^n)=0\) or \((y)=0\). Thus, \((x^n)=0 \implies x^n = 0\).

\underline{Proof of Claim}: Suppose \(a\in (y)\). So, \(a = cy\). Thus, \(ax=cxy=0\).

Thus, \(a\in (x^n) \implies a = bx^n \implies ax = bx^{n+1} \implies ax\in (x^{n+1})\)

Now, \(ax=0\). So, \(b\in Ann(x^{n+1})=Ann(x^n)\)

So, \(bx^n=0 \implies a = 0\).

\underline{Uniqueness of Primary Decomposition}

Example: Consider \(k[x,y]\). Then \((x)=(x)\cap (x^2)\) [How???]

\begin{definition}
    \(I = \bigcap Q_i\) is \underline{minimal} if:
    
    i: \(\sqrt{Q_i} \) distinct
    
    ii: \(\forall i\) we have \(Q_i \not\supset \cap_{j\neq i} Q_j\)  
\end{definition}

Fact: \(I\) has primary decomposition implies \(I\) has minimal primary decomposition.

Are minimal primary decomposition unique? NO.

Consider \(k[x,y]\) and \((x^2,xy)=(x)\cap (x,y)^2 = (x)\cap(x^2,y)\) 

\begin{theorem}
    (AM 4.5, 1st uniqueness theorem)

    \(I=\bigcap Q_i\) minimal primary decomposition implies \(P_i = \sqrt{Q_i}\) are uniquely determined upto reordering by \(I\).
    
\end{theorem}

\begin{definition}
    \(I = \bigcap Q_i\) minimal primary decomposition. Let \(P_i = \sqrt{Q_i}\).
    
    \(Q_i\) (or \(P_i\)) is \underline{isolated} if \(P_i\) is a minimal element of \(\{P_1,\cdots, P_n\}\) 
\end{definition}

eg in the above example \((x)\) is isolated.

\begin{theorem}
    (AM 4.10,4.11, 2nd Uniqueness Theorem)

    Let \(I = \bigcap Q_i\) be minimal primary decomposition.

    i: [corollary] Isolated \(Q_i\) are uniquely determined by \(I\).

    ii: If \(Q_{i_1},\cdots,Q_{i_m}\) are isolated (primary) then \(Q_{i_1}\cap\cdots\cap Q_{i_m}\) is uniquely determined.
\end{theorem}

\hrulefill

Class 37: 04/15

\section*{Dedekind Domain}

\begin{definition}
    A domain \(A\) is a dedekind domain if it satisfies the three following properties.

    \begin{enumerate}
        \item \(A\) is Noetherian
        \item Nonzero primes are maximal
        \item \(A\) is integrally closed
    \end{enumerate}
\end{definition}

\underline{Comment on ii:} ii is equivalent to saying the krull dimension \(\dim A \leq 1\).

A field is a dedekind domain, in that case \(\dim A = 0\). This is stupid.

\underline{Classical Example}: If we have a finite extension \(K\) of \(\mathbb{Q}\) and look at the ring of integers \(\mathcal{O}_K\):

\[
    \begin{tikzcd}
        \mathcal{O}_K \ar[r, no head] \ar[d, no head] & K \ar[d, no head, "\text{finite}"]\\ \mathbb{Z} \ar[r, hook] & \mathbb{Q}
    \end{tikzcd}
\]

So \(\mathcal{O}_K = IC(\mathbb{Z} \subset K)=K\cap\mathbb{A}\).

\begin{theorem}
    [AM 9.5] \(\mathcal{O}_K\) is a dedekind domain. 
\end{theorem}

\begin{proof}
    i: \(\mathcal{O}_K\) is noetherian by 5.17
    
    Review of proof: let \(n = \vert K : \mathbb{Q} \vert \).
    
    \(\exists u_1,\cdots, u_n\in \mathcal{O}_K\) and linearly independent over \(K\).
    
    Then \(\bigoplus \mathbb{Z} u_i \subset \mathcal{O}_K \subset \bigoplus \mathbb{Z} \hat{u}_j\)
    
    where \(\Tr(u_i \hat{u}_j)=\delta_{ij}\)
    
    Thus \(\mathcal{O}_K \cong \mathbb{Z}^n\) as \(\mathbb{Z}\)-module thus \(\mathcal{O}_K\) is noetherian.
    
    ii: First proof (AM): Take prime \(0 \neq P \triangleleft \mathcal{O}_K\).
    
    Claim: \(P\cap\mathbb{Z} \neq 0\)
    
    Proof of claim: if \(P\cap \mathbb{Z} = 0\) then \(0 \subset P\) both lie above \(0\) so by 5.8 \(0 = P\).
    
    5.9 gives \(P\) maximal \(\iff P \cap \mathbb{Z}\) maximal.

    2nd Proof: Let \(0\neq \alpha \in P\).
    
    norm \(N \alpha = \prod\) conjugates of \(\alpha\).
    
    \(N \alpha \neq 0\) 

    \(N \alpha \in \mathbb{Q} \cap \mathbb{A} = \mathbb{Z}\)
    
    \((N \alpha) / \alpha \in K \cap \mathbb{A} = O_\mathbb{K}\) 

    So, \(N \alpha = \left( \frac{N \alpha}{\alpha } \right) \alpha \in \mathcal{O}_K P = P \)
    
    (so \(P\cap\mathbb{Z} \neq 0\))

    Now, we have

    \(\mathcal{O}_K / P \leftarrowtail \mathcal{O}_K / \alpha \leftarrowtail \mathcal{O}_K / N_\alpha\)
    
    so, \(\mathcal{O}_K / P\) is finite domain hence \(\mathcal{O}_K / P\) is a field.
    
    Thus, \(P\) is maximal.

    Also, integrally closed by 5.5

\end{proof}

\begin{theorem}
    (Main Theorem) Let \(A\) be a domain which is not a field. Then TFAE:

    \begin{enumerate}
        \item \(A\) is a Dedekind Domain.
        \item Every ideal \(I \triangleleft A\) factors uniquely as a product of prime ideals \(I = P_1^{e_1}\cdots P_r^{e_r}\) 
        \item Every fractional ideal is invertible
        \item If \(I \subset J \triangleleft A, \exists K \triangleleft A\) such that \(I = JK\). `To contain is to divide'
        \item \(\forall\) nonzero prime \(P \triangleleft A, A_P\) is a DVR 
        \item Every ideal of \(A\) is a projective \(A\)-module 
        \item Every submodule of \(A^n\) is projective 
    \end{enumerate}
\end{theorem}

\begin{definition}
    [Fractional Ideal] Suppose \(K = \Frac(A)\). Let \(M = yI \subset K\) where \(y\in K^\times, I \triangleleft A\). Then \(M\) is a fractional ideal. \(M\) is invertible if there exists \(N\) such that \(M\cdot N = A\)    
\end{definition}

Lets talk about 2:

Recall ``Ideals'' is `ideal numbers'

Numbers factor into product of primes, similarly ideals factor into product of ideals.

Suppose \(K = \mathbb{Q} [\sqrt{2}]\) over \(\mathbb{Q}\)

Then \(\mathcal{O}_K = \mathbb{Z} [\sqrt{2}]\) 

Take \(p\in\mathbb{Z}\)

How does \(p \mathcal{O}_K\) factor?

\(2\mathbb{Z} [\sqrt{2}]=(\sqrt{2}\mathbb{Z}[\sqrt{2}])^2 = Q^2\) 

Suppose \(p\) odd, \(p \not\equiv 1\pmod 8\)

Then \(p \mathbb{Z} [\sqrt{2}] = Q\) prime

If \(p \equiv \pm 1\pmod 8\) then \(p\mathbb{Z} [\sqrt{2}]=Q_1 Q_2, Q_1\neq Q_2\)  

Note: \(\mathbb{C}[x,y]\) is not dedekind domain since \(\dim \mathbb{C} [x,y]=2\) or \((x)\) a prime ideal is not maximal.

\underline{Fact}: suppose \(A\) is a Dedekind Domain.

Ten \(A\) PID \(\iff A\) UFD

Suppose \(K = \mathbb{Q}[\sqrt{-5}]\)

Then \(\mathcal{O}_K = \mathbb{Z} [\sqrt{-5}]\) 

Not a UFD since \(2\cdot 3 = 6 = (1+\sqrt{-5})(1-\sqrt{-5})\) 

So, \(\mathbb{Z}[\sqrt{-5}]\) is a DD that is not a PID.

Note: \(\mathbb{Z}[\sqrt{8}]\) is not Integrally Closed and hence not a Dedekind Domain.

If \(A\) is a domain and \(K = \Frac(A)\)

\(M \subset K\) is a \underline{fractional ideal} if it is a fraction times an ideal: \(M = yI\) where \(I \triangleleft A, y\in K^\times\)  

eg \(\frac{2}{5}\mathbb{Z} \subset \mathbb{Q}\) is fractional ideal.

AM equivalent definition: \(M\) is an \(A\)-submodule of \(K\) such that \(\exists x\in K^\times\) such that \(xM \subset A\)  

\underline{Lemma}: If \(M\) is a finitely generated \(A\) submodule of \(K = \Frac(A)\) then \(M\) is a fractional ideal.

Suppose \(M = \sum A \frac{w_i}{z_i}\). If \(x = \prod z_i\) then \(M = \sum A \frac{y_i}{x}\) so \(xM \subset A\)   

\underline{Claim}: if \(A\) is Noetherian, every fractional ideal is finitely generated: \(M = y I\) and \(I\) is finitely generated

If \(M,N \subset K\) are fractional ideals, then \(MN = \{ \sum_{i} m_i n_i \} \) is a fractional ideal.

\begin{definition}
    \(M\) is invertible \(\exists N\) such that \(MN=A\)  
\end{definition}

Here \(\left( \frac{2}{5}\mathbb{Z} \right) ^{-1} = \frac{5}{2}\mathbb{Z} \) 

\underline{Proof of 4 \(\implies\) 3}

Let \(M\) be a fractional ideal. Then \(M = yJ\) for some \(J \triangleleft A\)

Choose \(0\neq a\in J\), by 3 we have \(\exists L\) such that \((aA)=JL\)

Then \(M ^{-1} = y^{-1} a^{-1} L\) 

\hrulefill Class 38: 04/17

Recall:

\begin{definition}
    Domain \(A\) is a dedekind domain if:

    i: \(A\) is noetherian

    ii: nonzero primes are maximal

    iii: \(A\) is IC
\end{definition}

\begin{theorem}
    Let \(A\) be a domain. TFAE:

    i: \(A\) is DD

    ii: \(\forall\) nonzero ideal \(I\), \(I= P_1^{e_1}\cdots P_r^{e_r}\) [uniquely]. The equivalence is true with or without uniqueness.
    
    iii: Every nonzero fractional ideal is invertible.

    iv: To contain is to divide: \(I \subset J \subset A\) means \(\exists L \triangleleft A\) such that \(I = JL\)   

    v: \(\forall 0 \neq P \triangleleft A\) means \(A_P\) is a DVR [ring with unique nonzero prime ideal]
    
    vi: Every ideal of \(A\) is a projective module

    v: Every submodule of \(A^n\) is a projective module
\end{theorem}

Yesterday we did iii \(\iff\) iv which is easy.

ii\(^{\prime} \): \(\forall\) non-zero fractional ideal, \(\exists ! M = P_1^{e_1}\cdots P_r^{e_r}\) 

Different from ii in the sense that instead of ideal we have fractional ideal, and we allow negative exponent.

Recall:

\begin{definition}
    A subset \(M \subset K = \Frac(A)\) is a fractional ideal if \(M = yI\) where \(y\in K^\times\) and \(I\) is an ideal of \(A\) 
\end{definition}

A f.i. is invertible if \(\exists\) f.i \(M ^{-1}\) such that \(M M ^{-1} = A\)

\begin{definition}
    \(M\) is principal f.i if \(M = yA\) for some \(y\in K\) 
\end{definition}

\begin{definition}
    Ideal class group of dedekind domain \(A\):

    \[
        Cl(A) = \frac{(\text{nonzero frac ideal},\cdot)}{(\text{nonzero prime f.i.})}
    \]
\end{definition}

We have \(Cl(\mathbb{Z})=0\)

Also \(Cl(A)=0\iff A\) is PID

\section*{Survey of Algebraic Number THeory}

Clasically it's about finite extensions.

\[
    \begin{tikzcd}
        K\cap\mathbb{A} = \mathcal{O}_K^{IC} \ar[r, no head] \ar[d, no head] & K \ar[d, no head, "\text{finite}"] \\ \mathbb{Z} \ar[r, no head] & \mathbb{Q}
    \end{tikzcd}
\]

\begin{theorem}
    \(Cl(\mathcal{O}_K)\) finite.
\end{theorem}

\(h_K = \vert Cl(\mathcal{O}_K) \vert \) 

Let prime \(p\in\mathbb{Z}\) 

\(p \mathcal{O}_K = P_1^{e_1}\cdots P_r^{e_r}\)

\(f_i = \vert \mathcal{O}_K / P_i : \mathbb{F}_P \vert \)

\(\vert K : \mathbb{Q} \vert = \sum e_i f_i \) 

Suppose \(K\) integral over \(\mathbb{Q}\) and galois.

Then for all \(\phi:K\hookrightarrow\mathbb{C}\) we have \(\phi(K)=K\) 

\newcommand{\Gal}{\operatorname{Gal}}

In \(\Gal(K/\mathbb{Q})\)  permute \(P_i\) 

\(e = e_1 = \cdots = e_r\) 

\(f = f_1=\cdots=f_r\) 

Then \(\vert K : \mathbb{Q} \vert = efr \) 

\(e\) is ramification index

Suppose \(K\) over \(\mathbb{Q}\) galois abelian.

Class field theory implies: \(\exists N\) such that factorization of \(p\) in \(\mathcal{O}_K\) depends on \(p\pmod{N}\) 

There exists \(n\) such that \(K \subset \mathbb{Q}(\zeta_n)\) 

\(K / \mathbb{Q}\) abelian

quadratic \(\mathbb{Q}[\sqrt{d}] / \mathbb{Q}\), \(d\) squarefree

If \(d \equiv 2,3\pmod 4\) then \(\mathcal{O}_K = \mathbb{Z}[\sqrt{d}]\) 

If \(d \equiv 1\pmod 4\) then \(\mathcal{O}_K = \mathbb{Z}\left[ \frac{1+\sqrt{d}}{2} \right] \) 

cyclotomic extension \(\mathbb{Q}(\zeta_n)\)

What is the ring of integers?

\(\mathcal{O}_{\mathbb{Q}(\zeta_n)}=\mathbb{Z}[\zeta_n]\) 

We have: for odd prime \(p\)

\[
    \begin{tikzcd}
        & \mathbb{Q}(\zeta_p) & \\
        \mathbb{Q} [\sqrt{p}] \ar[ur, no head, "p\equiv 1 \pmod 4"] & & \mathbb{Q}[\sqrt{-p}] \ar[ul, no head, "p\equiv 3\pmod 4",swap]
    \end{tikzcd}
\]

\(Cl(\mathcal{O}_{\mathbb{Q}[\sqrt{-d}]}) = 0\)

\(d > 0\) iff \(d = 1,2,7,11,19,43,67,163\) 

\(Cl(\mathbb{Z}[\zeta_p])=0\) if and only if \(p < 23\) 

Complex geometry:

Suppose \(X\) is compact complex 1-manifold. ``Riemann Surface''

Riemann sphere \(\mathbb{C} \cup \infty\) is one example.

Let \(A\) be a ring of holomorphic functions of \(X\) 

\(X \to \mathbb{C}\) 

points in \(X \leftrightarrow\) nonzero prime ideals \(A\)

Explicitly,

\(x_0 \mapsto \{ f\in A \mid f(x_0)=0 \} = P_{x_0}\) 

We can think about local rings:

\(A_{P_{x_0}}=\) ``germs of holomorphic functions at \(x_0\) '' 

Claim: only ideals of \(A_{P_{x_0}}\) are \((P_{x_0}^n)\) 

\(f\in A_{P_{x_0}}\) then \(n = \text{order}_{x_0} f\)

Implies \(A_{P_{x_0}}\) is a DVR

So \(A\) is a Dedekind domain.

Divisor class group

\(Cl(A)= \frac{\mathbb{Z}[x]}{\{ \sum_{x\in X} \text{order}_x f \mid f: X \to \mathbb{C} \cup \infty \text{ meromorphic} \} }\) 

\(Cl(S^2)=\mathbb{Z}\)

for higher genus, \(Cl\) = uncountable.

\(Cl(A)\) is useful for characterizing the zeroes and poles of \(f: X \to \mathbb{C} \cup \{ \infty \} \) 

\hrulefill

Class 39: 04/19

Today we prove:

\begin{proposition}
    domain \(A\) is DD \(\iff\) every ideal is a projective module 
\end{proposition}

What is a projective module?

\begin{definition}
    Module \(P\) is \underline{projective} if \(\forall \) epimorphism \(f: N \rightarrowtail M, \forall g : P \to M\) there exists \underline{lift} \(h:P \to N\) such that \(f \circ h = g\)
    
    \[
        \begin{tikzcd}
            & P \ar[dl, dashed, "h", swap] \ar[d, "g"] & \\
            N \ar[r, "f"] & M \ar[r] & 0
        \end{tikzcd}
    \]
\end{definition}

Nonexample: \(\mathbb{Z} / 2\) is not projective \(\mathbb{Z}\) module

\[
    \begin{tikzcd}
        & \mathbb{Z} / 2 \ar[dl, dashed, "\not\exists", swap] \ar[d] & \\
        \mathbb{Z} \ar[r] & \mathbb{Z}/2 \ar[r] & 0
    \end{tikzcd}
\]

\(A = \mathbb{Z} / 6 \mathbb{Z}\) 

\(\mathbb{Z} / 6 \mathbb{Z}  \overset{A-\text{modules}}{=}  \underbrace{2 \mathbb{Z} / 6\mathbb{Z}}_{P} \oplus \underbrace{3\mathbb{Z} / 3\mathbb{Z}}_{Q} \) 

\(P\) projective not free.

\underline{Exercise}: module \(P\) projective \(\iff\) \(\exists Q\) such that \(P \oplus Q\) free \(\iff \forall\) SES \(0 \to A \to B \to P \to 0\) splits.

Discrete Valuation Ring (DVR): local version of DD

\begin{definition}
    DVR is a PID with a unique nonzero maximal ideal.
\end{definition}

Example: \(\mathbb{Z}_{(p)}=\{ \frac{a}{b}p^k : (b,p) = 1 \} \subset \mathbb{Q} \) 

\(p^k \mathbb{Z}_{(p)}\) 

\(M = p \mathbb{Z}_{(p)}\) 

eg DVR:

suppose prime \(0\neq P \triangleleft A\) PID

Then \(A_P\) is DVR

iireducible \(f\in k[x]\) 

Then \(k[x]_{(f)}\) is DVR

\begin{definition}
    A discrete valuation on a field \(K\):

    Homomorphism \(\nu : K^\times \rightarrowtail \mathbb{Z}\) 

    \(\nu(x+y) \geq \min (\nu(x),\nu(y))\) 
\end{definition}

eg: \(\nu_p : \mathbb{Q}^\times \to \mathbb{Z}\)

\(\nu_p (\frac{a}{b}p^k) = k\), \(p\not\mid a,b\) 

Extend: \(\nu(0)=+\infty\) 

\underline{Lemma}: Let \(A\) be a domain. TFAE:

a: \(A\) DVR

b: [AM definition] \(\exists\) DV \(\nu:\Frac(A) \to \mathbb{Z}\) such that \(A = \nu^{-1} [0,\infty]\)   

\begin{proof}
    \(\implies\): \(A\) PID means \(A\) UFD
    
    Let \(x\in A\) be `unique' [upto unit] irreducible in \(A\).

    Define \(\nu(u x^k)=k\) for \(u\in A^\times\)
    
    \(\impliedby\) Claim: \(u\in K\)
    
    \(u\in A^\times \iff \nu(u)=0\) 

    IE \(A^\times = \nu ^{-1} 0\) 

    proof: \(0 = \nu(1) = \nu (u u ^{-1}) = \nu (u) + \nu(u ^{-1})\) tells us \(\nu(u)=0\)
    
    \(\nu(u) = 0 \implies \nu(u ^{-1})=-0=0 \implies u, u ^{-1} \in A\) // Claim

    \(b \implies a\) 

    Let \(\nu : K \to \mathbb{Z}\) be a DV

    Choose \(x\in K\) such that \(\nu(x) = 1\) 
    
    \(x\) ``uniformizing parameter''

    Then, \(\forall a, a = u x ^{\nu(a)}, u in A^\times\) (claim)

    \(\implies A\) is PID with unique maximal ideal \((x)\) 
    
    
    
    
\end{proof}

\begin{proposition}
    [AM 9.2] Let \(A\) be noetherian local with \(\dim A = 1\)
    
    TFAE:

    i: \(A\) is DVR
    
    ii: \(A\) IC

    iii: The maximal ideal is principal

    iv: resudue field \(k = A / M\), \(\dim_k (M / M^2) = 1\)
    
    v: Every nonzero ideal is a power of \(M\) 

    vi: \(\exists x\in A\) such that \(I\) is nonzero ideal, then \(I = (x^k)\)   

\end{proposition}

Example of noetherian local domain with \(\dim A = 1\) but not DVR 

Example:

\[
    \left( \frac{k[x,y]}{(y^2 - x^3)} \right)_{(x,y)} 
\]

\underline{Lemma} [p95]: Let \(A\) be local, noetherian, \(\dim A = 1\)

A: If \(0 \neq I \triangleleft A\) proper, \(\exists m\) such that \(I \supset M^m\)

B: \(\forall n, M^n \neq M^{n+1}\) 

Question: is \(\bigcap M^n\) 0 for a local ring?

No, but Yes for DVR.

\begin{proof}
    B: It is a consequence of \underline{Nakayama's Lemma} 

    Recall: Nakayama: For \(M\) finitely generated \(A\)-module, \(I \subset J(A) = \bigcap_{\text{maximal}} M\) then \(IM = M \implies M = 0\)
    
    We use contradiction.

    Assume \(M^n = M^{n+1}\) 

    So, \(M(M^n)=M^n\)

    \(M\) finitely generated since Noetherian

    Nakayama's Lemma means \(M^n = 0\) 

    Contradiction since \(A\) is a domain
    
\end{proof}

Add:

Chapter 4: Definition of Primary along with Proposition 4.1, Definition of Primary Decomposition

??: Uniqueness? Probably not.

Chapter 7: Theorem 7.13, Dedekind domains and DVRs

\hrulefill

Class 40: 04/22

\begin{proposition}
    [AM 7.14]

    Let \(A\) be Noetherian and \(I \triangleleft A\). Then \(\exists m\) such that \((\sqrt{I})^m \subset I\)   
\end{proposition}

\begin{proof}
    
    Since \(A\) is noetherian, \(\sqrt{I}\) is finitely generated.

    \(\sqrt{I} = (x_1,\cdots, x_k)\) 

    \(\forall i, \exists n_i\) such that \(x_i^{n_i}\in I\) 

    \((x_1,\cdots,x_k)^m\) is finitely generated. It's generators are monomials: \(x_1^{r_1}\cdots x_k^{r_k}\) where \(\sum_{i} r_i = m\)
    
    Let \(m = (\sum_{i} (n_i - 1))+1\)
    
    Then \(\exists i\) such that \(r_i \geq n_i\)
    
    So, \(x_i^{r_i}\in I\)
    
    Thus, \((\sqrt{I})^m \subset I\) 

\end{proof}

\underline{Corollary 7.13} If \(A\) is Noetherian, then the `ideal of nilpotents is nilpotent'.

\(\exists m\) such that the nilradical \((\operatorname{rad} A)^m = 0\)  

In other notation, \(\sqrt{0}^m = 0\)

Consider the polynomial ring with infinitely many variables and take a quotient.

\[
    A = \frac{\mathbb{Z}[x_1, x_2, x_3,\cdots]}{(x_1, x_2^2, x_3^3,\cdots)}
\]

Now, \(\forall m, (\operatorname{rad} A)^m \neq 0\) 

Counterexample!

\underline{Lemma} (p95) Let \(A\) be local domain, noetherian, \(\dim A = 1\)

Note that \(\dim A = 1\) means nonzero primes are maximal.

Let \(M\) be a maximal ideal of \(A\) 

A: \(0 \neq I \triangleleft A\) [proper] implies \(\exists n\) such that \(M^n \subset I\)

B: \(M^n \neq M^{n+1} \forall n\) 

\begin{proof}
    A: \(\sqrt{I} = \bigcap_{I \subset P} P\) 

    Since \(\dim A = 1\) we have \(\sqrt{I} = M\)
    
    Thus, by 7.14 we're done.

    B: We already did with nakayama's lemma


\end{proof}

\begin{proposition}
    [AM 9.2] Let \(A\) be a local noetherian domain, \(\dim A = 1\). Let \(k = A/M\). TFAE:
    
    i: DVR
    
    ii: \(A\) IC

    iii: \(M\) principal

    iv: \(\dim_K M / M^2 = 1\)
    
    v: \(0 \neq I \triangleleft A \implies I = M^k\)
    
    vi: \(\exists x\in A\) s.t. (\(0 \neq I \triangleleft A \implies I = (x^k)\) for some \(k\) )  
    
\end{proposition}

Canonical example: \(A = \mathbb{Z}_{(p)}\) where \(x = p\) 

\begin{proof}
    i \(\implies\) ii: 5.18, `valuation ring is IC'

    ii \(\implies\) iii: ``hardest part'' 

    Choose \(0 \neq a \in M\)

    \(A \implies \exists n\) such that \(M^n \subset (a), M^{n-1} \not\subset (a)\)  

    Choose \(b\in M^{n-1}, b \notin (a)\) 

    Let \(x = \frac{a}{b}\in \Frac(A)\)
    
    Claim: \(M = Ax\) 

    \begin{proof}
        \(b\notin (a)\implies x ^{-1} \notin A\)
        
        So \(x ^{-1}\)  not integral over \(A\) 

        \(\implies x ^{-1} M \not\subset M\) by 5.1 contrapositive

        \(x ^{-1} M = \frac{b}{a} M \subset \frac{M^n}{a} \subset A\)
        
        Thus, \(x ^{-1} M\) is an ideal not contained in a max ideal
        
        So, \(x ^{-1} M = A\) and thus \(M = Ax\)
        
        iii \(\iff\) iv:
        
        \(M\) principal implies \(\dim_k M / M^2 M=1\)
        
        by B say \(M \neq M^2;\) thu \(\dim-k M / M^2 = 1\) 
        
        \(\impliedby\) 2.8
        
        iii \(\implies\) v: 8.8                                          
        
        postponed.

        vi \(\implies\) i:
        
        Note that \(M = (x)\) 

        Note that \((x^k \neq x^{k+1})\) by B
        
        Define \(\nu : A -\) 


        Define \(\nu: A - \{ 0 \} \to \mathbb{Z}_{\geq 0}\) 
        
        \(\nu(a)=k\) if \(a \in (x^k)-(x^{Pk+1} )\)  

        define \(\nu : K^\times -\mathbb{Z}\)
        
        \(\nu(\frac{a}{b})=\nu (a)-\nu (b)\) 

    \end{proof}
\end{proof}

\underline{Correction to DD theorem}

Recall: if \(A\) is domain, tfae: a: \(A\) DD, b: i: \(A\) is noetherian, ii: \(\dim A = 1\), iii: \(\forall P, A_P\) is DVR  

\begin{theorem}
    [AM 9.3]: Let \(A\) be a noetherian domain with \(\dim A = 1\) 

    a: \(A\) is IC

    b: \(\forall P, A_P\) is DVR 
\end{theorem}

\begin{proof}
    [\(A \implies B\)]: proof comes from 9.2, 5.13.

    5.13: integrably closed is a local property

    
\end{proof}

\hrulefill

Class 41: 04/24

\section*{Limits, Colimits, Completion}

Functor \(X : I \to \mathcal{C}\) ``\(I\)-diagram in \(\mathcal{C}\) '' 

\begin{definition}
    \(\lim_{I} X\) is an object in \(\mathcal{C}\) with map \(\lim_{I} X \to X(i) \forall i\in I\) such that:
    
    1: \(\forall i \to i^{\prime} \) we have:
    
    \[
        \begin{tikzcd}
            & \lim_I X \ar[dl] \ar[dr]& \\
            X(i) \ar[rr] & & X(i^{\prime})
        \end{tikzcd}
    \]

    2: `Initial'

    \(\forall (L\in \mathcal{C},\{ L \to X(i) \}_{i\in I} )\)
    
    \[
        \begin{tikzcd}
            & L \ar[dl] \ar[dr] & \\
            X(i) \ar[rr] & & X(i^{\prime})
        \end{tikzcd}
    \]

    \(\exists ! L \to \lim_{I} X\) such that:
    
    \[
        \begin{tikzcd}
            & L \ar[dl] \ar[dr, "\exists!", dotted] & \\
            X(i) & & \lim_I X \ar[ll]
        \end{tikzcd}
    \]
\end{definition}

Define \(\operatorname{colim}_I X\) by reverse arrow.

\(X(i)\longrightarrow \operatorname{colim}_I X \)

\[
    \begin{tikzcd}
        & L \ar[ldd] \ar[rdd] \ar[d, "\exists!", dotted] & \\
        & \lim_I L \ar[dl] \ar[dr] & \\
        X(i) \ar[rr] & & X(i^{\prime})
    \end{tikzcd}
\]

If (co)limit exist, they are unique (upto isomorphism)

(co)limits exists:

\(\mathcal{C} = Ab, A\)-mod, Group, Ring, CRing, Top, Set etc

(co)liit doesn't exist: \(\mathcal{C}=\) Field. 

Let \(I = ., \mathcal{C} = A\)-mod

\(\lim_I X = X_0 \times X_1\)

\(\operatorname{colim}_I X = X_0 \oplus X_1 \) 

\underline{Example}: \(I\) discrete category (only morphisms are id)

\(\lim_I X = \prod X(i)\)

\(\operatorname{colim}_I X = \bigoplus X(i) \) 

Note that \(\prod X(i) \supset \bigoplus X(i)\), inclusion strict for infinite \(I\) 

\begin{theorem}
    If \(\mathcal{C} = A\)-mod then (co)limits exist. 
\end{theorem}

\begin{proof}
    Consider functor \(X : I \to \mathcal{C}\)
    
    We can construct the limit as a submodule:

    \[
        \lim_I X \subset \prod_i X(i)
    \]

    `submodule of compactible tuples'

    \(\lim_I X = \{ (x_i) \mid X(i \to i^{\prime})(x_i)=x_{i^{\prime}} \}\)
    
    \(\operatorname{colim}_I X = \frac{\bigoplus X_i}{(\{ x_i - X(i \to i^{\prime})x_i \} )} \) 
    

\end{proof}

Example:

Suppose \(I =(\mathbb{N} , \geq)\) 

Then we have `morphisms' \(5 \to 3\) aka \(5 \geq 3\)  

Then, functor means we hve morphisms:

\[
    \cdots \to X_4 \to X_3 \to X_2 \to X_1
\]

Then, \(\lim_{(\mathbb{N} , \geq )} X\) is called \(\lim_{n \to \infty} X(n)\)

This is called an ``inverse limit''

Note: \(\lim_{n\to \infty} X(n) \to X_k\) for all \(k\) 

Think of limits as `intersections' and colimits as `unions'

If the morphisms are `inclusion':

\[
    \cdots \subset X_3 \subset X_2 \subset X_1 \subset X_0
\]

Then \(\lim_{n \to \infty} X_n = \bigcap_{n=1}^{\infty} X_n\)

Classic example: p-adic

\[
    \cdots \to  \mathbb{Z} / p^3 \to \mathbb{Z} / p^2 \to \mathbb{Z} / p
\]

Then \(p\)-adic integers \(\hat{\mathbb{Z}}_p = \lim_{n \to \infty} \mathbb{Z} / p^n\) 

Note that all these are rings so essentially \(I \to\) CRing.

\(I \triangleleft A\)

We can have \(\cdots\to A / I^3 \to A / I^2 \to A / I\) 

Then \(\hat{A}_I = \lim_{n \to \infty} A / I^n\)

\underline{Application}:

\underline{Krull's Intersection Theorem} Let \(A\) be a noetherian and \(A\) is either local or a domain.

Then \(A \to \hat{A}_I\) is injective, ie \(\cap I^n = 0\) 

Consider categories:

\[
    \begin{tikzcd}
        \cdot \ar[r] \ar[d] & \cdot \\
        \cdot
    \end{tikzcd}
\]

colim is called pushout

\[
    \begin{tikzcd}
        & \cdot \ar[d]\\
        \cdot \ar[r] & \cdot
    \end{tikzcd}
\]

limit is called pullback

Now suppose \(X : I \to \mathcal{C}\) and consider SES:

\[
    0 \to X^{\prime} \to X \to X^{\prime\prime} \to 0
\]

Question: colimit and limit SES??

\section*{Galois Theory}

Suppose

\[
    \begin{tikzcd}
        & E \ar[dl, no head, "Galois"] \\
        F &
    \end{tikzcd}
\]

What if extension not finite?

Example of not Galois:

\[
    \begin{tikzcd}
        & \mathbb{Q} [\sqrt[3]{2} ] \ar[dl, no head] \\
        \mathbb{Q} 
    \end{tikzcd}
\]

Not Galois since not contain all roots.

But we also have \(\overline{\mathbb{Q}} / \mathbb{Q}, \overline{\mathbb{F}_p} / \mathbb{F}_p  \) 

Makes sense as union of finite extensions:

\(\Gal(E / F) \coloneqq \lim_{E / L /_{finite} F} \Gal(L / F)\). Note:

\[
    \begin{tikzcd}
        & & \lim_K \Gal(L / K) \ar[dl, no head]\\
        & L \ar[dl, "\text{finite galois}", no head] & \\
        F
    \end{tikzcd}
\]

Fundamental Theorem of Galois Theory: \(\exists\) 1-1 correspondence between closed subgroups of \(\Gal(E / F)\) and \(E / L / F\)

by \(H \mapsto E^H\)

Note:

\[
    \Gal(\overline{\mathbb{F}_p} / \mathbb{F}_p) = \hat{\mathbb{Z}} = \prod_p \mathbb{Z}_p
\]


\end{document}
