\documentclass{article}
\usepackage{amsmath, amsthm, amssymb, amsfonts, mathtools,enumitem, stmaryrd,physics, cancel, tikz-cd, graphicx, float, booktabs}
\usetikzlibrary{arrows}
\usepackage{geometry}
    \geometry{
        a4paper,
        left = 40mm,
        top = 20mm,
        right = 40mm,
        bottom = 30mm
    }
\setlength{\parindent}{0pt}

\theoremstyle{definition}
\newtheorem{problem}{Problem}
\newtheorem{solution}{Solution}
\newtheorem*{example}{Example}
\newtheorem*{exercise}{Exercise}
\newtheorem*{definition}{Definition}
\newtheorem{theorem}{Theorem}
\newtheorem*{theorem*}{Theorem}
\newtheorem{proposition}[theorem]{Proposition}
\newtheorem*{proposition*}{Proposition}
\newtheorem{lemma}[theorem]{Lemma}
\newtheorem*{lemma*}{Lemma}
\newtheorem{corollary}[theorem]{Corollary}
\newtheorem*{corollary*}{Corollary}
\newtheorem*{remark}{Remark}

\title{M623 Geometric Topology I}
\author{Thanic Nur Samin}
\date{}


\begin{document}
    \maketitle

    \section*{Monday, 8/25/2025}

    Textbook: \textit{Characteristic Classes} by Milnor and Stasheff. Hereafter referred by MS.

    \underline{Read Chapter 1 and 2 of MS.}

    \begin{definition}
        [\(n\)-manifold]

        Two different variants: embedded and abstract.

    

        Abstract: \((M, \mathcal{A})\) where \(\mathcal{A}\) is an atlas.
    \end{definition}

    Embedded: \(M \subset \mathbb{R}^A\). Here, \(A =\) index set, \(\mathbb{R}^A = \text{func}(A,\mathbb{R})\) with the product topology.

    \(M\) Hausdorff space, \(U \subset M\) open, \(V \subset \mathbb{R}^n\) open.
        
    Chart \(\phi : U \xrightarrow{\approx} V\) homeomorphism.

    Parameterization (ptz) \(h: V \xrightarrow{\approx} U\)

    We want some calculus.

    Let open \(V \subset \mathbb{R}^n\).

    A function \(f: V \to \mathbb{R}\) is \textit{smooth} if all partials of all orders exist: \(\frac{\partial^k f}{\partial x_{i_1} \cdots \partial x_{i_{p}}}\).

    \(f: V \to \mathbb{R}^A\) is \textit{smooth} if \(f_\alpha\) smooth \(\forall \alpha \in A\).

    \[
        \begin{tikzcd}
            V \ar[rr, bend right,"f_\alpha"'] \ar[r,"f"] & \mathbb{R}^A \ar[r,"pr_{\alpha}"] & \mathbb{R}
        \end{tikzcd}
    \]

    We can go from abstract manifold to embedded manifold.

    Let \(A = C^{\infty} (M,R)\).

    \(M \xrightarrow{i}\mathbb{R}^A\) where \(i(x) = (f \mapsto f(x))\).

    We can go to the reverse direction easily once we have all the definitions.

    \begin{definition}
        Two charts \((\phi_1: U_1 \to V_1)\) and \((\phi_2: U_2 \to V_2)\) are compatible (or smoothly compatible) if \(\phi_2 \circ \phi_1 ^{-1}\) is smooth. Explicitly,

        \(\phi_1(U_1 \cap U_2) \xrightarrow{\phi_2 \circ \phi_1 ^{-1}} \phi_2(U_1 \cap U_2)\) needs to be smooth. 
    \end{definition}

    \begin{definition}
        Parameterization \(h: V \to U\) is \textit{smooth} (assume \(M \subset \mathbb{R}^A\)) if:

        \[
            \begin{tikzcd}
                V \ar[r,"h"] \ar[rrr, bend right,"h"] & U \ar[r,hook] & M \ar[r,hook] & \mathbb{R}^n
            \end{tikzcd}
        \]

        is smooth.

        and has rank \(n\). ie, \(\forall v\in V\) the Jacobian:

        \[
            dh = \left( \frac{\partial_\alpha h}{\partial x_j}(v) \right)
        \]
    
        has rank \(n\).

        eg \(x \mapsto x^3\) is a parameterization which is not smooth, since the Jacobian has rank \(0\) at \(0\).
    \end{definition}

    Now we can properly define manifolds.

    \begin{definition}
        [Embedded Smooth \(n\)-Manifold] \(M \subset \mathbb{R}^A\) so that \(\forall x\in M\) there exists a smooth rank \(n\) parameterization \(h : V \to U \ni x\).

        We assume \(M\) is Hausdorff.
    \end{definition}

    We can now define a Cateogry of Embedded Manifolds.

    \begin{definition}
        [Category of Embedded Manifolds] \(\operatorname{Embmfld}\).

        \underline{Objects}: embedded \(M \subset \mathbb{R}^A\) of \(\dim n\) for some \(n\).

        \underline{Morphisms}: Smooth Maps (has to be defined carefully, restricting in Euclidean space).

        Diffeomorphism = invertible morphism.
    \end{definition}

    Let \((M \subset \mathbb{R}^A), (N \subset \mathbb{R}^B)\). \(f: M \to N\) is smooth if \textit{locally smooth}, meaning \(\forall x\in M, \exists\) smooth parameterization \(h: V \to U \ni x\) such that \(V \to U \hookrightarrow M \xrightarrow{f} N \to \mathbb{R}^B\) is smooth.

    Now we can define abstract manifold independend of embedded manifolds.

    \begin{definition}
        [Abstract Manifold] Let \(M\) be Hausdorff. An \(n\)-\textit{atlas} on \(M\) is a set \(\mathcal{A}=\left\{ \phi_\alpha : U_\alpha \xrightarrow{\approx} V_\alpha \subset \mathbb{R}^n \right\}\) of compactible \(n\)-charts such that \(\{ U_\alpha \} \) covers \(M.\)

        Atlas \(\mathcal{A}\) and \(\mathcal{A}^{\prime}\) are \underline{compatible} if all charts are.

        Fact: Every atlas is contained in a unique maximal atlas.

        Then an abstract manifold is \((M,\mathcal{A})\) with a maximal \(n\)-atlas.
    \end{definition}

    \section*{Wednesday, 8/27/2025}
    
    Recall: embedded \(n\)-manifold \(M \subset \mathbb{R} ^ A\): \(\forall x\in M, \exists\) smooth, rank \(n\) parameterization \(h: V \to U \subset M\) such that \(x\in U\). We assume \(M\) is Hausdorff.

    Abstract \(n\)-manifold: \((M, \mathcal{A})\) where \(\mathcal{A}\) is an \(n\)-atlas, so \(\mathcal{A}  = \{ \text{charts } \phi_\alpha : U_\alpha \xrightarrow{\cong} V_\alpha \}\) such that \(\{ U_\alpha \}\) cover \(M\) and \(\{ \phi_\alpha \} \) smoothly compatible. We assume \(M\) is Hausdorff.
    
    \begin{remark}
        If we have an abstract manifold we have a surjective map \(\coprod V_\alpha \overset{\coprod \phi_\alpha ^{-1}}{\twoheadrightarrow} M\).
        
        Then we can define \(M \cong \frac{\coprod V_\alpha}{\sim}\). This gives us another definition of a manifold. 
    \end{remark}

    \begin{exercise}
        Define smooth \(f: (M,\mathcal{A}) \to (N,\mathcal{B})\).

        Not hard, just annoying to get the definitions right!
    \end{exercise}

    \begin{theorem}
        Categories of abstract manifolds and embedded manifolds are equivalent.

        \[
            \text{EmbMflds} \simeq \text{absMflds}  
        \]
    \end{theorem}

    Recall equivalent categories:

    \begin{definition}
        Categories \(\mathcal{C}\) and \(\mathcal{D}\) are equivalent (Notation: \(\mathcal{C} \simeq \mathcal{D}\)): If there are functors \(\mathcal{C} \xrightarrow{F} \mathcal{D}\) and \(\mathcal{D} \xrightarrow{G} \mathcal{C}\) such that \(F \circ G\) and \(G \circ F\) are naturally isomorphic to the respective identities.
    \end{definition}

    We need some more definitons.

    \begin{definition}
        A skeleton of \(\mathcal{C}\) is \(\operatorname{Sk} \mathcal{C} \subset \mathcal{C}\) is a full subcategory \(\forall c\in \mathcal{C}, \exists ! c^{\prime} \in \operatorname{Sk} \mathcal{C}\) such that \(c \cong c^{\prime}\).

        \(\mathcal{A} \subset \mathcal{B}\) is full if \(\forall a,a^{\prime} \in \operatorname{Ob} \mathcal{A}, \mathcal{A}(a,a^{\prime}) \xrightarrow{\cong} \mathcal{B}(a,a^{\prime})\) 
    \end{definition}

    For example, let \(\mathcal{C} =\) finite sets. Then \(\operatorname{Sk} \mathcal{C} = \{ 1 \}, \{ 1,2 \}, \{ 1,2,3 \}, \cdots\) 

    \begin{theorem}
        \(\mathcal{C} \simeq \mathcal{D} \iff \operatorname{Sk} \mathcal{C} \cong \operatorname{Sk} \mathcal{D}\).
    \end{theorem}

    Note that \(\mathcal{C} \simeq \operatorname{Sk} \mathcal{C}\) so one direction is trivial.

    \begin{lemma}
        [1.1] Let \(h\) and \(h^{\prime}\) be smooth rank \(n\) on \(M \subset \mathbb{R}^A\). Then \(h ^{-1} \circ  h^{\prime}\) is smooth (thus a diffeomorphism).
        
        Let \(V\) and \(V^{\prime}\) be the domain of \(h\) and \(h^{\prime}\) respectively. Then \(h ^{-1} \circ h^{\prime} : (h^{\prime})^{-1} (V\cap V^{\prime}) \to h ^{-1} (V \cap V^{\prime})\) 
    \end{lemma}

    \begin{corollary}
        \(\mathcal{A} = \{ h ^{-1} \mid h \text{ parameterization} \}\) is \(n\)-atlas on \(M\).

        This gives us \(\text{EmbMflds}\to \text{AbstMflds}\).
    \end{corollary}

    \begin{proof}
        This is the proof of lemma 1.1, lemma 3 in the notes.

        Assume \(V = V^{\prime}\). WTS: \((h^{\prime})^{-1} V \to h ^{-1} (V)\) is smooth.

        For \(x\in V\) choose \(\alpha_1, \cdots , \alpha_n \in A\) such that \(\det \left( \frac{\partial \alpha_i}{\partial x_j} (x) \right)  \not\equiv 0\). 

        We have:

        \[
            \begin{tikzcd}
                M \ar[r, hook] & \mathbb{R}^A \ar[d, "pr_{\alpha_1 \cdots \alpha_n}"] \\ V \ar[u,"h"] \ar[r,dotted] & \text{subset of } \mathbb{R}^n
            \end{tikzcd}
        \]

        Then, by the inverse function theorem, the dotted map is locally invertible.

        \(h^{-1} \circ h^{\prime} = (pr \circ inc \circ h)^{-1} \circ inc \circ pr \circ h^{\prime}\) near \(h ^{-1} x\).

    \end{proof}

    Given abstract \((M,\mathcal{A})\), let \(A = C^{\infty} (M,\mathbb{R})\) smooth functions.
    
    \(i: M \to \mathbb{R}^A, x \mapsto (f \to f(x))\).
    
    Let \(M_1 = i(M)\).
    
    \begin{lemma}
        [1.5] \(M_1 \subset \mathbb{R}^A\) is \(\text{EmbMfld}\). \(M \xrightarrow{i} M_1\) is diffeomorphism.
    \end{lemma}

    \subsection*{Definition of tangent vector, tangent space and tangent bundle}

    \begin{definition}
        [Tangent Vector] is velocity vector of a curve.

        We have defined morphisms. Consider the embedded case: suppose we have smooth \(\gamma : \mathbb{R} \to M \subset \mathbb{R}^A\). Then,

        \[
            \gamma^{\prime} (0) = \lim_{h \to 0} \frac{\gamma (h) - \gamma(0)}{h} \in \mathbb{R}^A
        \]

        is a tangent vector
    \end{definition}

    \begin{definition}
        [Tangent Space] Suppose \(x\in M \subset \mathbb{R}^A\), an \(n\)-dim embedded manifold. \(T_x M =\) tangent space of \(M\) at \(x\). This is:
        
        \[
            \left\{ \gamma ^{\prime} (0) \mid \gamma(0) = x \right\} \subset \mathbb{R}^A
        \]

        an \(n\)-dim subspace.
    \end{definition}

    We are going to bundle this together.

    \begin{definition}
        [Tangent Bundle]

        \(TM = \left\{ (x,v) \in M \times \mathbb{R}^A \mid v\in T_x M \right\}\).

        By definition, \(TM \subset M \times \mathbb{R}^A\) so this is in fact a topological space.

        We have a projection map \(TM \xrightarrow{\pi} M\) by \((x,v) \mapsto x\).
    \end{definition}

    \begin{remark}
        Fibers of \(\pi\), \(\pi ^{-1} (x)\) are vector spaces: \(\pi ^{-1} (x) = T_x M\).

        Then, \(TM = \bigcup_{x\in M} \{ x \} \times T_x M\).

        Abuse of notation lets us write this as \(\bigcup T_x M\).

        Thus, tangent bundle is in fact a bundle of tangents.
    \end{remark}

    What about abstract manifolds \((M,\mathcal{A})\)?

    We can define \(TM\) as follows:

    \begin{itemize}
        \item \(M \subset \mathbb{R}^{C^{\infty}(M,\mathbb{R})}\).
        \item \(TM = \frac{\coprod V_\alpha \times \mathbb{R}^n}{\sim}\)
        \item \(T_x M =\) velocity vector of curves.
        \item derivations.
    \end{itemize} 

    Suppose we have smooth function between manifolds \(f: M \to N\). \(\forall x\in M\) we can define linear \(\mathrm{d}f_x : T_x M \to T_{f(x)} N\), \(\gamma ^{\prime} (0) \mapsto (f \circ \gamma)^{\prime} (0)\). \(\mathrm{d} f_x\) is a map between vector spaces, so it is a linear transformation. It is the `Jacobian'.
    
    Then we have \(\mathrm{d} f: TM \to TN\) such that \(\mathrm{d} f(x,v) = \mathrm{d} f_x(v)\).

    We also have the chain rule: \(\mathrm{d} (f \circ g) = \mathrm{d} f \circ \mathrm{d} g\) 

    \section*{Friday, 8/29/2025}
    
    No class next week!

    Manifold constructed by:

    \begin{itemize}
        \item open subset of \(\mathbb{R}^n\)
        \item Subset double torus \(\subset \mathbb{R}^3\)
        \item Quotients: \(P^n = \mathbb{R} P^n = S^n / x \sim -x\)
        \item Lie groups/ matrix group, eg closed subgroups of \(\operatorname{GL}_n\mathbb{R} \underset{\text{open}}{\subset} M_n \mathbb{R} = \mathbb{R}^{n^2}\) 
        \item Zero sets.
        \begin{itemize}
            \item regular values
            \item transversality
            \item smooth varieties 
        \end{itemize} 
    \end{itemize} 

    \begin{definition}
        \(t_0\in \mathbb{R}\) is a regular value of \(f: M \to \mathbb{R}\) if \(\forall x\in f ^{-1} t_0, \mathrm{d}f_x\) is onto.
    \end{definition}

    \(f^{-1}(\text{regular value})\) is a submanifold of \(M\).

    Consider \(S^n \subset \mathbb{R}^{n+1}\), and \(f: \mathbb{R}^{n+1} \to \mathbb{R}\) given by \(x \mapsto x_1^2 + \cdots + x_{n+1}^2\).

    \(1\) is a regular value \(f ^{-1} 1 = S^n\).

    \begin{definition}
        Let \(f: M \to N \supset X\) submanifold.

        \(f\pitchfork X\), \(f\) is \textit{transverse} to \(X\) if \(\forall m \in f ^{-1} X, T_{f(m)}N = T_{f(m)} X + \mathrm{d}f_m (T_m M)\).
    \end{definition}

    \begin{figure}[H]
        \centering
        \includegraphics[width=0.8\textwidth]{img/transverse}
        \caption{}
        \label{fig:}
    \end{figure}

    \begin{theorem}
        \(f ^{-1} X\) is a submmanifold of \(M\).

        Furthermore, \(\dim N - \dim X = \dim M - \dim f ^{-1} X\).
    \end{theorem}

    In fact, \(\nu (f ^{-1} X \hookrightarrow M) \to \nu (X \hookrightarrow N)\) as vector space isomorphism on fibers.

    [insert picture later]

    Now, suppose \(F\) is a topological space.

    \begin{definition}
        A fiber bundle with fiber \(F\):

        Let \(E \xrightarrow{\pi} B\) be a continuous map suuch that \(\forall b\in B, \exists\) open \(b\in U \subset B\) and:

        \[
            \begin{tikzcd}
                U \times F \ar[rr, "h", "\approx"'] \ar[rd, "pr_U"] && \pi ^{-1} U \ar[ld,"\pi"] \\ & U 
            \end{tikzcd}
        \]

        \(h\) fiber preserving homeomorphism. \(\forall b^{\prime} \in U\), \(F \cong F \times b^{\prime} \xrightarrow{\approx} F_{b^{\prime}} \coloneqq \pi ^{-1} (b^{\prime})\).
    \end{definition}

    Write: \( \begin{tikzcd}
        F \ar[r] & E \ar[d] \\ & B 
        \end{tikzcd} \)

    \[
        \begin{tikzcd}
            I \ar[r] & Mob \ar[d] \\ & S^1
        \end{tikzcd}
    \]

    eg \(B \times F \to B\) trivial bundle.

    \section*{Chapter 2 of MS}

    \begin{definition}
        A real vector bundle \(\xi\) over \(B\) is:

        \[
            \xi = \left( \begin{tikzcd} E \ar[d,"\pi"] \\ B \end{tikzcd}, \forall b\in B, \pi ^{-1} b = F_b \text{ is a fin. dim vector space.} \right) 
        \]

        \(F_b \times F_b \to F, \mathbb{R} \times F_b \to F\) satisfies \(8\) axioms s.t.

        \(\forall b\in B, \exists b\in U \subset B\) and \(n\geq 0\) and \(\begin{tikzcd}U \times \mathbb{R}^n \ar[rr,"h","\approx"'] \ar[rd] && \pi ^{-1} U \ar[ld] \\ & U\end{tikzcd}\).
    
        \(\mathbb{R}^n \cong b \times \mathbb{R}^n \xrightarrow[\approx]{h} \pi ^{-1} b\) is an isomorphism of vector spaces.
    \end{definition}

    If \(B\) is connected then \(n\) is constant.

    `rank \(n\) vector bundle'.

    \(n\)-plane bundle.

    Another thing MS does is write this: \(\xi = \begin{tikzcd} E(\xi) \ar[d,"\pi(\xi)"] \\ B(\xi)\end{tikzcd}\) for vector bundle which is very precise.

    \subsection*{Isomorphism of vector bundles over \(B\)}.

    Consider two bundles \(\xi\) and \(\eta\) and we have the homeomorphism

    \[
        \begin{tikzcd}
            E(\xi) \ar[rr, "\approx"] \ar[rd] && E(\eta) \ar[ld] \\ & B
        \end{tikzcd}
    \]

    vector space isomorphism on the fibers.

    \subsection*{Examples of vector bundles}

    We have the trivial bundle \(\begin{tikzcd} B \times \mathbb{R}^n \ar[d,"\underline{\mathbb{R}^n} = \underline{\mathbb{R}}^n_B = \varepsilon^n_B ="'] \\ B \end{tikzcd}\) 

    We have tangent bundles:

    \[
        \tau_M = \left\{ \begin{tikzcd} TM \ar[d,"\pi"] \\ M \end{tikzcd} , T_x M \right\} 
    \]

    \begin{definition}
        \(M\) is parallelizable if \(\tau_M\) is trivial.
    \end{definition}

    \(S^1\) is paralellizable.

    Lie groups are parallelizable eg \(S^3\).

    \(S^2\), or \(S^{2n}\) in general not parallelizable via the hairy ball theorem.

    We also have normal bundles. Consider \(M \subset \mathbb{R}^N\).

    \(\nu(M \subset \mathbb{R}^n) = \{ (x,v) \in M \times \mathbb{R}^n \mid x \in M, v \in (T_x M)^{\perp}\}\)
    
    \(\nu (S^2 \hookrightarrow S^3) \leftarrow S^2 \times \mathbb{R}\) is trivial, the map is \((x,tx) \mapsfrom (x,t)\).

    Tautological bundle over \(P^n\): \(\gamma_n^1 = \begin{tikzcd}
        \mathbb{R} \ar[r] & E(\gamma_n^1) \ar[d] \\ & P^n
    \end{tikzcd}\) 

    Note that \(P^n = S^n / x \sim -x =\) lines through \(O\) in \(\mathbb{R}^{n+1}\).

    \(E(\gamma_n^1) = \left\{ (\{ x,-x \}, v) \in P^n \times \mathbb{R}^{n+1} \mid v \in \mathbb{R} x \right\}\).

    \(E(\gamma_n^1) \xrightarrow{\pi} P^n, (\{ x,-x \} \mapsto \{ x,-x \})\). Essentially, point on line \(\mapsto\) line. 

    This tautological bundle is non-trivial.

    \section*{Monday, 9/8/2025}
    
    Last week was a break.

    HWK: an exercise from ch2. (C, D, E are recommended).

    Recall: a vector bundle \(\xi\) is \begin{tikzcd} \mathbb{R} \ar[r] & E \ar[d,"\pi"] \\ & B\end{tikzcd} meaning fibers of \(\pi\) are \(k\)-dimensional vector spaces.

    \begin{definition}
        A section of \(\xi\) is actually a section of \(\pi\).

        \(s: B \to E\) such that \(\pi \circ s = \operatorname{id}_{B}\).
    \end{definition}

    Section looks like this:

    \[
        \begin{tikzcd}
            \mathbb{R}^k \ar[r] & E \ar[d,"\pi"] \\ & B \ar[u, dotted, bend left, "s"]
        \end{tikzcd}
    \]

    Section of \(TM \eqqcolon\) vector field.

    There's also the zero section \(z: B \to E\) given by \(b \mapsto 0 \in \pi ^{-1} b\).

    \(\begin{tikzcd} E \ar[d,"\pi"] \\ B \ar[u,bend left, dotted,"z"] \end{tikzcd}\) homotopy inverses.

    Now we show there is some twisting.

    \(\begin{tikzcd} E_0 \ar[d] & = E - z(B) \\ B \end{tikzcd}\). \(B\) trivial implies \(E_0 \cong B \times (\mathbb{R}^k \setminus  e) \simeq B \times S^{k-1}\).

    We have the tautological line bundle:

    \[
        \begin{tikzcd}
            R \ar[r] & E \ar[d] & = \{ ([x],v) \mid v\in \mathbb{R} x \} & \subset P^n \times \mathbb{R}^{n+1} \\ & P^n & = S^n / x \sim -x 
        \end{tikzcd}
    \]

    We can think of it like \((\text{line}, \text{point on line}) \in E\).
    
    For example, consider \(P^1\). This gives us the open mobius strip.

    \begin{theorem}
        [2.1] \(\gamma_n^1\) is nontrivial for \(n \geq 1\).
    \end{theorem}

    \begin{proof}
        \(E(\gamma_n^1)_0\) is connected \(\iff \not\simeq P^n \times S^0\).
    \end{proof}

    \begin{figure}[H]
        \centering
        \includegraphics[height=0.8\textwidth, angle=90]{img/gamma11.pdf}
        \caption{}
    \end{figure}

    \begin{definition}
        A metric on a vector bundle \(\xi \) is \(g: E \times_B E \to \mathbb{R}\) such that \(\forall b\in B, \pi ^{-1} b \times \pi ^{-1} b \to \mathbb{R}\) is an inner product.
    \end{definition}

    Recall: pullback of \(\begin{tikzcd}
        & B\ar[d,"\beta"] \\ A \ar[r,"\alpha"] & C
    \end{tikzcd}\) is \(A \times_C B = \left\{ (a,b) \mid \alpha (a) = \beta(b) \right\} \subset A \times B\).

    Also see: a vector bundle \(E \to B\) needs all fibers to be vector spaces. For a metric we want them to be inner product spaces.

    A bundle with metric is often callled a Euclidean vector bundle.

    Examples: A \textit{Riemannian manifold} is \(TM\) with a smooth metric [\(g\) is smooth].

    If \(M^n \subset \mathbb{R}^N\) we can use the inner product inherited from \(\mathbb{R}^N\) so it is a riemannian manifold.

    eg the trivial bundle has a metric: \((B \times \mathbb{R}^n) \times_B (B \times \mathbb{R}^n) \to \mathbb{R}^n\) which looks like \(((b,v),(b,w)) \mapsto v \cdot w\). 

    If \(M^n \subset \mathbb{R}^N\), \(TM = \left\{ (x,v) \in M \times \mathbb{R}^N \mid v = \gamma^{\prime} (0), \gamma(0) = x \right\} \) 

    \(\lVert (x,v) \rVert = \lVert v \rVert, g((x,v),(x,w)) = v \cdot w\).

    Then \(\lVert \cdot \rVert : E \to \mathbb{R}_{\geq 0}\) given by \(\lVert v \rVert \coloneqq \sqrt{g(v,v)}\).

    \begin{theorem}
        [Exercises, ch2] Suppose \(B\) is paracompact. We can look at Isomorphism classes of Euclidean vector bundles over \(B\), forget the metric to get isomorphism classes of vector bundles over \(B\):

        \[
            \begin{Bmatrix}
                \text{iso class of euclidean} \\
                \text{vector bundle over \(B\)} 
            \end{Bmatrix} \xrightarrow{\text{forget } g}\begin{Bmatrix}
                \text{iso class of} \\
                \text{vector bundle over \(B\)} 
            \end{Bmatrix} 
        \]

        This is an isomorphism.
    \end{theorem}

    \begin{definition}
        Sections \(s_1, \cdots , s_n\) of rank \(n\) vector bundle \(\begin{tikzcd}
            E \ar[d] \\ B \ar[u, bend right, dotted, "s_i", swap]
        \end{tikzcd}\) are llinearly independent (l.i) if \(\forall b\in B, \left\{ s_1(b), \cdots , s_n(b) \right\} \) is linearly independent in \(\pi ^{-1} (b)\).  
    \end{definition}

    \begin{theorem}
        [2.2] rank \(n\) vector bundle \(\xi \) is trivial iff \(\xi \) has \(n\) l.i. sections.
    \end{theorem}

    \begin{proof}
        \(\implies: s_i(b) \coloneqq (b,\underline{e_i}) \in B \times \mathbb{R}^n\).
        
        \(\impliedby:\) define \(f: B \times \mathbb{R}^n \to E\) by \(\left( b, \sum a_i e_i \right) \mapsto \sum a_i s_i(b)\) 
    \end{proof}

    eg \(T^2\) has \(2\) l.i. sections, thus \(TT^2 \cong T^2 \times \mathbb{R}^2\).

    \begin{figure}[H]
        \centering
        \includegraphics[width=0.3\textwidth]{img/trivialtorus}
        \caption{}
    \end{figure}

    \section*{Wednesday, 9/10/2025}
    
    \section*{Chapter 3: New bundles}

    Homeowrk: pick up problems from chapter 3 (and chapter 2).

    \textit{Abstract definition of bundle} (Steenrod, see D-Kirk 5.2).

    Let \(G\) be a topological group, \(F\) a space, \(G \curvearrowright F\)

    Topological group meaning: \(G\) topological group means \(G\) is a group and a space such that \(G \times G \to G, (a,b) \mapsto ab\) and \(G \to G, a \mapsto a ^{-1}\) are continuous.

    Action of \(G\) on \(F\): \(G \times F \to F\) given by \(ef = f\) and \((gg^{\prime})f = g (g^{\prime} f)\).

    \begin{definition}
        A fiber bundle with structure group \(G\) and fiber \(F\) [\((G,F)\)-bundle] is a map with:
        
        Map \(\begin{tikzcd}
            E \ar[d] \\ F
        \end{tikzcd}\)
        
        Atlas \(\mathcal{A} = \{ \phi : U_{\phi} \times F  \xrightarrow{\approx} \pi ^{-1} U_\phi \} \) 

        Transition functions \(\Theta = \left\{ \theta_{\phi,\psi} : U_{\phi} \cap U_{\psi} \to G \mid \phi,\psi \in \mathcal{A} \right\} \) 

        such that:

        \begin{enumerate}[label=\arabic*)]
            \item \(\{ U_{\phi} \} \) open cover of \(B\).
            \item Fiber preserving homeomorphism:
            
            the following diagram commutes: \(\begin{tikzcd}
                U_{\phi} \times F \ar[rr,"\approx"] \ar[rd] && \pi ^{-1} U_\phi\ar[ld] \\ & U_{\phi} 
            \end{tikzcd}\)   
            \item \(b \in U_{\phi}\cap U_{\psi}, f\in F \implies \psi(b,f) = \phi(b,\theta_{\phi, \psi}(b)f)\) 
            \item \(\theta_{\phi,\psi}(b) = \theta_{\phi, \chi}(b) \theta_{\chi,\psi}(b)\) 
        \end{enumerate} 
    \end{definition}

    Examples:

    \(G\) trivial group implies the bundle is a trivial bundle, \(\begin{tikzcd}
        B \times F \ar[d] \\ B
    \end{tikzcd}\) 

    \(G = \operatorname{GL}(n,\mathbb{R}), F = \mathbb{R}^n\) gives us the rank \(n\) vector bundle. Let \(b\in B\), choose \(\phi , b\in U_{\phi}\). Use the atlas to find bijection \(\pi ^{-1} b \cong \mathbb{R}^n\). This gives us a vector space on \(\pi ^{-1} b\) independent of the choice of \(U_{\phi}\) by the 3rd condition.

    If the \(G\)-action on \(F\) is \textit{effective}, meaning every non-trivial action does something, meaning there is \(f\in F\) such that \(gf\neq f\) for every \(g\in G \setminus \{ e \}\), then we don't need condition 4.

    If \(G=\operatorname{O}(n)\) and \(F=\mathbb{R}^n\) then we have a vector bundle with a metric.
    
    If \(G = \operatorname{GL}(n,\mathbb{R})^+\) and \(F\) is \(\mathbb{R}^n\) then we have an oriented vector bundle.

    If \(G = S_F = \operatorname{Aut}(F)\) where \(F\) is discrete,  then we have a cover.

    For discrete \(G\) with \(F = G\) then we have a regular \(G\)-cover.

    If \(G = \operatorname{Spin}(n), F = \mathbb{R}^n\) then we have a vector bundle with spin structure. 

    Now we start chapter 3. We can do a lot of things on vector spaces, like tensor products. This lets us do stuff with vector bundles as well.

    Some basic constructions involving vector bundles:

    \begin{enumerate}[label=\arabic*)]
        \item Restriction: Let \(\xi\) be a vector bundle, \(\overline{b} \hookrightarrow B\). Then we can let \(\eval{\xi}_{\overline{B}}^{} = \begin{tikzcd}
            \pi ^{-1} \overline{B} \ar[d] \\ \overline{B}
        \end{tikzcd}\)  

        \[
            \begin{tikzcd}
                & \xi \ar[d,phantom,"||"] \\ & E \ar[d, "\pi"] \\ \overline{B} \ar[r, hook] & B 
            \end{tikzcd}
        \]

        \item Induced bundles (= Pullback bundle) Let \(\xi\) be a vector bundle, and \(B_1 \xrightarrow{f} B\). We can \textit{pullback} the bundle and get \(f ^{\ast} \xi\):
        
        \[
            \begin{tikzcd}
                f^{\ast} E = B_1 \times_B E \ar[r] \ar[d] & E \ar[d] \\ B_1 \ar[r,"f"] & B
            \end{tikzcd}
        \]

        in fact \(\eval{\xi}_{\overline{B} }^{} = \operatorname{inc}^{\ast} \xi\).
    \end{enumerate} 

    \begin{definition}
        Bundle map \(g: \eta \to \xi\) [both \(n\)-plane] is given by a commutative diagram which is isomorphism on fibers:

        \[
            \begin{tikzcd}
                E(\eta) \ar[r,"g"] \ar[d] & E(\xi) \ar[d] \\ B(\eta) \ar[r,"\overline{g}"] & B(\xi)
            \end{tikzcd}
        \]
    \end{definition}

    \begin{lemma}
        [3.1] \(\eta \cong \overline{g}^{\ast}\)  as vector bundle over \(B(\eta)\).

        \[
            \begin{tikzcd}
                E(\eta) \ar[rr,"\approx"] \ar[rd] & & \overline{g}^{\ast} E(\xi) \ar[ld] \\ & B(\eta)
            \end{tikzcd}
        \]
    \end{lemma}

    \begin{proof}
        We just need to define the map.

        \[
            E(\eta) \to B(\eta) \times_{B(\xi)} E(\xi)
        \]

        \[
            e \mapsto (\pi(e), g(e))
        \]
    \end{proof}

    pullback stuff works for \((G,F)\)-bundles.

    \section*{Friday, 9/12/2025}
    
    Today we finish chapter 3.

    We can study construction of new vector bundles in the following ways:

    \begin{enumerate}[label=\alph*)]
        \item \textit{Restriction}: \(\eval{\xi}_{\overline{B}}^{}\) for \(\overline{B} \subset B \leftarrow E\)
        \item \textit{Pullback}: \(f^{\ast} \xi\) for \(\overline{B} \xrightarrow{f} B \leftarrow E\)
        \item \textit{Product}: \(\xi_1 \times \xi_2\).
        
        \[
            \begin{tikzcd}
                F_b(\xi_1) \times F_b(\xi_2) \ar[r] & E(\xi_1) \times E(\xi_2) \ar[d] \\ & B(\xi_1) \times B(\xi_2)
            \end{tikzcd}
        \]

        eg \(T(M_1 \times M_2) = TM_1 \times TM_2\).
        \item \textit{Whitney Sum}: We keep the base space the same. Let \(\xi_1, \xi_2\) be vector bundles over the same base space \(B\). Then we can define the whitney sum as the pullback of the diagonal map to the product:
        
        \[
            \xi_1 \oplus \xi_2 \coloneqq \Delta^{\ast} (\xi_1 \times \xi_2)
        \]

        \(B \xrightarrow{\Delta} B \times B\) is \(b \mapsto (b,b)\).

        For example, in \(S^2 \hookrightarrow \mathbb{R}^3\), the whitney sum of the tangent bundle and the normal bundle gives us the trivial bundle: \(\varepsilon ^3_{S^2} = TS^2 \oplus \nu(S^2 \hookrightarrow \mathbb{R}^3)\).

        \item \textit{Subbundles, Quotients and Orthogonal Complements}: A subbundle \(\eta\) of \(\xi\) is \(E(\eta) \subset E(\xi)\) such that \(\eval{\pi}_{E(\eta)}^{} \) is a vector bundle.
        
        \[
            \begin{tikzcd}
                F_b(\eta) \ar[rr, hook] \ar[d] & & F_b(\xi) \ar[d] \\ E(\eta) \ar[rr, hook] \ar[rd] & & E(\xi) \ar[ld] \\ & B
            \end{tikzcd}
        \]

        In order to study quotient, we need \textit{bundle morphisms}. We want the following diagram to be commutative and also want the map to be linear on fibers:

        \[
            \begin{tikzcd}
                E(\eta) \ar[r] \ar[d] & E(\xi) \ar[d] \\ B(\eta) \ar[r] & B(\xi)
            \end{tikzcd}
        \]

        \textit{Bundle morphism over \(B\)} is different: we want the following commutative diagram to be linear on fibers:

        \[
            \begin{tikzcd}
                E(\eta) \ar[rr] \ar[rd] & & E(\xi)  \ar[ld] \\ & B
            \end{tikzcd}
        \]

        An example: suppose we have smooth \(f: M \to N\). Then we have bundle morphism:

        \[
            \begin{tikzcd}
                M \ar[d] \ar[r,"df"] & TN \ar[d] \\ M \ar[r,"f"] & N
            \end{tikzcd}
        \]

        and the bundle morphism \(/ M\):

        \[
            \begin{tikzcd}
                TM \ar[rr,"\cong"] \ar[rd] & & f^{\ast} TN \ar[ld] \\ & M
            \end{tikzcd}
        \]

        We can define quotient bundles from subbundles: subbundle \(\eta\) of \(\xi\) there exists quotient bundle \(\xi / \eta\) so that \(F_b(\xi / \eta)\) are \(F_b(\xi) / F_b(\eta)\). We have bundle map over \(B\) \(\xi \to \xi / \eta\) 

        Bundles \(/ B\) fform abelian category. We have the SES:

        \[
            0 \to \eta \to \xi \to \xi / \eta \to 0
        \]

        We now define normal bundles. Normal bundle of submanifold \(M\) of \(N\) is given by \(\nu (M \hookrightarrow N) = \frac{\left( \eval{TN}_{M}^{} \right) }{TM}\).
        
        \begin{figure}[H]
            \centering
            \includegraphics[width=0.4\textwidth]{img/normal}
            \caption{}
        \end{figure}

        If \(N \subset \mathbb{R}^k\) (or \(N\) Riemannian metric space) then \((TM)^\perp \subset \eval{TN}_{M}\).

        \[
            \begin{tikzcd}
                (TM)^\perp \ar[rr, bend right, "\cong"] \ar[r] & \eval{TN}_{M}^{} \ar[r] & \nu(M \hookrightarrow N) 
            \end{tikzcd}
        \]

        We have \((TN)_M = TM \oplus (TM)^\perp\).

        If \(\xi\) is a bundle with metric and \(\eta\) is a subbundle then \(\xi = \eta \oplus \eta^\perp\) and \(\eta^\perp \cong  \xi / \eta\).

    \end{enumerate}

    If \(B\) is paracompact [eg \(B \subset W\)] then bundles over \(B\) form an exact category [meaning all SES split].
    
    Reason: consider the following SES:

    \[
        0 \to \alpha  \to \beta \to \gamma \to 0
    \]

    Since \(B\) is paracompact we can give \(\beta\) a metric. \(\alpha^ \perp \xrightarrow{\approx} \gamma\) so it splits.
    
    This tells us: if \(M \subset N\) and \(N\) has a Riemannian metric, then,
    
    \(\eval{TN}_{M}^{} = TM \oplus TM^ \perp \cong TM \oplus \nu(M \hookrightarrow N)\).

    \begin{definition}
        Smooth \(f: M \to N\) is a immersion/submersion if \(\forall x\in M\), \(df_x\) is injective/surjective.
    \end{definition}

    For example, consider \(S^1 \to \mathbb{R}^2\) given by \(\bigcirc \to \infty\) is an immersion, since it's locally an embedding.

    \(TS^2 \to S^2\) is a submersion.

    Let \(f: M\to N\) be an immersion. Then, \(\nu(f) = \frac{f^{\ast} TN}{TM}\).

    If \(N\) has a metric then \(TM \cong \eval{TN}_{M}^{} \oplus \nu (f)\).

    \section*{Tuesday, 9/16/2025}

    \subsection*{UCT, Cup and Cap Prodcuts}
    
    Let \(M\) be an abelian group. Then we have homology \(H_i(X,A;M)\) and cohomology \(H^i(X,A;N)\) abelian groups.
    
    The cohomology \(H^i(X,A;N)\) is the cohomology of the following cochain complex: \(H^i(\operatorname{Hom} (S_\bullet(X,A),N))\) 

    `Cohomology eats homology' via the following \textit{Kronecker Pairing}:

    \[
        \langle , \rangle: H^i(X,A;N) \otimes H_i(X,A;M) \to N \otimes_\mathbb{Z} M
    \]

    \[
        [\phi] \otimes \left[ \sum_{i} k_i \sigma_i \otimes m_i \right] \mapsto \sum_{i} k_i \varphi (\sigma_i) \otimes m_i
    \]

    Now we do UCT. Let \(R = \mathbb{Z}\) and \(M = \mathbb{Z}\)-module, i.e. abelian group.
    
    If \(X = \mathbb{R} P^n\) then the cellular chain complex of \(\mathbb{R} P^n\) is:

    \[
        C_\bullet X = \mathbb{Z} \xrightarrow[0 \text{ \(n\) odd}]{2 \text{ \(n\) even}}\cdots \to \mathbb{Z} \xrightarrow{2} \mathbb{Z} \xrightarrow{0} \mathbb{Z}
    \]

    Thus, if \(n\) odd, then \(H_i \mathbb{R} P^n = \begin{dcases}
        \mathbb{Z} , &\text{ if } i = 0,n ;\\
        \mathbb{Z} _2, &\text{ if } i \text{ odd, } 0 < i < n ;\\
        0, &\text{ otherwise} .
    \end{dcases}\)

    If coefficients are in \(\mathbb{Z}_2\) then,
    
    \[
        C_\bullet X \otimes \mathbb{Z}_2 \xrightarrow{0} \mathbb{Z}_2 \xrightarrow{0} \cdots \xrightarrow{0} \mathbb{Z}_2
    \]

    Thus \(H_i (\mathbb{R} P^n;\mathbb{Z}_2) = \mathbb{Z}_2\) for \(0 \leq i \leq n\).

    UCT states that the following is a split short exact sequence:

    \[
        0 \to H_i X \otimes M \to H_i(X;M) \to \operatorname{Tor} (H_{i-1} X,M) \to 0
    \]

    We can say three things about Tor:

    Tor is a functor, \(\operatorname{Tor}: \operatorname{Ab} \times \operatorname{Ab} \to \operatorname{Ab}\).

    If \(M,N\) are f.g. then \(\operatorname{Tor} (M,N) \cong (\text{torsion } M) \otimes_\mathbb{Z} (\text{torsion } N)\) 

    \begin{definition}
        Find an exact sequence of free groups as follows:

        \[
            0 \to F_1 \to F_0 \to M \to 0
        \]

        Then \(\operatorname{Tor}(M,N) = H_1(F_1 \otimes N \to F_0 \otimes N)\).
    \end{definition}

    For example, \(\operatorname{Tor}(\mathbb{Z}_2, \mathbb{Z}_2)\), we have following free groups:

    \[
        0 \to \mathbb{Z} \xrightarrow{\times 2} \mathbb{Z} \xrightarrow{\text{mod } 2} \mathbb{Z}_2 \to 0
    \]

    Tensoring with \(\mathbb{Z}_2\) to get the following: \(\mathbb{Z}_2 \xrightarrow{0} \mathbb{Z}_2\). Then \(H_1\) is the kernel.

    So, \(\operatorname{Tor}(\mathbb{Z}_2, \mathbb{Z}_2) = \mathbb{Z}_2\).

    Now we go back to geometry.

    Suppose we have space \(X\) such that \(H_{i-1} X = \mathbb{Z}_2 \oplus ?\)

    This gives us \(H_i(X) \to \mathbb{Z}_2 \subset H_i(X;\mathbb{Z}_2)\).

    Geometrically, consider \(H_i(X;\mathbb{Z}_2) \to \operatorname{Tor}(H_{i-1} (X); \mathbb{Z}_2)\).
    
    If there is \([a] \in \operatorname{Tor}(H_{i-1} X; \mathbb{Z}_2)\) with \(2a = \partial b\) then section given by \([b] \mapsfrom [a]\)  

    UCT works even if we change \(\mathbb{Z}\) with a PID. For any PID \(R\) we can talk about \(R\)-modules \(M\), then \(H_i(X;M) \cong H_i(X;R) \otimes M \oplus \operatorname{Tor}^R(H_{i-1}(X;R),M)\).

    We want the analogue of UCT for cohomology. This gives us the split exact sequence:

    \[
        0 \to \operatorname{Ext}(H_{i-1} X, M) \to H^i(X;M) \to \operatorname{Hom}(H_i X, M) \to 0
    \] 

    Again, for \(n\) odd consider the chain complex:

    \[
        C_\bullet \mathbb{R} P^n = \mathbb{Z} \xrightarrow{0} \mathbb{Z} \to \cdots \mathbb{Z} \xrightarrow{2} \mathbb{Z} \xrightarrow{0} \mathbb{Z} \to 0
    \]

    For cochain complex we'd simply reverse the arrows:

    \[
        C^\bullet \mathbb{R} P^n = \mathbb{Z} \xleftarrow{0} \mathbb{Z} \leftarrow \cdots \mathbb{Z} \xleftarrow{2} \mathbb{Z} \xleftarrow{0} \mathbb{Z} \leftarrow 0
    \]

    \(H_i \mathbb{R}P^n = \mathbb{Z}\) for \(i=0,n\) and \(\mathbb{Z}_2\) for \(0 < i < n, n\) odd.

    \(H^i (\mathbb{R} P^n; \mathbb{Z} ) = \mathbb{Z}\) for \(i = 0,n\) and \(\mathbb{Z}_2\) for \(0 < i < n, n\) even.
    
    We have: \(\operatorname{Ext}(\text{Free}, M) = 0\).
    
    In general, \(\operatorname{Ext}(A,B)\) is given by: resolve \(A\), apply \(\operatorname{Hom}(-,B)\) cohomolgy.

    Suppose \(0 \to F_1 \to F_0 \to A \to 0\).

    Then, \(\operatorname{Hom}(F_1, B) \xleftarrow{\partial^1} \operatorname{Hom}(F_0, B)\).

    Thus \(\operatorname{Ext}(A,B) = \operatorname{coker} \partial^1\).

    If \(A, B\) are finitely generated then \(\operatorname{Ext}(A,B) \cong (\operatorname{torsion} A) \otimes B\).

    Now, suppose \(R\) is a commutative ring.

    Then \(H^i(X;R) = H^i(\operatorname{Hom}_\mathbb{Z}(X_\bullet X, R))\)
    
    But might be more in the spirit of how we are doing this to do the following:

    \(H^i(X;R) = H^i(\operatorname{Hom}_R(S_\bullet(X;R),R))\)
    
    For \(R\)-modules \(M\),

    \(H^i(X;M) = H^i(\operatorname{Hom}_\mathbb{Z}(S_\bullet X, M)) = H^i(\operatorname{Hom}_R(S_\bullet(X;R),M))\)
    
    Then, \(H^{\ast} (X;R)\) is a graded commutative ring under the cup product.

    \(H^{\ast} (X;R)\) is a graded commutative ring meaning we can write:
    
    \(H^{\ast} (X;R) = \bigoplus_{i \geq 0} H^i(X;R)\) and we have \(H^i(X;R) \otimes_R H^j(X;R) \to H^{i+j} (X;R)\)

    Commutative graded ring meaning \(\alpha \cup \beta = (-1)^{\vert \alpha  \vert \vert \beta \vert} \beta \cup \alpha\).
    
    For De Rham cohomology,

    \(H^i_{\text{DR}}(M;\mathbb{R}) \otimes H^j_{\text{DR}}(M;\mathbb{R})\) we have \(\alpha  \otimes \beta  \mapsto [\alpha \wedge \beta]\) 

    We also have: \(H_{\ast} (M;R)\) is a graded module over \(H^{\ast} (M;R)\) w.r.t.\ cap product.

    For \(\alpha \in H^i(M;R)\) and \(z\in H_j(M;R)\) then \(\alpha \cap z \in H_{j-i}(M;R)\).

    So, cap product by \(\alpha\) eats \(i\) dimensions from \(z\).

    We also have \(\langle \alpha \cup \beta, z \rangle = \langle \alpha , \beta \cap z \rangle \).

    If \(f: X\to Y\) is continuous, we have a ring map \(f^{\ast} : H^{\ast} (Y;R) \to H^{\ast} (X;R)\) by \(f^{\ast} (\alpha \cup \beta) = f^{\ast} \alpha \cup f^{\ast} \beta\).

    Poincar\'e Duality: if \(M^n\) is closed and oriented and connected then \(H_n M \cong \mathbb{Z}\). Choose generator \([M] \in H_n M\).

    Then we have isomorphism \(\cap [M]: H^i M \xrightarrow{\cong} H_{n-i} M  \) 

    Another fact:

    \[
        \frac{H^i M}{\text{torsion}} \otimes \frac{H^{n-i} M}{\text{torsion}} \to \mathbb{Z}
    \]

    is a nonsingular perfect pairing: \(\alpha \otimes \beta\) is given by \((\alpha \cup \beta)[M] \in \mathbb{Z}\).

    Recall \(A \times B \to \mathbb{Z}\) is perfect \(\iff A \xrightarrow{\cong} \operatorname{Hom}(B,\mathbb{Z})\) and \(B \xrightarrow{\cong} \operatorname{Hom}(A,\mathbb{Z})\) are isomorphism.

    In \(\mathbb{C} P^n = e^0 \cup e^2 \cup \cdots \cup e^{2n}\) we have \(H^{\ast} \mathbb{C} P^n \cong \mathbb{Z} [\alpha] / \alpha^{n+1}\), with \(\deg \alpha = 2\).

    This is a truncated polynomial ring.

    We can prove this by Poincar\'e duality and induction on \(n\).

    We also have Kunneth Theorem. If \(R\) is a field, then:

    \[
        H^{\ast} (X;R) \otimes H^{\ast} (X;R) \xrightarrow{\cong} (X \times Y; R)
    \]

    It is only an injection for general ring.

\end{document}