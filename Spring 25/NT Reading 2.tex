\documentclass{article}
\usepackage{amsmath, amsthm, amssymb, amsfonts, mathtools, mathrsfs, enumitem, stmaryrd,physics, cancel, tikz-cd, graphicx, float, booktabs}
\usetikzlibrary{arrows}
\usepackage{geometry}
    \geometry{
        a4paper,
        left = 40mm,
        top = 20mm,
        right = 40mm,
        bottom = 30mm
    }
\setlength{\parindent}{0pt}

\theoremstyle{definition}
\newtheorem{problem}{Problem}
\newtheorem{solution}{Solution}
\newtheorem*{example}{Example}
\newtheorem*{exercise}{Exercise}
\newtheorem*{definition}{Definition}
\newtheorem{theorem}{Theorem}
\newtheorem*{theorem*}{Theorem}
\newtheorem{proposition}[theorem]{Proposition}
\newtheorem*{proposition*}{Proposition}
\newtheorem{lemma}[theorem]{Lemma}
\newtheorem*{lemma*}{Lemma}
\newtheorem{corollary}[theorem]{Corollary}
\newtheorem*{corollary*}{Corollary}
\newtheorem*{remark}{Remark}

\title{NT Reading 2}
\author{Thanic Nur Samin}
\date{\vspace{-5ex}}

\begin{document}
    \maketitle

    \tableofcontents

    \setcounter{section}{8}

    \section{Thursday, 1/16/2025, Cartan Subalgebras (CSA) by Rostyslav}

    \begin{corollary}
        [15.3] Let \(L\) be semisimple. CSA's of \(L\) are precisely the maximal toral subalgebras of \(L\).
    \end{corollary}

    \begin{proof}
        \(\implies\): Let \(H\) be a maximal toral subalgebra.

        \(\implies\) \(H\)-abelian \(\implies H\)-nilpotent, \(N_L(H) = H\) since \(L = H + \bigsqcup_{\alpha\in \Phi} L_\alpha\) with \([H,L_\alpha] = L_\alpha\) for \(\alpha \in \Phi \implies H\)-CSA.
        
        \(\impliedby\): Let \(H\)-CSA. \(x = x_s + x_h\) by Jordan decomposition.

        \(\implies L_0(\operatorname{ad} x_s) \subset L_0(\operatorname{ad} x)\) for \(x\in L\) semisimple.

        \(L_0(\operatorname{ad} x_s) = C_L(x_s)\) since \(\operatorname{ad} x\) is diagonal.
        
        \(H\)- minimal Engel.

        By defintion, \(\implies L_0(\operatorname{ad} x_s) = C_L(x_s) = H\).
        
        But \(C_L(x_g)\) contains maximal toral subalgebra which is CSA. Thus it itself is minimal Engel. \(H\)-maximal toral.

        Detaills:

        \(L \subset H\). \(H\) is CSA \(\iff H\) is nilpotent, \(N_L(H)=H\).


        \(L_0(\operatorname{ad}(x_s)) \subset L_0(\operatorname{ad}(x)).\)
        
        \(L_0(\operatorname{ad}(x_s)) = \{ y\in L \mid \operatorname{ad}(x_s)(y) = 0\} = \{ y\in L \mid [x_s,y] = 0 \} \).
        
        \([x,y] = \underbrace{[x_s,y]}_{=0} + [x_n,y]\).

        \(\operatorname{ad}(x)^m(y) = \operatorname{ad}(x_n)^m(y) = 0\) for \(m \gg 0\).
        
    \end{proof}

    \begin{lemma}
        [15.4.B] Let \(\phi: L \to L^{\prime}\) [epimorphism]. Let \(H^{\prime}\) be CSA of \(L^{\prime}\). Then, any CSA of \(\phi ^{-1} (H^{\prime})\) is also a CSA of \(L\).
    \end{lemma}

    \begin{definition}
        \(x\in L\) is callled \underline{strongly ad-nilpotent} if \(\exists y\in L\) and \(\exists a\neq 0\) eigenvalue of \(\operatorname{ad} y\) such that \(x\in L_a(\operatorname{ad} y)\).

        \[
            [L_a(\operatorname{ad} y), L_b(\operatorname{ad} y)] \subset L_{a,b}(\operatorname{ad} y)
        \]
    \end{definition}

    \begin{remark}
        By \underline{Lemma 15.1} if \(x\) is strongly ad-nilpotent then \(x\) is \(\operatorname{ad}\)-nilpotent.  

        Recall: \(x\in L_a(\operatorname{ad} y)\) means \(\exists m > 0 : (\operatorname{ad} y - a \cdot \operatorname{id}_{})^m (x) = 0\).
    
        This implies that \(\operatorname{ad} x\) is nilpotent after some calculation.
    \end{remark}

    \begin{definition}
        \(\mathcal{N} (L)\) is the set of strongly ad-nilpotent elements. 
    \end{definition}

    \begin{definition}
        \(\mathscr{E} (L) < \operatorname{Int} L \coloneqq \langle \exp (\operatorname{ad} x) \mid x \text{ is ad-nilpotent}  \rangle \) generated by \(\forall \text{exp} \operatorname{ad} x\) where \(x\in \mathcal{N} (L)\).
    \end{definition}

    \begin{remark}
        \(\mathcal{N}(L)\) is stable under \(\forall x \in \operatorname{Aut}(L)\).
        
        Thus, \(\mathscr{E}(L) \trianglelefteq \operatorname{Aut}(L)\).

        \(K \subset L \implies \mathcal{N}(K) \subset \mathcal{N}(L)\).

        Then,
    \end{remark}

    \begin{definition}
        \(\mathscr{E} (L,K)\) is generated by \(\exp \operatorname{ad} x \forall x\in \mathscr{E}(K)\) 
    \end{definition}

    Then, \(\mathscr{E}(K) = \mathscr{E}(L,K)\).

    If \(\phi : L \to L^{\prime}\) is an epimorphism then \(\phi(L_a(\operatorname{ad} y)) = L^{\prime}_a(\operatorname{ad} \phi(y))\).
    
    \(\implies  \phi (\mathcal{N}(L)) = \mathcal{N}(L^{\prime})\).

    \begin{lemma}
        [16.1] Let \(\phi: L \to L^{\prime}\) be an epimorphism. If \( \sigma'^{\prime} \in \mathscr{E}(L^{\prime}) \implies \exists \sigma\in \mathscr{E}(L)\) such that:

        \begin{center}
            \begin{tikzcd}
                L \ar[r,"\phi"] \ar[d,"\sigma"] & L^\prime \ar[d,"\sigma^\prime"] \\ L \ar[r,"\phi"] & L'
            \end{tikzcd}
        \end{center}

    \end{lemma}

    \begin{proof}
        If \(\sigma ^{\prime} = \exp \operatorname{ad} x^{\prime}\) \(x^{\prime} \in \mathcal{N}(L^{\prime})\) \(\exists x\) s.t. \(\phi(x)=x^{\prime}\).
        
        \(\forall z\in L, (\phi \circ \exp \operatorname{ad}_L x)(z) = \phi(z + [x,z] + [x,[x,z]]) = \phi(z) + [x^{\prime}, \phi(z)] + [x^{\prime}, [x^{\prime},\phi(z)]]\)
        
        \(= (\exp \operatorname{ad}_L x^{\prime})(\phi(z)) \implies\) QED. 
        
    \end{proof}

    \begin{theorem}
        [16.2] Let \(L\) be solvable. Let \(H_1, H_2\) be CSA's of \(L\). 
        
        Then, \(H_1\) is conjugate with \(H_2\) by an element of \(\mathscr{E}(L)\).
    \end{theorem}

    \section{Thursday, 1/23/2025, Cartan Subalgebra (CSA) by Rostyslav}

    \begin{proof}
        Induction on \(\dim L\).

        Base case: suppose \(\dim L=1\). Since \(L\) is nilpotent \(L\) must be trivial.

        Assume that \(L\) is not nilpotent. \(L\)--solvable \(\implies\) \(L\) has non-zero abelian ideals. eg least non-zero term of the derived series. Choose such \(A\) of least possible dimension.

        Set \(L^{\prime} = L / A\). We have \(\phi: L \to L / A = L^{\prime}\) given by \(x \mapsto x^{\prime}\).

        Lemma 15.4(image of CSA is CSA) implies \(H_1^{\prime} , H_2^{\prime}\) are CSAs of the solvable algebra \(L^{\prime}\). By induction \(\exists \sigma \in \mathscr{E} (L^{\prime})\) such that \(\sigma(H_1^{\prime})=H_2^{\prime}\).

        Lemma 16.1(the commutative diagram) implies \(\exists \sigma \in \mathscr{E} (L)\) such that the diagram commutes. So, \(\sigma\) maps \(K_1 = \phi ^{-1} (H_1^{\prime})\) to \(\phi ^{-1} (H_2^{\prime}) = K_2\)
        
        But now \(H_2\) and \(\sigma(H_2)\) are both CSA''s of \(K_2\).

        If \(K_2\) is smaller than \(L\) induction allows us to find \(\tau ^{\prime} \in \mathscr{E} (K_2)\) such that \(\tau ^{\prime} \sigma (H_1)=H_2\).

        But \(\mathscr{E}(K_2)\) consists of restrictions of \(\mathscr{E}(L,K_2)\) to \(K_2\).

        \(\exists \tau\) such that \(\tau \sigma (H_1) = H_2\) for \(\tau \in \mathscr{E} (L) \implies\) done.

        Otherwise \(L = K_2 = \sigma(L_1)\).

        \(K_2 = K_1\) and \(L = H_2 + A = H_1 + A\).

        Theorem 15.3 \(\implies\) CSA \(H_2 = L_0(\operatorname{ad} x)\) for suitable \(x\in L\).

        \(A\) being \(\operatorname{ad} x\) stable, so by lemma 15.1,

        \[
            A = A_0 (\operatorname{ad} x) \oplus A_{\ast} (\operatorname{ad} x)
        \]

        and each summand is stable under \(H_2 + A\).

        Since \(A\)  is minimal, \(A = A_0 (\operatorname{ad} x)\) or \(A = A_{\ast}(\operatorname{ad} x)\)  

        \(A\) cannot be equal to \(A_0(\operatorname{ad} x)\) since in that case \(A \subset H_2, L = H_2\).

        But since \(L\) is \underline{not} nilpotennt we have a contradiction.

        Thus, \(A = A_{\ast} (\operatorname{ad} x) \implies A = L_{\ast} (\operatorname{ad} x)\). Since \(L = H_1 + A\) we can write \(x = y+z\) with \(y\in H_1, z \in A_{\ast} (\operatorname{ad} x)\).
        
        Since \(\operatorname{ad} x\) is invertible on \(L_{\ast} (\operatorname{ad} x)\) we can write \(z = [x,z^{\prime}]\) where \(z^{\prime} \in L_{\ast} (\operatorname{ad} x)\).
        
        \(A\)-abelian \(\implies (\operatorname{ad} z^{\prime})^2 = 0\).
        
        Thus, \(\exp \operatorname{ad} z^{\prime} = 1_L + \operatorname{ad} z^{\prime}\).

        Applying to \(x\) we have, \(x - z = y\).

        \(\implies H_0 = L_0(\operatorname{ad} y)\) must also be a CSA of \(L\). Since \(y\in H_1, H \supset H_1\) and both minimal Engel, \(H = H_1\).

        \(H_1\) is conjuate to \(H_2\) using \(\exp \operatorname{ad} z^{\prime}\).

        We only need to show that, \(\exp \operatorname{ad} z^{\prime} \in \mathscr{E} (L)\).

        \(z^{\prime} \) can be written as sum of strongly \(\operatorname{ad}\)-nilpotent elements of \(A = L_{\ast} (\operatorname{ad} x)\)
        
        \(\implies A\)-abelian so \(\exp \operatorname{ad} z^{\prime} = \prod \exp \operatorname{ad} z_i \in \mathscr{E} (L)\).

        So we're done.

    \end{proof}

    Consider \(B = \) upper triangular matrices. It is a lie algebra. What is a CSA of this?

    Attempt: we have \(H =\) upper triangular matrices with \(0\) diagonal. \(N_B (H) = B\). But it is not nilpotent so it doesn't work.

    However, attempt 2: we can take \(H = \) diagonal matrices.

    In fact, if \(\mathfrak{g} \subset \mathfrak{gl}_n\) is a subalgebra and \(\mathfrak{g}\) contains a diagonal matrix with all entries different, then the subalgebra \(\mathfrak{h}\) of \(\mathfrak{g}\) containing all diagonal matrices on \(\mathfrak{g}\) is a CSA.

    \begin{definition}
        A maximal solvable subalgebra of a lie algebra \(L\) is called a \underline{Borel} \underline{subalgebra}.
    \end{definition}

    \begin{lemma}
        [16.3.A] If \(B\) is a borel subalgebra of \(L\) then \(B = N_L(B)\). Aka, Borel subalgebras are self normalizing.
    \end{lemma}

    \begin{proof}
        Let \(x\in N_L(B)\). Then, \(B + Fx\) is a subalgebra of \(L\). It is solvable since \([B+Fx,B+Fx] \subset B\). Since \(B\) is maximal, we must have \(x\in B\).
    \end{proof}

    \begin{lemma}
        [16.3.B] If \(\operatorname{Rad} L \neq L\) then there is a bijection between the sets of Borel subalgebras of \(L\) and Borel subalgebras of \(L / \operatorname{Rad} L\).
    \end{lemma}

    \begin{proof}
        \(\operatorname{Rad} L\) is a solvable ideal of \(L\).

        Therefore, \(B + \operatorname{Rad} L\) is a solvable subalgebra of \(L\).

        \(\implies\) by maximality, we're done.
    \end{proof}

    \begin{definition}
        Let \(H\) be a CSA in a semisimple lie algebra \(L\), \(\Phi\) a root system of \(L\) relative to \(H\). Fix a base \(\Delta\) and a set of positive roots.

        Set \(B(\Delta) = H \sqcup_{\alpha > 0} L_\alpha\)

        And \(N(\Delta) = \sqcup_{\alpha \not> 0} L_\alpha\).

        Then \(B(\Delta)\) is the standard Borel subalgebra relative to \(H\).

        \(N(\Delta)\) is the derived algebra of \(B(\Delta)\).
    \end{definition}

    \begin{lemma}
        [16.3.C.1]

        \(N(\Delta)\) is nilpotent.

        If \(x\in L_\alpha (\alpha > 0)\) then,

        Application of \(\operatorname{ad} x\) to root vector increases the \underline{height} by at least \(1\).

        \(\implies\) decreasing central series goes to zero.

        Thus, \(B(\Delta)\) is solvable.

        Let \(K \supset B(\Delta)\). Then, \(K\) is stable under \(\operatorname{ad} H\).

        Then \(K\) must include some \(L_\alpha\) with \(\alpha < 0\).

        Thus, simple \(S_\alpha \subset K \implies K\) is not solvable.

    \end{lemma}

    Note: \(S_\alpha = \langle L_\alpha , L_{-\alpha}, H \rangle \) 

    \begin{lemma}
        [16.3.C2] All standard Borel subalgebras of \(L\) relative to \(H\) are conjugate under \(\mathscr{E} (L)\).
    \end{lemma}

    \begin{proof}
        By 14.3 the reflection \(\sigma_\alpha\) acting on \(H\) may be extended to an inner automorphism \(\tau_\alpha\) of \(L\) which is, by construction, in \(\mathscr{E}(L)\).

        \(\tau_\alpha\) would send \(B(\Delta)\) to \(B(\sigma\Delta)\).

        The \underline{Weyl Group} is generated by those reflections, so we see that \(\mathscr{E}(L)\) will act transitively on standard Borel subalgebras relative to \(H\).

    \end{proof}

    \begin{theorem}
        [16.4] The Borel subalgebras of an arbitrary Lie algebra \(L\) are all conjugate under \(\mathscr{E}(L)\).
    \end{theorem}

    We omit the proof for now.

    \begin{corollary}
        All the CSAs of lie algebra \(L\) are all conjugate under \(\mathscr{E}(L)\). 
    \end{corollary}

    \begin{proof}
        Let \(H,H^{\prime}\) be CSAs. They're nilpotent by definition. Therefore, they're solvable. 

        Therefore, \(H \subset B, H^{\prime} \subset B^{\prime}\) where \(B,B^{\prime}\) are some borel subalgebra.

        By the previous theorem, \(\exists \sigma \in \mathscr{E}(L)\) such that \(\sigma(B) = B^{\prime}\).

        Thus, \(\sigma(H)\) and \(H^{\prime}\) are CSAs of \(B^{\prime}\).

        By theorem 16.2, \(\exists \tau^{\prime} \in \mathscr{E}(B^{\prime})\) such that \(\tau ^{\prime} \sigma (H) = H^{\prime}\).

        But \(\tau ^{\prime} \) is a restriction of some \(\tau \in \mathscr{E}(L,B^{\prime}) \subset \mathscr{E}(L)\). Therefore,

        \[
            \tau \sigma (H) = H^{\prime}, \, \tau \sigma \in \mathscr{E}(L)
        \]

    \end{proof}

\end{document}