\documentclass{article}
\usepackage{amsmath, amsthm, amssymb, amsfonts, mathtools,enumitem, stmaryrd,physics, cancel, tikz-cd, graphicx, float, booktabs}
\usetikzlibrary{arrows}
\usepackage{geometry}
    \geometry{
        a4paper,
        left = 40mm,
        top = 20mm,
        right = 40mm,
        bottom = 30mm
    }
\setlength{\parindent}{0pt}

\theoremstyle{definition}
\newtheorem{problem}{Problem}
\newtheorem{solution}{Solution}
\newtheorem*{example}{Example}
\newtheorem*{exercise}{Exercise}
\newtheorem*{definition}{Definition}
\newtheorem{theorem}{Theorem}
\newtheorem*{theorem*}{Theorem}
\newtheorem{proposition}[theorem]{Proposition}
\newtheorem*{proposition*}{Proposition}
\newtheorem{lemma}[theorem]{Lemma}
\newtheorem*{lemma*}{Lemma}
\newtheorem{corollary}[theorem]{Corollary}
\newtheorem*{corollary*}{Corollary}
\newtheorem*{remark}{Remark}

\title{M623 Geometric Topology I}
\author{Thanic Nur Samin}
\date{}


\begin{document}
    \maketitle

    \section*{Monday, 8/25/2025}

    Textbook: \textit{Characteristic Classes} by Milnor and Stasheff. Hereafter referred by MS.

    \underline{Read Chapter 1 and 2 of MS.}

    \begin{definition}
        [\(n\)-manifold]

        Two different variants: embedded and abstract.

    

        Abstract: \((M, \mathcal{A})\) where \(\mathcal{A}\) is an atlas.
    \end{definition}

    Embedded: \(M \subset \mathbb{R}^A\). Here, \(A =\) index set, \(\mathbb{R}^A = \text{func}(A,\mathbb{R})\) with the product topology.

    \(M\) Hausdorff space, \(U \subset M\) open, \(V \subset \mathbb{R}^n\) open.
        
    Chart \(\phi : U \xrightarrow{\approx} V\) homeomorphism.

    Parameterization (ptz) \(h: V \xrightarrow{\approx} U\)

    We want some calculus.

    Let open \(V \subset \mathbb{R}^n\).

    A function \(f: V \to \mathbb{R}\) is \textit{smooth} if all partials of all orders exist: \(\frac{\partial^k f}{\partial x_{i_1} \cdots \partial x_{i_{p}}}\).

    \(f: V \to \mathbb{R}^A\) is \textit{smooth} if \(f_\alpha\) smooth \(\forall \alpha \in A\).

    \[
        \begin{tikzcd}
            V \ar[rr, bend right,"f_\alpha"'] \ar[r,"f"] & \mathbb{R}^A \ar[r,"pr_{\alpha}"] & \mathbb{R}
        \end{tikzcd}
    \]

    We can go from abstract manifold to embedded manifold.

    Let \(A = C^{\infty} (M,R)\).

    \(M \xrightarrow{i}\mathbb{R}^A\) where \(i(x) = (f \mapsto f(x))\).

    We can go to the reverse direction easily once we have all the definitions.

    \begin{definition}
        Two charts \((\phi_1: U_1 \to V_1)\) and \((\phi_2: U_2 \to V_2)\) are compatible (or smoothly compatible) if \(\phi_2 \circ \phi_1 ^{-1}\) is smooth. Explicitly,

        \(\phi_1(U_1 \cap U_2) \xrightarrow{\phi_2 \circ \phi_1 ^{-1}} \phi_2(U_1 \cap U_2)\) needs to be smooth. 
    \end{definition}

    \begin{definition}
        Parameterization \(h: V \to U\) is \textit{smooth} (assume \(M \subset \mathbb{R}^A\)) if:

        \[
            \begin{tikzcd}
                V \ar[r,"h"] \ar[rrr, bend right,"h"] & U \ar[r,hook] & M \ar[r,hook] & \mathbb{R}^n
            \end{tikzcd}
        \]

        is smooth.

        and has rank \(n\). ie, \(\forall v\in V\) the Jacobian:

        \[
            dh = \left( \frac{\partial_\alpha h}{\partial x_j}(v) \right)
        \]
    
        has rank \(n\).

        eg \(x \mapsto x^3\) is a parameterization which is not smooth, since the Jacobian has rank \(0\) at \(0\).
    \end{definition}

    Now we can properly define manifolds.

    \begin{definition}
        [Embedded Smooth \(n\)-Manifold] \(M \subset \mathbb{R}^A\) so that \(\forall x\in M\) there exists a smooth rank \(n\) parameterization \(h : V \to U \ni x\).

        We assume \(M\) is Hausdorff.
    \end{definition}

    We can now define a Cateogry of Embedded Manifolds.

    \begin{definition}
        [Category of Embedded Manifolds] \(\operatorname{Embmfld}\).

        \underline{Objects}: embedded \(M \subset \mathbb{R}^A\) of \(\dim n\) for some \(n\).

        \underline{Morphisms}: Smooth Maps (has to be defined carefully, restricting in Euclidean space).

        Diffeomorphism = invertible morphism.
    \end{definition}

    Let \((M \subset \mathbb{R}^A), (N \subset \mathbb{R}^B)\). \(f: M \to N\) is smooth if \textit{locally smooth}, meaning \(\forall x\in M, \exists\) smooth parameterization \(h: V \to U \ni x\) such that \(V \to U \hookrightarrow M \xrightarrow{f} N \to \mathbb{R}^B\) is smooth.

    Now we can define abstract manifold independend of embedded manifolds.

    \begin{definition}
        [Abstract Manifold] Let \(M\) be Hausdorff. An \(n\)-\textit{atlas} on \(M\) is a set \(\mathcal{A}=\left\{ \phi_\alpha : U_\alpha \xrightarrow{\approx} V_\alpha \subset \mathbb{R}^n \right\}\) of compactible \(n\)-charts such that \(\{ U_\alpha \} \) covers \(M.\)

        Atlas \(\mathcal{A}\) and \(\mathcal{A}^{\prime}\) are \underline{compatible} if all charts are.

        Fact: Every atlas is contained in a unique maximal atlas.

        Then an abstract manifold is \((M,\mathcal{A})\) with a maximal \(n\)-atlas.
    \end{definition}

    \section*{Wednesday, 8/27/2025}
    
    Recall: embedded \(n\)-manifold \(M \subset \mathbb{R} ^ A\): \(\forall x\in M, \exists\) smooth, rank \(n\) parameterization \(h: V \to U \subset M\) such that \(x\in U\). We assume \(M\) is Hausdorff.

    Abstract \(n\)-manifold: \((M, \mathcal{A})\) where \(\mathcal{A}\) is an \(n\)-atlas, so \(\mathcal{A}  = \{ \text{charts } \phi_\alpha : U_\alpha \xrightarrow{\cong} V_\alpha \}\) such that \(\{ U_\alpha \}\) cover \(M\) and \(\{ \phi_\alpha \} \) smoothly compatible. We assume \(M\) is Hausdorff.
    
    \begin{remark}
        If we have an abstract manifold we have a surjective map \(\coprod V_\alpha \overset{\coprod \phi_\alpha ^{-1}}{\twoheadrightarrow} M\).
        
        Then we can define \(M \cong \frac{\coprod V_\alpha}{\sim}\). This gives us another definition of a manifold. 
    \end{remark}

    \begin{exercise}
        Define smooth \(f: (M,\mathcal{A}) \to (N,\mathcal{B})\).

        Not hard, just annoying to get the definitions right!
    \end{exercise}

    \begin{theorem}
        Categories of abstract manifolds and embedded manifolds are equivalent.

        \[
            \text{EmbMflds} \simeq \text{absMflds}  
        \]
    \end{theorem}

    Recall equivalent categories:

    \begin{definition}
        Categories \(\mathcal{C}\) and \(\mathcal{D}\) are equivalent (Notation: \(\mathcal{C} \simeq \mathcal{D}\)): If there are functors \(\mathcal{C} \xrightarrow{F} \mathcal{D}\) and \(\mathcal{D} \xrightarrow{G} \mathcal{C}\) such that \(F \circ G\) and \(G \circ F\) are naturally isomorphic to the respective identities.
    \end{definition}

    We need some more definitons.

    \begin{definition}
        A skeleton of \(\mathcal{C}\) is \(\operatorname{Sk} \mathcal{C} \subset \mathcal{C}\) is a full subcategory \(\forall c\in \mathcal{C}, \exists ! c^{\prime} \in \operatorname{Sk} \mathcal{C}\) such that \(c \cong c^{\prime}\).

        \(\mathcal{A} \subset \mathcal{B}\) is full if \(\forall a,a^{\prime} \in \operatorname{Ob} \mathcal{A}, \mathcal{A}(a,a^{\prime}) \xrightarrow{\cong} \mathcal{B}(a,a^{\prime})\) 
    \end{definition}

    For example, let \(\mathcal{C} =\) finite sets. Then \(\operatorname{Sk} \mathcal{C} = \{ 1 \}, \{ 1,2 \}, \{ 1,2,3 \}, \cdots\) 

    \begin{theorem}
        \(\mathcal{C} \simeq \mathcal{D} \iff \operatorname{Sk} \mathcal{C} \cong \operatorname{Sk} \mathcal{D}\).
    \end{theorem}

    Note that \(\mathcal{C} \simeq \operatorname{Sk} \mathcal{C}\) so one direction is trivial.

    \begin{lemma}
        [1.1] Let \(h\) and \(h^{\prime}\) be smooth rank \(n\) on \(M \subset \mathbb{R}^A\). Then \(h ^{-1} \circ  h^{\prime}\) is smooth (thus a diffeomorphism).
        
        Let \(V\) and \(V^{\prime}\) be the domain of \(h\) and \(h^{\prime}\) respectively. Then \(h ^{-1} \circ h^{\prime} : (h^{\prime})^{-1} (V\cap V^{\prime}) \to h ^{-1} (V \cap V^{\prime})\) 
    \end{lemma}

    \begin{corollary}
        \(\mathcal{A} = \{ h ^{-1} \mid h \text{ parameterization} \}\) is \(n\)-atlas on \(M\).

        This gives us \(\text{EmbMflds}\to \text{AbstMflds}\).
    \end{corollary}

    \begin{proof}
        This is the proof of lemma 1.1, lemma 3 in the notes.

        Assume \(V = V^{\prime}\). WTS: \((h^{\prime})^{-1} V \to h ^{-1} (V)\) is smooth.

        For \(x\in V\) choose \(\alpha_1, \cdots , \alpha_n \in A\) such that \(\det \left( \frac{\partial \alpha_i}{\partial x_j} (x) \right)  \not\equiv 0\). 

        We have:

        \[
            \begin{tikzcd}
                M \ar[r, hook] & \mathbb{R}^A \ar[d, "pr_{\alpha_1 \cdots \alpha_n}"] \\ V \ar[u,"h"] \ar[r,dotted] & \text{subset of } \mathbb{R}^n
            \end{tikzcd}
        \]

        Then, by the inverse function theorem, the dotted map is locally invertible.

        \(h^{-1} \circ h^{\prime} = (pr \circ inc \circ h)^{-1} \circ inc \circ pr \circ h^{\prime}\) near \(h ^{-1} x\).

    \end{proof}

    Given abstract \((M,\mathcal{A})\), let \(A = C^{\infty} (M,\mathbb{R})\) smooth functions.
    
    \(i: M \to \mathbb{R}^A, x \mapsto (f \to f(x))\).
    
    Let \(M_1 = i(M)\).
    
    \begin{lemma}
        [1.5] \(M_1 \subset \mathbb{R}^A\) is \(\text{EmbMfld}\). \(M \xrightarrow{i} M_1\) is diffeomorphism.
    \end{lemma}

    \subsection*{Definition of tangent vector, tangent space and tangent bundle}

    \begin{definition}
        [Tangent Vector] is velocity vector of a curve.

        We have defined morphisms. Consider the embedded case: suppose we have smooth \(\gamma : \mathbb{R} \to M \subset \mathbb{R}^A\). Then,

        \[
            \gamma^{\prime} (0) = \lim_{h \to 0} \frac{\gamma (h) - \gamma(0)}{h} \in \mathbb{R}^A
        \]

        is a tangent vector
    \end{definition}

    \begin{definition}
        [Tangent Space] Suppose \(x\in M \subset \mathbb{R}^A\), an \(n\)-dim embedded manifold. \(T_x M =\) tangent space of \(M\) at \(x\). This is:
        
        \[
            \left\{ \gamma ^{\prime} (0) \mid \gamma(0) = x \right\} \subset \mathbb{R}^A
        \]

        an \(n\)-dim subspace.
    \end{definition}

    We are going to bundle this together.

    \begin{definition}
        [Tangent Bundle]

        \(TM = \left\{ (x,v) \in M \times \mathbb{R}^A \mid v\in T_x M \right\}\).

        By definition, \(TM \subset M \times \mathbb{R}^A\) so this is in fact a topological space.

        We have a projection map \(TM \xrightarrow{\pi} M\) by \((x,v) \mapsto x\).
    \end{definition}

    \begin{remark}
        Fibers of \(\pi\), \(\pi ^{-1} (x)\) are vector spaces: \(\pi ^{-1} (x) = T_x M\).

        Then, \(TM = \bigcup_{x\in M} \{ x \} \times T_x M\).

        Abuse of notation lets us write this as \(\bigcup T_x M\).

        Thus, tangent bundle is in fact a bundle of tangents.
    \end{remark}

    What about abstract manifolds \((M,\mathcal{A})\)?

    We can define \(TM\) as follows:

    \begin{itemize}
        \item \(M \subset \mathbb{R}^{C^{\infty}(M,\mathbb{R})}\).
        \item \(TM = \frac{\coprod V_\alpha \times \mathbb{R}^n}{\sim}\)
        \item \(T_x M =\) velocity vector of curves.
        \item derivations.
    \end{itemize} 

    Suppose we have smooth function between manifolds \(f: M \to N\). \(\forall x\in M\) we can define linear \(\mathrm{d}f_x : T_x M \to T_{f(x)} N\), \(\gamma ^{\prime} (0) \mapsto (f \circ \gamma)^{\prime} (0)\). \(\mathrm{d} f_x\) is a map between vector spaces, so it is a linear transformation. It is the `Jacobian'.
    
    Then we have \(\mathrm{d} f: TM \to TN\) such that \(\mathrm{d} f(x,v) = \mathrm{d} f_x(v)\).

    We also have the chain rule: \(\mathrm{d} (f \circ g) = \mathrm{d} f \circ \mathrm{d} g\) 

    \section*{Friday, 8/29/2025}
    
    

\end{document}