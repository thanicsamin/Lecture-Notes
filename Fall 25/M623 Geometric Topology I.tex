\documentclass{article}
\usepackage{amsmath, amsthm, amssymb, amsfonts, mathtools, enumitem, stmaryrd,physics, cancel, tikz-cd, graphicx, float, booktabs}
\usetikzlibrary{arrows}
\usepackage{geometry}
    \geometry{
        a4paper,
        left = 20mm,
        top = 20mm,
        right = 20mm,
        bottom = 30mm
    }
\setlength{\parindent}{0pt}
\setlength{\parskip}{\baselineskip}%

\theoremstyle{definition}
\newtheorem{problem}{Problem}
\newtheorem{solution}{Solution}
\newtheorem*{example}{Example}
\newtheorem*{exercise}{Exercise}
\newtheorem*{definition}{Definition}
\newtheorem{theorem}{Theorem}
\newtheorem*{theorem*}{Theorem}
\newtheorem{proposition}[theorem]{Proposition}
\newtheorem*{proposition*}{Proposition}
\newtheorem{lemma}[theorem]{Lemma}
\newtheorem*{lemma*}{Lemma}
\newtheorem{corollary}[theorem]{Corollary}
\newtheorem*{corollary*}{Corollary}
\newtheorem*{remark}{Remark}

\title{M623 Geometric Topology I}
\author{Taught by: Dr. James Davis \\ Written by: Thanic Nur Samin}
\date{}


\begin{document}
    \maketitle

    \section*{Monday, 8/25/2025}

    Textbook: \textit{Characteristic Classes} by Milnor and Stasheff. Hereafter referred by MS.

    \underline{Read Chapter 1 and 2 of MS.}

    \begin{definition}
        [\(n\)-manifold]

        Two different variants: embedded and abstract.

    

        Abstract: \((M, \mathcal{A})\) where \(\mathcal{A}\) is an atlas.
    \end{definition}

    Embedded: \(M \subset \mathbb{R}^A\). Here, \(A =\) index set, \(\mathbb{R}^A = \text{func}(A,\mathbb{R})\) with the product topology.

    \(M\) Hausdorff space, \(U \subset M\) open, \(V \subset \mathbb{R}^n\) open.
        
    Chart \(\phi : U \xrightarrow{\approx} V\) homeomorphism.

    Parameterization (ptz) \(h: V \xrightarrow{\approx} U\)

    We want some calculus.

    Let open \(V \subset \mathbb{R}^n\).

    A function \(f: V \to \mathbb{R}\) is \textit{smooth} if all partials of all orders exist: \(\frac{\partial^k f}{\partial x_{i_1} \cdots \partial x_{i_{p}}}\).

    \(f: V \to \mathbb{R}^A\) is \textit{smooth} if \(f_\alpha\) smooth \(\forall \alpha \in A\).

    \[
        \begin{tikzcd}
            V \ar[rr, bend right,"f_\alpha"'] \ar[r,"f"] & \mathbb{R}^A \ar[r,"pr_{\alpha}"] & \mathbb{R}
        \end{tikzcd}
    \]

    We can go from abstract manifold to embedded manifold.

    Let \(A = C^{\infty} (M,R)\).

    \(M \xrightarrow{i}\mathbb{R}^A\) where \(i(x) = (f \mapsto f(x))\).

    We can go to the reverse direction easily once we have all the definitions.

    \begin{definition}
        Two charts \((\phi_1: U_1 \to V_1)\) and \((\phi_2: U_2 \to V_2)\) are compatible (or smoothly compatible) if \(\phi_2 \circ \phi_1 ^{-1}\) is smooth. Explicitly,

        \(\phi_1(U_1 \cap U_2) \xrightarrow{\phi_2 \circ \phi_1 ^{-1}} \phi_2(U_1 \cap U_2)\) needs to be smooth. 
    \end{definition}

    \begin{definition}
        Parameterization \(h: V \to U\) is \textit{smooth} (assume \(M \subset \mathbb{R}^A\)) if:

        \[
            \begin{tikzcd}
                V \ar[r,"h"] \ar[rrr, bend right,"h"] & U \ar[r,hook] & M \ar[r,hook] & \mathbb{R}^n
            \end{tikzcd}
        \]

        is smooth.

        and has rank \(n\). ie, \(\forall v\in V\) the Jacobian:

        \[
            dh = \left( \frac{\partial_\alpha h}{\partial x_j}(v) \right)
        \]
    
        has rank \(n\).

        eg \(x \mapsto x^3\) is a parameterization which is not smooth, since the Jacobian has rank \(0\) at \(0\).
    \end{definition}

    Now we can properly define manifolds.

    \begin{definition}
        [Embedded Smooth \(n\)-Manifold] \(M \subset \mathbb{R}^A\) so that \(\forall x\in M\) there exists a smooth rank \(n\) parameterization \(h : V \to U \ni x\).

        We assume \(M\) is Hausdorff.
    \end{definition}

    We can now define a Cateogry of Embedded Manifolds.

    \begin{definition}
        [Category of Embedded Manifolds] \(\operatorname{Embmfld}\).

        \underline{Objects}: embedded \(M \subset \mathbb{R}^A\) of \(\dim n\) for some \(n\).

        \underline{Morphisms}: Smooth Maps (has to be defined carefully, restricting in Euclidean space).

        Diffeomorphism = invertible morphism.
    \end{definition}

    Let \((M \subset \mathbb{R}^A), (N \subset \mathbb{R}^B)\). \(f: M \to N\) is smooth if \textit{locally smooth}, meaning \(\forall x\in M, \exists\) smooth parameterization \(h: V \to U \ni x\) such that \(V \to U \hookrightarrow M \xrightarrow{f} N \to \mathbb{R}^B\) is smooth.

    Now we can define abstract manifold independend of embedded manifolds.

    \begin{definition}
        [Abstract Manifold] Let \(M\) be Hausdorff. An \(n\)-\textit{atlas} on \(M\) is a set \(\mathcal{A}=\left\{ \phi_\alpha : U_\alpha \xrightarrow{\approx} V_\alpha \subset \mathbb{R}^n \right\}\) of compactible \(n\)-charts such that \(\{ U_\alpha \} \) covers \(M.\)

        Atlas \(\mathcal{A}\) and \(\mathcal{A}^{\prime}\) are \underline{compatible} if all charts are.

        Fact: Every atlas is contained in a unique maximal atlas.

        Then an abstract manifold is \((M,\mathcal{A})\) with a maximal \(n\)-atlas.
    \end{definition}

    \section*{Wednesday, 8/27/2025}
    
    Recall: embedded \(n\)-manifold \(M \subset \mathbb{R} ^ A\): \(\forall x\in M, \exists\) smooth, rank \(n\) parameterization \(h: V \to U \subset M\) such that \(x\in U\). We assume \(M\) is Hausdorff.

    Abstract \(n\)-manifold: \((M, \mathcal{A})\) where \(\mathcal{A}\) is an \(n\)-atlas, so \(\mathcal{A}  = \{ \text{charts } \phi_\alpha : U_\alpha \xrightarrow{\cong} V_\alpha \}\) such that \(\{ U_\alpha \}\) cover \(M\) and \(\{ \phi_\alpha \} \) smoothly compatible. We assume \(M\) is Hausdorff.
    
    \begin{remark}
        If we have an abstract manifold we have a surjective map \(\coprod V_\alpha \overset{\coprod \phi_\alpha ^{-1}}{\twoheadrightarrow} M\).
        
        Then we can define \(M \cong \frac{\coprod V_\alpha}{\sim}\). This gives us another definition of a manifold. 
    \end{remark}

    \begin{exercise}
        Define smooth \(f: (M,\mathcal{A}) \to (N,\mathcal{B})\).

        Not hard, just annoying to get the definitions right!
    \end{exercise}

    \begin{theorem}
        Categories of abstract manifolds and embedded manifolds are equivalent.

        \[
            \text{EmbMflds} \simeq \text{absMflds}  
        \]
    \end{theorem}

    Recall equivalent categories:

    \begin{definition}
        Categories \(\mathcal{C}\) and \(\mathcal{D}\) are equivalent (Notation: \(\mathcal{C} \simeq \mathcal{D}\)): If there are functors \(\mathcal{C} \xrightarrow{F} \mathcal{D}\) and \(\mathcal{D} \xrightarrow{G} \mathcal{C}\) such that \(F \circ G\) and \(G \circ F\) are naturally isomorphic to the respective identities.
    \end{definition}

    We need some more definitons.

    \begin{definition}
        A skeleton of \(\mathcal{C}\) is \(\operatorname{Sk} \mathcal{C} \subset \mathcal{C}\) is a full subcategory \(\forall c\in \mathcal{C}, \exists ! c^{\prime} \in \operatorname{Sk} \mathcal{C}\) such that \(c \cong c^{\prime}\).

        \(\mathcal{A} \subset \mathcal{B}\) is full if \(\forall a,a^{\prime} \in \operatorname{Ob} \mathcal{A}, \mathcal{A}(a,a^{\prime}) \xrightarrow{\cong} \mathcal{B}(a,a^{\prime})\) 
    \end{definition}

    For example, let \(\mathcal{C} =\) finite sets. Then \(\operatorname{Sk} \mathcal{C} = \{ 1 \}, \{ 1,2 \}, \{ 1,2,3 \}, \cdots\) 

    \begin{theorem}
        \(\mathcal{C} \simeq \mathcal{D} \iff \operatorname{Sk} \mathcal{C} \cong \operatorname{Sk} \mathcal{D}\).
    \end{theorem}

    Note that \(\mathcal{C} \simeq \operatorname{Sk} \mathcal{C}\) so one direction is trivial.

    \begin{lemma}
        [1.1] Let \(h\) and \(h^{\prime}\) be smooth rank \(n\) on \(M \subset \mathbb{R}^A\). Then \(h ^{-1} \circ  h^{\prime}\) is smooth (thus a diffeomorphism).
        
        Let \(V\) and \(V^{\prime}\) be the domain of \(h\) and \(h^{\prime}\) respectively. Then \(h ^{-1} \circ h^{\prime} : (h^{\prime})^{-1} (V\cap V^{\prime}) \to h ^{-1} (V \cap V^{\prime})\) 
    \end{lemma}

    \begin{corollary}
        \(\mathcal{A} = \{ h ^{-1} \mid h \text{ parameterization} \}\) is \(n\)-atlas on \(M\).

        This gives us \(\text{EmbMflds}\to \text{AbstMflds}\).
    \end{corollary}

    \begin{proof}
        This is the proof of lemma 1.1, lemma 3 in the notes.

        Assume \(V = V^{\prime}\). WTS: \((h^{\prime})^{-1} V \to h ^{-1} (V)\) is smooth.

        For \(x\in V\) choose \(\alpha_1, \cdots , \alpha_n \in A\) such that \(\det \left( \frac{\partial \alpha_i}{\partial x_j} (x) \right)  \not\equiv 0\). 

        We have:

        \[
            \begin{tikzcd}
                M \ar[r, hook] & \mathbb{R}^A \ar[d, "pr_{\alpha_1 \cdots \alpha_n}"] \\ V \ar[u,"h"] \ar[r,dotted] & \text{subset of } \mathbb{R}^n
            \end{tikzcd}
        \]

        Then, by the inverse function theorem, the dotted map is locally invertible.

        \(h^{-1} \circ h^{\prime} = (pr \circ inc \circ h)^{-1} \circ inc \circ pr \circ h^{\prime}\) near \(h ^{-1} x\).

    \end{proof}

    Given abstract \((M,\mathcal{A})\), let \(A = C^{\infty} (M,\mathbb{R})\) smooth functions.
    
    \(i: M \to \mathbb{R}^A, x \mapsto (f \to f(x))\).
    
    Let \(M_1 = i(M)\).
    
    \begin{lemma}
        [1.5] \(M_1 \subset \mathbb{R}^A\) is \(\text{EmbMfld}\). \(M \xrightarrow{i} M_1\) is diffeomorphism.
    \end{lemma}

    \subsection*{Definition of tangent vector, tangent space and tangent bundle}

    \begin{definition}
        [Tangent Vector] is velocity vector of a curve.

        We have defined morphisms. Consider the embedded case: suppose we have smooth \(\gamma : \mathbb{R} \to M \subset \mathbb{R}^A\). Then,

        \[
            \gamma^{\prime} (0) = \lim_{h \to 0} \frac{\gamma (h) - \gamma(0)}{h} \in \mathbb{R}^A
        \]

        is a tangent vector
    \end{definition}

    \begin{definition}
        [Tangent Space] Suppose \(x\in M \subset \mathbb{R}^A\), an \(n\)-dim embedded manifold. \(T_x M =\) tangent space of \(M\) at \(x\). This is:
        
        \[
            \left\{ \gamma ^{\prime} (0) \mid \gamma(0) = x \right\} \subset \mathbb{R}^A
        \]

        an \(n\)-dim subspace.
    \end{definition}

    We are going to bundle this together.

    \begin{definition}
        [Tangent Bundle]

        \(TM = \left\{ (x,v) \in M \times \mathbb{R}^A \mid v\in T_x M \right\}\).

        By definition, \(TM \subset M \times \mathbb{R}^A\) so this is in fact a topological space.

        We have a projection map \(TM \xrightarrow{\pi} M\) by \((x,v) \mapsto x\).
    \end{definition}

    \begin{remark}
        Fibers of \(\pi\), \(\pi ^{-1} (x)\) are vector spaces: \(\pi ^{-1} (x) = T_x M\).

        Then, \(TM = \bigcup_{x\in M} \{ x \} \times T_x M\).

        Abuse of notation lets us write this as \(\bigcup T_x M\).

        Thus, tangent bundle is in fact a bundle of tangents.
    \end{remark}

    What about abstract manifolds \((M,\mathcal{A})\)?

    We can define \(TM\) as follows:

    \begin{itemize}
        \item \(M \subset \mathbb{R}^{C^{\infty}(M,\mathbb{R})}\).
        \item \(TM = \frac{\coprod V_\alpha \times \mathbb{R}^n}{\sim}\)
        \item \(T_x M =\) velocity vector of curves.
        \item derivations.
    \end{itemize} 

    Suppose we have smooth function between manifolds \(f: M \to N\). \(\forall x\in M\) we can define linear \(\mathrm{d}f_x : T_x M \to T_{f(x)} N\), \(\gamma ^{\prime} (0) \mapsto (f \circ \gamma)^{\prime} (0)\). \(\mathrm{d} f_x\) is a map between vector spaces, so it is a linear transformation. It is the `Jacobian'.
    
    Then we have \(\mathrm{d} f: TM \to TN\) such that \(\mathrm{d} f(x,v) = \mathrm{d} f_x(v)\).

    We also have the chain rule: \(\mathrm{d} (f \circ g) = \mathrm{d} f \circ \mathrm{d} g\) 

    \section*{Friday, 8/29/2025}
    
    No class next week!

    Manifold constructed by:

    \begin{itemize}
        \item open subset of \(\mathbb{R}^n\)
        \item Subset double torus \(\subset \mathbb{R}^3\)
        \item Quotients: \(P^n = \mathbb{R} P^n = S^n / x \sim -x\)
        \item Lie groups/ matrix group, eg closed subgroups of \(\operatorname{GL}_n\mathbb{R} \underset{\text{open}}{\subset} M_n \mathbb{R} = \mathbb{R}^{n^2}\) 
        \item Zero sets.
        \begin{itemize}
            \item regular values
            \item transversality
            \item smooth varieties 
        \end{itemize} 
    \end{itemize} 

    \begin{definition}
        \(t_0\in \mathbb{R}\) is a regular value of \(f: M \to \mathbb{R}\) if \(\forall x\in f ^{-1} t_0, \mathrm{d}f_x\) is onto.
    \end{definition}

    \(f^{-1}(\text{regular value})\) is a submanifold of \(M\).

    Consider \(S^n \subset \mathbb{R}^{n+1}\), and \(f: \mathbb{R}^{n+1} \to \mathbb{R}\) given by \(x \mapsto x_1^2 + \cdots + x_{n+1}^2\).

    \(1\) is a regular value \(f ^{-1} 1 = S^n\).

    \begin{definition}
        Let \(f: M \to N \supset X\) submanifold.

        \(f\pitchfork X\), \(f\) is \textit{transverse} to \(X\) if \(\forall m \in f ^{-1} X, T_{f(m)}N = T_{f(m)} X + \mathrm{d}f_m (T_m M)\).
    \end{definition}

    \begin{figure}[H]
        \centering
        \includegraphics[width=0.8\textwidth]{img/transverse}
    \end{figure}

    \begin{theorem}
        \(f ^{-1} X\) is a submmanifold of \(M\).

        Furthermore, \(\dim N - \dim X = \dim M - \dim f ^{-1} X\).
    \end{theorem}

    In fact, \(\nu (f ^{-1} X \hookrightarrow M) \to \nu (X \hookrightarrow N)\) as vector space isomorphism on fibers.

    [insert picture later]

    Now, suppose \(F\) is a topological space.

    \begin{definition}
        A fiber bundle with fiber \(F\):

        Let \(E \xrightarrow{\pi} B\) be a continuous map suuch that \(\forall b\in B, \exists\) open \(b\in U \subset B\) and:

        \[
            \begin{tikzcd}
                U \times F \ar[rr, "h", "\approx"'] \ar[rd, "pr_U"] && \pi ^{-1} U \ar[ld,"\pi"] \\ & U 
            \end{tikzcd}
        \]

        \(h\) fiber preserving homeomorphism. \(\forall b^{\prime} \in U\), \(F \cong F \times b^{\prime} \xrightarrow{\approx} F_{b^{\prime}} \coloneqq \pi ^{-1} (b^{\prime})\).
    \end{definition}

    Write: \( \begin{tikzcd}
        F \ar[r] & E \ar[d] \\ & B 
        \end{tikzcd} \)

    \[
        \begin{tikzcd}
            I \ar[r] & Mob \ar[d] \\ & S^1
        \end{tikzcd}
    \]

    eg \(B \times F \to B\) trivial bundle.

    \section*{Chapter 2 of MS}

    \begin{definition}
        A real vector bundle \(\xi\) over \(B\) is:

        \[
            \xi = \left( \begin{tikzcd} E \ar[d,"\pi"] \\ B \end{tikzcd}, \forall b\in B, \pi ^{-1} b = F_b \text{ is a fin. dim vector space.} \right) 
        \]

        \(F_b \times F_b \to F, \mathbb{R} \times F_b \to F\) satisfies \(8\) axioms s.t.

        \(\forall b\in B, \exists b\in U \subset B\) and \(n\geq 0\) and \(\begin{tikzcd}U \times \mathbb{R}^n \ar[rr,"h","\approx"'] \ar[rd] && \pi ^{-1} U \ar[ld] \\ & U\end{tikzcd}\).
    
        \(\mathbb{R}^n \cong b \times \mathbb{R}^n \xrightarrow[\approx]{h} \pi ^{-1} b\) is an isomorphism of vector spaces.
    \end{definition}

    If \(B\) is connected then \(n\) is constant.

    `rank \(n\) vector bundle'.

    \(n\)-plane bundle.

    Another thing MS does is write this: \(\xi = \begin{tikzcd} E(\xi) \ar[d,"\pi(\xi)"] \\ B(\xi)\end{tikzcd}\) for vector bundle which is very precise.

    \subsection*{Isomorphism of vector bundles over \(B\)}.

    Consider two bundles \(\xi\) and \(\eta\) and we have the homeomorphism

    \[
        \begin{tikzcd}
            E(\xi) \ar[rr, "\approx"] \ar[rd] && E(\eta) \ar[ld] \\ & B
        \end{tikzcd}
    \]

    vector space isomorphism on the fibers.

    \subsection*{Examples of vector bundles}

    We have the trivial bundle \(\begin{tikzcd} B \times \mathbb{R}^n \ar[d,"\underline{\mathbb{R}^n} = \underline{\mathbb{R}}^n_B = \varepsilon^n_B ="'] \\ B \end{tikzcd}\) 

    We have tangent bundles:

    \[
        \tau_M = \left\{ \begin{tikzcd} TM \ar[d,"\pi"] \\ M \end{tikzcd} , T_x M \right\} 
    \]

    \begin{definition}
        \(M\) is parallelizable if \(\tau_M\) is trivial.
    \end{definition}

    \(S^1\) is paralellizable.

    Lie groups are parallelizable eg \(S^3\).

    \(S^2\), or \(S^{2n}\) in general not parallelizable via the hairy ball theorem.

    We also have normal bundles. Consider \(M \subset \mathbb{R}^N\).

    \(\nu(M \subset \mathbb{R}^n) = \{ (x,v) \in M \times \mathbb{R}^n \mid x \in M, v \in (T_x M)^{\perp}\}\)
    
    \(\nu (S^2 \hookrightarrow S^3) \leftarrow S^2 \times \mathbb{R}\) is trivial, the map is \((x,tx) \mapsfrom (x,t)\).

    Tautological bundle over \(P^n\): \(\gamma_n^1 = \begin{tikzcd}
        \mathbb{R} \ar[r] & E(\gamma_n^1) \ar[d] \\ & P^n
    \end{tikzcd}\) 

    Note that \(P^n = S^n / x \sim -x =\) lines through \(O\) in \(\mathbb{R}^{n+1}\).

    \(E(\gamma_n^1) = \left\{ (\{ x,-x \}, v) \in P^n \times \mathbb{R}^{n+1} \mid v \in \mathbb{R} x \right\}\).

    \(E(\gamma_n^1) \xrightarrow{\pi} P^n, (\{ x,-x \} \mapsto \{ x,-x \})\). Essentially, point on line \(\mapsto\) line. 

    This tautological bundle is non-trivial.

    \section*{Monday, 9/8/2025}
    
    Last week was a break.

    HWK: an exercise from ch2. (C, D, E are recommended).

    Recall: a vector bundle \(\xi\) is \begin{tikzcd} \mathbb{R} \ar[r] & E \ar[d,"\pi"] \\ & B\end{tikzcd} meaning fibers of \(\pi\) are \(k\)-dimensional vector spaces.

    \begin{definition}
        A section of \(\xi\) is actually a section of \(\pi\).

        \(s: B \to E\) such that \(\pi \circ s = \operatorname{id}_{B}\).
    \end{definition}

    Section looks like this:

    \[
        \begin{tikzcd}
            \mathbb{R}^k \ar[r] & E \ar[d,"\pi"] \\ & B \ar[u, dotted, bend left, "s"]
        \end{tikzcd}
    \]

    Section of \(TM \eqqcolon\) vector field.

    There's also the zero section \(z: B \to E\) given by \(b \mapsto 0 \in \pi ^{-1} b\).

    \(\begin{tikzcd} E \ar[d,"\pi"] \\ B \ar[u,bend left, dotted,"z"] \end{tikzcd}\) homotopy inverses.

    Now we show there is some twisting.

    \(\begin{tikzcd} E_0 \ar[d] & = E - z(B) \\ B \end{tikzcd}\). \(B\) trivial implies \(E_0 \cong B \times (\mathbb{R}^k \setminus  e) \simeq B \times S^{k-1}\).

    We have the tautological line bundle:

    \[
        \begin{tikzcd}
            R \ar[r] & E \ar[d] & = \{ ([x],v) \mid v\in \mathbb{R} x \} & \subset P^n \times \mathbb{R}^{n+1} \\ & P^n & = S^n / x \sim -x 
        \end{tikzcd}
    \]

    We can think of it like \((\text{line}, \text{point on line}) \in E\).
    
    For example, consider \(P^1\). This gives us the open mobius strip.

    \begin{theorem}
        [2.1] \(\gamma_n^1\) is nontrivial for \(n \geq 1\).
    \end{theorem}

    \begin{proof}
        \(E(\gamma_n^1)_0\) is connected \(\iff \not\simeq P^n \times S^0\).
    \end{proof}

    \begin{figure}[H]
        \centering
        \includegraphics[height=0.8\textwidth, angle=90]{img/gamma11.pdf}
        \caption{}
    \end{figure}

    \begin{definition}
        A metric on a vector bundle \(\xi \) is \(g: E \times_B E \to \mathbb{R}\) such that \(\forall b\in B, \pi ^{-1} b \times \pi ^{-1} b \to \mathbb{R}\) is an inner product.
    \end{definition}

    Recall: pullback of \(\begin{tikzcd}
        & B\ar[d,"\beta"] \\ A \ar[r,"\alpha"] & C
    \end{tikzcd}\) is \(A \times_C B = \left\{ (a,b) \mid \alpha (a) = \beta(b) \right\} \subset A \times B\).

    Also see: a vector bundle \(E \to B\) needs all fibers to be vector spaces. For a metric we want them to be inner product spaces.

    A bundle with metric is often callled a Euclidean vector bundle.

    Examples: A \textit{Riemannian manifold} is \(TM\) with a smooth metric [\(g\) is smooth].

    If \(M^n \subset \mathbb{R}^N\) we can use the inner product inherited from \(\mathbb{R}^N\) so it is a riemannian manifold.

    eg the trivial bundle has a metric: \((B \times \mathbb{R}^n) \times_B (B \times \mathbb{R}^n) \to \mathbb{R}^n\) which looks like \(((b,v),(b,w)) \mapsto v \cdot w\). 

    If \(M^n \subset \mathbb{R}^N\), \(TM = \left\{ (x,v) \in M \times \mathbb{R}^N \mid v = \gamma^{\prime} (0), \gamma(0) = x \right\} \) 

    \(\lVert (x,v) \rVert = \lVert v \rVert, g((x,v),(x,w)) = v \cdot w\).

    Then \(\lVert \cdot \rVert : E \to \mathbb{R}_{\geq 0}\) given by \(\lVert v \rVert \coloneqq \sqrt{g(v,v)}\).

    \begin{theorem}
        [Exercises, ch2] Suppose \(B\) is paracompact. We can look at Isomorphism classes of Euclidean vector bundles over \(B\), forget the metric to get isomorphism classes of vector bundles over \(B\):

        \[
            \begin{Bmatrix}
                \text{iso class of euclidean} \\
                \text{vector bundle over \(B\)} 
            \end{Bmatrix} \xrightarrow{\text{forget } g}\begin{Bmatrix}
                \text{iso class of} \\
                \text{vector bundle over \(B\)} 
            \end{Bmatrix} 
        \]

        This is an isomorphism.
    \end{theorem}

    \begin{definition}
        Sections \(s_1, \cdots , s_n\) of rank \(n\) vector bundle given by \(\begin{tikzcd}
            E \ar[d] \\ B \ar[u, bend right, dotted, "s_i", swap]
        \end{tikzcd}\) are linearly independent (l.i) if \(\forall b\in B, \left\{ s_1(b), \cdots , s_n(b) \right\} \) is linearly independent in \(\pi ^{-1} (b)\).  
    \end{definition}

    \begin{theorem}
        [2.2] rank \(n\) vector bundle \(\xi \) is trivial iff \(\xi \) has \(n\) l.i. sections.
    \end{theorem}

    \begin{proof}
        \(\implies: s_i(b) \coloneqq (b,\underline{e_i}) \in B \times \mathbb{R}^n\).
        
        \(\impliedby:\) define \(f: B \times \mathbb{R}^n \to E\) by \(\left( b, \sum a_i e_i \right) \mapsto \sum a_i s_i(b)\) 
    \end{proof}

    eg \(T^2\) has \(2\) l.i. sections, thus \(TT^2 \cong T^2 \times \mathbb{R}^2\).

    \begin{figure}[H]
        \centering
        \includegraphics[width=0.3\textwidth]{img/trivialtorus}
        \caption{}
    \end{figure}

    \section*{Wednesday, 9/10/2025}
    
    \section*{Chapter 3: New bundles}

    Homeowrk: pick up problems from chapter 3 (and chapter 2).

    \textit{Abstract definition of bundle} (Steenrod, see D-Kirk 5.2).

    Let \(G\) be a topological group, \(F\) a space, \(G \curvearrowright F\)

    Topological group meaning: \(G\) topological group means \(G\) is a group and a space such that \(G \times G \to G, (a,b) \mapsto ab\) and \(G \to G, a \mapsto a ^{-1}\) are continuous.

    Action of \(G\) on \(F\): \(G \times F \to F\) given by \(ef = f\) and \((gg^{\prime})f = g (g^{\prime} f)\).

    \begin{definition}
        A fiber bundle with structure group \(G\) and fiber \(F\) [\((G,F)\)-bundle] is a map with:
        
        Map \(\begin{tikzcd}
            E \ar[d] \\ F
        \end{tikzcd}\)
        
        Atlas \(\mathcal{A} = \{ \phi : U_{\phi} \times F  \xrightarrow{\approx} \pi ^{-1} U_\phi \} \) 

        Transition functions \(\Theta = \left\{ \theta_{\phi,\psi} : U_{\phi} \cap U_{\psi} \to G \mid \phi,\psi \in \mathcal{A} \right\} \) 

        such that:

        \begin{enumerate}[label=\arabic*)]
            \item \(\{ U_{\phi} \} \) open cover of \(B\).
            \item Fiber preserving homeomorphism:
            
            the following diagram commutes: \(\begin{tikzcd}
                U_{\phi} \times F \ar[rr,"\approx"] \ar[rd] && \pi ^{-1} U_\phi\ar[ld] \\ & U_{\phi} 
            \end{tikzcd}\)   
            \item \(b \in U_{\phi}\cap U_{\psi}, f\in F \implies \psi(b,f) = \phi(b,\theta_{\phi, \psi}(b)f)\) 
            \item \(\theta_{\phi,\psi}(b) = \theta_{\phi, \chi}(b) \theta_{\chi,\psi}(b)\) 
        \end{enumerate} 
    \end{definition}

    Examples:

    \(G\) trivial group implies the bundle is a trivial bundle, \(\begin{tikzcd}
        B \times F \ar[d] \\ B
    \end{tikzcd}\) 

    \(G = \operatorname{GL}(n,\mathbb{R}), F = \mathbb{R}^n\) gives us the rank \(n\) vector bundle. Let \(b\in B\), choose \(\phi , b\in U_{\phi}\). Use the atlas to find bijection \(\pi ^{-1} b \cong \mathbb{R}^n\). This gives us a vector space on \(\pi ^{-1} b\) independent of the choice of \(U_{\phi}\) by the 3rd condition.

    If the \(G\)-action on \(F\) is \textit{effective}, meaning every non-trivial action does something, meaning there is \(f\in F\) such that \(gf\neq f\) for every \(g\in G \setminus \{ e \}\), then we don't need condition 4.

    If \(G=\operatorname{O}(n)\) and \(F=\mathbb{R}^n\) then we have a vector bundle with a metric.
    
    If \(G = \operatorname{GL}(n,\mathbb{R})^+\) and \(F\) is \(\mathbb{R}^n\) then we have an oriented vector bundle.

    If \(G = S_F = \operatorname{Aut}(F)\) where \(F\) is discrete,  then we have a cover.

    For discrete \(G\) with \(F = G\) then we have a regular \(G\)-cover.

    If \(G = \operatorname{Spin}(n), F = \mathbb{R}^n\) then we have a vector bundle with spin structure. 

    Now we start chapter 3. We can do a lot of things on vector spaces, like tensor products. This lets us do stuff with vector bundles as well.

    Some basic constructions involving vector bundles:

    \begin{enumerate}[label=\arabic*)]
        \item Restriction: Let \(\xi\) be a vector bundle, \(\overline{b} \hookrightarrow B\). Then we can let \(\eval{\xi}_{\overline{B}}^{} = \begin{tikzcd}
            \pi ^{-1} \overline{B} \ar[d] \\ \overline{B}
        \end{tikzcd}\)  

        \[
            \begin{tikzcd}
                & \xi \ar[d,phantom,"||"] \\ & E \ar[d, "\pi"] \\ \overline{B} \ar[r, hook] & B 
            \end{tikzcd}
        \]

        \item Induced bundles (= Pullback bundle) Let \(\xi\) be a vector bundle, and \(B_1 \xrightarrow{f} B\). We can \textit{pullback} the bundle and get \(f ^{\ast} \xi\):
        
        \[
            \begin{tikzcd}
                f^{\ast} E = B_1 \times_B E \ar[r] \ar[d] & E \ar[d] \\ B_1 \ar[r,"f"] & B
            \end{tikzcd}
        \]

        in fact \(\eval{\xi}_{\overline{B} }^{} = \operatorname{inc}^{\ast} \xi\).
    \end{enumerate} 

    \begin{definition}
        Bundle map \(g: \eta \to \xi\) [both \(n\)-plane] is given by a commutative diagram which is isomorphism on fibers:

        \[
            \begin{tikzcd}
                E(\eta) \ar[r,"g"] \ar[d] & E(\xi) \ar[d] \\ B(\eta) \ar[r,"\overline{g}"] & B(\xi)
            \end{tikzcd}
        \]
    \end{definition}

    \begin{lemma}
        [3.1] \(\eta \cong \overline{g}^{\ast}\)  as vector bundle over \(B(\eta)\).

        \[
            \begin{tikzcd}
                E(\eta) \ar[rr,"\approx"] \ar[rd] & & \overline{g}^{\ast} E(\xi) \ar[ld] \\ & B(\eta)
            \end{tikzcd}
        \]
    \end{lemma}

    \begin{proof}
        We just need to define the map.

        \[
            E(\eta) \to B(\eta) \times_{B(\xi)} E(\xi)
        \]

        \[
            e \mapsto (\pi(e), g(e))
        \]
    \end{proof}

    pullback stuff works for \((G,F)\)-bundles.

    \section*{Friday, 9/12/2025}
    
    Today we finish chapter 3.

    We can study construction of new vector bundles in the following ways:

    \begin{enumerate}[label=\alph*)]
        \item \textit{Restriction}: \(\eval{\xi}_{\overline{B}}^{}\) for \(\overline{B} \subset B \leftarrow E\)
        \item \textit{Pullback}: \(f^{\ast} \xi\) for \(\overline{B} \xrightarrow{f} B \leftarrow E\)
        \item \textit{Product}: \(\xi_1 \times \xi_2\).
        
        \[
            \begin{tikzcd}
                F_b(\xi_1) \times F_b(\xi_2) \ar[r] & E(\xi_1) \times E(\xi_2) \ar[d] \\ & B(\xi_1) \times B(\xi_2)
            \end{tikzcd}
        \]

        eg \(T(M_1 \times M_2) = TM_1 \times TM_2\).
        \item \textit{Whitney Sum}: We keep the base space the same. Let \(\xi_1, \xi_2\) be vector bundles over the same base space \(B\). Then we can define the whitney sum as the pullback of the diagonal map to the product:
        
        \[
            \xi_1 \oplus \xi_2 \coloneqq \Delta^{\ast} (\xi_1 \times \xi_2)
        \]

        \(B \xrightarrow{\Delta} B \times B\) is \(b \mapsto (b,b)\).

        For example, in \(S^2 \hookrightarrow \mathbb{R}^3\), the whitney sum of the tangent bundle and the normal bundle gives us the trivial bundle: \(\varepsilon ^3_{S^2} = TS^2 \oplus \nu(S^2 \hookrightarrow \mathbb{R}^3)\).

        \item \textit{Subbundles, Quotients and Orthogonal Complements}: A subbundle \(\eta\) of \(\xi\) is \(E(\eta) \subset E(\xi)\) such that \(\eval{\pi}_{E(\eta)}^{} \) is a vector bundle.
        
        \[
            \begin{tikzcd}
                F_b(\eta) \ar[rr, hook] \ar[d] & & F_b(\xi) \ar[d] \\ E(\eta) \ar[rr, hook] \ar[rd] & & E(\xi) \ar[ld] \\ & B
            \end{tikzcd}
        \]

        In order to study quotient, we need \textit{bundle morphisms}. We want the following diagram to be commutative and also want the map to be linear on fibers:

        \[
            \begin{tikzcd}
                E(\eta) \ar[r] \ar[d] & E(\xi) \ar[d] \\ B(\eta) \ar[r] & B(\xi)
            \end{tikzcd}
        \]

        \textit{Bundle morphism over \(B\)} is different: we want the following commutative diagram to be linear on fibers:

        \[
            \begin{tikzcd}
                E(\eta) \ar[rr] \ar[rd] & & E(\xi)  \ar[ld] \\ & B
            \end{tikzcd}
        \]

        An example: suppose we have smooth \(f: M \to N\). Then we have bundle morphism:

        \[
            \begin{tikzcd}
                M \ar[d] \ar[r,"df"] & TN \ar[d] \\ M \ar[r,"f"] & N
            \end{tikzcd}
        \]

        and the bundle morphism \(/ M\):

        \[
            \begin{tikzcd}
                TM \ar[rr,"\cong"] \ar[rd] & & f^{\ast} TN \ar[ld] \\ & M
            \end{tikzcd}
        \]

        We can define quotient bundles from subbundles: subbundle \(\eta\) of \(\xi\) there exists quotient bundle \(\xi / \eta\) so that \(F_b(\xi / \eta)\) are \(F_b(\xi) / F_b(\eta)\). We have bundle map over \(B\) \(\xi \to \xi / \eta\) 

        Bundles \(/ B\) fform abelian category. We have the SES:

        \[
            0 \to \eta \to \xi \to \xi / \eta \to 0
        \]

        We now define normal bundles. Normal bundle of submanifold \(M\) of \(N\) is given by \(\nu (M \hookrightarrow N) = \frac{\left( \eval{TN}_{M}^{} \right) }{TM}\).
        
        \begin{figure}[H]
            \centering
            \includegraphics[width=0.4\textwidth]{img/normal}
            \caption{}
        \end{figure}

        If \(N \subset \mathbb{R}^k\) (or \(N\) Riemannian metric space) then \((TM)^\perp \subset \eval{TN}_{M}\).

        \[
            \begin{tikzcd}
                (TM)^\perp \ar[rr, bend right, "\cong"] \ar[r] & \eval{TN}_{M}^{} \ar[r] & \nu(M \hookrightarrow N) 
            \end{tikzcd}
        \]

        We have \((TN)_M = TM \oplus (TM)^\perp\).

        If \(\xi\) is a bundle with metric and \(\eta\) is a subbundle then \(\xi = \eta \oplus \eta^\perp\) and \(\eta^\perp \cong  \xi / \eta\).

    \end{enumerate}

    If \(B\) is paracompact [eg \(B \subset W\)] then bundles over \(B\) form an exact category [meaning all SES split].
    
    Reason: consider the following SES:

    \[
        0 \to \alpha  \to \beta \to \gamma \to 0
    \]

    Since \(B\) is paracompact we can give \(\beta\) a metric. \(\alpha^ \perp \xrightarrow{\approx} \gamma\) so it splits.
    
    This tells us: if \(M \subset N\) and \(N\) has a Riemannian metric, then,
    
    \(\eval{TN}_{M}^{} = TM \oplus TM^ \perp \cong TM \oplus \nu(M \hookrightarrow N)\).

    \begin{definition}
        Smooth \(f: M \to N\) is a immersion/submersion if \(\forall x\in M\), \(df_x\) is injective/surjective.
    \end{definition}

    For example, consider \(S^1 \to \mathbb{R}^2\) given by \(\bigcirc \to \infty\) is an immersion, since it's locally an embedding.

    \(TS^2 \to S^2\) is a submersion.

    Let \(f: M\to N\) be an immersion. Then, \(\nu(f) = \frac{f^{\ast} TN}{TM}\).

    If \(N\) has a metric then \(TM \cong \eval{TN}_{M}^{} \oplus \nu (f)\).

    \section*{Tuesday, 9/16/2025}

    \subsection*{UCT, Cup and Cap Prodcuts}
    
    Let \(M\) be an abelian group. Then we have homology \(H_i(X,A;M)\) and cohomology \(H^i(X,A;N)\) abelian groups.
    
    The cohomology \(H^i(X,A;N)\) is the cohomology of the following cochain complex: \(H^i(\operatorname{Hom} (S_\bullet(X,A),N))\) 

    `Cohomology eats homology' via the following \textit{Kronecker Pairing}:

    \[
        \langle , \rangle: H^i(X,A;N) \otimes H_i(X,A;M) \to N \otimes_\mathbb{Z} M
    \]

    \[
        [\phi] \otimes \left[ \sum_{i} k_i \sigma_i \otimes m_i \right] \mapsto \sum_{i} k_i \varphi (\sigma_i) \otimes m_i
    \]

    Now we do UCT. Let \(R = \mathbb{Z}\) and \(M = \mathbb{Z}\)-module, i.e. abelian group.
    
    If \(X = \mathbb{R} P^n\) then the cellular chain complex of \(\mathbb{R} P^n\) is:

    \[
        C_\bullet X = \mathbb{Z} \xrightarrow[0 \text{ \(n\) odd}]{2 \text{ \(n\) even}}\cdots \to \mathbb{Z} \xrightarrow{2} \mathbb{Z} \xrightarrow{0} \mathbb{Z}
    \]

    Thus, if \(n\) odd, then \(H_i \mathbb{R} P^n = \begin{dcases}
        \mathbb{Z} , &\text{ if } i = 0,n ;\\
        \mathbb{Z} _2, &\text{ if } i \text{ odd, } 0 < i < n ;\\
        0, &\text{ otherwise} .
    \end{dcases}\)

    If coefficients are in \(\mathbb{Z}_2\) then,
    
    \[
        C_\bullet X \otimes \mathbb{Z}_2 \xrightarrow{0} \mathbb{Z}_2 \xrightarrow{0} \cdots \xrightarrow{0} \mathbb{Z}_2
    \]

    Thus \(H_i (\mathbb{R} P^n;\mathbb{Z}_2) = \mathbb{Z}_2\) for \(0 \leq i \leq n\).

    UCT states that the following is a split short exact sequence:

    \[
        0 \to H_i X \otimes M \to H_i(X;M) \to \operatorname{Tor} (H_{i-1} X,M) \to 0
    \]

    We can say three things about Tor:

    Tor is a functor, \(\operatorname{Tor}: \operatorname{Ab} \times \operatorname{Ab} \to \operatorname{Ab}\).

    If \(M,N\) are f.g. then \(\operatorname{Tor} (M,N) \cong (\text{torsion } M) \otimes_\mathbb{Z} (\text{torsion } N)\) 

    \begin{definition}
        Find an exact sequence of free groups as follows:

        \[
            0 \to F_1 \to F_0 \to M \to 0
        \]

        Then \(\operatorname{Tor}(M,N) = H_1(F_1 \otimes N \to F_0 \otimes N)\).
    \end{definition}

    For example, \(\operatorname{Tor}(\mathbb{Z}_2, \mathbb{Z}_2)\), we have following free groups:

    \[
        0 \to \mathbb{Z} \xrightarrow{\times 2} \mathbb{Z} \xrightarrow{\text{mod } 2} \mathbb{Z}_2 \to 0
    \]

    Tensoring with \(\mathbb{Z}_2\) to get the following: \(\mathbb{Z}_2 \xrightarrow{0} \mathbb{Z}_2\). Then \(H_1\) is the kernel.

    So, \(\operatorname{Tor}(\mathbb{Z}_2, \mathbb{Z}_2) = \mathbb{Z}_2\).

    Now we go back to geometry.

    Suppose we have space \(X\) such that \(H_{i-1} X = \mathbb{Z}_2 \oplus ?\)

    This gives us \(H_i(X) \to \mathbb{Z}_2 \subset H_i(X;\mathbb{Z}_2)\).

    Geometrically, consider \(H_i(X;\mathbb{Z}_2) \to \operatorname{Tor}(H_{i-1} (X); \mathbb{Z}_2)\).
    
    If there is \([a] \in \operatorname{Tor}(H_{i-1} X; \mathbb{Z}_2)\) with \(2a = \partial b\) then section given by \([b] \mapsfrom [a]\)  

    UCT works even if we change \(\mathbb{Z}\) with a PID. For any PID \(R\) we can talk about \(R\)-modules \(M\), then \(H_i(X;M) \cong H_i(X;R) \otimes M \oplus \operatorname{Tor}^R(H_{i-1}(X;R),M)\).

    We want the analogue of UCT for cohomology. This gives us the split exact sequence:

    \[
        0 \to \operatorname{Ext}(H_{i-1} X, M) \to H^i(X;M) \to \operatorname{Hom}(H_i X, M) \to 0
    \] 

    Again, for \(n\) odd consider the chain complex:

    \[
        C_\bullet \mathbb{R} P^n = \mathbb{Z} \xrightarrow{0} \mathbb{Z} \to \cdots \mathbb{Z} \xrightarrow{2} \mathbb{Z} \xrightarrow{0} \mathbb{Z} \to 0
    \]

    For cochain complex we'd simply reverse the arrows:

    \[
        C^\bullet \mathbb{R} P^n = \mathbb{Z} \xleftarrow{0} \mathbb{Z} \leftarrow \cdots \mathbb{Z} \xleftarrow{2} \mathbb{Z} \xleftarrow{0} \mathbb{Z} \leftarrow 0
    \]

    \(H_i \mathbb{R}P^n = \mathbb{Z}\) for \(i=0,n\) and \(\mathbb{Z}_2\) for \(0 < i < n, n\) odd.

    \(H^i (\mathbb{R} P^n; \mathbb{Z} ) = \mathbb{Z}\) for \(i = 0,n\) and \(\mathbb{Z}_2\) for \(0 < i < n, n\) even.
    
    We have: \(\operatorname{Ext}(\text{Free}, M) = 0\).
    
    In general, \(\operatorname{Ext}(A,B)\) is given by: resolve \(A\), apply \(\operatorname{Hom}(-,B)\) cohomolgy.

    Suppose \(0 \to F_1 \to F_0 \to A \to 0\).

    Then, \(\operatorname{Hom}(F_1, B) \xleftarrow{\partial^1} \operatorname{Hom}(F_0, B)\).

    Thus \(\operatorname{Ext}(A,B) = \operatorname{coker} \partial^1\).

    If \(A, B\) are finitely generated then \(\operatorname{Ext}(A,B) \cong (\operatorname{torsion} A) \otimes B\).

    Now, suppose \(R\) is a commutative ring.

    Then \(H^i(X;R) = H^i(\operatorname{Hom}_\mathbb{Z}(X_\bullet X, R))\)
    
    But might be more in the spirit of how we are doing this to do the following:

    \(H^i(X;R) = H^i(\operatorname{Hom}_R(S_\bullet(X;R),R))\)
    
    For \(R\)-modules \(M\),

    \(H^i(X;M) = H^i(\operatorname{Hom}_\mathbb{Z}(S_\bullet X, M)) = H^i(\operatorname{Hom}_R(S_\bullet(X;R),M))\)
    
    Then, \(H^{\ast} (X;R)\) is a graded commutative ring under the cup product.

    \(H^{\ast} (X;R)\) is a graded commutative ring meaning we can write:
    
    \(H^{\ast} (X;R) = \bigoplus_{i \geq 0} H^i(X;R)\) and we have \(H^i(X;R) \otimes_R H^j(X;R) \to H^{i+j} (X;R)\)

    Commutative graded ring meaning \(\alpha \cup \beta = (-1)^{\vert \alpha  \vert \vert \beta \vert} \beta \cup \alpha\).
    
    For De Rham cohomology,

    \(H^i_{\text{DR}}(M;\mathbb{R}) \otimes H^j_{\text{DR}}(M;\mathbb{R})\) we have \(\alpha  \otimes \beta  \mapsto [\alpha \wedge \beta]\) 

    We also have: \(H_{\ast} (M;R)\) is a graded module over \(H^{\ast} (M;R)\) w.r.t.\ cap product.

    For \(\alpha \in H^i(M;R)\) and \(z\in H_j(M;R)\) then \(\alpha \cap z \in H_{j-i}(M;R)\).

    So, cap product by \(\alpha\) eats \(i\) dimensions from \(z\).

    We also have \(\langle \alpha \cup \beta, z \rangle = \langle \alpha , \beta \cap z \rangle \).

    If \(f: X\to Y\) is continuous, we have a ring map \(f^{\ast} : H^{\ast} (Y;R) \to H^{\ast} (X;R)\) by \(f^{\ast} (\alpha \cup \beta) = f^{\ast} \alpha \cup f^{\ast} \beta\).

    Poincar\'e Duality: if \(M^n\) is closed and oriented and connected then \(H_n M \cong \mathbb{Z}\). Choose generator \([M] \in H_n M\).

    Then we have isomorphism \(\cap [M]: H^i M \xrightarrow{\cong} H_{n-i} M  \) 

    Another fact:

    \[
        \frac{H^i M}{\text{torsion}} \otimes \frac{H^{n-i} M}{\text{torsion}} \to \mathbb{Z}
    \]

    is a nonsingular perfect pairing: \(\alpha \otimes \beta\) is given by \((\alpha \cup \beta)[M] \in \mathbb{Z}\).

    Recall \(A \times B \to \mathbb{Z}\) is perfect \(\iff A \xrightarrow{\cong} \operatorname{Hom}(B,\mathbb{Z})\) and \(B \xrightarrow{\cong} \operatorname{Hom}(A,\mathbb{Z})\) are isomorphism.

    In \(\mathbb{C} P^n = e^0 \cup e^2 \cup \cdots \cup e^{2n}\) we have \(H^{\ast} \mathbb{C} P^n \cong \mathbb{Z} [\alpha] / \alpha^{n+1}\), with \(\deg \alpha = 2\).

    This is a truncated polynomial ring.

    We can prove this by Poincar\'e duality and induction on \(n\).

    We also have Kunneth Theorem. If \(R\) is a field, then:

    \[
        H^{\ast} (X;R) \otimes H^{\ast} (Y;R) \xrightarrow{\cong} H^{\ast}(X \times Y; R)
    \]

    It is only an injection for general ring.

    \section*{Wednesday, 9/17/2025}
    
    HWK due 9/29.

    4 Exercises: 1 from Ch2, 1 from Ch3, 2 from Ch4.

    Today we finish chapter 3, construction of bundles.

    We skipped part f on Friday.

    \begin{table}[H]
        \centering
        \begin{tabular}{c|c}
            \toprule
                Vector Spaces & Vector Bundle \\
            \midrule
                \(V \otimes W\) & \(\xi \otimes \eta\) \\
                \(\operatorname{Hom}(V,W)\) & \(\operatorname{Hom}(\xi, \eta)\) \\
                \(V^{\ast} = \operatorname{Hom} (V,\mathbb{R})\) & \(\xi ^{\ast} = \operatorname{Hom} (\xi , \epsilon_B^1)\) \\
                \(\Lambda^k V\) & \(\Lambda ^k \xi\) \\
                \(\Lambda ^{\ast} V\) & \(\Lambda ^{\ast} \xi\)  \\
            \bottomrule
        \end{tabular}
        \caption{Anythong we can do on Vector Spaces, we can do in Vector Bundles.}
    \end{table}

    As for \(\operatorname{Hom}(\xi, \eta)\) we assume base space is the same:
    
    \[
        \begin{tikzcd}
            \operatorname{Hom}(\mathbb{R}^k, \mathbb{R}^l) \ar[r] & E \operatorname{Hom}(\overset{k}{\xi}, \overset{l}{\eta}) \ar[d] \\ & B
        \end{tikzcd}
    \]

    Here \(E \operatorname{Hom}(\xi, \eta) =\) [roughly] \(\bigcup_{b\in B} \operatorname{Hom}_{\mathbb{R}} (F_b(\xi), F_b(\eta))\)
    
    \(\coloneqq \coprod_{\text{open } U \subset B, \eval{\xi}_U, \eval{\eta}_U \text{ trivial}} U \times \operatorname{Hom}(\mathbb{R}^k, \mathbb{R}^l) / \sim\).
    
    \section*{Cotangent Bundle}

    Let \(M^n\) be a smooth \(n\)-manifold.

    \begin{definition}
        [Cotangent Bundle] Is dual to the tangent bundle: \(T^{\ast} M \coloneqq (TM)^{\ast}\).
    \end{definition}

    We can take exterior power to get differential \(k\) forms:

    \[
        \begin{tikzcd}
            \Lambda^k \mathbb{R} ^ n \ar[r] & \Lambda ^k T^{\ast} M \ar[d] \\ & M \ar[u, bend right, dotted]
        \end{tikzcd}
    \]

    Differential \(k\)-form on \(M, \omega \in \Gamma (\Lambda ^ k T^{\ast} M)\) smooth section.

    \(\begin{matrix}
        \Lambda ^{\ast} \mathbb{R}^n \to & \Lambda ^{\ast} T^{\ast} M \\
        & \downarrow \\
        & M \\
    \end{matrix} \leftarrow\) wedge product. 

    In fact, \(\Gamma (\Lambda ^{\ast} T^{\ast} M)\) is a graded algebra, \(\Omega ^{\ast} M\).
    
    \section*{Chapter 4}

    Now we start on Characteristic Classes.

    \begin{definition}
        [Stiefel-Whitney Classes] have these 4 axioms:

        \begin{enumerate}[label=\arabic*)]
            \item \(\forall\) vector bundle \(\xi\), assign \(\operatorname{w}_i(\xi) \in H^i(B(\xi);\mathbb{F}_2)\) so that \(\operatorname{w}_0(\xi) = 1\) and \(\operatorname{w}_i(\xi) = 0\) for \(i > n\) when \(\xi\) is a an \(n\)-plane bundle.
            \item \textit{Naturality}: For continuous \(f: B^{\prime} \to B(\xi)\), we have \(\operatorname{w}_i(f^{\ast} \xi) = f^{\ast} (\operatorname{w}_i \xi) \in H^i(B;\mathbb{F}_2)\). [First one is the pullback on the bundle, second one is the induced map on the cohomology.]
            \item \textit{Whitney Sum Formula}: If \(\xi , \eta\) are vector bundles over \(B\) we have: \(\operatorname{w}_k(\xi \oplus \eta) = \sum_{i+j=k} \operatorname{w}_i(\xi) \cup \operatorname{w}_j(\eta) \).
            \item \(0 \neq \operatorname{w}_1(\gamma^1_1) \in H^1(P^1;\mathbb{F}_2) = H^1(S^1; \mathbb{F}_2) = \mathbb{F}_2\).
        \end{enumerate} 

        This sequence of cohomology classes is called the Stiefel-Whitney Classes.

    \end{definition}

    Recall: \(\gamma^1_1\) for a mobius strip is the zero section, i.e. \(S^1\).

    Milnor-Stasheff says naturality a bit differently. Recall: If

    \[
        \begin{tikzcd}
            E(\eta) \ar[r, "\text{iso/fibers}"] \ar[d] & E(\xi) \ar[d] \\ B(\eta) \ar[r,"f"] & B(\xi)
        \end{tikzcd}
    \]

    then \(\eta = f^{\ast} \xi, \operatorname{w}_i(\eta) = f^{\ast} \operatorname{w}_i(\xi)\).

    Note: axioms 1 and 2 says \(\operatorname{w}_i\) are \textit{characteristic classes}. Characteristic Classes are cohomology classes respecting naturality. Meaning they respect nontriviality of bundles. Just like homology `classifies' upto homotopy in a sense, we need characteristic classes to capture the `twists' in a vector bundle.

    Axiom 1 and 2 implies:

    \begin{proposition}
        [1] \(\xi \cong \eta \implies \operatorname{w}_i(\xi) = \operatorname{w}_i(\eta)\).
    \end{proposition}

    Recall that vector bundles are isomorphic if:

    \[
        \begin{tikzcd}
            E(\xi) \ar[rr,"\cong"] \ar[rd] & & E(\eta) \ar[dl] \\ & B
        \end{tikzcd}
    \]

    \begin{proof}
        \(f = \operatorname{id}\).
    \end{proof}

    \begin{proposition}
        [2] \(\operatorname{w}_i(\epsilon^n_B) = 0\) for \(i > 0\).  
    \end{proposition}

    \begin{proof}
        \[
            \begin{tikzcd}
                B \times \mathbb{R}^n \ar[r] \ar[d] & \mathbb{R}^n \ar[d] \\ B \ar[r,"c"] & \text{pt}
            \end{tikzcd}
        \]

        \(\operatorname{w}_i(\epsilon^n_B) = \operatorname{w}_i (c^{\ast} \epsilon^n_{\text{pt}}) = c^{\ast} \operatorname{w}_i(\epsilon^n_{\text{pt}}) \in H^i(\text{pt};\mathbb{F}_2) = 0\).
        
        Thus, nontrivial Stiefel-Whitney Class implies nontrivial bundle.

    \end{proof}

    \begin{proposition}
        [3] If \(\epsilon\) trivial then \(\operatorname{w}_i(\epsilon \oplus \eta) = \operatorname{w}_i(\eta)\). In other words, \(\operatorname{w}_i\) stable characteristic classes.
    \end{proposition}

    \begin{proposition}
        [4] If \(\xi\) is an \(n\)-plane bundle with \(k\) linearly independent sections, then \(k\) of them vanishes:

        \[
            \operatorname{w}_{n-k+1}(\xi) = \cdots = \operatorname{w}_{n-1} (\xi) = \operatorname{w}_n(\xi) = 0
        \]
    \end{proposition}

    Most interesting case is \(k = 1\) contrapositive.

    \(\operatorname{w}_n(\xi) \neq 0 \implies \not\exists\) nowhere zero section. Hairy ball theorem!

    eg for \(n\) odd there exists a nowhere zero section of the tangent bundle \(TS^n\). Therefore, \(\operatorname{w}_n(TS^n) = 0\).
    
    Since \(n\) is odd \(n+1\) is even, and we can switch the coordiantes in pairs:

    \[
        \underline{x} = (x_1, \cdots , x_n) \mapsto (\underline{x}, -x_2, x_1, \cdots ,-x_{n+1}, x_n) \in TS^n \subset S^n \times \mathbb{R}^{n+1} 
    \]

    \(\operatorname{w}_4(T \mathbb{C} P^2)\neq 0, \not\exists\) nowhere vanishing vector field on \(\mathbb{C}P^2\).

    If \(M^n\) is a closed \(n\)-manifold then \(\operatorname{w}_n(TM^n) \equiv \xi(M) \pmod 2\).

    \begin{proof}
        The condition of \(k\) linearly independent section is equivalent to existence of a subbundle \(\epsilon^k_B \subset \xi\).

        Case 1: Suppose \(\xi\) has a metric.

        Then \(\xi = \epsilon^k_B \oplus (\epsilon^k_B)^{\perp}\). 

        \(\operatorname{w}_i(\xi) = \operatorname{w}_i(\epsilon^{k \perp}_B)\) by proposition 3. Note that \(\epsilon^{k \perp}_B\) is a \(n - k\) bundle, axiom 1 implies the statement.

        Case 2: \(B\) is a CW complex so \(B\) is paracompact which implies \(\xi\) has a metric.

        General case: suppose \(\begin{matrix} E(\xi) \\ \downarrow \\ B \end{matrix}\). Then \(\exists\) CW-approximation \(B^{\prime} \to B\) where \(B^{\prime}\) is a CW complex which is isomorphism on \(\pi_{\ast}\) which is isomorphism in homology and cohomology. This reduces to case 2.

    \end{proof} 

    \section*{Friday, 9/19/2025}
    
    Recap: Stiefel-Whitney-Classes:

    Suppose we have an \(n\)-plane bundle \(\begin{pmatrix}
        \mathbb{R}^n & \to & E \\
         &  & \downarrow \\
         &  & B \\
    \end{pmatrix} \) 

    Then \(\operatorname{w}_i E = \operatorname{w}_i (\xi) \in H^i(B;\mathbb{F}_2)\).

    We have some axioms:

    \begin{enumerate}[label=\arabic*)]
        \item \(\operatorname{w}_0(\xi) = 1, \operatorname{w}_i(\xi) = 0\) for \(i > n\)
        \item Naturality: if we have \(f: B^{\prime} \to B\) then \(\operatorname{w}_i(f^{\ast} \xi) = f^{\ast} \operatorname{w}_i(\xi) \in H^i(B^{\prime} ; \mathbb{F}_2)\).
        
        One way to rephrase it is as follows: \(f^{\ast} E\) is the pullback bundle in the following:

        \[
            \begin{tikzcd}
                f^{\ast} E \ar[r] \ar[d] & E \ar[d] \\ B^{\prime} \ar[r,"f"] & B
            \end{tikzcd}
        \]

        Another way: if we have a bundle map:

        \[
            \begin{tikzcd}
                E^{\prime} \ar[r] \ar[d] & E \ar[d] \\ B^{\prime} \ar[r] & B
            \end{tikzcd}
        \]

        which is an isomorphism on the fibers, then \(f^{\ast} E \cong E^{\prime}\). We have \(E^{\prime} \to B^{\prime} \) which is equal to \(f^{\ast} \xi\).

        In Milnor-Stasheff, if we have:

        \[
            \begin{tikzcd}
                E(\eta) \ar[r] \ar[d] & E(\xi) \ar[d] \\ B(\eta) \ar[r] & B(\xi)
            \end{tikzcd}
        \]

        \(\eta \to \xi\) in this case \(\operatorname{w}_i(\eta) = f^{\ast} \operatorname{w}_i(\xi)\).

        Note that properties 1 and 2 are called characteristic class on a bundle.

        \item Whitney Sum formula: \(\operatorname{w}_k(\xi \oplus \eta) = \sum_{i+j=k} \operatorname{w}_i(\xi) \operatorname{w}_i(\eta)\)
        \item \(\operatorname{w}_1(\gamma^1_1) \neq 0\).
    \end{enumerate} 

    Recall proposition 3: if \(\epsilon\) trivial then \(\operatorname{w}_i(\epsilon \oplus \eta) = \operatorname{w}_i(\eta)\).

    Proposition 4: \textit{obstruction to sections}: If \(\xi\) has \(k\)-linearly independent sections then the top \(k\) Stiefel-Whitney Classes vanish.

    \subsection*{Whitney Sum Inverses}

    \begin{definition}
        Suppose \(\xi \oplus \eta = \epsilon^N\). Then \(\xi\) and \(\eta\) are whitney sum inverses of each other.
        
        Example: Normal bundle and tangent bundle.
    \end{definition}

    Fact: \(\dim B < \infty\) implies every bundle has an inverse.

    Observation: \(\operatorname{w}_{\ast} (\xi)\) can be computed in terms of \(\operatorname{w}_{\ast} (\eta)\).

    \[
        0 = \operatorname{w}_1(\xi \oplus \eta) = \operatorname{w}_1(\xi) + \operatorname{w}_1 (\eta) \implies \operatorname{w}_1(\xi) = \operatorname{w}_1(\eta)
    \]
    
    \[
        0 = \operatorname{w}_2(\xi \oplus \eta) = \operatorname{w}_2(\xi) + \operatorname{w}_1(\xi) \operatorname{w}_1(\eta) + \operatorname{w}_2(\eta) \implies \operatorname{w}_2(\xi) = \operatorname{w}_1(\eta)^2 + \operatorname{w}_2(\eta)
    \]

    In Milnor Stasheff, they define a new ring:

    \[
        H^{\prod}  (B;\mathbb{F}_2) = \prod_i H^i(B;\mathbb{F}_2)
    \]

    This allows us to take infinite series:

    \[
        \operatorname{w}(\xi) = 1 + \operatorname{w}_1 \xi + \operatorname{w}_2 \xi + \cdots \in H^{\prod}  (B;\mathbb{F}_2)
    \]

    Then we can rephrase the Whitney sum theorem as follows:

    \(\operatorname{w}(\xi \oplus \eta) = \operatorname{w}(\xi) \cup w(\eta)\).

    \begin{lemma}
        \(\left\{ 1 + a_1 + a_2 + \cdots \in H^{\prod} (B;\mathbb{F}_2) \mid a_i \in H^i(B;\mathbb{F}_2) \right\}\)
    \end{lemma}

    \begin{proof}
        Due to `Euler':

        \((1 + a_1 + a_2 + \cdots ) ^{-1}  = \frac{1}{1+(a_1 + a_2 + \cdots)}\)
        
        \(= 1 + (a_1 + a_2 + \cdots ) + (a_1 + a_2 + \cdots)^2 + (a_1 + a_2 + \cdots)^3\)

        \(= 1 + a_1 + (a_2 + a_1^2) + (a_3 + a_1^3) + \cdots\) 
        
        
    \end{proof}

    Notation: Suppose \(\operatorname{w}(\xi) \in H^{\prod} (B;\mathbb{F}_2)\) then we can have the formal multiplicative inverse: \(\overline{\operatorname{w}}(\xi) \in H^{\prod } (B;\mathbb{F}_2)\) so that \(\operatorname{w} (\xi) \overline{\operatorname{w}}(\xi) = 1\)  

    This gives us the following observation: \(\xi \oplus \eta = \epsilon^N\) gives us \(\operatorname{w}(\xi) \operatorname{w}(\eta) = 1 \implies \operatorname{w}(\xi) = \overline{\operatorname{w}}(\eta)\).
    
    eg \(H^{\ast} (\mathbb{P}^{\infty} ;\mathbb{F}_2) = \mathbb{F}_2[a]\) then we have canonical line bundle \(\gamma^1\) then \(\operatorname{w} (\gamma^1) = 1+a\) so \((1+a)^{-1} = 1 + a + a^2 + \cdots\) which has infinitely many terms so the inverse might not exist! The line bundle doesn't have any whitney sum inverse.

    \begin{theorem}
        [Whitney Duality Theorem] Let \(M^n \subset \mathbb{R} ^N\) be a smooth manifold. Then,
        
        \[
            \operatorname{w}_i(TM) = \overline{\operatorname{w}}_i \left( \nu (M \hookrightarrow \mathbb{R}^N) \right)
        \] 
    \end{theorem}

    \begin{proof}
        \((TM \oplus \nu (M \hookrightarrow \mathbb{R}^N)) = \eval{T\mathbb{R}^N}_M\) 
    \end{proof}

    \begin{lemma}
        Suppose we have a closed codim \(1\) manifold: \(M^n \subset \mathbb{R}^{n+1}\). Then \(\operatorname{w} (TM) = 1\).
    \end{lemma}

    So Stiefel-Whitney Classes give an obstruction to submanifolds of codimension \(1\).

    \begin{proof}
        \(TM \oplus \nu (M \hookrightarrow \mathbb{R}^{n+1})\) is trivial, \(\nu (M \hookrightarrow \mathbb{R}^{n+1})\) gives nowhere zero section.
    \end{proof}

    \begin{corollary}
        Non-orientable submanifolds must have codimension at least \(2\).
    \end{corollary}

    Recall \(P^n = \mathbb{R} P^n = S^n / x \sim -x = \frac{S^n_+}{x \sim -x \text{ when } x \in S^{n-1}} =\) lines in \(\mathbb{R}^{n+1}\) through \(0\).
    
    \(P^n\) is a CW complex via the pushout:

    \[
        \begin{tikzcd}
            S^{n-1} \ar[r] \ar[d, hook] & P^{n-1} \ar[d] \\ D^n \ar[r] & P^n
        \end{tikzcd}
    \]

    \(P^0 \subset P^1 \subset \cdots \subset P^n\) is the skeleton.

    Essentially \(P^n = e^0 \cup e^1 \cup \cdots \cup e^n\) with \(e^i \cong \overset{\circ}{D^i}\).
    
    Cellular chain complex:

    \[
        C_\bullet(P^n; \mathbb{F}_2) = \mathbb{F}_2 \xrightarrow{0} \cdots \xrightarrow{0} \mathbb{F}_2
    \]

    Cochain complex:

    \[
        C^\bullet(P^n;\mathbb{F}_2) = \mathbb{F}_2 \leftarrow \cdots \leftarrow \mathbb{F}_2
    \]

    \(H_{\ast} (P^n;\mathbb{F}_2) = H^{\ast} (P;\mathbb{F}_2) = \{ \mathbb{F}_2: x \leq n \}\)

    Next: \(H^{\ast} (P^n;\mathbb{F}_2) = \frac{\mathbb{F}_2[a]}{a^{n+1}}\) truncated polynomial ring. 

    \section*{Monday, 9/22/2025}
    
    We do some computations today.

    Recall: \(P^n = S^n / x \sim -x = \underbrace{e^0 \cup e^1 \cup \cdots}_{P^{n-1} } \cup e^n\)

    Then \(H^{\ast} (P^n; \mathbb{F}_2) = \begin{dcases}
        \mathbb{F}_2, &\text{ if } \ast \leq n ;\\
        0, &\text{ otherwise} .
    \end{dcases}\) 

    Let \(0 \neq a \in H^1(P^n; \mathbb{F}_2)\).

    \begin{theorem}
        \(H^{\ast} (P^n; \mathbb{F}_2) = \frac{\mathbb{F}_2[a]}{a^{n+1}}\), truncated polynomial ring.
    \end{theorem}

    \begin{proof}
        Induction on \(n\) and Poincar\'e Duality.

        It is true for \(n = 1\).

        Now suppose it is true for \(n - 1\).

        We have injection \(i : P^{n-1} \hookrightarrow P^n\). Thus \(i^{\ast}\) is a ring map isomorphism on dimension \(\leq n - 1\).

        Thus \(a, a^2, \cdots , a^{n-1}\) non-zero.

        Question: do we have \(a^n \neq 0\)?
        
        We use Poincar\'e Duality to prove that.

        Suppose \([P^n] \in H_n(P^n;\mathbb{F}_2) \neq 0\).
        
        Then we have: \(\cap [P^n]: H^{n-1} (P^n;\mathbb{F}_2) \xrightarrow{\sim} H_1(P^n;\mathbb{F}_2)\).
        
        Then \(\langle a^n, [P^n] \rangle = \langle a^{n-1}, a\cap [P^n] \rangle \neq 0\) since UCT implies:
        
        \(H^{n-1}(P^n;\mathbb{F}_2) \xrightarrow{\approx} \operatorname{Hom} (H_{n-1}(P^n;\mathbb{F}_2),\mathbb{F}_2)\) by \(\beta \mapsto (b \mapsto \langle \beta , b \rangle )\) and both \(a^{n-1}\) and \(a\cap [P^n]\) are nonzero.
    \end{proof}

    Now we can look at SW classses of \(\gamma_n^1\) and \(TP^n\).

    \begin{proposition}
        \(\operatorname{w} (\gamma_n^1) = 1 + a \in H^{\ast} (P^n; \mathbb{F}_2)\).
    \end{proposition}

    \begin{proof}
        True for \(n=1\) by axiom \(4\).

        Now consider restriction: \(\eval{\gamma^1_n}_{P^1} = \gamma^1_1\).

        By the axiom we have \(1+a = \operatorname{w}(\gamma^1_1) = i^{\ast} \operatorname{w} (\gamma^1_n)\).
    \end{proof}

    Now let \(\gamma=\gamma^1_n = \left\{ \{ ([x],v) \} \mid v\in \mathbb{R}x \right\} \subset P^n \times \mathbb{R}^{n+1}\) be the tautological line bundle.

    \(\gamma \subset \epsilon^{n+1}_{P^n} \implies \gamma \oplus \gamma^{\perp} =\epsilon^{n+1} \).

    Therefore, \(\operatorname{w}(\gamma^{\perp}) = \overline{\operatorname{w}}(\gamma) = (1+a)^{-1} = 1 + a + \cdots + a^n \in H^{\ast} (P^n;\mathbb{F}_2)\). 

    Thus \(\gamma^{\perp}\) has no nonzero sections.

    \begin{corollary}
        \(\gamma^1_\infty\) over \(P^{\infty}\) has no W.SI.
    \end{corollary}

    Question: \(\operatorname{w}(TP^n) = ?\) 

    Recall: \(G \curvearrowright X\) then orbit space \(X / G = X / x \sim gx, S^n / C_2 = P^n\).

    \begin{theorem}
        \begin{enumerate}[label=\roman*)]
            \item \(TP^n \oplus \epsilon^1 = \underbrace{\gamma \oplus \cdots \oplus \gamma}_{n+1}\).
            \item \(\operatorname{w} (TP^n) = (1+a)^{n+1} = \sum_{j=0}^n \binom{n+1}{j} a^j \in H^{\ast} (P^n;\mathbb{F}_2) \)
        \end{enumerate} 
    \end{theorem}

    \begin{proof}
        Apply the antipodal map to:

        \[
            TS^n \oplus \nu = \epsilon^{n+1} = \epsilon ^1 \oplus \cdots \oplus \epsilon^1 \tag{\(\ast\)}
        \]

        To get the following:

        \[
            TP^n \oplus \epsilon = \gamma \oplus \cdots \gamma \tag{\(\ast \ast\)}
        \]

        where \(C_2 \curvearrowright S^n \times \mathbb{R}^{n+1}\) by \((x,v) \mapsto (-x,-v)\).

        Note: \(TP^n = (TS^n) / C_2\) since \(S^n\) is a covering space of \(TP^n\).

        Note: \(\nu(S^n \hookrightarrow \mathbb{R}^{n+1}) \cong \epsilon^1_{S^n}\)
        
        Note: \(\nu(S^n \hookrightarrow \mathbb{R}^{n+1}) / C_2 \cong \epsilon^1_{P^n}\)
        
        Note: \(\epsilon^1_{S^n} / C_2 \cong \gamma\) since \(\frac{S^n \times \mathbb{R}}{C_2} \cong E(\gamma)\) by \([(x,t)] \mapsto ([x],tx)\) 

        This proves \((\ast \ast)\).

        Now we prove i \(\implies\) ii.

        \(\operatorname{w}(P^n) = \operatorname{w}(TP^n \oplus \epsilon) = \operatorname{w}((n+1)\gamma) = \operatorname{w}(\gamma)^{n+1} = (1+a)^{n+1}\) 

    \end{proof}

    MS shows \(TP^n \cong \operatorname{Hom} (\gamma , \gamma^{\perp})\).

    \subsection*{Parallelizable Manifolds}
    
    \begin{definition}
        A manifold \(M^n\) is \textit{parallelizable} if \(TM^n = \epsilon^n_M \) [i.e. if there exists \(n\) linearly independent vector fields]
    \end{definition}

    eg \(S^{2n}\) is not parallelizable via the hairy ball theorem.

    Lie Groups are parallelizable: note that \(T_e G^n\) has basis \(e_1, \cdots , e_n\), and for \(g\in G\) we have \(\ell_g: G \to G\) given by \(h \mapsto gh\).

    We then have \(g \mapsto (d \ell_{g \ast})(e_i)\) giving \(n\) linearly independent vector fields. 

    Thus, \(\operatorname{w}_i(TM^n) \neq 0\) for \(i > 0\) implies \(M\) is not a lie group.

    \(S^0, S^1, S^3, P^0, P^1, P^3 (=\operatorname{SO}(3))\) are lie groups.

    \section*{Wednesday, 9/24/2025}
    
    \begin{corollary}
        [4.6i] \(\operatorname{w}_n(P^n) \neq 0 \iff n\) even.

        (ii). \(\operatorname{w}(P^n) = 1 \iff n+1 = 2^r\)
    \end{corollary}

    \begin{corollary}
        \(n\) even implies \(P^n\) has no nowhere zero vector field.

        \(P^n\) parallelizable [i.e. \(TP^n\) trivial] implies \(n = 2^r - 1\). 
    \end{corollary}

    \begin{proof}
        4.6i: \(\operatorname{w}_n(P^n) \neq 0 \iff \binom{n+1}{n}a^n \neq 0 \iff n+1 \neq 0 \iff n+1\) odd.

        4.6ii: \(\operatorname{w}(P^{2^r - 1}) = (1+a)^{2^r} = 1 + a^{2^r} = 1\) gives one direction. For other direction, if \(n+1 = 2^r m\) for odd \(m > 1\) then \(\operatorname{w}(P^n) = (1+a)^{2^r m} = (1 + a^{2^r})^m = 1 + m a^{2^r} + \cdots\).

    \end{proof}

    \begin{theorem}
        [4.7 Stiefel] Suppose \(\exists\) bilinear map \(p: \mathbb{R}^n \times \mathbb{R}^n \to \mathbb{R}^n\) without zero divisor [meaning \(p(x,y) = 0 \implies x=0\) or \(y=0\)].

        Then \(P^{n-1}\) is parallelizabl [thus \(n = 2^r\)].
    \end{theorem}

    e.g. \(\mathbb{R}, \mathbb{C} ,\mathbb{H} , \mathbb{O}\). Theorem by Adams states \(n = 1,2,4,8\).

    \begin{proof}
        Let \(\{ b_1, \cdots , b_n \} \) be basis for \(\mathbb{R}^n\). Define \(v_i\):

        \[
            \begin{tikzcd}
                \mathbb{R}^n \ar[rr, bend right, "v_i"] & \mathbb{R}^n \ar[l, "{p(-,b_1)}", swap] \ar[r, "{p(-,b_i)}"] & \mathbb{R}^n
            \end{tikzcd}
        \]

        Then \(x\neq 0 \implies p(x,b_1),\cdots ,p(x,b_n)\) are linearly independent, thus \(v_1(x), \cdots , v_n(x)\) linearly independent.

        Note that \(v_1(x) = x\).

        Define linearly independent sections \(s_2, \cdots , s_n\) of \(TP^{n-1}\).

        \(s_i[x] = [x,  \operatorname{pr}_{(\mathbb{R} x)^{\perp}}(v_i(x))]\in TP^{n-1} = (TS^{n-1}) / C_2\). 
    \end{proof}

    \section*{Stiefel-Whitney Numbers}

    We want to prove the following theorem:

    \begin{theorem}
        A closed manifold is a boundary \(\iff\) Stiefel-Whitney numbers are all zero.
    \end{theorem}

    We need to talk about first fundamental class.

    If \(M^n\) is a closed connected manifold [since we have \(\mathbb{F}_2\) coefficient we don't worry about orientation] then the fundamental class \([M] \in H_n(M;\mathbb{F}_2) \cong H^0(M;\mathbb{F}_2) = \mathbb{F}_2\).

    We dont really need connectedness. If \(M^n = M_1 \sqcup \cdots \sqcup M_k\) where each \(M_j\) are connected then the fundamental class \([M] = i_{1 \ast} [M_1] + \cdots + i_{k \ast} [M_k] \in H_n(M;\mathbb{F}_2) = \mathbb{F}_2^k\).

    \begin{definition}
        A partitition of \(n\) is \(r_1, \cdots , r_n \in \mathbb{Z}_{\geq 0}\) such that \(r_1 + 2r_2 + 3r_3 + \cdots + nr_n = n\).

        Let \(\Pi(n) =\) set of partitions of \(n\).

        For example, \(\Pi(4) = \left\{ (0,0,0,1),(0,2,0,0),(1,0,1,0),(2,1,0,0),(4,0,0,0)\right\} \) 
    \end{definition}

    \begin{definition}
        [Stiefel-Whitney Number] Given \((r_i) \in \Pi(n)\) the Stiefel-Whitney Number is defined by:

        \[
            \operatorname{w}_1^{r_1} \cdots \operatorname{w_n}^{r_n} [M] \coloneqq \left\langle \operatorname{w}_1(TM)^{r_1} \cup \cdots \operatorname{w}_n(TM)^{r_n}, [M] \right\rangle \in \mathbb{F}_2
        \]
    \end{definition}

    For example we find Stiefel-Whitney numbers of \(P^2\).

    \(\operatorname{w}(P^2) = \operatorname{w}(TP^2) = (1+a)^3 = 1+a+a^2\).

    \(\operatorname{w}_1^2 [P^2] = \langle a^2, P^2 \rangle = 1\) 

    \(\operatorname{w}_2 [P^2] = \langle a^2, P^2 \rangle = 1\)

    Thus \(P^2\) is not the boundary of a \(3\)-manifold.

    We can see this more easily since the characteristic of \(P^2\) is odd.

    \section*{Friday, 9/26/2025}
    
    Homeowork Due Monday.

    Ch2: 1 Exercise
    Ch3: 1 Exercise
    Ch4: 2 Exercise

    \section*{Manifolds with Boundary}

    Classic examples: disk \(D^n\), cylinder \(S^{n-1} \times I\)

    \begin{definition}
        \textit{Local Model} is the upper half-space \(H^n = \left\{ (x_1, \cdots , x_n) \in \mathbb{R}^n \mid x_1 \geq c \right\} \).
    \end{definition}

    \begin{definition}
        Let \(M \subset \mathbb{R}^A\). An \(n\)-\textit{manifold with boundary} such that \(\forall x\in M, \exists\) smooth homeomorphism (parameterization) \(h: V \to U\) where \(V \subset H^n\) and \(x \in U \subset M\) open such that \(\forall y\in V, \mathrm{d}h_y:\mathbb{R}^n \to \mathbb{R}^A\) has rank \(n\).
    \end{definition}

    \begin{definition}
        \(\operatorname{Int} M \coloneqq \left\{ x\in M \mid \exists \text{nbhd } U \cong \mathbb{R}^n \right\} \).
        
        \(\partial M \coloneqq M - \operatorname{Int} M\)

        \(M = \partial M \cup \operatorname{Int} M\) 
        
        \(D^n = S^{n-1} \cup \operatorname{Int} D^n\)
    \end{definition}

    \(n\)-manifold is \(n\)-manifold with boundary.

    manifold with nonempty interior is not a manifold.

    \(M\) is an \(n\)-manifold with boundary \(\implies \operatorname{Int} M\) is a \(n\)-manifold and \(\partial M\) is a \(n-1\) manifold.

    \(M \simeq \operatorname{Int} M\).

    Now consider tangent space:

    \(\begin{matrix}
        \mathbb{R}^n & \rightarrow & TM & = \{ (x,v) \mid x\in M, v= \gamma^{\prime} (0), \gamma(0)=x  \\ & & \downarrow & , \gamma : [0,\infty) \to M \lor \gamma :(-\infty,0] \to M \} \\ & & M
    \end{matrix}\) 

    Then \(\eval{TM}_{\partial} \cong T \partial M \oplus \epsilon^1\) where \(\epsilon^1\) is the outward poinitng normal, the nowhere zero section of \(\eval{TM}_\partial \).

    \subsection*{Poincar\'e-Lefschetz Duality}

    (PL duality).

    \begin{theorem}
        \(H_n(M, \partial M; \mathbb{F}_2) = \mathbb{F}_2\).
    \end{theorem}

    \begin{definition}
        Fundamental class \([M] \in H_n(M, \partial M; \mathbb{F}_2)\).
    \end{definition}

    \begin{theorem}
        [PL Duality]  \(\cap [M]: H^i(M, \partial M; \mathbb{F}_2) \xrightarrow{\approx} H_{n-i} (M;\mathbb{F}_2)\).

        \(\cap [M]: H^i(M,\mathbb{F}_2) \xrightarrow{\approx} H_{n-i} (M, \partial M, \mathbb{F}_2)\).
    \end{theorem}

    Exercise: Work this out for \(D^n\).

    Furthermore, if we look at the long exact sequence of a pair:

    \[
        H_n(M, \partial M; \mathbb{F}_2) \xrightarrow{\partial} H_{n-1}(\partial M;\mathbb{F}_2) \to H_{n-1}(M;\mathbb{F}_2)  
    \]

    then \(\partial [M] = [\partial M]\).

    \begin{theorem}
        [MS 4.9, Pontryagin] Suppose \(M\) is a compact \(n+1\)-manifold with boundary. Then the Stiefel Whitney numbers of \(\partial M\) are \(0\).
    \end{theorem}

    \begin{proof}
        WLOG \(M\) is connected. Let \(r_i \in \Pi(n)\) [thus \(\sum_{i} r_i i = n\)].

        Then \(\left\langle \operatorname{w}_1 (T \partial M)^{r_1} \cup \cdots \operatorname{w}_n(T \partial M)^{r_n}, [\partial M] (= \partial [M]) \right\rangle\)
        
        \(= \left\langle \delta (\operatorname{w}_1(T \partial M)^{r_1} \cdots \operatorname{w}_n(T \partial M)^{r_n}), [M] \right\rangle \).

        Now, recall:
    
        \[
            H^n(M;\mathbb{F}_2) \xrightarrow{i^{\ast}} H^n(\partial M, \mathbb{F}_2) \xrightarrow{\delta} H^{n+1} (M, \partial M; \mathbb{F}_2)
        \]

        WTS: \(\operatorname{w}_1(T \partial M)^{r_1} \cdots \operatorname{w}_n(T \partial M)^{r_n} \in \operatorname{im} i^{\ast}\).

        Note that it is equal to:

        \[
            \operatorname{w}_1 \left( i^{\ast} (TM) \right)^{r_1} \cdots \operatorname{w}_n(i^{\ast} (TM))^{r_n} = i^{\ast} \left( \operatorname{w}_1(TM)^{r_1} \cdots \operatorname{w}_n(TM)^{r_n} \right)  
        \]

    \end{proof}

    \begin{theorem}
        [P-Thom] A closed \(n\)-manifold is the boundary of a compact \(n\)-manifold iff all Stiefel Whitney numbers vanish.
    \end{theorem}

    Note that all manifolds are boundary of a not necessarily compact manifold, just take \(M \times [0,\infty)\) 

    \begin{definition}
        [Bordism Groups] Two closed \(n\)-dimenstional manifolds \(M_1, M_2\) are \textit{bordant} if \(\exists\) a compact \(W^{n+1}\) manifold with boundary such that \(\partial W \underset{\text{diff}}{\cong} M_1 \coprod M_2\). W is called the \textit{cobordism}.

        Easy exercise: Bordism is an equivalence relation. Canonical example: Pant \(\implies S^1 \sim S^1 \coprod S^1\).

        One can get a group \(\Omega^o_n = (\text{bordism classes of closed \(n\)-manifold}, \coprod)\).
        
        This is called the unoriented bordism group.

    \end{definition}

    Note that \(2 \Omega^o_n = 0\) since \(\partial (M \times I) = M \coprod M\), \(-[M] = [M]\).

    \begin{theorem}
        [Collar Neighborhood] \(\exists\) neighborhood \(U\) of \(\partial W\) and a diffeomorphism \(h: U \xrightarrow{i} \partial W \times [0,\infty)\) such that \(h(x,0)\) for \(x\in \partial W\).
    \end{theorem}

    Note that \(\Omega^o_{\ast}\) is a graded ring with cartesian product.

    \begin{table}[H]
        \centering
        \begin{tabular}{c|c|c}
            \toprule
                \(n\) & \(\Omega^n_o\) & \(\Pi(n)\) \\
            \midrule
                \(0\) & \(\mathbb{Z} / 2\) pt & \(1\) \\
                \(1\) & \(0\) & \(1\) \\
                \(2\) & \(\mathbb{Z}/2\, P^2\) & \(2\) \\
                \(3\) & \(0\) & \(0\) \\
                \(4\) & \(\mathbb{Z}/2 \oplus \mathbb{Z}/2\, P^4, P^2 \times P^2\) & \(5\) \\
                \(5\) & \(\mathbb{Z} / 2\) Wu-\(n\)-manifold \(SU(3) / SO(3)\) & \(7\) \\
            \bottomrule
        \end{tabular}
        \caption{Bordism Group Calculations}
    \end{table}

    \begin{theorem}
        [PT Theorem]

        \[
            \Omega^o_n \overset{\text{SW\#}}{\rightarrowtail} (\mathbb{F}_2)^{\Pi(n)}
        \]
    \end{theorem}

    \section*{Monday, 9/29/2025}
    
    Applications:

    Let \(M \to \overline{M}\) is a \(k\) to \(1\) covering map with \(k\) odd. Then,

    \(0 = [M]\in \Omega^o_n \iff 0 = [\overline{M}] \in \Omega^o_n\) 

    eg Lens spaces \(L(k)\) with \(k\) odd are boundaries.

    \begin{proof}
        \(H_n(M;\mathbb{F}_2) \xrightarrow[\cong]{\cdot k} H_n(\overline{M};\mathbb{F}_2)\).
        
        SW numbers of \(M =\) SW numbers of \(\overline{M}\).
    \end{proof}

    MS poses the question:

    Why is \(P^{2k-1}\) a boundary?

    \begin{proof}
        First proof: 

        We explicitly calculate the SW numbers.

        \(\operatorname{w}(P^{2k-1}) = (1+a)^{2k} = (1+a^2)^k\).
        
        Thus, for \(i\) odd, \(\operatorname{w}_i(P^{2k-1}) = 0\).

        Thus, since \(\sum_{i} ir_i\) is odd:
        
        Taking mod \(2 \to \sum_{i \text{ odd}} r_i\) is odd so some odd \(r_i\) is nonzero. Thus, \(\operatorname{w}_1^{r_1} \cdots \operatorname{w}_{2k-1}^{r_{2k-1}}[P^{2k-1}] = 0\).
        
        Second proof:

        If \(\exists\) free \(C_2\)-action on \(M\) then \(M\) is a boundary.

        Proof: \(\partial (M \times_{C_2} [1,-1]) = M \times_{C_2} \{ -1, 1 \} = M\).

        Or: \(\begin{tikzcd} S^0\ar[r] & M \ar[d] \\ & \overline{M} \end{tikzcd}\), change fiber \(D^1\) gives us \(\begin{tikzcd} D^1 \ar[r] & W = M \times_{C_2} [-1,1] \ar[d] \\ & \overline{M} \end{tikzcd}\) which gives us \(\partial W = M\). 

        Lens space \(L(4)\) with \(\pi_1 = C_4\) then covered by \(P^{2k-1}\).
    \end{proof}

    Conjecture by Farrell/Yau:

    Almost flat manifolds are boundaries.

    \begin{theorem}
        [Gromov] Almost flat \(\iff\) infranil \(\overset{def}{\iff } \begin{matrix} \text{nilmanifold} \\ \downarrow & \text{finite cover} \\ M \end{matrix}\) 
    \end{theorem}

    Nilmanifold is a simply connected lie group modulo a lattice. Example: \(\begin{bmatrix}
        1 & \ast & \ast  \\
        0 & 1 & \ast  \\
        0 & 0 & 1 \\
    \end{bmatrix} \), lattice is where \(\ast\) are integers.

    \begin{theorem}
        [D-Fang] Yes if finite cover is \(2^k\)-to-\(1\).

        \(N / \Gamma \to M\) if \(2^k\)-to-\(1\) implies \(M = \partial W\). 
    \end{theorem}

    \section*{Chapter 5}

    \(\mathbb{R} P^{k-1} =\) lines in \(\mathbb{R}^k\). By lines we mean \(1\)-dim spaces through the origin. Easier to think of \(\frac{S^{k-1}}{x \sim -x}\) usually.

    We have the tautological line bundle given by \(E(\gamma) = \{ \text{(line, point on line)} \} \subset \mathbb{R}P^{k-1} \times \mathbb{R}^k\).

    \[
        \begin{tikzcd}
            \mathbb{R} \ar[r] & E(\gamma)\ar[d] \\ & P^{k-1}
        \end{tikzcd}
    \]

    Instead of lines we can think about higher dimensional vector spaces through the origin which gives us the Grassmanian.

    \subsection*{Grassmanian or Grassmanian Manifold of \(n\)-planes in \(\mathbb{R}^k\)}

    Notation: \(G_n(\mathbb{R}^k)\) is the Grassmanian. Points are \(n\)-dim subspaces of \(\mathbb{R}^k\).

    \(X\in G_n(\mathbb{R}^k) \implies X = n\)-dim subspaces of \(\mathbb{R}^k\).

    Example: planes through the origin in \(\mathbb{R}^n\).

    We have a tautological \(n\)-plane bundle \(E(\gamma^n) = \{ \text{point, point on plane} \} \) 

    \(\begin{tikzcd} \mathbb{R}^n \ar[r] & E(\gamma^n) = \{ (X,v) \in G_n(\mathbb{R}^k) \times \mathbb{R}^k \mid v\in X \} \ar[d] \\ & G_n(\mathbb{R}^k) \end{tikzcd}\) 

    Suppose \(M^n \subset \mathbb{R}^k\). Then we have \(M \to G_n(\mathbb{R}^k), p \mapsto T_p M\).

    We in fact have a bundle map:

    \[
        \begin{tikzcd}
            TM \ar[d] \ar[r] & E(\gamma^n) \ar[d] \\ M \ar[r] & G_n(\mathbb{R}^k)
        \end{tikzcd}
    \]

    \[
        \begin{tikzcd}
            (p,v) \ar[r,mapsto] \ar[d,mapsto] & (T_{\pi(v)}M, v) \ar[d,mapsto] \\ p \ar[r, mapsto] & T_p M
        \end{tikzcd}
    \]

    We can do the same for the normal bundle.

    \[
        \begin{tikzcd}
            \nu(M \hookrightarrow \mathbb{R}^k) \ar[r] \ar[d] & E(\gamma^{k-n}) \ar[d] \\ M \ar[r] & G_{k-n}(\mathbb{R}^k)
        \end{tikzcd}
    \]

    \subsection*{Topology on \(G_n(\mathbb{R}^k)\)}

    We need to find an atlas. What is the dimension?

    \begin{definition}
        [Stiefel Manifold] \(V_n(\mathbb{R}^k) =\) orthonormal \(n\)-frames in \(\mathbb{R}^k\)

        \(= \left\{ (v_1, \cdots , v_n) \in \mathbb{R}^k \times \cdots \mathbb{R}^k \mid v_i \cdot v_j = \delta_{ij} \right\} \).

        This is a closed, bounded subsset of \((\mathbb{R}^k)^n \implies\) it is compact.
    \end{definition}

    Thus this has a topology.
    
    Now, we have an onto map \(q:V_n(\mathbb{R}^k) \twoheadrightarrow G_n(\mathbb{R}^k)\) with \(q(v_1, \cdots , v_n) = \operatorname{Span}\{ v_1, \cdots , v_n \}\).

    Give \(G_n(\mathbb{R}^k)\) the quotient topology, i.e. \(U \subset G_n(\mathbb{R}^k)\) is open iff \(q ^{-1} U\) is open.

    \begin{lemma}
        [5.1] \(G_n(\mathbb{R}^k)\) is a compact smooth manifold of dimension \(n(k-n)\). Furthermore, there is a diffeomorphism \(G_n(\mathbb{R}^k) \to G_{k-n}(\mathbb{R}^k)\) by \(X \mapsto X^{\perp}\). 
    \end{lemma}

    \section*{Wednesday, 10/1/2025}
    
    \(O(n)\to V_n(\mathbb{R}^k)\) Stiefell, On n

    \(\downarrow q\)
    
    \(G_n(\mathbb{R}^k) = n\) planes in \(\mathbb{R}^k\), Grassmanian.

    \(q(v_1, \cdots , v_n) = \operatorname{Span}(v_1, \cdots , v_n)\).
    
    Given \(V_n(\mathbb{R}^k) \subset (\mathbb{R}^k)^n\) subspace topology.

    We give \(G_n(\mathbb{R}^k)\) quotient topology.

    \begin{lemma}
        \(G_n(\mathbb{R}^k)\) is a compact smooth manifold of \(\dim n(k-n)\).
    \end{lemma}

    \begin{proof}
        hausdorff?

        \(X\in G_n(\mathbb{R} ^k)\) 

        \(v\in \mathbb{R}^{k} \) 

        \(d(x,v) = ^{-1} d(x,v)\) 
        
        \(V_n(\mathbb{R}^k) \xrightarrow{q} G_n(\mathbb{R}^k) \xrightarrow{} R\) 

        \(d(-,v) \circ q\) continuous.

        \(d(-,v)\) continuous.
        
        If \(X\neq Y\) choose \(v\in Y - X\).
        
        Let \(d = d(X,v)\).

        Separate \(X\) and \(Y\) by:

        \(d(-,v)^{-1} (-\infty , \frac{d}{2})\) and \(d(-,v)^{-1} (\frac{d}{2} \infty )\)  
    \end{proof}

    Atlas? Euclidean Neighborhoods?

    \(X \in G_n(\mathbb{R}^k)\)

    \(U = U_X = \{ y\in G_n(\mathbb{R}^k) \mid X^{\perp}  = \{ 0 \} \} \) open and dense.

    \(\Gamma : \operatorname{Hom} (X, X^{\perp}) \to U\)
    
    \(f \mapsto \operatorname{graph}(f) \subset \mathbb{R}^k = X \oplus X^{\perp } (\cong X \times X^{\perp})\).

    \(\operatorname{graph}(f) \coloneqq \{v + f(v) \mid v\in x\} \) 

    \[
        \begin{tikzcd}
            U \ar[r, "\Gamma^{-1}"] \ar[rr, bend right, "\phi"] & \operatorname{Hom} (X, X^{\perp}) \ar[r, "\cong"] & \mathbb{R}^{n(k-n)}
        \end{tikzcd}
    \]

    Coordinates show \(\phi\) is homeomorphism.

    Atlas \(\{ (U, \phi) \} \) 

    ANother proof:

    \(O(k) \curvearrowright G_n(\mathbb{R}^k)\) transitively, \((A,X) \mapsto AX\) 

    Isotopy at \(\mathbb{R} \times \{ 0_{k-n} \} \):

    is \(O(n) \times O(k-n)\).

    Thus \(G_n(\mathbb{R}^k) = O(k) / O(n) \times O(k-n)\).

    If \(G\) is a compact lie group and \(H\) is a closed subgroup then \(G / H\) is a manifold.

    \[
        \begin{tikzcd}
            O(n) = \frac{O(n) \times O(n-k)}{O(n-k)} \ar[r] & V_n(\mathbb{R}^k) \ar[d] \ar[r,"=", phantom] & O(k)/O(n-k)  \\
            & G_n(\mathbb{R}^k) \ar[r,"=",phantom] & O(k) / O(n) x,O(n-k)
        \end{tikzcd}
    \]

    Associated \(\mathbb{R}^n\) bundle is \(\gamma^n\).

    \(E(\gamma^n) = V_n(\mathbb{R}^k) \times_{O(n)} \mathbb{R}^n\).
    
    \section*{Friday, 10/3/2025}
    
    \begin{lemma}
        [5.2] The tautological bundle is a bundle:

        \[
            \begin{tikzcd}
                E(\gamma^n_k) \ar[r, phantom, "="] \ar[d,"\pi"] & \{ (X,v) \mid v\in X \} \subset G_n(\mathbb{R}^k) \times \mathbb{R}^k \\
                G_n(\mathbb{R}^k)
            \end{tikzcd}
        \]

        is a rank \(n\) v.b.
    \end{lemma}

    \begin{proof}
        \(\pi ^{-1} X\) is a vector space: \((X,v)+(X,w) = (X,v+w), c(X,v) = (X,cv)\).

        We also want local triviality. Consider \(X\in G_n(\mathbb{R}^k)\). Let \(U = \{ Y \mid Y\cap X^{\perp} = 0\} \).
        
        \[
            \begin{tikzcd}
                U \times \mathbb{R}^n \ar[rr,"h"] \ar[rd] & & \pi ^{-1} U \ar[ld] \\ & U
            \end{tikzcd}
        \]

        is a fiberwise isomorphism where \(h\) is a homeomorphism.

        Then \(U \times \mathbb{R}^n \cong U \times X\) by choosing a basis for \(X\). Furthermore, \(U \times X \xleftarrow{\Gamma \times \operatorname{id}_{X}} \operatorname{Hom} (X, X^{\perp}) \times X\) and \(\operatorname{Hom} (X, X^{\perp}) \times X \to \pi ^{-1} U\) by \((f,v) \mapsto (\operatorname{graph} f, v + f(v))\).  
    \end{proof}

    \begin{lemma}
        [5.3] Any \(n\)-plane bundle \(\xi\) over a compact Hausdorff manifold, \(\exists\) a bundle map to the tautological bundle \(G_n(\mathbb{R}^k)\):

        \[
            \begin{tikzcd}
                E(\xi) \ar[r,"\widetilde{c}"] \ar[d] & E(\gamma^n_k) \ar[d] \\ B \ar[r,"c"] & G_n(\mathbb{R}^k)
            \end{tikzcd}
        \]

        for \(k\) large.

        So the tautological bundle is final.
    \end{lemma}

    Note that we knew this for embedded manifold and tangent bundle:

    \[
        \begin{tikzcd}
            TM \ar[d] \\ M \ar[r] & G_n(\mathbb{R}^k) \\ p \ar[r, mapsto] & T_p M
        \end{tikzcd}
    \]

    \(c\) is called `classifying group' and \(\gamma^n\) is the universal bundle.

    By defintion, a bundle map \(\xi \to \gamma_k^n\) is the same as a fiberwise isomorphism:
    
    \(\begin{tikzcd}E(\xi) \ar[r] \ar[d] & E(\gamma_k^n) \ar[d] \\ B \ar[r] & G_n(\mathbb{R}^k)\end{tikzcd}\) which is by definition the same as a fiberwise monomorphism \(\hat{c}: E(\xi) \to \mathbb{R}^k\).

    Let \(F_b = \pi ^{-1} b\). Then \(c(b) = \hat{c}(F_b) \mapsfrom \hat{c}\).

    Then \(\widetilde{c}(e) = (\hat{c}(F_b), \hat{c}(e))\).
    
    Now we prove lemma 5.3.

    \begin{proof}
        Compact, so choose open cover \(U_1, \cdots , U_r\) of \(B\) such that \(\eval{\xi}_{U_i}\) is trivial.
        
        Choose open \(W_i \subset V_i \subset U_i\) such that \(\overline{W}_i \subset V_i, \overline{V}_i \subset U_i\), and \(\{ W_i \}\) and \(\{ V_i \}\) still cover \(B\).

        Note that \(\overline{W}_i\) and \(B - V_i\) are disjoint closed sets. Thus \(\exists\) continuous \(\lambda_i: B \to [0,1]\) such that \(\lambda_i(\overline{W}_i) = 1, \lambda_i(B - V_i) = 0\) by Urysohn's lemma.

        \(\eval{\xi}_{U_i}\) trivial \(\iff\) fiberwise isomorphism \(h_i: \pi ^{-1} U_i \to \mathbb{R}^n\) by sections \(s_j(b) \mapsto e_i\).

        Define \(\hat{c}: E(\xi) \to \underbrace{\mathbb{R}^n \oplus \cdots \oplus \mathbb{R}^n}_{r \text{ times}}\).

        \[
            \hat{c}(e) = (\lambda_1(\pi(e)) h_1(e), \cdots , \lambda_r(\pi(e)) h_r(e))
        \]

    \end{proof}

    \begin{corollary}
        [Not in MS] Every vector bundle \(\xi\) over a compact Hausdorff space \(B\) has a whitney sum inverse.
    \end{corollary}

    \begin{proof}
        [What we need is a finite locally trivial cover].

        Let \(\xi = c^{\ast} (\xi^n_k)\). Consider \(\xi \oplus c^{\ast} (\gamma^{\perp}) = c^{\ast} (\gamma \oplus \gamma^{\perp}) = c^{\ast} (\epsilon^k_{G_n(\mathbb{R}^k)})\) which is trivial. 
    \end{proof}

    Contrast this with the fact that \(\gamma^1_\infty\) has no whitney sum inverse.

    Comment: \(k^{\prime} \leq k \implies \begin{tikzcd} E(\gamma^n_{k^{\prime}}) \ar[r, hook] \ar[d] & E(\gamma^n_k) \ar[d] \\ G_n(\mathbb{R}^{k^{\prime}})\ar[r, hook] & G_n(\mathbb{R}^k) \end{tikzcd}\) 

    \begin{theorem}
        If \(f,g: \xi \to \gamma^n_k\) bunndle maps then \(f \simeq g: \xi \to \gamma^n_{2k}\).

        So ``classifying map unique upto homotopy.''
    \end{theorem}

    \begin{proof}
        WTS: \(\hat{f} \simeq \hat{g} : E(\xi) \to \mathbb{R}^{2k}\) fiberwise monomorphism.
        
        Special case: \(\forall e\in E(\xi), \forall \lambda > 0, \hat{f}(e) \neq -\lambda \hat{g}(e)\). \(h_t(e) = (1-t) \hat{f}(e) + t \hat{g} (e)\).

        General case: define embeddings \(d_0, d_1, d_2:\mathbb{R}^k \to \mathbb{R}^{2k}\).

        \(d_0(e_i) = e_i, d_1(e_i) = e_{2i - 1}, d_2(e_i) = e_{2i}\).

        Then \(d_0 \circ \hat{f} \simeq d_1 \circ \hat{f} \simeq d_2 \circ  \hat{g} \simeq d_0 \circ \hat{g}\).
    \end{proof}

    \section*{Monday, 10/6/2025}
    
    Recall lemma 5.3: all vector bundle \(\xi\) over compact hausdorff \(B\) there exists a bundle map:

    \[
        \begin{tikzcd}
            E(\xi) \ar[r, "\widetilde{c}"] \ar[d] & E(\gamma^n_k) \ar[d] \\ B \ar[r,"c"] & G_n(\mathbb{R}^k)
        \end{tikzcd}
    \]

    for \(k\) sufficiently large.

    \begin{theorem}
        [5.7] If \(B\) is compact Hausdroff and \(f,g: \xi \to \gamma^n_k\) are bundle maps, then \(f\simeq g: \xi \to \gamma^n_{2k}\).
    \end{theorem}

    Recall the proofs required \(E(\xi) \to \mathbb{R}^k\) fiber monomorphism.

    \begin{theorem}
        [Covering Homotopy Theorem] Slogan: ``Homotopy Invariance of Pullback.''.

        Suppose we have compact hausdorff manifolds and maps:
        
        \[
            \begin{tikzcd}
                & E(\xi^{\prime}) \ar[d] \\ B \ar[r,"f \simeq g"] & B^{\prime}
            \end{tikzcd}
        \]

        Then \(f^{\ast} \xi^{\prime} \cong g^{\ast} \xi ^{\prime} \).
    \end{theorem}

    We can `replace \(k\) by \(\infty\) and compact hausdorff by paracompact Hausdorff.'

    For 5.2, we use \(\infty \cdot n = \infty\).

    For 5.7, we use \(\infty + \infty = \infty\).

    5.3, 5.7 and CHT implies: \(B\) paracompact Hausdroff implies there is a bijection between homotopy classes \([B, G_n(\mathbb{R}^{\infty})]\) and \([\text{iso class of \(n\)-plane v.b. over \(B\)}]\).

    \(f \mapsto f^{\ast} \gamma^n\).

    This is why the Grassmanian is a classifying space, it classifies all bundles.

    e.g. for sphere \(B = S^l\) then \(\pi_l(G_n(\mathbb{R}^{\infty})) = \frac{\begin{Bmatrix}\mathbb{R}^n & \to & E \\ & & \downarrow \\ & & S^l\end{Bmatrix}}{\text{iso}}\).

    Let \(A\) be an abelian group and \(\operatorname{w} \in H^l(G_n(\mathbb{R}^{\infty}), A)\).

    Then we get characteristic class of \(n\)-plane bundle over \(B\) a CW complex. Recall CW complexes are paracompact Hausdorff!

    Thus, in order to get characteristic classes, we only need:

    \[
        \begin{tikzcd}
            E(\xi) \ar[r] \ar[d] & E(\gamma^n) \ar[d] \\ B \ar[r,"c"] & G_n(\mathbb{R}^{\infty})
        \end{tikzcd}
    \]

    Then the characteristic class is defined to be \(\operatorname{w}(\xi) = c^{\ast} \operatorname{w} \in H^l(B;A)\).
    
    Then if we have \(\begin{tikzcd} & E\ar[d] \\ B^{\prime} \ar[r,"f"] & B\end{tikzcd}\) then \(f^{\ast} \operatorname{w} (\xi) = \operatorname{w} (f^{\ast} \xi)\).

    \begin{theorem}
        [Future Theorem] \(H^{\ast} (G_n(\mathbb{R}^{\infty});\mathbb{F}_2) = \mathbb{F}_2[\operatorname{w}_1, \operatorname{w}_2, \cdots ,\operatorname{w}_n]\).
    \end{theorem}

    For example, for \(n = 1\), this theorem states that \(H^{\ast} (P^{\infty}, \mathbb{F}_2) = \mathbb{F}_2[a]\).

    First we talk about \(\mathbb{R}^{\infty}\) and \(G_n(\mathbb{R}^{\infty})\). We talk about colimits for that.

    \section*{Colimit}

    Consider Category \(\mathcal{C}\).

    \begin{definition}
        A \textit{directed system} (ds):

        \[
            X_0 \to X_1 \to X_2 \to X_3 \to \cdots
        \]
    \end{definition}

    \begin{definition}
        A \textit{cocone} of a directed system is an object \(X\) with maps so that:

        \[
            \begin{tikzcd}
                X_0 \ar[rrrd] \ar[r] & X_1 \ar[r] \ar[rrd] & X_2 \ar[r] \ar[rd] & X_3 \ar[r] \ar[d] & \cdots \ar[ld] \\
                & & & X
            \end{tikzcd}
        \]
    \end{definition}

    \begin{definition}
        A \textit{colimit} of a directed system is an \textit{initial cocone}:

        \[
            \begin{tikzcd}
                X_0 \ar[rrrd] \ar[rrrdd] \ar[r] & X_1 \ar[r] \ar[rrd] \ar[rrdd] & X_2 \ar[r] \ar[rd] \ar[rdd] & X_3 \ar[r] \ar[d] \ar[dd, bend right] & \cdots \ar[ld] \ar[ldd] \\
                & & & C \ar[d, dotted, "\exists!"] \\
                & & & X
            \end{tikzcd}
        \]

    \end{definition}

    Colimits may not exist. If they exist they are unique upto isomorphism. We write \(C = \operatorname{colim}_{n\to \infty} X_n\) 

    Colimit is kind of a `generalized union'.

    Colimits are generally `quotients of coproducts'.

    In the category \(R\)-mod, 

    \[
        \operatorname{colim}_{n\to \infty} X_n = \frac{\bigoplus_{n} X_n}{\langle X_n - \operatorname{im} (X_{n}) \rangle }
    \]

    Thus \(\mathbb{R}^{\infty} \coloneqq \operatorname{colim}_{n\to \infty} \mathbb{R}^n\), if basis \(e_1, e_2, e_3, \cdots\) then almost all coordinates are zero: \((a_1, a_2, \cdots , a_n, 0, 0, \cdots)\) 

    In \(\operatorname{Top}\) or \(\operatorname{Set}\),

    \[
        \operatorname{colim}_{n\to \infty} X_n = \frac{\coprod X_n}{X_n \sim \operatorname{im} (X_n)}
    \]

    Then \(G_n(\mathbb{R}^{\infty}) = \operatorname{colim}_{n\to \infty} G_n(\mathbb{R}^k) =\) set of \(n\)-planes in \(\mathbb{R}^{\infty}\) with a particular topology. In some sense, it is \(\bigcup_{k} G_n(\mathbb{R}^k)\).

    \section*{Stiefel Manifolds}

    Recall: we have Stiefel Manifolds:

    \[
        \begin{tikzcd}
            O(n) \ar[r] & V_n(\mathbb{R}^{\infty}) \ar[d] & \text{orthonormal \(n\)-frames in \(\mathbb{R}^{\infty}\)} \\ & G_n(\mathbb{R}^{\infty}) 
        \end{tikzcd}
    \]

    \(V_n(\mathbb{R}^{\infty}) \times_{O(n)} \mathbb{R}^n = E(\gamma^n)\).

    \begin{theorem}
        \(V_n(\mathbb{R}^{\infty})\) is contractible. eg for \(n=1\) we have \(S^{\infty} \simeq \ast\).
    \end{theorem}

    We need some facts from algebraic topology:

    \begin{enumerate}[label=\arabic*)]
        \item \(V_n \mathbb{R}^{\infty}\) is a CW complex and \(V_n \mathbb{R}^k \subset V_n \mathbb{R}^{\infty}\) are subcomplexes.
        \item Whitehead's Theorem: if \(X\) is CW then \(X\simeq \ast \iff \pi_{\ast} X = 0\).
        \item Given fibration \(\begin{matrix} F & \to & E \\ & & \downarrow \\ & & B\end{matrix}\) (e.g. a \((G,F)\)-bundle) there exists long exact sequence:
        
        \[
            \cdots \to \pi_i F \to \pi_i E \to \pi_i B \to \pi_{i-1} F \to \cdots
        \]
    \end{enumerate} 

    Now we can prove the theorem:

    \begin{proof}
        \(1\implies\)  \(\pi_i(V_n(\mathbb{R}^{\infty})) = \operatorname{colim}_{k \to \infty} \pi_i(V_n(\mathbb{R}^k))\)  

        \(3 \implies\) for \(i \leq l, \pi_i O(l) \xrightarrow{\approx} \pi_i O(l+1)\).
        
        \(\begin{matrix}O(l) & \to & O(l+1) & A \\ & & \downarrow & \downarrow \\ & & S^l & A e_{l+1}\end{matrix}\)

        Then \(\begin{matrix}O(k-n) & \to & O(k) & A \\ & & \downarrow \\ & & V_n(\mathbb{R}^k) & A e_1, \cdots , A e_n\end{matrix}\) 

        \(\implies i < k-n, \pi_i (V_n \mathbb{R}^k) = 0 \overset{(2)}{\implies} \) the theorem. 
    \end{proof}

    \section*{Monday, 10/13/2025}
    
    Note:

    Schubert Symbol: \(\sigma = (\sigma_1, \cdots , \sigma_n)\).

    \(1 \leq \sigma_1 < \cdots < \sigma_n\).

    Dimension \(d = d(\sigma) = \sum_{i} \sigma_i - i\)

    Partition of \(d = \sigma - (1,2,\cdots , n)\).

    Recap:

    \(G_n(\mathbb{R}^{\infty}) = BGL(n,\mathbb{R})\).

    It is a classifying space.

    Proof 1: representative object.

    Proof 2: \(\begin{tikzcd}O(n) \ar[r] & V_n(\mathbb{R}^{\infty}) \simeq \ast \ar[d] \\ & G_n(\mathbb{R}^{\infty})\end{tikzcd}\).

    Thus \(G_n(\mathbb{R}^{\infty}) = BO(n), BO(n) = BGL(n,\mathbb{R}), O(n) \simeq GL(n,\mathbb{R})\).

    Preview of Chapter 6/7:

    \begin{itemize}
        \item Find CW structure on \(G_n \mathbb{R}^{\infty}\).
        \item Show \(\mod 2\) cellular chain complex has zero differentials. [So this is just like \(\mathbb{R}P^{\infty}\)].
    \end{itemize} 

    Then \(H_k(G_n(\mathbb{R}^{\infty}); \mathbb{F}_2) = C_k(G_n \mathbb{R}^{\infty}) \otimes \mathbb{F}_2 = \mathbb{F}_2^{\text{\# of \(k\)-cells}}\).

    We use the following two definitions of CW-complexes.

    \begin{definition}
        [Using Pushouts] A topological space \(X\) together with the filtration \(\{ X^n \}_{n=0}^{\infty}\) called skeleton, written \(\left( X, \{ X^n \}_{n=0}^{\infty} \right) \) so that,

        \[
            X^0 \subset X^1 \subset \cdots \subset X = \bigcup_{n=0}^{\infty} X^n 
        \]

        such that,

        \begin{enumerate}[label=\arabic*)]
            \item \(\forall n, \exists\) pushout diagram:
            
            \[
                \begin{tikzcd}
                    \coprod S^{n-1} \ar[r] \ar[d] & X^{n-1} \ar[d] \\ \coprod D^n \ar[r] & X^n
                \end{tikzcd}
            \]

            \item \(X = \operatorname{colim}_{n\to \infty} X^n\).
        \end{enumerate} 
    \end{definition}

    \begin{definition}
        [Whitehead] Instead of a filtration we have a partition with cells \(e_\alpha\).

        Let \(X\) be a Hausdorff space. Consider \(\left( (X, \{ e_\alpha \}) \right)\) so that,

        \(\{ e_\alpha \} \) form partition of \(X\). i.e. \(X = \bigcup_{\alpha} e_\alpha, e_\alpha \cap e_\beta = \varnothing\) so that,

        \begin{enumerate}[label=\arabic*)]
            \item \(\forall \alpha, \exists\) characteristic map \(\chi_\alpha : D^n \to \overline{e_\alpha}\) such that \(\eval{\chi_\alpha}_{\overset{\circ}{D^n}}: \overset{\circ}{D^n}\xrightarrow{\approx} e_\alpha\) homeomorphism.
            \item \(\chi_\alpha (S^{n-1}) \subset\) finite union of \(n-1\) cells.
            \item \(B \subset X\) closed \(\iff \forall \alpha, B \cap \overline{e_{\alpha}}\) closed in \(\overline{e_\alpha}\).    
        \end{enumerate} 
    \end{definition}

    We can get the skeleton from the cells in the following way: \(X^n = \bigcup_{\dim e_\alpha \leq n} e_\alpha\).

    Also note \(2^{\prime}\) alternate: \(\overline{e_\alpha} - e_\alpha \subset\) finite union of \(n-1\) cells.

    For skeleton to cell, note that \(X^n - X^{n-1}\) is topologically \(\coprod_{n \text{-cells}} e_\alpha\).
    
    We want to figure out the CW complex of the Grassmanian. This is connected to combinatorics.

    \begin{definition}
        [Schubert Symbol] The cells will be indexed by Schubert Symbol, which will be increasing sequence of integers: \(\sigma  = (\sigma_1, \cdots , \sigma_n)\) so that \(1 \leq \sigma_1 < \sigma_2 < \cdots < \sigma_n\). This will index a `Schubert cell' of \(G_n \mathbb{R}^k\) if \(\sigma_n \leq k\):

        \[
            e(\sigma) = \left\{ X \in G_n \mathbb{R}^k \mid \forall i, \dim (X \cap \mathbb{R}^{\sigma_i}) = i, \dim (X \cap \mathbb{R}^{\sigma_{i-1}}) = i - 1 \right\} 
        \]

        So we have a dimension jump at \(\mathbb{R}^{\sigma_i}\).

        \(\dim e(\sigma) = \sum_{i} \sigma_i - i\)
    \end{definition}

    \begin{theorem}
        [6.4] \((G_n \mathbb{R}^k, \{ e(\sigma)\})\) is a CW complex [note: \(1 \leq \sigma_1 < \cdots < \sigma_n \leq k\)], and \(\dim e(\sigma) = d(\sigma)\).

        It also holds for \(k=\infty\), i.e. \(G_n \mathbb{R}^{\infty}, \{ e(\sigma) \}\) where \(1 \leq \sigma_1 < \cdots < \sigma_n\) is a CW complex.
    \end{theorem}

    Example: \(G_1(\mathbb{R}^3)\). \(\sigma = (1), (2), (3)\).

    Thus \(G_1 \mathbb{R}^3 = e^0 \cup e^1 \cup e^2\).

    \(e^{(1)}\) is the line given by the \(x\)-axis.

    \(e^{(2)}\) is the set of lines through origin in the \(xy\)-plane except the \(x\)-axis.

    \(e^{(3)}\) is the set of lines through origin that are not contained in \(xy\)-plane.

    Now consider \(G_2(\mathbb{R}^3). \sigma =(1,2),(1,3),(2,3)\).

    \(e(1,2)\) is the \(xy\)-plane.
    
    \(e(1,3)\) are the planes with one basis \(x\)-axis, other basis not the \(y\)-axis.

    \(e(2,3)\) are the planes that doesn't contain the \(x\)-axis.

    Now consider \(G_2(\mathbb{R}^4)\). Then \(\sigma = (1,2) [d=0],(1,3) [d=1],(1,4) [d=2], (2,3) [d=2], (2,4) [d=3], (3,4) [d=4]\).

    \begin{table}[H]
        \centering
        \begin{tabular}{c|c|c}
            \toprule
                \(\sigma\) & \(\dim, d = d(\sigma)\) & \(\sigma - (1,2,\cdots , n)\) \\
            \midrule
                \((1)\) & \(0\) & \(0\) \\
                \((2)\) & \(1\) & \(1\)  \\
                \((3)\) & \(2\) & \(2\) \\
                \((4)\) & \(3\) & \(3\) \\
                \((1357)\) & \(6\) & \(0\,1\,2\,3\)  \\
            \bottomrule
        \end{tabular}
        \caption{Schubert Symbol Dimensions}
    \end{table}

    \begin{corollary}
        [6.7] \(\#\) of \(d\)-cells in \(G_n\mathbb{R}^k = \#\) of paritions of \(d\) into at most \(n\) integers \(\leq k - n\).
    \end{corollary}

    \section*{Wednesday, 10/15/2025}
    
    Chapter 7 assumes existence of SW classes satisfying axioms 1-4.

    Abbreviate \(G_n = G_n(\mathbb{R}^{\infty})\). We have bundles:

    \[
        \begin{tikzcd}
            \mathbb{R}^n \ar[r] & E(\gamma^n) \ar[d] \ar[r,"\subset", phantom] & G_n \times \mathbb{R}^{\infty} \\ & G_n \ar[r,"\coloneqq", phantom] & G_n(\mathbb{R}^{\infty})
        \end{tikzcd}
    \]

    Notation: \(\operatorname{w}_k \coloneqq \operatorname{w}_k(\gamma^n)\).

    \(H^{\ast} X = H^{\ast} (X;\mathbb{F}_2)\). `\(\mathbb{F}_2\)-coefficients understood'.

    \begin{theorem}
        [7.1]

        \[
            H^{\ast} G_n = \mathbb{F}_2[w_1, \cdots , \operatorname{w}_n]
        \]
    \end{theorem}
    
    
    The free polynomial ring on generators of degs \(1, 2, \cdots , n\).

    \(\iff\) There is no polynomial relationship between them: if \(p\) is a polynomial in \(n\) variables and \(p(\operatorname{w}_1, \cdots , \operatorname{w}_n) = 0\), we must have \(p \equiv 0\).
        
    \(\iff \operatorname{w}_1, \cdots , \operatorname{w}_n\) are \textit{algebraically independent}.
    
    \begin{lemma}
        Recall \(\gamma^1\) is the tautological line bundle.

        Let \(\xi = \underbrace{\gamma^1 \times \cdots \gamma^1}_{n \text{ times}}\).

        \begin{enumerate}[label=\roman*)]
            \item \(\operatorname{w}_1(\xi), \cdots , \operatorname{w}_n(\xi)\) are algebraically independent.
            \item \(\operatorname{w}_1, \cdots , \operatorname{w}_n\) are algebraically independent.  
        \end{enumerate} 
    \end{lemma}

    \begin{proof}
        \[
            \begin{tikzcd}
                \mathbb{R}^n \ar[r] & E(\xi) = E(\gamma^1) \times \cdots \times E(\gamma^1) \ar[d] \\ & \mathbb{P}^{\infty} \times \cdots \times \mathbb{P}^{\infty}
            \end{tikzcd}
        \]

        \(H^{\ast} \mathbb{P}^{\infty} = \mathbb{F}_2[a]\) by Poincar\'e duality.

        Thus \(H^{\ast} (\mathbb{P}^{\infty} \times  \cdots \times \mathbb{P}^{\infty}) = H^{\ast} \mathbb{P} ^{\infty} \otimes_{\mathbb{F}_2} \cdots \otimes_{\mathbb{F}_2} H^{\ast} \mathbb{P}^{\infty} = \mathbb{F}_2[a_1, \cdots , a_n]\) by K\"unneth Theorem.

        By exercise, \(\operatorname{w} (\xi) = \operatorname{w} (\pi_1^{\ast} \gamma^1 \oplus \cdots \oplus \pi_n^{\ast} \gamma^1) = \prod_k \operatorname{w} (\pi_k^{\ast} \gamma^1) = (1+a_1) \cdots (1+a_n)\).
        
        Then \(\operatorname{w}_k(\xi) = \sigma_k(a_1, \cdots , a_n)\) the \(k\)'th elementary symmetric function.

        \(\sigma_1, \cdots , \sigma_n\) are algebraically independent [Newton].

        ii follows from this. We have:

        \[
            \begin{tikzcd}
                E(\xi) \ar[r] \ar[d] & E(\gamma^n) \ar[d] \\ \mathbb{P}^{\infty} \times \cdots \mathbb \times {P}^{\infty} \ar[r,"c"] & G_n
            \end{tikzcd}
        \]

        Suppose \(p(\operatorname{w}_1, \cdots , \operatorname{w}_n) = 0\). Apply \(c^{\ast}\) to see \(p(\operatorname{w}_1(\xi), \cdots , \operatorname{w}_n(\xi)) = 0 \implies p = 0\).
    \end{proof}

    Now we finally prove theorem 7.1. We need to prove that the polynomials on SW classes generate the cohomology.

    \begin{proof}
        We have:

        \[
            \mathbb{F}_2[\operatorname{w}_1, \cdots , \operatorname{w}_n] \subset H^{\ast} (G_n)
        \]

        Let \(\mathbb{F}_2[\operatorname{w}_1, \cdots , \operatorname{w}_n]^d\) be the subspace of degree \(d\) polynomials on the \(\operatorname{w}\)'s.

        \[
            \mathbb{F}_2[\operatorname{w}_1, \cdots , \operatorname{w}_n]^d \subset H^d(G_n)
        \]

        \(H^d(G_n)\) is a \textit{subquotient} of \(C^d(G_n)\). Meaning it is quotient of a subgroup / subgroup of a quotient [same thing].

        Note that:

        \[
            \dim_{\mathbb{F}_2} \mathbb{F}_2[\operatorname{w}_1, \cdots , \operatorname{w}_n]^d \leq \dim_{\mathbb{F}_2} H^d(G_n) \leq \dim_{\mathbb{F}_2} C^d(G_n)
        \]

        We will show this is an equality.

        Note that \(\dim_{\mathbb{F}_2} \mathbb{F}_2[\operatorname{w}_1, \cdots , \operatorname{w}_n]^d\) is the number of monomials \(\operatorname{w}_1^{r_1} \cdots \operatorname{w}_n^{r_n}\) of degree \(d\), meaning we need \(r_1 + 2r_2 + \cdots + n r_n = d\).

        \(\dim_{\mathbb{F}_2} C^d(G_n)\) is the number of schubert symbols \(1 \leq \sigma_1 < \sigma_2 < \cdots < \sigma_n\) of dimension \(d\), meaning \(d = \sum_i (\sigma_i - i)\).

        We claim they are in bijection as follows:

        \[
            r_n + 1 < r_n + r_{n-1} + 2 < \cdots < r_n + r_{n-1} + \cdots + r_1 - n
        \]

        Thus all three dimensions are equal. Therefore,

        \[
            \mathbb{F}_2 [\operatorname{w}_1, \cdots , \operatorname{w}_n] = H^{\ast} G_n
        \]

        Furthermore, we can deduce that \(\partial \equiv = 0 \pmod 2\) in \(C^{\ast} G_n\).
    \end{proof}

    \begin{corollary}
        We have a classifying map:

        \[
            H^{\ast} (G_n) \xrightarrow{c^{\ast}} H^{\ast} (\mathbb{P}^{\infty} \times \cdots \times \mathbb{P}^{\infty})
        \]

        \[
            \operatorname{w}_k \mapsto \sigma_k(a_1, \cdots , a_n)
        \]

        Thus, \(H^{\ast} (G^n) \cong H^{\ast} (\mathbb{P}^{\infty} \times \cdots \times \mathbb{P}^{\infty})^{S_n}\)

        \(c^{\ast}\) is injective.
    \end{corollary}

    \begin{theorem}
        [7.3 Uniqueness]

        If \(\operatorname{w} (\eta) = 1 + \operatorname{w}_1(\eta) + \cdots\) and \(\widetilde{\operatorname{w}}(\eta) = 1 + \widetilde{\operatorname{w}}_1(\eta) + \cdots \) satisfying axioms 1-4, then \(\operatorname{w} = \widetilde{\operatorname{w}}\) 
    \end{theorem}

    \begin{proof}
        Step 1: By axiom 4, \(\operatorname{w} (\gamma^1_1) = \widetilde{\operatorname{w}} (\gamma^1_1)\).

        Step 2: we have

        \[
            \begin{tikzcd}
                E(\gamma^1_1) \ar[r] \ar[d] & E(\gamma^1) \ar[d] \\ P^1 \ar[r,"c"] & P^{\infty}
            \end{tikzcd}
        \]

        Recall \(c^{\ast} : H^1 \mathbb{P}^{\infty}  \rightarrowtail H^1 \mathbb{P}^1\) is an injection so \(\operatorname{w}(\gamma^1) = \widetilde{\operatorname{w}}(\gamma^1)\).

        Step 3: Set \(\xi = \gamma^1 \times \cdots \times \gamma^1\). Then \(\operatorname{w} (\xi) = \widetilde{\operatorname{w}}(\xi)\).

        To see this, \(\xi = \pi_1 ^{\ast} \gamma^1 \oplus \cdots \oplus \pi^n_{\ast} \gamma^1\).

        \(\operatorname{w} (\xi) = \prod_i (1 + a_i) = \widetilde{\operatorname{w} } (\xi)\).
        
        Step 4: \(\operatorname{w} (\gamma^n) = \widetilde{\operatorname{w}}(\gamma^n) \).
        
        \[
            \begin{tikzcd}
                E(\xi) \ar[r] \ar[d] & E(\gamma^n) \ar[d] \\ \mathbb{P}^{\infty} \times \cdots \times \mathbb{P} ^{\infty} \ar[r,"c"] & G_n
            \end{tikzcd}
        \]

        \(c^{\ast}\) is injective on \(H^{\ast}\). \(\operatorname{w}(\xi) = \widetilde{\operatorname{w}}(\xi)\) so \(c^{\ast} \operatorname{w} (\xi) = c^{\ast} \widetilde{\operatorname{w}} (\xi) \implies \operatorname{w} (\gamma^n) = \widetilde{w}(\gamma^n)\).

        Step 5: \(\operatorname{w}(\eta) = \widetilde{\operatorname{w}} (\eta)\) when \(B(\eta)\) is CW complex.

        To see this, just check:

        \[
            \begin{tikzcd}
                E(\eta) \ar[r] \ar[d] & E(\gamma^n) \ar[d] \\ B(\eta) \ar[r] & G_n
            \end{tikzcd}
        \]

        Step 6: \(\operatorname{w} (\eta) = \widetilde{\operatorname{w}} (\eta)\) for alll \(\eta\).

        Take CW approximation:

        \[
            \begin{tikzcd}
                E \ar[r] \ar[d] & E(\eta) \ar[d] \\ B \ar[r] & B(\eta)
            \end{tikzcd}
        \]

        \(\operatorname{w} (E) = \widetilde{\operatorname{w}} (E)\) so \(\widetilde{\operatorname{w}} (\eta) = \widetilde{\operatorname{w}} (\eta)\). 
    \end{proof}

    \section*{Friday, 10/17/2025}
    
    \subsection*{Existence of SW Classes following Thom}

    Uses two things: Thom isomorphism theorem and Steenrod squares.

    \(\mathbb{F}_2\)-coefficients understood.

    Consider a rank \(n\) vector bundle \(\begin{tikzcd}
        \mathbb{R}^n \ar[r] & E \ar[d,"z"] \\ & B
    \end{tikzcd}\).

    Then we have \(\begin{tikzcd}
        \mathbb{R}^n - 0 \ar[r] & E_0 \ar[d] \ar[r,"=", phantom] & E - z(B) \\ & B
    \end{tikzcd}\) 

    \(z(B)\) zero section.

    \(b\in B, F_b = \pi ^{-1} b, F_{b_0} = \pi ^{-1} b - \{ 0 \}\).

    \begin{remark}
        \(H^{\ast} (F_b, F_{b_0}) \cong H^{\ast} (\mathbb{R}^n, \mathbb{R}^n - 0) \cong H^{\ast} (D^n, S^{n-1}) \cong \widetilde{H}^{\ast} (D^n / S^{n-1}) = \begin{dcases}
            \mathbb{F}_2, &\text{ if } \ast = n ;\\
            0, &\text{ otherwise} .
        \end{dcases} \) 
    \end{remark}

    \begin{theorem}
        [8.1, Thom] \(\exists ! u\in H^n(E,E_0)\) such that \(\forall b\in B\), 
        
        \(i_b^{\ast} u \neq 0 \in H^n(F_b, F_{b_0}) = \mathbb{F}_2\).

        \(\forall k\in \mathbb{Z}, H^k E \xrightarrow{\approx} H^{k+n} (E,E_0), x \mapsto x \cup u\) is an isomorphism.
    \end{theorem}

    `Every bundle behaves like the trivial bundle'.

    \begin{corollary}
        \(H^i(E,E_0)=0\) for \(i < n\).
    \end{corollary}

    \begin{definition}
        \(u \in H^n(E,E_0)\) Thom class \(u = u_E\).
    \end{definition}

    \begin{theorem}
        [Thom Isomorphism]

        We have the following isomorphism:

        \[
            \phi : H^k B \to H^{k+n} (E,E_0)
        \]

        \[
            \phi(X) = \pi^{\ast} x \cup u
        \]
    \end{theorem}

    \begin{exercise}
        Prove 8.1 for trivial bundle. [Use K\"unneth theorem]
    \end{exercise}

    What is \(\langle u, \text{relative cycle} \rangle\)? This is inner product \(H^k(E,E_0) \otimes H_k(E,E_0) \to \mathbb{F}_2\). It `counts' the number of intersections with the zero sections.

    \begin{figure}[H]
        \centering
        \includegraphics[width=0.8\textwidth]{img/zerosection}
    \end{figure}    

    \subsection*{Steenrod Squares (Generalizes Cup Products)}

    Axioms:

    \begin{enumerate}[label=\arabic*)]
        \item \(\operatorname{Sq}^i: H^n(X,Y) \to H^{n+i}(X,Y) \) homology of abelian groups \(\forall n, i \geq 0\).
        \item `naturality' \(f: (X,Y) \to (X^{\prime} , Y^{\prime})\) then \(\operatorname{Sq}^i \circ f^{\ast} = f^{\ast} \operatorname{Sq}^i\).
        \item \(a\in H^n(X,Y)\).
        
        \begin{align*}
            \operatorname{Sq}^0 a &= a \\
            \operatorname{Sq}^n a &= a \cup a\\
            \operatorname{Sq}^i a &= 0 \text{ when } i > n
        \end{align*}

        \item Cartan formula
        
        \[
            \operatorname{Sq}^k(a\cup b) = \sum_{i+j = n} \operatorname{Sq}^i a \cup \operatorname{Sq}^j b
        \]
    \end{enumerate} 

    These axioms look like the axioms of SW classes.

    \begin{definition}
        [SW Classes, Thom]

        Let \(\phi\) be the Thom isomorphism. Then,

        \[
            \operatorname{w}_i(\xi) = \phi ^{-1} \operatorname{Sq}^i \phi (1) = \phi ^{-1} \operatorname{Sq}^i u
        \]
    \end{definition}

    So, when \(n\) is the rank of the bundle,

    \[
        \begin{tikzcd}
            u \ar[rrr, mapsto] & & & \operatorname{Sq}^i u \ar[ddd, mapsto, "\phi^{-1}"] \\ & H^n(E,E_0) \ar[r,"\operatorname{Sq}^i"] & H^{n+i} (E,E_0) \\ & H^0 B \ar[u] \ar[r] & H^i(B) \ar[u] \\ 1 \ar[uuu, mapsto] \ar[rrr] & & & \operatorname{w}_i(\xi)
        \end{tikzcd}
    \]

    Goal: SW classes satisfy axioms.

    Total Steenrod square: \(\operatorname{Sq}(a) = a + \operatorname{Sq}^1 a + \operatorname{Sq}^2 a + \cdots + \operatorname{Sq}^n a, a \in H^n(X,Y)\).

    Then \(\operatorname{Sq}: H^{\ast} (X,Y) \to H^{\ast} (X,Y), \operatorname{Sq} = 1 + \operatorname{Sq}^1 + \operatorname{Sq}^2 + \cdots\).

    Cartan: \(\operatorname{Sq}(a\cup b) = \operatorname{Sq}(a) \cup \operatorname{Sq}(b)\). 

    Axioms for SW classes:

    Axiom 1: \(\operatorname{w}_0 \xi = 1, \operatorname{w}_i \xi = 0\) for \(i > \operatorname{rank} \xi\) follows from 3.

    Axiom 2: Naturality:

    \[
        \begin{tikzcd}
            E \ar[r,"f"] \ar[d] & E^{\prime} \ar[d] \\ B \ar[r,"\overline{f}"] & B^{\prime}
        \end{tikzcd}
    \]

    \(f: (E,E_0) \to (E^{\prime} , E_0^{\prime})\).

    Thom class is natural [meaning \(f^{\ast} u_{E^{\prime}} = u_E\) since \(f\) is isomorphism on fibers].

    Thom isomorphism is natural: \(f^{\ast} \circ \phi_{E^{\prime}} = \phi_E \circ \overline{f}^{\ast}\).
    
    Thus, \(\overline{f}^{\ast} \operatorname{w}_i(\xi^{\prime}) = \overline{f}^{\ast} \phi ^{-1} \operatorname{Sq}^i \phi(u_{E^{\prime}}) = \phi_E ^{-1} f^{\ast} \operatorname{Sq}^i \phi_{E^{\prime}}(u_{E^{\prime}}) =\) [some calculations] \(=\operatorname{w}_i(\overline{f}^{\ast} \xi ^{\prime} )\).

    \section*{Monday, 10/20/2025}
    
    Review: \(\mathbb{F}_2\)-coefficients understood. We have vector bundle \(\xi: \mathbb{R}^n \to E \xrightarrow{\pi} B\). We defined \(E_0 = E - z(B)\), the complement of the zero section. We defined the Thom class \(u = u_E \in H^n(E,E_0)\) so that \(i_b^{\ast} u \neq 0 \in H^n(F_b , F_{b_0})\) for all \(b\in B\).
    
    Thom isomorphism theorem: \(\phi_E = \phi : H^{\ast} B \to H^{\ast + n} (E,E_0)\) given by \(\phi(x) = (\pi^{\ast} X) \cup u_E\) is an isomorphism.

    Then we can define SW class of a bundle: \(\operatorname{w} _i \xi = \phi ^{-1} \operatorname{Sq}^i u\).

    Recall that \(\operatorname{Sq}^i: H^{\ast} (E,E_0) \to H^{\ast + i} (E, E_0)\).

    We also have a total version: \(\operatorname{w} (\xi) = \phi ^{-1} \operatorname{Sq} u_E\) where \(\operatorname{Sq} = 1 + \operatorname{Sq}^1 + \operatorname{Sq}^2 + \cdots\)

    \begin{lemma}
        \(\operatorname{w} (\xi \oplus \xi^{\prime}) = \operatorname{w} (\xi) \cup \operatorname{w} (\xi^{\prime})\).
    \end{lemma}

    ALso recall we have the cross product: \(H^i X \otimes H^j Y \to H^{i+j} (X \times Y)\) by \(a \otimes b \mapsto a \times b\).

    This comes from: if we have an \(n\)-simplex on \(X \times Y\) given by \(\sigma : \Delta^n \to X \times Y\), then \((a \times b)(\sigma) = a(_i(p_X \circ \sigma)) b((p_Y \circ \sigma)_j)\) where we have the front \(i\) and back \(j\) face maps and \(p_X , p_Y\) are projections.

    Then, \(a \times b = (p_X ^{\ast} a) \cup (p_Y^{\ast} b)\) and \(a\cup b = \Delta ^{\ast} (a \times b)\).
    
    Now, suppose we have two bundles \(\xi: \mathbb{R}^n \to E \to B\) and \(\xi^{\prime}: \mathbb{R}^{n^{\prime}} \to E^{\prime} \to B^{\prime}\).
    
    Then we can have the cross version of the lemma:

    \begin{lemma}[X-lemma]
        \(\operatorname{w} (\xi) \times \operatorname{w} (\xi^{\prime}) = \operatorname{w} (\xi \times \xi^{\prime})\).
    \end{lemma}

    Claim: X-lemma implies the lemma.

    \begin{proof}
        \(\operatorname{w} (\xi \oplus \xi^{\prime}) = \operatorname{w} (\Delta^{\ast} (\xi \times \xi^{\prime})) = \Delta^{\ast} \operatorname{w} (\xi \times \xi^{\prime}) = \Delta^{\ast} (\operatorname{w} (\xi) \times \operatorname{w} (\xi^{\prime})) = w(\xi) \cup w(\xi^{\prime})\).
    \end{proof}

    Now we prove the X-lemma.

    \begin{proof}
        \(\operatorname{w} (\xi \times \xi^{\prime}) = \phi_{E \times E^{\prime}}^{-1} (\operatorname{Sq}(u_{E \times E^{\prime}})) = \phi_{E \times E^{\prime}}^{-1} (\operatorname{Sq}(u_E \times u_{E^{\prime}}))\).
        
        Cartan \(\implies \operatorname{Sq} (a \cup b) = \operatorname{Sq} \cup \operatorname{Sq} b\), applying \(\Delta^{\ast}\) we see that \(\operatorname{Sq} (a \times b)n = \operatorname{Sq} \times \operatorname{Sq} b\).

        Thus, \(= \Phi^{-1} _{E \times E^{\prime}} (\operatorname{Sq} u_E \times \operatorname{Sq} u_{E^{\prime}}) = (\phi_E \times \phi_{E^{\prime}})^{-1} (\operatorname{Sq} u_E) \times (\operatorname{Sq} u_{E^{\prime}})\).
        
        \(= \operatorname{w} (\xi) \times \operatorname{w} (\xi^{\prime})\).
    \end{proof}

    Recall Axiom 4: \(\operatorname{w}_1(\gamma^1_1) \neq 0\). We want to prove that.

    \begin{proof}
        Let \(M\) be the M\"obius strip. Then we have \((E,E_0)\). We also have \((M, \partial M)\). We can collapse the boundary of the m\"obius strip to a point which gives us \(\mathbb{P}^2\). i.e. we have:
        
        \[
            H^{\ast} (E, E_0) \xrightarrow[\text{htpy invariance}]{\approx} H^{\ast} (M, \partial M) \xleftarrow[\text{good pair}]{\approx} H^{\ast} (M / \partial M, \ast) \cong H^{\ast} (\mathbb{P}^2, \ast)
        \]

        Recall \(E = E(\gamma^1_1) \subset \mathbb{P}^2 \times \mathbb{R}^3, [-1,1] \times \mathbb{R} / \sim , (x,t) \sim (-x,-t)\).

        \(u_E \neq 0\) by definition and \(H^1(E,E_0) \cong H^1(P^2), u \leftrightarrow a\).
        
        Then, \(\operatorname{Sq}^1 a = a \cup a \neq 0 \implies \operatorname{Sq}^1 u \neq 0\).

        Thus, \(\operatorname{w}_1(\gamma^1_1) = \phi ^{-1} (\operatorname{Sq}^1 u_E) \neq 0\). 

    \end{proof}

    \subsection*{Chapter 9}

    For this chapter, \(\mathbb{Z}\)-coefficients understood.

    We want to talk about orientation. Let \(V\) be a \(\dim n\) vector space. Let \(V_0 = V - \{ 0 \}\).

    \begin{definition}
        An orientation for \(V\) is a generator \(\mu_V \in H_n(V, V_0)\).
    \end{definition}

    This corresponds to the linear algebra definition for \(V\).
    
    Orientation of \(V\) corresponds to \(\frac{\text{ordered bases \((b_1, ..., b_n)\) for \(V\)}}{(b_1, \cdots, b_n) \sim (b_1^{\prime}, \cdots, b_n^{\prime})\text{ if determinant of change of basis matrix is positive}}\).
    
    Then, the class of \([b_1 , \cdots , b_n]\) maps to the orientation in homology given by \(\sigma : \Delta^n \to V\) where \(\sigma(t_0 , \cdots , t_n) = \sum_{i=1}^n (t_i - t_{i-1}) b_i\).
    
    Now suppose \(\xi : \mathbb{R}^n \to E \to B\) is a vector bundle.

    \begin{definition}
        Orientation for \(\xi\) is an assignment \(b \mapsto \mu_{F_b} \in H_n(F_b , F_{b_0} ; \mathbb{Z})\) that is `continuous in \(b\)'. Meaning, \(\forall b \in B\), \(\exists (U, h)\) where \(b\in U\) and,
        
        \[
            \pi ^{-1} U \xrightarrow{h} U \times \mathbb{R}^n
        \]

        \(\forall x\in U, F_x \to \{ x \} \times \mathbb{R}^n \cong \mathbb{R}^n\) is o.p.
    \end{definition}

    If there exists such an orientation we call \(\xi\) is orientizable.

    \begin{theorem}
        [Thom Isom, 9.1] Let \(\xi: \mathbb{R}^n \to E \to B\) be an oriented vector bundle.

        \begin{enumerate}[label=\roman*)]
            \item \(\exists ! u = u_E \in H^n(E,E_0)\) such that \(\forall b, i^{\ast} _b u \in H^n(F_b , F_{b_0}) \cong \mathbb{Z}\) is a generator. We call \(u\) the Thom class.
            \item \(\phi = \phi_E : H^{\ast} B \xrightarrow{\approx} H^{\ast + n} (E, E_0)\) given by \(\phi(x) = \pi ^{\ast} x \cup u\), this is the THom isomorphism.  
        \end{enumerate} 
    \end{theorem}

    \begin{corollary}
        \(H^k(E,E_0) = 0\) for \(k < n\).

        \(H^n(E, E_0) \cong \mathbb{Z}\) if \(B\) is path connected.
        
        e.g. \(\gamma^1_1\) is path connected.

        \(H^1(E,E_0 ; \mathbb{Z}) = H^1(\mathbb{P}^2 ; \mathbb{Z}) = 0\) 
    \end{corollary}

    \section*{Wednesday, 10/22/2025}
    
    Let \(\xi : \mathbb{R}^n \to E \to B\). Recall that an orientation on \(\xi\) is a `continuous assignment of a point' \(b \mapsto \mu_{F_b} \in H_n(F_b , F_{b_0} ; \mathbb{Z})\).

    Equivalently, continuous assignment of \([b_1 , \cdots , b_n]\) an equivalence class of ordered basis of \(F_b\).
    
    \[
        \begin{tikzcd}
            M^n \text{ manifold} & \text{local homology} \\
            \text{cont } x \ar[r,mapsto] & \mu_x \in H_n(M, M-x) \cong \mathbb{Z}
        \end{tikzcd}
    \]

    Puzzle: \(M^n\) is smooth, orientable on \(M \leftrightarrow\) orientation on \(TM\) how?
    
    Note that there is \(\exp_x : T_x M \to M\) which is a diffeomorphism near \(x\). Patch them up with orientation preserving atlas on \(M\). Meaning, \((M,\mathcal{A})\) where transition maps \(\Phi_\beta \circ \Phi_\alpha ^{-1}\) are oritentation preserving, meaning their determinant is positive.

    \[
        \begin{tikzcd}
            \text{orientation on } M \ar[rr] \ar[rd] & & \text{orientation on } TM \ar[ll] \ar[ld] \\ & \text{o.p. atlas on } M \ar[ur] \ar[ul]
        \end{tikzcd}
    \]

    Exercise 12A: \(\operatorname{w}_1(\xi) =  0 \iff \xi\) orientable.
    
    \begin{theorem}
        \(\xi\) orientable \(\iff \operatorname{w}_1(\xi) \in H^1(B;\mathbb{F}_2)\) is \(0\).
    \end{theorem}

    Note that, \(\forall n, \exists\) two \(n\)-plane bundles over \(S^1\) given by \(\epsilon^n\) and \(\gamma^1_1 \oplus \epsilon^{n-1}\).

    \[
        \begin{tikzcd}[column sep = large]
            \text{bundles over } S^1 \ar[r, "\text{clutching}"] & \pi_0(\text{GL}_n(\mathbb{R})) \xrightarrow[\det]{\approx} \{ \pm 1 \}
        \end{tikzcd}
    \]

    Bundles over \(I\) are trivial.

    \(\xi : \mathbb{R}^n \to E \to B\) homomorphism [orientation character] \(\widetilde{\operatorname{w}}: \pi_1 B \to \{ \pm 1 \}\).

    \[
        \begin{tikzcd}
            \epsilon^n \ar[r] \ar[d] & E(\gamma) \ar[d] \\ I \ar[r,"\exp"] & S^1
        \end{tikzcd}
    \]

    \(E(\gamma) \cong \frac{I \times \mathbb{R}^n}{(0,v) \sim (1,Av)}\) where \(A\in \text{GL}_n(\mathbb{R})\).

    \[
        \widetilde{\operatorname{w}} [\gamma] = \begin{dcases}
            +1, &\text{ if } \gamma ^{\ast} \epsilon \text{ trivial}  ;\\
            -1, &\text{ if } \gamma^{\ast} \epsilon \text{ non-trivial}.
        \end{dcases}
    \]

    Essentially, given a loop we walk around it to see if my right hand becomes my left hand.

    By UCT and Hurewicz theorem,

    \[
        H^1(B;\mathbb{F}_2) \cong \operatorname{Hom} (H_1 B, \mathbb{F}_2) \cong \operatorname{Hom} (\pi_1 B, \{ \pm 1 \})
    \]

    \[
        \operatorname{w}_1(\xi) \longleftrightarrow \widetilde{\operatorname{w}} 
    \]

    They correspond for \(\gamma^1_1\) so they correspond for \(\gamma\).

    \(\mathbb{P}^{\infty} = G_1(\mathbb{R}^{\infty}) \to G_n(\mathbb{R}^{\infty})\) is isomorphism on \(\pi_1\). Meaning,

    \[
        \pi_1 \mathbb{P}^{\infty} \to \pi_1 G_n (\mathbb{R}^{\infty})
    \]

    by cellular approximation [they have the same \(1\)-skeleton and thus \(1\)-cells. Recall the \(1\)-skeleton contains some Schubert cells with dimension \(1\). So any path on \(G_n(\mathbb{R}^{\infty})\) is homotopic to one in \(G_1(\mathbb{R}^{\infty})\)]. It is \(1-1\) because of \(\operatorname{w}_1\).

    Therefore, they correspond for \(\gamma^n\). Thus, \(\operatorname{w} \rightsquigarrow \widetilde{\operatorname{w}}\) for general \(\xi\).

    Milnor-Stasheff uses oriented grasmanian \(\widetilde{G}_n(\mathbb{R}^{\infty})\) to show that \(H^1(\widetilde{G}_n(\mathbb{R}^{\infty});\mathbb{F}_2) = 0\). 

    \begin{theorem}
        [Thom Isomorphism Theorem]

        \(\xi : \mathbb{R}^n \to E \to B\). We have a \(\mathbb{Z}\)-coefficient version and a \(\mathbb{F}_2\)-coefficient version. In other words, we have a general manifold version and an oriented manifold version.

        We can then state the theorem in a fancier way:

        \(H^{\ast} (E,E_0)\) is a free rank \(1\) module over \(H^{\ast} B\) with a generator [the thom class] in \(\deg n\). This works for both \(\mathbb{Z}\) and \(\mathbb{F}_2\) coefficients.

        The module action is given by the cup product. For \(x\in H^{\ast} B\) and \(a\in H^{\ast} (E, E_0)\), we can first take the pullback \(\pi ^{\ast} x\) of \(x\) into \(H^{\ast} E\). Then,

        \(x \cdot a \coloneqq \underbrace{(\pi ^{\ast} x)}_{H^{\ast} E} \cup \underbrace{a}_{H^{\ast} (E,E_0)}\).
        
        Then \(H^{\ast} B \cong H^{\ast + n} (E,E_0) = H^{\ast} B \cup u_E\) 
    \end{theorem}

    \begin{proof}
        [Proof 1]

        We use the Serre Spectral Sequence. We look at the relative fibration:

        \[
            \begin{tikzcd}
                (F,F_0) \ar[r] & (E,E_0) \ar[d] \\ & B
            \end{tikzcd}
        \]

        We then have the machine that computes the cohomology of the total space in terms of the cohomology of the base with coefficients in the fiber:

        \[
            E^2 = H^p(B;H^q(F,F_0)) \implies H^{p+q}(E,E_0)
        \]

        \begin{table}[H]
            \centering
            \begin{tabular}{c||ccccc}
                \toprule
                     &  &  & \(0\)  &  &  \\
                \midrule
                    \(n\)  &  &  & \(H^{\ast}(F,F_0)\)  &  &  \\
                \midrule
                     &  &  &  &  &  \\
                     &  &  & \(0\)  &  &  \\
                     &  &  &  &  &  \\
                \bottomrule
            \end{tabular}
        \end{table}

        [Take M622 for more information].
    \end{proof}

    \section*{Friday, 10/24/2025}
    
    \(\xi : \mathbb{R}^n \to E \to B\)

    \begin{theorem}
        [Thom Isomorphism] Here \(\mathbb{Z}\)-coefficient if oriented, else \(\mathbb{F}_2\).

        Then \(H^{\ast} (E,E_0)\) is free rank \(1\) module over \(H^{\ast} B\). Generator \(u_E \in H^n(E,E_0)\).
    \end{theorem}

    \begin{theorem}
        [Thom Isomorphism]

        \(\phi : H^{\ast} B \xrightarrow{\approx} H^{\ast + n} (E,E_0)\), \(\phi (y) = \pi ^{\ast} y \cup u_E\).
    \end{theorem}

    \begin{theorem}
        [Thom Isomorphism for Homology]

        \[
            H_{\ast} B \xleftarrow{\approx} H_{\ast + n} (E,E_0)
        \] 

        It is given by cap product with the Thom class.

    \end{theorem}

    We did first proof via spectral sequences.

    Second proof: Mayer-Vietoris.

    \begin{proof}
        Case 1: Trivial bundle.

        Here \((E, E_0) = B \times (\mathbb{R}^n , \mathbb{R}^n_0)\). In \((E,E_0)\) we have \(H^{\ast} = H^{\ast} B \otimes H^{\ast} (\mathbb{R}^n, \mathbb{R}^n_0)\) is free rank \(1\) by K\"unneth theorem.

        Case 2: \(B = B^{\prime} \cup B^{\prime\prime}\) open cover. Assume Thom Isomorphism Theorem holds for \(\eval{\xi}_{B^{\prime}},\eval{\xi}_{B^{\prime\prime}} \text{ and } \eval{\xi}_{B^{\prime} \cap B^{\prime\prime}}\).
        
        Write \(B^\cap \coloneqq B^{\prime} \cap B^{\prime\prime}\). Let \(E^\cap = \pi ^{-1} (B^\cap)\) and \(E_0^\cap = \pi_0 ^{-1} (B^\cap)\).
        
        Question: why is this a thom class?

        We have the relative Mayer-Vietoris exact sequence:

        \[
            \begin{tikzcd}[column sep = small]
                \cdots \ar[r] & H^n(E,E_0) \ar[r] & H^n(E^{\prime}, E^{\prime}_0) \oplus H^n(E^{\prime\prime}_0 ,  E^{\prime\prime} _0) \ar[r] & H^n(E^\cap , E_0^\cap) \ar[r] & \cdots \\
                & u \ar[r, mapsto] & (u^{\prime}, u^{\prime\prime}) \ar[r, mapsto] & 0
            \end{tikzcd}
        \]

        Thus we must have \(u^{\prime} \mapsto u^\cap \mapsfrom u^{\prime\prime}\).
        
        Now we can use a \(5\)-lemma argument:
        
        \[
            \begin{tikzcd}
                H^i B \ar[r] \ar[d] & H^i B^{\prime} \oplus H^i B^{\prime\prime} \ar[r] \ar[d, "\phi", "\cong"'] & H^i B^\cap \ar[d, "\phi", "\cong"'] \\
                H^{i+n} (E,E_0) \ar[r] & H^{i+n} (E^{\prime} , E^{\prime}_0) \oplus H^{i+n} (E^{\prime\prime}, E^{\prime\prime}_0) \ar[r] & H^{i+n} (E^\cap, E_0^\cap)
            \end{tikzcd}
        \]

        So, \(H^i B \xrightarrow{\phi, \cong} H^{i+n}(E,E_0)\).

        Case 3: Finite cover \(B = B_1 \cup \cdots \cup B_k\) such that \(\eval{\xi}_{B_i}\) is trivial for \(\forall i\).
        
        Use induction and Case 2: \((B_1 \cup \cdots \cup B_{k-1})\cup B_k\).
        
        Thus Thom isomorphism holds if \(B\) is compact.

        Case 4: General case. Then use limits. Too hard.

    \end{proof}

    \begin{proof}
        Third proof. Assume \(\xi\) is smooth \(n\)-bundle, \(B\) is a \(k\)-dimensional smooth closed manifold. 

        We can give \(\xi\) a metric \(\lVert \cdot \rVert : E \to \mathbb{R}\).

        Disk bundle \(D(E) = \{ e\in E \mid \lVert e \rVert \leq 1 \}\).

        \(S(E) = \{ e\in E \mid \lVert e \rVert = 1 \}\).

        Then \((D(E), S(E)) \to (E,E_0)\) gives isomorphism \(H_{\ast}\) and \(H^{\ast}\)
        
        \begin{figure}[H]
            \centering
            \includegraphics[width=0.8\textwidth]{img/DESE_example}
        \end{figure}

        \(D(E)\) is a compact manifold with \(\partial D(E) = S(E)\).

        Let \(PD_B \text{ and } PD_{DE}\) be Poincar\'e (Lefschetz) duality isomorphims. Define thom class to P-L dual of zero section.
        
        \[
            u_E \coloneqq PD_{DE} (z_{\ast} [B]) \in H^n (DE, SE)
        \]

        Here \(z\) is the zero section.

        \(\phi(y) = \pi^{\ast} y \cap u_E \overset{\text{claim}}{=} PD_{DE} (\operatorname{inc}_{\ast} PD_B y)\).

        For the claim see Bredon's topology and geometry book page 369.

    \end{proof}

    \section*{Monday, 10/27/2025}
    
    \subsection*{Chapter 9}

    Consider an oriented vector bundle \(\xi : \begin{tikzcd} \mathbb{R}^n \ar[r] & E \ar[d,"\pi"] \\ & B \end{tikzcd}\).

    \begin{definition}
        \textit{Euler Class} \(e(\xi) \in H^n(B;\mathbb{Z})\) is the image of the Thom class:

        \[
            \begin{tikzcd}
                H^n(E,E_0) \ar[r] & H^n E & H^n B \ar[l,"\pi^{\ast}"',"\cong"] \\ u \ar[u,phantom,"\in", sloped] & & e(\xi) \ar[u,phantom,"\in", sloped]
            \end{tikzcd}
        \]
    \end{definition}

    Three uses:

    \begin{proposition}
        [11.12] \(M^n\) closed, oriented manifold then,

        \[
            \langle e(TM), [M] \rangle = \chi(M)
        \]

        Where \(\chi\) is the Euler characteristic. 
    \end{proposition}

    \begin{proposition}
        Euler class is the first obstruction to the existence of a nowhere zero section.

        Thus, \(\dim B < n \implies \xi\) has a nowhere zero section.

        \(\dim B = n, e(\xi) = 0 \implies \xi\) has a nowhere zero section.

        Thus, \(M^n\) closed, oriented, \(\chi(M) = 0 \implies \exists\) nowhere zero vector field.
    \end{proposition}

    If \(X^n \subset M^{2n}\) closed, oriented then,

    \[
        \langle e(\nu(X \hookrightarrow M)), i_{\ast} [X] \rangle = \text{self intersection \(\#\) of } X
    \]

    \[
        = X \cdot X = \langle PD_M [X], [X] \rangle 
    \]

    For example, \(\langle e(\mathbb{C} P^n \hookrightarrow \mathbb{C} P^2), [\mathbb{C} P^1] \rangle = 1\) 

    Non-oriented clase:

    \(\langle e(\nu (X \hookrightarrow M)), i_{\ast}[X] \rangle = X \cdot X \mod 2\).

    Example: consider \(M = \mathbb{R} P^2\).
    
    \begin{figure}[H]
        \centering
        \includegraphics[width=0.5\textwidth]{img/nonorientedcase}
    \end{figure}

    [We pertrube a bit since considering the intersection doesn't really make sense]
    
    Then \(e(\xi) \mod 2 = \operatorname{w}_n(\xi)\).

    Note that \(X \cdot X \mod  2 = X \cdot X^{\prime} \mod 2\).

    \subsection*{Basic Properties, Milnor-Stasheff 9.2}

    \begin{enumerate}[label=\roman*)]
        \item 9.2 \(e(\xi)\) is natural. i.e. It is a characteristic class. If \(f: \xi^{\prime} \to \xi\) is a bundle map [meaning there is an isomorphism on the fibers]
        
        \[
            \begin{tikzcd}
                E^{\prime} \ar[r,"\widetilde{f}"] \ar[d] & E \ar[d] \\ B^{\prime} \ar[r,"\overline{f}"] & B
            \end{tikzcd}
        \]

        Then \(e(f^{\ast} \xi) = f^{\ast} e(\xi)\).

        \item 9.3 \(\overline{\xi}\) reversing orientation on \(\xi\) gives us \(e(\overline{\xi}) = - e(\xi)\).
        \item 9.4 \(n\) odd \(\implies 2 e(\xi) = 0, \xi \cong \overline{\xi}\) [oriented vector bundle]. \(v \mapsto -v\), then \(e(\xi) \underset{9.2}{=} e(\overline{\xi }) = -e(\xi)\).  
    \end{enumerate} 

    If \(M^n\) is closed and oriented, then \(\xi(M^n) = 0, e(TM) = 0\).
    
    \(\chi(\mathbb{R}P^2) = 1, e(T \mathbb{R} P^2) \neq 0\) 

    So, if \(H^n B \) is torsion free and \(n\) is odd, then \(e(\xi) = 0\).

    If \(e(\xi) \neq 0, n\) odd then \(e(\xi) \in H^n(B)\) has order \(2\). Thus there exists a nontrivial torsion summand of \(H^n B\).

    Question: does there exists unique oriented \(\xi: \mathbb{R}^n \to E \to B\) with \(n\) odd so that \(e(\xi) \neq 0\)?

    \begin{proposition}
        9.4.\(\frac{1}{2}\): \(e(\xi) = \phi ^{-1} (u \cup u)\).
    \end{proposition}

    \begin{proof}
        \(\phi(e(\xi)) = \pi^{\ast} e(\xi) \cup u = \eval{u}_{E} \cup u = u \cup u\).

        \[
            \begin{tikzcd}[row sep = small]
                u \quad u & u \cup u \\
                H^n(E,E_0) \otimes H^n(E, E_0; \mathbb{F}_2) \ar[d] \ar[r] & H^{2n}(E,E_0; \mathbb{F}_2) \\ H^n E \otimes H^n(E,E_0; \mathbb{F}_2) \ar[ru] \\ \eval{u}_{E} \quad u
            \end{tikzcd}
        \]
    \end{proof}

    \begin{proposition}
        \(H^n(B;\mathbb{Z}) \to H^n(B,\mathbb{F}_2)\) has \(e(\xi) \mapsto \operatorname{w}_n(\xi)\).
    \end{proposition}

    \begin{proof}
        \(e(\xi) \mapsto \phi ^{-1} (u \cup u) = \phi ^{-1} (\operatorname{Sq}^n u) = \operatorname{w}_n(\xi)\) 
    \end{proof}

    \begin{proposition}
        [9.6]

        \begin{enumerate}[label=\alph*)]
            \item \(e(\xi \times \xi^{\prime})= e(\xi) \times e(\xi^{\prime})\).
            \item \(e(\xi \oplus \xi^{\prime}) = e(\xi) \cup e(\xi^{\prime})\).
        \end{enumerate} 
    \end{proposition}

    \begin{proof}
        \begin{enumerate}[label=\alph*)]
            \item Follows from \(u_{E \times E^{\prime}} = u_E \times u_{E^{\prime} }\).
            \item Apply \(\Delta^{\ast}\) to a.  
        \end{enumerate} 
    \end{proof}

    \begin{proposition}
        [9.7] If \(\xi\) has a nowhere zero section then \(e(\xi) = 0\).
    \end{proposition}

    \begin{proof}
        If \(B\) is paracompact we can choose a metric. Then, \(\xi = \epsilon^1 \oplus (\epsilon^1)^\perp \to e(\xi) = 0 \cup e((\epsilon^1)^\perp) = 0\).
        
        We use CW approximation for general case.
    \end{proof}

    In general, \(e(\xi \oplus \epsilon^1) = 0\). Thus, the Euler class is not stable, in contrast to the Stiefel-Whitney classes, where they are stable w.r.t.\ `adding' trivial bundles.
    
    \section*{Wednesday, 10/29/2025}
    
    \subsection*{Crash Course in Intersection Theory}

    \begin{itemize}
        \item Transversality
        \item Isotopy invariance
        \item Intersection numbers
        \item Thom transversality theorem
        \item Tubular neighborhood theorem
        \item Explicit PD
        \item Alg Int \(\# =\) Gem Int \(\# \).  
    \end{itemize} 

    \subsection*{Transversality}

    Consider submanifolds \(A, B\) of \(M\).

    \begin{definition}
        \(A \pitchfork B\) [\(A\) and \(B\) intersect Transfversly] means \(\forall x\in A \cap B\), \(T_x A + T_x B = T_x M\). 
    \end{definition}

    \begin{figure}[H]
        \centering
        \includegraphics[width=0.8\textwidth]{img/transverse_example}
        \caption{Transverse}
    \end{figure}

    \begin{figure}[H]
        \centering
        \includegraphics[width=0.8\textwidth]{img/not_transverse}
        \caption{Not transverse}
    \end{figure}

    \begin{theorem}
        \(A \pitchfork B\). Then,

        \begin{itemize}
            \item \(A \cap B\) is a manifold.
            \item \(\nu (A \cap B \hookrightarrow A) \cong \eval{\nu(B \hookrightarrow M)}_{A\cap B}\). 
        \end{itemize}
        
        Furthermore, \(\dim A - \dim A \cap B = \dim M - \dim B\).
    \end{theorem}

    Recall that \(\nu(B \hookrightarrow M) = (TB)^\perp \subset \eval{TM}_{B}\).

    \(\nu(B \hookrightarrow M) = \frac{\eval{TM}_{B}}{TB}\).

    \begin{theorem}
        All submanifolds \(A, B\) where \(A\) is isotopic to \(A^{\prime}, A^{\prime} \pitchfork B\).

        Slogan: ``Transversality is generic''. i.e. it is a dense open condition.
    \end{theorem}

    We can pertrube \(A\) to make it transverse.

    Recall isotopy means homotopy through embeddings.

    \subsection*{Intersection Numbers}

    Assume now that \(A ^n \pitchfork B^k \subset M^{n+k}\).

    This implies that \(T_x A \oplus T_x B = T_x M\). Assume further that \(\vert A\cap B \vert < \infty\). e.g. \(M\) is compact.

    Then we can define the \(\mod 2\) interserction number: \(\vert A\cap B \vert \mod 2\).
    
    Now assume \(A, B, M\) are all oriented.

    For \(x\in A \cap B\) we can define:

    \[
        \epsilon_x = \begin{dcases}
            +1, &\text{ if orientation of \(T_x A \oplus T_x B\) and \(T_x M\) match} ;\\
            -1, &\text{ otherwise}.
        \end{dcases}
    \]

    \[
        A \cdot B = \sum_{x\in A \cap B} \epsilon_x\;.
    \]

    \begin{figure}[H]
        \centering
        \includegraphics[width=0.8\textwidth]{img/intersectionmod2}
    \end{figure}

    There \(M = T^2, A \cdot B = 1 - 1 + 1\).

    \begin{theorem}
        \(A, B, M\) are closed then \(A \cdot B\) is isotopy invariant.
    \end{theorem}

    \begin{proof}
        [First Proof] `Geometric'
    \end{proof}

    \begin{proof}
        [Second Proof] `Homological'.

        \[
            A \cdot B = \langle PD_M [A] \cup PD_M[B], [M] \rangle \in \mathbb{Z}
        \]
    \end{proof}

    Observe that \(A\) not transverse to \(B\) can derive that \(A \cdot B \coloneqq A^{\prime} \cdot B^{\prime}\).
    
    Consider \(M = \mathbb{R}^2, A = S^1\) and \(B = I\). Then, \(A \cdot B\) is not isotopy invariant.

    \begin{figure}[H]
        \centering
        \includegraphics[width=0.8\textwidth]{img/nonisotopyinvariant}
        \caption{\(A \cdot B\) is not isotopy invariant in this case}
    \end{figure}

    Suppose \(\partial M \neq \varnothing\), submanfiold \(A\) of \(F\) is called \textit{proper} if \(\partial A = A \cap \partial M\).

    \begin{theorem}
        If \(A^n, B^k\) are proper submanifolds of \(M^{n+k}\) where \(B\) is closed and \(A,M\) are compact, and suppose that \(A\pitchfork B\), then \(A \cdot B\) is isotopy invariant.
    \end{theorem}

    \begin{figure}[H]
        \centering
        \includegraphics[width=0.4\textwidth]{img/cylinder}
        \caption{Here \(A \cdot B = 1\)}
    \end{figure}

    \begin{corollary}
        For closed \(A, B \subset M, A \cdot B\) is isotopy invariant.

        Warning: \(A, B \subset M\) proper then \(A \cdot B\) is not isotopy invariant.
    \end{corollary}

    \begin{theorem}
        [Thom Intersection Theorem] Suppose we have a smooth bundle \(\xi: \mathbb{R}^n \to E \to B^k\) with metric on \(\xi\) and \(B\) closed.

        Recall that the thom class \(u_E = PD_E z_{\ast}[B] \in H^n(DE, SE) = H^n(E, E_0)\) where \(z\) is a zero section.
        
        If \(A^n \subset D(E)\) is a proper compact submanifold, then,

        \[
            A \cdot z(B) = \langle u_E , z_{\ast} [B] \rangle \in \begin{dcases}
                \mathbb{Z}, &\text{ if oriented}  ;\\
                \mathbb{F}_2, &\text{ otherwise}.
            \end{dcases}
        \]
    \end{theorem}
    
    \begin{proof}
        After isotopy of \(A\),  assume \(\exists\) neighborhood of \(z(B)\) such that each component \(A \cap V\) lies in a fiber.

        \begin{figure}[H]
            \centering
            \includegraphics[width=0.4\textwidth]{img/thomintersection}
        \end{figure}
    \end{proof}

    \section*{Friday, 10/31/2025}
    
    We are moving on to chapter 11.

    Let \(M^n \subset A^{n+k}\) submanifold.

    \begin{theorem}
        [11.1 Tubular Neighborhood Theorem]

        \(\exists\) embedding \(\nu(M \hookrightarrow A) \hookrightarrow A\) which is `identity' on \(M\).
    \end{theorem}

    \begin{proof}
        (When \(A\) is compact): Give \(TM\) a metric. Consider \(\exp : TM \to M\) as follows:

        \(\exp(v) = \gamma^{\prime}(1)\) where \(\gamma:[0,1] \to M\) geodesic where \(\gamma(0) = \pi(v)\) and \(\gamma^{\prime}(c) = v\) 

        We start at the base point and run in the direction of \(v\). 

        \(\exists \epsilon > 0\) such that \(\eval{\exp}\overset{\circ}{D}_{\epsilon}(\nu) \hookrightarrow A\).

        Note that \(E(\nu) \cong \overset{\circ}{D}_{\epsilon}(v)\) by scaling. 

        \(E(\nu) \hookrightarrow A, (-\epsilon, \epsilon) \cong \mathbb{R}\).

    \end{proof}

    \begin{corollary}
        [11.2] If \(M\) is losed in \(A\) then restriction maps are isomorphisms:

        \[
            H^{\ast}(A, A-M) \xrightarrow[\text{excision}]{\cong} H^{\ast}(N, N-M) \xrightarrow[TMT]{\cong} H^{\ast}(E(\nu), E(\nu)_0)
        \]

        Here \(N\) is the tubular neighborhood: \(\operatorname{im} (E(\nu) \subset A)\).

    \end{corollary}

    \begin{definition}
        Thom class \(u_M \in H^n(A, A-M)\) maps to \(u_\nu\).

        \(u_M \in H^n(-;\mathbb{F}_2)\)
        
        \(u_M \in H^n(-;\mathbb{Z})\) if \(\nu\) is oriented, e.g. \(A,M\) are oriented.
    \end{definition}

    \begin{remark}
        \(X\pitchfork M\). \([x] \in H_n(A,A-M)\).
        
        \(\langle u_M , [x] \rangle \in M \cdot X\).

        \begin{figure}[H]
            \centering
            \includegraphics[width=0.4\textwidth]{img/imgthmclss}
        \end{figure}
    \end{remark}

    \begin{theorem}
        [11.3] 
        \[
            H^k(A, A-M) \xrightarrow{i^{\ast}} H^k A \xrightarrow{j^{\ast}} H^k M
        \]

        \begin{enumerate}[label=\alph*)]
            \item If \(M\) is closed in \(A\) then,
            \[
                j^{\ast} i^{\ast} u_A = \begin{Bmatrix} \operatorname{w}_k(\nu) \\ e(\nu) \end{Bmatrix} \text{ if } \nu\text{ is } \begin{Bmatrix} \\ \text{oriented} \end{Bmatrix} 
            \]

            \item If \(M \subset A\) but closed in manifolds,
            \[
                i^{\ast} u_M = PD[M] \in \begin{dcases}
                    H^k(A;\mathbb{F}_2), &\text{ if }  ;\\
                    H^k A, &\text{ if } A \text{ and } M \text{ both oriented}.
                \end{dcases}
            \]
        \end{enumerate} 
    \end{theorem}

    \begin{proof}
        b: explicit Poincar\'e Duality: Poincar\'e Dual of submanifold in the image of Thom class of its normal bundle.

        \[
            \begin{tikzcd}
                H^n A \ar[r] \ar[d,"\in",sloped,phantom] & H^k A \\
                {[M]} \ar[r] & \operatorname{im} u_M 
            \end{tikzcd}
        \]

        a: oriented case `Essentially definition of Euler class'

        \[
            \begin{tikzcd}
                & u_M \\
                H^k(N, N-M) \ar[d,"TNT"] & H^k(A , A - M) \ar[l, "\cong"'] \ar[r,"i^{\ast}"] & H^k A \ar[d,"j^{\ast}"] \\
                H^k(E(\nu) , E(\nu)_0) \ar[r] & H^k(E(\nu)) \ar[r,"\cong"] & H^k M \\
                u_\nu & & e(U)
            \end{tikzcd}
        \]

    \end{proof}

    In the non-oriented case, with \(\mathbb{F}_2\)-coefficients, need:

    \[
        \begin{tikzcd}
            H^k(E(\nu), E(\nu)_0 ; \mathbb{F}_2) \ar[r] & H^k(M;\mathbb{F}_2) \\
            u_\nu \ar[r,mapsto] & \operatorname{w}_k(\nu)
        \end{tikzcd}
    \]

    [See 95]

    Applications: 

    \begin{corollary}
        [11.3a] \(\implies\) Cor 11.4. \(M^n \subset \mathbb{R}^{n+k}\) closed subset then,
        
        \(0 = \operatorname{w}_k(\nu) = \overline{\operatorname{w}}_k(TM)\).

        If \(M \subset \mathbb{R}^{n_k}\) is oriented, then \(e(\nu) = 0\).
    \end{corollary}

    Recall that \(\overline{\operatorname{w}}(\xi) \operatorname{w}(\xi) = 1\).

    \(\overline{\operatorname{w}}(\xi) = \operatorname{w}(\xi) ^{-1} = \frac{1}{1 + (\operatorname{w}_1 + \operatorname{w}_2 + \cdots)}\)
    
    \(= 1 + (\operatorname{w}_1 + \operatorname{w}_2 + \cdots) + (\operatorname{w}_1 + \operatorname{w}_2 + \cdots)^2 + \cdots\)

    Recall \(M^n \hookrightarrow \mathbb{R}^{n+k}\) immersion implies \(\overline{\operatorname{w}}_l(TM) = 0\) for \(l > k\).

    When \(n = 2^l\), \(\operatorname{w}(TP^n) = 1 + a + a^n\).
    
    \(\operatorname{w}(TP^n) = 1 + a + \cdots + a^n\).

    Therefore, \(\mathbb{R}P^n\) does not immerse into \(\mathbb{R}^{2n - 2}\).
    
    We can go down one further dimension \(\mathbb{R}P^n\) doesn't embed in \(\mathbb{R}^{2n-1}\). In particular, \(\mathbb{R}P^2 \not \hookrightarrow \mathbb{R}^3\).

    Now, consider the open M\"obius strip \(M\).

    \(M \hookrightarrow \mathbb{R}^3\) but \(\operatorname{w}_1 (TM) \neq 0 \implies \overline{\operatorname{w}}_1(TM) \neq 0\)

    This means \(M \not \hookrightarrow \mathbb{R}^3\) as closed subset.

    \section*{Monday, 11/3/2025}
    
    \subsection*{Chapter 11}

    Goals: Euler class of a closed manifold integrated over the whole manifold is the Euler characteristic:

    \[
        \langle e(T), [M] \rangle = \chi(M)
    \]

    Another goal: Wu's formula for \(\operatorname{w}_k (TM)\).

    Review:

    Euler class \(e(\xi) \in H^n(B;\mathbb{Z})\) is the image of the Thom class:

    \[
        u \in H^n(E,E_0) \to H^n E \xleftarrow[\cong]{\pi^{\ast}} H^n B \ni e(\xi)
    \]

    Submanifold \(M^n \subset A^{n+k}\).

    11.2: If \(M\) is closed in \(A\), then,

    \[
        u_\nu \in H^k(E(\nu), E(\nu)_0) \underset{T.N.T.}{\cong} H^k(N, N-M) \xleftarrow{\cong} H^k(A, A-M) \ni u_M
    \]

    Milnor-Stasheff class \(u_M\) as \(u^{\prime}\).

    Intution for \(u_M\): \(\langle u_M , [X] \rangle = M \cdot X\).

    11.3: \(u_M \in H^k(A, A-M)\xrightarrow{i^{\ast}} H^k A \xrightarrow{j^{\ast}} H^k M\)
    
    \begin{enumerate}[label=\alph*)]
        \item \(M\) closed in \(A\) implies \(u_M \mapsto \operatorname{w}_k(\nu)\). \(\nu\) oriented implies \(u_M \mapsto e(\nu)\).
        \item \(M, A\) closed manifolds implies \(u_M \mapsto PD_A[M]\). 
    \end{enumerate} 

    Application of 11.3(b): \(X^k \pitchfork M^n \subset A\), all closed and oriented. In that case,
    
    \(M \cdot X = \langle u_M , [X] \rangle = \langle PD[M] , X \rangle = \langle PD[M] \cup PD[X], [A] \rangle\), the algebraic intersection number.
    
    In the case \(A^{n+k}\) closed and oriented, then, we have algebraic integral pairing:

    \[
        \frac{H^n A}{\text{tor}} \otimes \frac{H^k A}{\text{tor}} \to \mathbb{Z}
    \]

    \[
        a \otimes b \mapsto \langle a \cup b, [A] \rangle 
    \]

    Choose \(\mathbb{Z}\)-basis \(\{ e_k \}\), that gives us \(\langle e_i \otimes e_j , [A] \rangle \). It's a symmetric matrix, and P.D. implies \(\det = \pm 1\).

    \subsubsection*{Tangent Bundle}

    `Normal bundle of the diagonal bundle is the tangent bundle of the manifold.'

    Define diagonal map \(\Delta: M \to M \times M, \Delta(x) = (x,x)\).
    
    \begin{figure}[H]
        \centering
        \includegraphics[width=0.4\textwidth]{img/diagonalmap}
        \caption{Diagonal Map}
    \end{figure}

    Consider curve \(\alpha : \mathbb{R} \to M \times M\). Then we in fact have two maps: \(\alpha = (\alpha_1 , \alpha_2)\) where \(\alpha_i : \mathbb{R} \to M\).

    Therefore, \(T(M \times M) = TM \times TM\).

    Notice that for any curve \(\gamma: \mathbb{R}\to M\) we can find a new curve
    \((\gamma(t), \gamma(-t)) : \mathbb{R} \to M \times M\).

    These give us lemma 11.5

    \begin{lemma}
        [11.5] \(\exists\) bundle map:

        \[
            \begin{tikzcd}
                v \ar[r, mapsto] & (v,-v) \\
                TM \ar[r] \ar[d] & \nu(\Delta(M) \hookrightarrow M \times M) \ar[d] \\ M \ar[r,"\Delta", "\cong"'] & M \times M
            \end{tikzcd}
        \]

        Therefore \(TM = \nu(\Delta \hookrightarrow M \times M)\).
    \end{lemma}

    Now we jump into the algebraic topology.

    \[
        H^n(M \times M , M \times M - \Delta M) \to H^n(M \times M)
    \]

    \[
        u_{\Delta M} \mapsto u^{\prime\prime}
    \]

    Here \(u^{\prime\prime}\) is the `diagonal cohomology class'. \(u^{\prime\prime} = PD_{M \times M} [\Delta]\).

    \begin{lemma}
        [11.8] \((1 \times a) \cup u^{\prime\prime} = (a \times 1) \cup u^{\prime\prime}\) for \(a\in H^{\ast} M\).
    \end{lemma}

    \begin{proof}
        [Sketch] \(\Delta M \hookrightarrow M \times M\) is symmetric in the two factors. 
    \end{proof}

    \begin{lemma}
        [11.9] When \(M\) is closed, if we take the `slant product' then \(u^{\prime\prime} / [M] = 1 \in H^0 M\) 
    \end{lemma}

    Proof ommitted.

    \subsection*{Products}

    Recall: Cup products \(\leftrightarrow\) cross products. Implies cohomology is a ring.

    Cap products imply homology is a module over cohomology ring. It corresponds to `slant prduct'.

    \[
        / : H^{p+q} (X \times Y) \otimes H_q Y \to H^p X
    \]

    \[
        a \otimes z \mapsto a / z
    \]

    It is supposed to be like a fraction.

    It is also related to the cross product: \((a \times b) / \beta = \langle b, \beta \rangle a\).

    This can work as a definition if coefficients are in a field. Theorem for general coefficients.

    \begin{definition}
        [Slant Product] At the cochain level: take \(f\in H^{p+q}(X \times Y)\) and \(\sigma : \Delta^q \to Y\), then for any \(p\)-chain \(\tau\),

        \[
            (f / \sigma)(\tau) = f(_p \sigma \times \tau)
        \]

        [Note: this is not quite right]
    \end{definition}

    \section*{Wednesday, 11/5/2025}
    
    Recap:

    Slant product \(/ : H^{p+q}(X \times Y) \otimes H_q X \to H^p Y\). \(p \otimes \beta \mapsto p / \beta\).

    Main idea: if \(a\in H^p X , b \in H^q Y\) then \((a \times b) / \beta = \langle b , \beta \rangle a\).

    \(- / \beta\) is \(H^{\ast} X\)-linear: \(((a \times 1) \cup p) / \beta = a \cup (p / \beta)\).
    
    If \(M\) is oriented assume field coefficient \(F\). Otherwise assume \(\mathbb{F}_2\)-coefficients.

    Now assume that \(M^n\) is closed and smooth. \(H^n (M \times M , M \times M - \Delta) \ni u_\Delta\), the thom class of the diagonal. \(u_\Delta\) maps to \(u^{\prime\prime} \in H^n(M \times M)\). It is called the diagonal cohomology class, which is the Poincar\'e dual to \(\Delta M\).

    Recall when \(n\in \dim B - \dim A\), we have \(H^n(B , B - A) \cong H^n(E(\nu) , E(\nu)_0)\) where \(\nu\) is the normal bundle by excision and tubular neighborhood theorem.
    
    11.8: \(\forall a\in H^{\ast} M , (a \times 1) \cup u^{\prime\prime} = (1 \times a) \cup u^{\prime\prime}\), symmetry.

    11.9: \(u^{\prime\prime} / [M] = 1 \in H^0 M\).

    Proof omitted.

    11.10: Duality Theorem: \(\forall\) basis \(b_1 , \cdots , b_r\) for \(H^{\ast} M\) there exists dual basis \(b_1^\# , \cdots , b_r^\#\) so that \(\langle b_i \cup b_j^\# , [M] \rangle = \delta_{ij}\).

    11.11 \(u^{\prime\prime} = \sum_{i} (-1)^{\vert b_i \vert} b_i \times b_i^\# \in H^n(M \times M)\).
    
    11.10 \(\iff\) \(I : H^{\ast} M \otimes_F H^{\ast} M \to F\) given by \(a \otimes b \mapsto \langle a \cup b , [M] \rangle \) is a perfect pairing, thus \(\dim H_p M = \dim H^{n-p} M = \dim H_{n-p} M\). 

    Suppose \(A, B\) are \(\Lambda\)-modules where \(\Lambda\) is a commutative ring. then \(A \otimes_\Lambda B \to C\) is perfect pairing if \(A \xrightarrow{\approx} \operatorname{Hom} (B, C)\) and \(B \xrightarrow{\approx} \operatorname{Hom} (A, C)\). In our example the perfect pairing comes from the bilinear map.
    
    \begin{proof}
        We prove 11.10 and 11.11.

        By K\"unneth theorem we can write \(H^n (M \times M) \ni u^{\prime\prime} = b_1 \times c_1 + \cdots + b_r \times c_r\).

        11.8 \(\implies (a \times 1) \cup u^{\prime\prime} = (1 \times a) \cup u^{\prime\prime}\). By taking slat with fundamental class,

        \(((a \times 1) \cup u^{\prime\prime}) / [M] = ((1\times a) \cup u^{\prime\prime}) / [M]\)
        
        \(a \cup (u^{\prime\prime} / [M]) = (1 \times a) \cup \left( \sum_{j} b_j \times c_j \right) / [M]\)

        \(a = \left( \sum_{j} (-1)^{\vert a \vert \vert b_j \vert} (1 \cup b_j) \times (a \cup c_j) \right) / [M]\).
        
        \(a = \sum_{j} (-1)^{\vert a \vert \vert b_j \vert} \langle a \cup c_j , [M] \rangle b_j\).
        
        Now take \(a = b_i\). The \(b_i\) are a basis. Therefore, taking \(a = b_i\) we see:

        \[
            b_i = \sum_{j} (-1)^{\vert b_i \vert \vert b_j \vert} \langle b_i \cup c_j , [M] \rangle = \delta_{ij}
        \]

        Define \(b_j^\# = (-1)^{b_j} c_j\).

        \(\langle b_i \cup b_j^\# , [M] \rangle = \delta_{ij}\)
        
        \(u^{\prime\prime} = \sum_{i} (-1)^{\vert b_i \vert } b_i \times b_i^\#\) 

        When \(M = \mathbb{R} P^2 , u^{\prime\prime} = 1 \times a^2 + a \times a + a^2 \times 1 \in H^2(\mathbb{R} ^2 \times \mathbb{R} P^2)\).

    \end{proof}

    \begin{corollary}
        When \(M^n\) is closed, smooth and oriented, \(\langle e(TM) , [M] \rangle = \chi(M)\).

        When \(M^n\) is closed and smooth, \(\langle \operatorname{w}_n(TM) , [M] \rangle \cong \chi \xi (M) \mod 2\),

    \end{corollary}


    \begin{proof}
        Oriented case: claim \(e(TM) = \Delta^{\ast} u^{\prime\prime}\).

        \[
            \begin{tikzcd}
                u_{TM} & & e(TM) \\
                H^n(TM , TM_0) \ar[rr] \ar[d,"\cong"] & & H^n (M \times M) \\
                {H^n(E(\nu : \Delta \hookrightarrow M \times M), E(v)_0)} \ar[r, "\cong"] & H^n(M \times M , M \times M - \Delta) \ar[r] & H^{\ast} (M \times M) \ar[u,"\Delta^{\ast}"] \\
                & u_\Delta & u^{\prime\prime} 
            \end{tikzcd}
        \]
    \end{proof}

    \begin{figure}[H]
        \centering
        \includegraphics[width=0.8\textwidth]{img/pertrubediagonal}
    \end{figure}

    Let \(\Delta^{\prime}\) be isotopic copy of \(\Delta\) such that \(\Delta^{\prime} \pitchfork \Delta\).

    Then \(\Delta^{\prime} \cdot \Delta = \langle e(\nu), [\Delta] \rangle - \langle e(TM) , [M] \rangle = \chi(M)\) 

    THus, \(\chi(M)\) is the self intersection number of the diagonal \(\Delta M \hookrightarrow M \times M\) 

    \begin{corollary}
        If \(M\) has a nowhere zero vector field then \(\chi(M) = 0\).
    \end{corollary}

    \begin{proof}
        Suppose otherwise. Then \(M\) has a non-zero vector field implies \(\Delta M\) has a non-zero normal vector field. ``Flow'' implies \(\exists \Delta^{\prime} \) such that \(\Delta^{\prime} \cap \Delta = \varnothing\).
    \end{proof}

    Thus, \(\chi(M) \neq 0 \implies\) can't comb hairy \(M\).

    Recall \(\chi(M) = (-1)^i \dim H_i(M , \mathbb{Q}) = \sum_{i} (-1)^i (\# \text{-of \(i\)-cells})\).
    
    \(= \sum_{i} (-1)^i \dim H_i(M , \mathbb{F}_p)\).

    \section*{Friday, 11/7/2025}
    
    \subsection*{Wu classes / Wu Formula / Wu Theorem}

    Coefficients in \(\mathbb{F}_2\) understood.

    Wu classes are polynomials of whitney classes.

    \(\operatorname{v}_0 = \operatorname{w}_0 = 1\)
    
    \(\operatorname{v}_1 = \operatorname{w}_1\) 

    \(\operatorname{v}_2 = \operatorname{w}_1^2 + \operatorname{w}_2\)
    
    \(\operatorname{v}_3 = \operatorname{w}_1 \operatorname{w}_2\).

    They're defined as following:

    \begin{definition}
        [Total Wu Class]

        \[
            \operatorname{v} =  \operatorname{v}_0 + \operatorname{v}_1 + \operatorname{v}_2 + \cdots
        \]

        \[
            \operatorname{w} = \operatorname{Sq} v
        \]
    \end{definition}

    i.e. \(v = \operatorname{Sq}^{-1} \operatorname{w} = (1 + \operatorname{Sq}^1 + \operatorname{Sq}^2 + \cdots)^{-1} \operatorname{w}\).
    
    \begin{proposition}
        [Wu's Formula, Execrise 8A]

        \(\operatorname{Sq}^k \operatorname{w}_m\) is `something in the cohomology of the Grassmanian', so it must be some polynomial over Stiefel Whitney Classes.

        \[
            \operatorname{Sq}^k \operatorname{w}_m = \sum_{i} \binom{k-m}{i} \operatorname{w}_{k-i} \operatorname{w}_{m+i}
        \]
    \end{proposition}

    Hint on 8A:

    \[
        H^{\ast} (G_n) \cong H^{\ast} (P^m \times \cdots \times P^m)^{S_n}
    \]

    \[
        \operatorname{w}_i \mapsto \sigma_i (a_1 , \cdots , a_n)
    \]

    Compute \(\operatorname{Sq}^i\) using Cartan.

    eg \(\operatorname{Sq}^1 \operatorname{w}_2 = \operatorname{w}_1 \operatorname{w}_2 + \operatorname{w}_3\).

    Computation:

    \[
        \operatorname{w} = \operatorname{Sq} \operatorname{v} = (1 + \operatorname{Sq}^1 + \operatorname{Sq}^2 + \cdots)(v_0 + v_1 + v_2 + \cdots)
    \]

    Then, \(1 = \operatorname{w}_0 = \operatorname{v}_0\).

    \(\operatorname{w}_1 = \operatorname{v}_1\)

    \(\operatorname{w}_2 = \operatorname{Sq}^1 \operatorname{v}_1 + \operatorname{v}_2 \implies \operatorname{w}_2 = \operatorname{w}_1^2 + \operatorname{v}_2\)
    
    \(\operatorname{w}_3 = \cancel{\operatorname{Sq}^2 \operatorname{v}_1} + \operatorname{Sq}^1 \operatorname{v}_2 + \operatorname{v}_3 = \operatorname{Sq}^1 \operatorname{w}_1^2 + \operatorname{Sq}^1 \operatorname{w}_2 + \operatorname{v}_3\)
    
    \(= \underbrace{\cancel{\operatorname{Sq}^0 \operatorname{w}_1 \operatorname{Sq}^1 \operatorname{w}_1 + \operatorname{Sq}^1 \operatorname{w}_1 \operatorname{Sq}^0 \operatorname{w}_1}}_{\text{cartan}} + \underbrace{\operatorname{w}_1 \operatorname{w}_2 + \operatorname{w}_3}_{\text{Wu Formula}} + \operatorname{v}_3\) 

    Now, suppose we have \(M^n\) a closed \(n\)-manifold.

    \begin{theorem}
        [Wu Theorem]

        Let \(\operatorname{v}(TM)\) be the total Wu class of a tangent bundle.

        \[
            \langle \operatorname{v}(TM) \cup - , [M] \rangle = \langle \operatorname{Sq}(-), [M]\rangle 
        \] 

        i.e. if \(x\in H^{n-k} M\) then \(\operatorname{v}_k(TM) \cup x = \operatorname{Sq}^k x\).
        
        i.e. \(\langle \operatorname{v}_k(TM) \cup x , [M] \rangle = \langle \operatorname{Sq}(-), [M] \rangle\).
    \end{theorem}

    \begin{corollary}
        Let \(M \underset{h}{\simeq} M^{\prime}\) be homotopy equivalent manifolds. Then, \(\operatorname{w}(TM) = h^{\ast} \operatorname{w}(TM^{\prime})\).
    \end{corollary}

    \begin{proof}
        [Sketch] Wu classes are determined by algebraic topology. Thus, homotopy equivalent implies same algebraic topology which implies same Wu class which implies same Stiefel-Whitney class.
    \end{proof}

    We can connect this to intersection forms.

    \begin{definition}
        [Algebraic Intersection Form] \(I_M : H^{\ast} M \otimes H^{\ast} M \to \mathbb{F}_2\).

        \[
            I_M(a \otimes b) = \langle a\cup b , [M] \rangle
        \]

        We write \(a \cdot b = I_M(a \otimes b)\). By Poincar\'e duality it is a \textit{perfect pairing}, thus it is a \textit{non-singular pairing}.
    \end{definition}

    \subsubsection*{Key application of Wu's Theorem}

    Suppose \(n = 2k\). \(M\) is a closed \(n\)-dimensional manifold.
    
    \[
        \langle v_k (TM) \cup x, [M] \rangle = \langle \operatorname{Sk}^k x, [M] \rangle = \langle x\cup x, [M] \rangle 
    \]

    Thus, for \(x\in H^k M\):
    
    \[
        \operatorname{v}_k(TM) \cdot x = x \cdot x\;.
    \]

    Now we restrict to the middle dimensional homology.

    \[
        \widehat{I_M} : H^k M^{2k} \otimes H^k M^{2k} \to \mathbb{F}_2
    \]

    \begin{definition}
        \(\widehat{I_M}\) is even if \(\forall a, \widehat{I_M} (a \otimes a) = 0\).

        \(\iff\) if \(\beta_i\) is a basis for \(H^k M\) then the matrix \((\beta_i \cdot \beta_j)\) has even \(\#\) on the diagonal.
    \end{definition}

    Then,

    \begin{theorem}
        [Wu's Theorem]

        \[
            \operatorname{v}_k(TM^{2k}) = 0 \iff \widehat{I_M} \text{ is even} 
        \]
    \end{theorem}

    Example: Suppose \(n = 2\). Then \(\operatorname{v}_1 = \operatorname{w}_1\).

    \(\operatorname{v}_1 = 0 \iff M^2\) orientable \(\iff \widehat{I_M}\) is even (eg Torus).

    Matrix: \(\begin{bmatrix}
        \begin{bmatrix}
            0 & 1 \\
            1 & 0 \\
        \end{bmatrix}  &  &  \\
         & \ddots &  \\
         &  & \begin{bmatrix}
            1 & 0 \\
            0 & 1 \\
         \end{bmatrix} \\
    \end{bmatrix} \) 

    \(\operatorname{v}_1 \neq 0 \iff \widehat{I_M}\) is odd. e.g. \(\mathbb{R} P^1 \cdot \mathbb{R} P^1 = 1\) in \(\mathbb{R} P^2\).

    Let \(K\) be the Klein bottle. Then \(\widehat{I_K}\) has matrix \(\begin{bmatrix}
        0 & 1 \\
        1 & 1 \\
    \end{bmatrix}\) since \(b \cdot b = 1\) and \(a \cdot a = 0\).

    \begin{figure}[H]
        \centering
        \includegraphics[width=0.3\textwidth]{img/klein}
        \caption{Klein Bottle}
    \end{figure}

    e.g. \(\mathbb{R} P^4: \operatorname{w}_1 \neq 0, \operatorname{w}_2 = 0, \operatorname{v}_2 \neq 0\) thus \(\mathbb{R}P^2 \cdot \mathbb{R} P^2 = 1\).

    Further example: \(\mathbb{C} P^1 \cdot \mathbb{C} P^1 = 1\).

    \begin{corollary}
        Orientable \(4\)-manifold: \(\widehat{I_M}\) is even \(\iff \operatorname{w}_2 (TM) = 0 \iff \operatorname{v}_2(TM) = 0\).
    \end{corollary}

    To prove Wu's theorem we need an additional lemma:

    \begin{lemma}
        [11.3]

        \[
            \operatorname{w}(TM) = \operatorname{Sq}(u^{\prime\prime}) / [M]
        \]

        Where \(u^{\prime\prime} \in H^n(M \times M)\) the diagonal cohomology class dual to \(\Delta M\).
    \end{lemma}

    \begin{proof}
        We assume the lemma is true. In that case, 

        \(I_M\) is perfdect pairingm thus non-singular, thus \(\exists ! \hat{v} \in H^{\ast} M\) such that \(\langle \hat{v} \cup - , [M] \rangle = \langle \operatorname{Sq}(-) , [M] \rangle : H^{\ast} M \to \mathbb{F}_2\).

        WTS: \(\hat{v} = \operatorname{v}(TM)\).

        WTS: \(\operatorname{Sq}\hat{v} = \operatorname{w}(TM)\).

        Choose basis \(b_i\) for \(H^{\ast} M\) and dual basis \(b_i^\sharp\) i.e. \(b_i \cdot b_j^\sharp = \delta_{ij}\) [11.10]

        Then, 11.11 \(\implies u^{\prime\prime} = \sum_{i} b_i \times b_i^\sharp\).

        11.10: \(\hat{v} = \left( \sum_{i} \hat{v} \cdot b_i^\sharp \right) b_i = \sum_{i} \langle \operatorname{Sq} (b_i^\sharp) , [M] \rangle b_i\)

        \(\implies \operatorname{Sq} \hat{v} = \sum_{i} \langle \operatorname{Sq} (b_i^\sharp) , [M] \rangle \operatorname{Sq} b_i\)
        
        Cartan and 11.11 implies,

        \(\operatorname{Sq} \hat{v} = \sum_{i} (\operatorname{Sq}(b_i) \times \operatorname{Sq}(b_i^\sharp)) / [M] = \operatorname{Sq}(u^{\prime\prime}) / [M] = \operatorname{w}(TM)\). 

    \end{proof}

    \section*{Monday, 11/10/2025}
    
    Recap: Wu classes: \(\operatorname{Sq} v = \operatorname{w}\).

    Wu formula:

    \[
        \operatorname{Sq}^k \operatorname{w}_m = \sum_{i} \binom{k-m}{i} \operatorname{w}_{k-i} \operatorname{w}_{k+i}
    \]

    Using these, we can find out: \(v_1 = \operatorname{w}_1 , v_2 = \operatorname{w}_1^2 + \operatorname{w}_2 , v_3 = \operatorname{w}_1 \operatorname{w}_2\).

    Wu's Theorem: If \(M\) is a closed manifold and \(x\in H^{\ast} (M;\mathbb{F}_2)\) then,

    \[
        \langle v(TM) \cup x , [M]\rangle = \langle \operatorname{Sq}^k (x), [M] \rangle
    \]

    \begin{corollary}
        If \(k > \frac{\dim M}{2}\) then \(v_k(TM) = 0\).
    \end{corollary}

    \begin{proof}
        \(\forall x\in H^{n-k} (M;\mathbb{F}_2)\)

        \[
            \langle v_k(TM) \cup x , [M] \rangle = \langle \operatorname{Sq}^k(x) , [M] \rangle = \langle 0, [M] \rangle = 0
        \]
    \end{proof}

    If \(k = \frac{\dim M}{2}\) then \(\langle v_k(TM)\cup x, [M] \rangle = \langle x\cup x, [M] \rangle \) which is the `self intersection' number.

    \subsubsection*{Application to 3-manifolds}

    Let \(M^3\) be closed, \(\operatorname{w}_i = \operatorname{w}_i(M) , v_i = v_i(TM)\).

    \begin{theorem}
        \begin{enumerate}[label=\alph*)]
            \item All SW numbers of \(M^3\) vanish.
            \item \(M^3\) orienatable implies \(\operatorname{w}_1 = \operatorname{w}_2 = \operatorname{w}_3 = 0\). 
        \end{enumerate} 
    \end{theorem}

    \begin{proof}
        \(\frac{\dim M}{2} \implies v_2 = 0, v_3 = 0\). Then \(\operatorname{w}_1^2 = \operatorname{w}_2\) and \(\operatorname{w}_1 \operatorname{w}_2 = 0\). So \(\operatorname{w}_1^3 = 0\). \(\chi(M^3) = 0 \implies \operatorname{w}_3 = 0\) [recall \(\chi(M^n) \equiv \langle \operatorname{w}_n(TM) , [M] \rangle \pmod 2\), apply PD].
        
        For part b: \(\operatorname{w}_1 = 0 \implies  \operatorname{w}_2 = 0, \operatorname{w}_3 = 0\).
    \end{proof}

    a + Thom's theorem \(\implies M^3 = \partial W^4\) compact, i.e. every \(3\)-manifold is the boundary of a compact \(4\)-manifold.

    b + obstruction theorem \(\implies\) oriented closed \(3\)-manifold \(M^3\)  has trivial tangent bundle, ``paralellizable'' [Problem 12-13].

    \subsection*{Gysin Sequence}

    It's a long exact sequence. Consider the vector bundle \(\begin{tikzcd}\mathbb{R}^n \ar[r] & E\ar[d,"\pi"] \\ & B\end{tikzcd}\).

    \begin{enumerate}[label=\alph*)]
        \item \(\exists\) LES:
        
        \[
            \cdots \to H^{j-n} (B;\mathbb{F}_2) \xrightarrow{-\cup \operatorname{w}_n} H^j(B;\mathbb{F}_2) \xrightarrow{\pi^\sharp} H^j (E_0 ; \mathbb{F}_2) \to H^{j-n+1}(B;\mathbb{F}_2) \to \cdots 
        \]

        \item If oriented, \(\exists\) LES:
        
        \[
            \cdots \to H^{j-n} \xrightarrow{-\cup e} H^j B \to H^j E_0 \to H^{j-n+1} B \to \cdots 
        \]

        \item If oriented with metric,
        
        \[
            \cdots \to H^{j-n} B \xrightarrow{-\cup e} H^j B \to H^j (S(E)) \to \cdots 
        \]
    \end{enumerate} 

    Recall, suppose we have a trivial bundle. \(H^{\ast} E_0 = H^{\ast} (B \times (\mathbb{R}^n - 0)) = H^{\ast} (B \times S^{n-1}) = H^{\ast} B \oplus H^{\ast + n - 1} B\) [K\"unneth]. Since in trivial bundle, \(-\cup e\) is \(0\) this works!

    \begin{proof}
        b: LES of pair \((E, E_0)\):

        \[
            \begin{tikzcd}
                H^j(E,E_0) \ar[r] & H^j E \ar[r] & H^j E_0 \ar[r] & H^{j+1}(E,E_0) \\ H^{j-n} \ar[u, "\cong", "-\cup u"'] \ar[ur,"\eval{-\cup u}_E"] & H^j B\ar[u,"\cong"] \\ H^{j-n}B \ar[u,"\cong"] \ar[ur, "-\cup e"]
            \end{tikzcd}
        \]

        2nd proof: SSS tp \(\begin{tikzcd}S^1 \ar[r] & E \ar[d] \\ & S^2 \end{tikzcd}\).
        
        Classified by \(e\in H^2(S^2)\). eg \(E = S^3 , S^1 \times S^2 , L_n\) lens spaces, \(e = 0,1,n\).
    \end{proof}

    \begin{corollary}
        [12.3] Any \(2\)-fold cover \(\begin{tikzcd} \widetilde{B} \ar[d,"\pi"] \\ B\end{tikzcd}\) implies: \(\exists \xi = \begin{tikzcd}\mathbb{R} \ar[r] & E \ar[d] \\ & B\end{tikzcd}\) such that,

        \[
            \begin{tikzcd}
                SE \ar[rr,"\cong"] \ar[rd] & & \widetilde{B} \ar[ld] \\ & B
            \end{tikzcd}
        \]

        and LES:

        \[
            \cdots \to H^{j-1}(B;\mathbb{F}_2) \xrightarrow{-\cup \operatorname{w}_1} H^j(B;\mathbb{F}_2) \to H^j(\widetilde{B},\mathbb{F}_2) \to \cdots 
        \]

        `Smith exact sequence, Hatcher'
    \end{corollary}

    \begin{proof}
        Let \(E \coloneqq \frac{\widetilde{B} \times \mathbb{R}}{(x,t) \sim (x^{\prime}, -t)}\)

        Where \(\pi(x) = \pi(x^{\prime}), x\neq x^{\prime}\).

        Use Gysin. e.g. \(\begin{tikzcd} S^2 \ar[d] \\ P^2\end{tikzcd}\) or \(\begin{tikzcd} T^2 \ar[d] \\ K^2 \end{tikzcd}\).
    \end{proof}

    \(\widetilde{G}_n(\mathbb{R}^{{n+k}})=\) oriented \(n\)-planes in \(\mathbb{R}^{n+k}\). This is \(V_n(\mathbb{R}^{n+k}) / SO(n)\).

    \[
        V_n(\mathbb{R}^{n+k}) = \{ (v_1 , \cdots , v_n) \mid v_i \in \mathbb{R}^{n+k} ; v_i \cdot v_j = \delta_{ij} \} \subset \mathbb{R}^{n+k} \times \cdots \times \mathbb{R}^{n+k}
    \]

    \(SO(n) = \{ A \in M_n\mathbb{R} \mid A A^t = I, \det A = 1 \}\).

    \[
        \begin{tikzcd}
            \widetilde{G}_n \ar[d,"\text{double cover}"] \ar[r,"=",phantom] & \widetilde{G}_n(\mathbb{R}^{\infty}) \ar[r,"=",phantom] & BSO(n) \\ G_n \ar[r,"=",phantom] & G_n(\mathbb{R}^{\infty} ) \ar[r,"=",phantom] & BO(n)
        \end{tikzcd}
    \]

    Then we will have 12.3 (Gysin):

    \[
        H^{\ast} (\widetilde{G}_n ; \mathbb{F}_2) = \mathbb{F}_2 [ \operatorname{w}_2 , \operatorname{w}_3 , \cdots ]
    \]

    \section*{Friday, 11/14/2025}
    
    Today: \(\widetilde{G_n}\) and \(\mathbb{C}\) vector bundles.

    \begin{definition}
        [Oriented Grassmanian]

        \(\widetilde{G_n}(\mathbb{R}^{n+k}) =\) oriented \(n\)-planes in \(\mathbb{R}^{n+k}\)

        \[
            = \frac{\text{Orthonormal } n \text{-frames in }\mathbb{R}^{n+k}}{\text{Orientation Preserving Rigit motions}} = \frac{V_n(\mathbb{R}^{n+k})}{\operatorname{SO}(n)}
        \]
    \end{definition}

    Then there's a double cover:

    \[
        \begin{tikzcd}
            \widetilde{G}_n(\mathbb{R}^{n+k}) \ar[d] \\ G_n(\mathbb{R}^{n+k})
        \end{tikzcd}
    \]

    The double cover is not trivial. \(k > 0, \widetilde{G}_n(\mathbb{R}^{n+k})\) is connected.

    There is a tautological bundle over this space.

    \[
        \begin{tikzcd}
            E(\widetilde{\gamma}_n) \ar[d, no head]\\ \widetilde{G}_n(\mathbb{R}^{n+k})
        \end{tikzcd}
    \]

    Definition 1: \(E(\widetilde{\gamma}_n) \subset \widetilde{G}_n (\mathbb{R}^{n+k}) \times \mathbb{R}^{n+k}\).

    Definition 2: Pullback:

    \[
        \begin{tikzcd}
            E(\widetilde{\gamma}_n) \ar[d] \ar[r] & E(\gamma_n ) \ar[d] \\ \widetilde{G}_n(\mathbb{R}^{n+k}) \ar[r] & G_n(\mathbb{R}^{n+k})
        \end{tikzcd}
    \]

    \(\widetilde{G}_n = G_n(\mathbb{R}^{\infty}) = \underset{k \to \infty}{\operatorname{colim}} \widetilde{G}_n(\mathbb{R}^{n+k})\) 

    \begin{theorem}
        \(\begin{tikzcd}E(\widetilde{\gamma}_n) \ar[d] \\ \widetilde{G}_n \end{tikzcd}\) classifies oriented vector bundles over \(B\) CW. i.e.

        \[
            \begin{tikzcd}
                {[B, \widetilde{G}_n]} \ar[r] & {\begin{Bmatrix}\text{iso class of} \\ \text{oriented \(n\)-planes} \\ \text{bundles }/ B \end{Bmatrix}} \ar[l] \\
                f \ar[r, mapsto] & f^{\ast} \widetilde{\gamma}_n 
            \end{tikzcd}
        \]
    \end{theorem}

    \(H^{\ast} (\widetilde{G}_n) \to H^{\ast} B\). Here \(\widetilde{G}_n\) classifying space, \(\widetilde{\gamma}_n\) universal bundle.
    
    \begin{proof}
        First: If \(\xi \) oriented then any bundle map \(\xi  \to \gamma_n\) lifts uniquely to o.p. bundle map \(\xi \to \widetilde{\gamma}_n\)

        Second: Presentation

        \[
            \begin{tikzcd}
                SO(n) \ar[r] & V_n(\mathbb{R}^{\infty}) \ar[d] \ar[r, phantom, "\simeq"] & \ast \\ & \widetilde{G}_n
            \end{tikzcd}
        \]

        Then \(\widetilde{G}_n = BSO(n) = BGL_n^+(\mathbb{R})\).
    \end{proof}

    \(\widetilde{G}_n \xrightarrow{\pi} G_n\): non-trivial \(2\)-fold cover. Let \(\gamma_\pi\) be the associated line bundle to the double cover. 

    \(H^1(G_n , \mathbb{F}_2) = \mathbb{F}_2\). This is \(\operatorname{w}_1\).

    Therefore, \(\operatorname{w}_1(\gamma_\pi) = \operatorname{w}_1(\gamma_n)\).

    We can change the fiber:

    \[
        \begin{tikzcd}
            & & & \widetilde{G}_n \times_{C_2} \mathbb{R} \ar[d,"=", phantom, sloped] \\ 
            S^0 \ar[r] & \widetilde{G}_n \ar[d] & \mathbb{R} \ar[r] & E(\gamma_\pi) \ar[d] \\ & G_n & & G_n
        \end{tikzcd}
    \]

    Recall 12.3: Gysin sequence for \(\gamma_\pi\).

    \[
        \xrightarrow{0} H^{j-1} (G_n ; \mathbb{F}_2) \xrightarrow{- \cup \operatorname{w}_1} H^j (G_n ; \mathbb{F}_2) \to H^j(\widetilde{G}_n ;\mathbb{F}_2) \xrightarrow{0} H^j(G_n ; \mathbb{F}_2) \xrightarrow{-\cup \operatorname{w}_1}
    \]

    \(H^{\ast}(G_n ; \mathbb{F}_2) = \mathbb{F}_2 [\operatorname{w}_1 , \cdots , \operatorname{w}_n] \). This is a polynomial ring, so multiplying by \(\operatorname{w}_1\) is injective.

    Thus, \(- \cup \operatorname{w}_1\) is injective.

    \begin{theorem}
        [12.4] \(H^{\ast}(\widetilde{G}_n ; \mathbb{F}_2) / \langle \operatorname{w}_1 \rangle = \mathbb{F}_2 [\operatorname{w}_2 , \cdots , \operatorname{w}_3]\) 
    \end{theorem}

    Remark: there also exists Euler class \(e(\widetilde{\gamma}_n) \in H^n(\widetilde{G}_n ; \mathbb{Z})\) 

    If we have an oriented v.b. \(\xi\), then \(e(\xi) \in H^n(B; \mathbb{Z})\). \(n\) odd means \(2e(\xi) = 0\).

    Q(Davis): Find example where \(e(\xi) \neq 0 , n\) odd.

    A(Mandell): \(\xi = \widetilde{\gamma}_3\), oriented grassmanian of \(3\)-planes in \(\mathbb{R}^{\infty}\). 

    \(e(\widetilde{\gamma}_3) \in H^3(\widetilde{G}_3 ; \mathbb{Z})\)
    
    \(0 \neq e(\widetilde{\gamma}_3) \xrightarrow{mod} \operatorname{w}_3 = \operatorname{w}_3 (\widetilde{\gamma}_3) \neq 0\).

    Puzzles:

    1. What \(2\)-dimensional real planes in \(\mathbb{C}^n\) are complex lines?

    2. P176:

    \[
        \begin{tikzcd}
            & \text{real} \ar[rdd, bend left] \\ \\ \text{oriented} \ar[uur, bend left] & & \text{complex} \ar[ll, bend left]    
        \end{tikzcd}
    \]

    \subsection*{\(\mathbb{C}^n\)-bundle}

    \[
        \begin{tikzcd}
            w & \mathbb{C}^n \ar[r] & E \ar[d,"\pi"] \\ & & B
        \end{tikzcd}
    \]

    MS Definition, of Steenrod \(\operatorname{GL}_n(\mathbb{C}, \mathbb{C}^n)\)-bundle

    Complex projective space \(\mathbb{C} P^n = G_1(\mathbb{C}^{n+1})\).

    Complex Grassmanian \(G_n(\mathbb{C}^{n+k})\)

    Tautological bundle:

    \[
        \begin{tikzcd}
            \mathbb{C}^n \ar[r] & E(\gamma_n) \ar[r, "\subset", phantom] \ar[d] & G_n(\mathbb{C}^{n+k}) \times \mathbb{C}^{n+k} \\ & G_n(\mathbb{C}^{n+k})
        \end{tikzcd}
    \]

    Universal bundle:
    
    \[
        \begin{tikzcd}
            \mathbb{C}^n \ar[r] & E(\gamma^n) \ar[d] \\ & G_n(\mathbb{C}^{\infty})
        \end{tikzcd}
    \]

    \(H^{\ast}(G_n \mathbb{C}^{\infty})\) characterstic classes, \(\mathbb{C}^n\)-bundle.

    \(H^{\ast}(G_n \mathbb{C}^{\infty}, \mathbb{Z}) = \mathbb{Z}[c_1 , c_2 , \cdots , c_n]\) are called Chern Classes.

    \(\vert c_i \vert = 2i\).

    \[
        \begin{tikzcd}
            \mathbb{C}^n \text{-bundle} \ar[r] & \mathbb{R}^{2n} \text{-bundle} \\
            w \ar[r, mapsto] & \eval{w}_{\mathbb{R}}  
        \end{tikzcd}
    \] 

    \begin{definition}
        A complex structure on \(\xi: \mathbb{R}^n \to E \to B\) is a bundle map:

        \[
            \begin{tikzcd}
                E(\xi) \ar[rr,"J"] \ar[rd] & & E(\xi) \ar[ld] \\ & B 
            \end{tikzcd}
        \]

        such that \(J^2 = -\operatorname{id}\). i.e. \(J(J(v)) = -v\). 
    \end{definition}

    \[
        \begin{tikzcd}
            \text{complex vector bundle} \ar[r] & \text{real vector bundle with complex structure} \ar[l]  
        \end{tikzcd}
    \]

    \section*{Monday, 11/17/2025}
    
    1. \(\operatorname{Spin} (n) \to \operatorname{SO} (n)\)

    2. BoH Periodicity:

    \[
        \pi_i O = \begin{dcases}
            \mathbb{Z} / 2, &\text{ if } i \equiv 0 (8) ;\\
            \mathbb{Z} / 2, &\text{ if } i \equiv  1(8) ;\\
            0, &\text{ if } i \equiv 2 (8); \\
            \mathbb{Z}, &\text{ if } i \equiv 3 (8); \\
            0, &\text{ if } i \equiv 4 (8); \\
            0, &\text{ if } i \equiv 5 (8); \\
            0, &\text{ if } i \equiv 6 (8); \\
            \mathbb{Z}, &\text{ if } i \equiv 7 (8); \\
        \end{dcases}
    \]

    \(\pi_1 U = \begin{dcases}
        0, &\text{ if } i\equiv 0(2) ;\\
        \mathbb{Z}, &\text{ if } i\equiv 1(2).
    \end{dcases}\) 

    3. Splitting principal

    \[
        \begin{tikzcd}
            L_1 \oplus \cdots \oplus L_n \ar[r] \ar[d] & E \ar[d] \\
            B^{\prime} \ar[r,"f"] & B
        \end{tikzcd}
    \]

    \(f^{\ast}\) injective.

    \subsection*{Homotopy}

    \(\pi_i(X , x_0) = [(S^i , \ast) , (X, x_0)]\).

    \(i = 0: \pi_0 \leftrightarrow\) path-component of \(X\).

    \(i \geq 2\): Abelian group.

    Suppose \(X\) is path connected.

    Path \(\gamma : I \to X\) with \(\gamma_{\ast} : \pi_i(X, \gamma(0)) \xrightarrow{\approx} \pi_i(X, \gamma(1))\). So we can omit \(x_0\) from the definition. We can go wrong sometimes, but we won't worry about it.

    Addition structure:

    \begin{figure}[H]
        \centering
        \includegraphics[width=0.6\textwidth]{img/homotopygroup}
    \end{figure}

    \(\pi_i GL_n(\mathbb{R}) = \operatorname{Vect}_n(S^{i+1})\) isomorphisom classes.
    
    \(\operatorname{Vect}_n(S^{i+1})\) is \(\begin{tikzcd} \mathbb{R}^n \ar[r] & E \ar[d] \\ & S^{i+1}  \end{tikzcd}\) 

    \begin{proof}
        [Proof 1] Clutching.

        \(\eval{\xi}_{H^{i+1}_+}\) and \(\eval{\xi}_{H^{i+1}_-}\) are trivial. \(\xi\) is given, \(S^i \to GL_n(\mathbb{R})\) by gluing.
    \end{proof}

    \begin{proof}
        [Proof 2]

        \[
            \begin{tikzcd}
                GL_n \ar[r] & EGL_n \ar[d] \ar[r,"\simeq", phantom] & \ast \\ & BGL_n
            \end{tikzcd}
        \]

        Then \(\operatorname{Vect}_n(S^{i+1}) \overset{C.S.}{=} \pi_{i+1} BGL_n \underset{LES}{\cong} \pi_i GL_n\) 
    \end{proof}

    In general \([X, BG] \cong\) Iso class of \((G,F)\)-bundle \(/ X\).

    \subsection*{Classifying Spaces}

    We have the following groups:

    \[
        \begin{tikzcd}
            \operatorname{SO}(n) \ar[r, hook] \ar[d, hook] & \operatorname{GL}_n^+(\mathbb{R}) \ar[d,hook] \\ O(n) \ar[r,hook] & \operatorname{GL}_n(\mathbb{R})
        \end{tikzcd}
    \]

    \(\operatorname{GL}_n^+(\mathbb{R})\) corresponds to orientable bundles.

    \(\operatorname{O}(n)\) corresponds to metrics.

    Claim: the horizontal maps are homotopy equivalent

    \begin{proof}
        Polar decomposition: \(A \in \operatorname{GL}_n(\mathbb{R}) \implies A = PO\) where \(P\) is `positive' [i.e. symmetric and positive definite] and \(O \in \operatorname{O}(n)\).

        Then \(\operatorname{O}(n)\) is a deformation retract of \(\operatorname{GL}_n\) by

        \[
            ((1-t) P + t I)O
        \]
    \end{proof}

    \begin{corollary}
        \(\operatorname{BO}(n) \simeq \operatorname{BGL}_n\mathbb{R}\) 
    \end{corollary}

    Every bundle over CW-complex admit a metric / unique upto isometry.

    \begin{theorem}
        \(\operatorname{SO}(n)\) is path-connected, \(\pi_0 \operatorname{O}(n) \xrightarrow{\approx}[\det] \{ \pm 1 \} \).        
    \end{theorem}

    \begin{proof}
        Pick \(0 \neq a \in \mathbb{R}^n\). Look at reflection through \(a^{\perp}\). Call it \(R_a\).
        
        Then \(R_a : \mathbb{R}^n \to \mathbb{R}^n , \eval{R_a}_{a^{\perp}} = \operatorname{id}, R_a(a) = -a\).

        First, if \(O \in \operatorname{O}(n)\) then \(O\) is a product of reflection.
        
        Second, if \(S \in \operatorname{SO}(n)\) then \(S\) is a product of even number of reflection.

        Third, if \(a,b\) are linearly independent then \(R_a \simeq R_b\) via \(R_{ta+(1-t)b}\).

        Fourth, \(R_a R_b \simeq R_a R_a =\operatorname{id}\).

        This proves the problem. Note that \(A A^t = 1 \implies (\det A)^2 = 1 \implies \det A \in \{ \pm 1 \} \).
    \end{proof}

    Then \(\operatorname{SO}(n)\) is path-connected and \(\operatorname{O}(n)\) has two path components.

    \section*{Wednesday, 11/19/2025}
    
    Let \(R = \begin{bmatrix}
        -1 &  &  &  \\
         & 1 &  &  \\
         &  & \ddots &  \\
         &  &  & 1 \\
    \end{bmatrix} \) 

    Then we have the following split exact sequence:

    \[
        \begin{tikzcd}
            1 \ar[r] & \operatorname{SO}(n) \ar[r] & \operatorname{O}(n) \ar[r,"\det"] & \{ \pm 1 \} \ar[r] \ar[l, bend left, dotted] & 1\\
            & & R & -1 \ar[l, mapsto]
        \end{tikzcd}
    \]

    Then \(\operatorname{O}(n) = \operatorname{SO}(n) \rtimes \{ \pm 1 \}.\) 

    \(\pi_0 \operatorname{O}(n) = \{ \pm 1 \}\).

    \(\operatorname{SO}(1) = \{ 1 \}, \operatorname{O}(1) = \{ \pm 1 \}\).

    \(\operatorname{SO}(2) = S^1 , \operatorname{O}(2) = S^1 \rtimes \{ \pm 1 \}\) the dihedral group.

    \begin{lemma}
        \(\operatorname{SO}(3) \cong \mathbb{R}P^3\).
    \end{lemma}

    \begin{proof}
        [Proof 1] \(A\in \operatorname{SO}(3)\). Then the characteristic polynomial is of degree \(3\). Thus, \(A\) has a real eigenvalue.
        
        Since \(A\in \operatorname{SO}(3)\) the eigenvalue \(\lambda = \pm 1\).
        
        Case 1: all eigenvalues are real. \(\begin{bmatrix}
            -1 &  &  \\
             & -1 &  \\
             &  & -1 \\
        \end{bmatrix}\notin \operatorname{SO}(3)\).
        
        Case 2: Other eigenvalues are non-real. Then \(\lambda = \pm 1, \mu , \overline{\mu}\) with \(\lambda \mu \overline{\mu} = 1 \implies \lambda = 1\).

        Thus, there exists `axis' \(v\) such that \(Av = v\) with \(\overline{v} = 1\).

        i.e. \(A\) is a rotation about axis \(v\) through angle \(0 \leq \theta \leq \pi\).

        \[
            \operatorname{SO}(3) \xrightarrow{\approx} D^3 / \sim = \mathbb{R} P^3
        \]

        \[
            A \mapsto \frac{\theta}{\pi} v
        \]
    \end{proof}

    \begin{proof}
        [Proof 2]

        \(S^3 =\) unit quarternions \(= \{ a + bi + cj + dk \mid a^2 + b^2 + c^2 + d^2 = 1 \}\).

        Claim: \(S^3\) is a double cover of \(\operatorname{SO}(3)\). We essentially have to prove that:

        \[
            \begin{tikzcd}
                1 \ar[r] & \{ \pm 1 \} \ar[r] & S^3 \ar[r] & \operatorname{SO}(3) \ar[r] & 1 \\
                & & z \ar[r, mapsto] & (bi + cj + dk \mapsto z(bi + cj + dk) \overline{z})
            \end{tikzcd}
        \]
    \end{proof}

    \begin{lemma}
        [Stability Lemma]

        Recall \(\operatorname{SO}(n) \hookrightarrow \operatorname{SO}(n+1) \hookrightarrow \cdots\) by \(A \mapsto \begin{bmatrix}
            A & 0 \\
            0 & 1 \\
        \end{bmatrix} \).

        \begin{enumerate}[label=\alph*)]
            \item \(\pi_{n-1} \operatorname{SO}(n) \twoheadrightarrow \pi_{n-1} \operatorname{SO}(n+1) \xrightarrow{\approx} \pi_{n-1} \operatorname{SO}(n+2) \xrightarrow{\approx}\) 
            \item \(\pi_n \operatorname{BSO}(n) \twoheadrightarrow \pi_n \operatorname{BSO}(n+1) \xrightarrow{\approx} \pi_r \operatorname{BSO}(n+2) \xrightarrow{\approx} \)  
        \end{enumerate} 
    \end{lemma}

    \begin{proof}
        Fiber bundle.

        \[
            \begin{tikzcd}
                \operatorname{SO}(n) \ar[r] & \operatorname{SO}(n+1) \ar[d] & A \ar[d,mapsto] \\ & S^n & A \begin{bmatrix}
                    0 \\
                    \vdots \\
                    0 \\
                    1 \\
                \end{bmatrix}.
            \end{tikzcd}
        \]

        LES on \(\pi_{\ast}\) and \(\pi_1 S^n = 0\) for \(i < n\).

        \begin{enumerate}[label=\alph*)]
            \item LES on \(\pi_{\ast}\) and \(\pi_1 S^n = 0\) for \(i < n\).
            \item  
            \[
                \begin{tikzcd}
                    \operatorname{SO}(n) \ar[r] & \operatorname{ESO}(n) \ar[r,phantom,"\simeq"] \ar[d] & \ast \\ & \operatorname{BSO}(n)
                \end{tikzcd}
            \]

            \(\pi_i \operatorname{BSO}(n) \xrightarrow{\approx} \pi_{i-1} \operatorname{SO}(n)\).
        
        \end{enumerate} 
    \end{proof}

    Example:

    \[
        \begin{tikzcd}
            \pi_1 \operatorname{SO}(2) \ar[r] & \pi_1 \operatorname{SO}(3) \ar[r,"\approx"] & \pi_1 \operatorname{SO}(4) \ar[r] & \;\\
            TS^2 \ar[r,mapsto] & 0
        \end{tikzcd}
    \]

    \(\pi_1 \operatorname{SO(n)} = \mathbb{Z}_2 \) for \(n > 2\). \(\pi_1 \operatorname{SO}(2) = \mathbb{Z}\).
    
    We define \(\operatorname{Spin}(n)\) as connected double group of \(\operatorname{SO}(n)\).
    
    \(\operatorname{Spin}(3) = S^3\).

    \[
        \begin{tikzcd}
            0 \ar[r] & \{ \pm 1 \} \ar[r,"\Delta"] & S^3 \times S^3 \to \operatorname{SO}(4) \ar[r] & 1 \\
            & & (z,w) \ar[r, mapsto] & (v \mapsto zw \overline{v})
        \end{tikzcd}
    \]

    \(\operatorname{Spin(4)} = S^3 \times S^3\).
    
    Spin structure on \(\xi = \mathbb{R}^n \to E \to B\) or vector bundle with metrics where \(B\) is path-connected.

    \(P_{\operatorname{SO}} = \left\{ (e_1 , \cdots , e_n) \mid \pi (e_i) = \pi(e_i) = \pi(e_j), \text{ orthonormal}  \right\} \) 

    Then we can define spin structure to \(\operatorname{Spin}n\). i.e.

    principal \(\operatorname{Spin}(n)\):
    
    \[
        \begin{tikzcd}
            \operatorname{Spin}(n) \ar[r] & P_{\operatorname{Spin} (n)} \ar[d] \\ & B
        \end{tikzcd}
    \]

    Furthermore,

    \[
        \begin{tikzcd}
            P_{Spin} \times_{Spin} \operatorname{SO} \ar[rr, "\approx"] \ar[rd] && \operatorname{P}_{\operatorname{SO}} \ar[ld] \\ & B
        \end{tikzcd}
    \]

    \(\iff\) the following happens:

    \[
        \begin{tikzcd}
            P \ar[rr] \ar[rd] & & P_{\operatorname{SO} } \ar[ld] \\ & B
        \end{tikzcd}
    \]
   
    \[
        \begin{tikzcd}
            \operatorname{Spin} \ar[rr, dash] \ar[d] & & \operatorname{SO} \ar[d] \\
            P_{\operatorname{spin}} \ar[rr] \ar[rd] & & P_{\operatorname{SO}} \ar[dl] \\
            & B
        \end{tikzcd}
    \]

    Deine: \(\operatorname{Spin}(n)\) as conected double cover of \(\operatorname{Spin}(n)\) 

    \begin{theorem}
        \(\xi\) admits a spin structure \(\iff \operatorname{w}_2 \xi = 0\).

        If \(\xi\) admits a spin structure then,

        \[
            \text{spin structures} \leftrightarrow H^1(B;\mathbb{Z}_2) 
        \]
    \end{theorem}

    \begin{proof}
        \[
            \begin{tikzcd}
                \mathbb{R} P^{\infty} \ar[r] & \operatorname{BSpin}(n) \ar[d] \\
                B \ar[ru, dotted] \ar[r] & \operatorname{BSO}(n) 
            \end{tikzcd}
        \]
    \end{proof}

    \section*{Monday, 12/1/2025}
    
    Let \(\xi = \mathbb{R}^n \to E \to B\) be oriented with metric.

    \begin{theorem}
        \(\xi\) admits a spin structure iff \(\operatorname{w}_2(\xi) = 0\).

        If so, spin structure on \(\xi \leftrightarrow H^1(B;\mathbb{Z}_2)\).
    \end{theorem}

    Consider the \textit{frame bundle}.

    \[
        \begin{tikzcd}
            \operatorname{SO}(n) = \operatorname{SO} \ar[r] & P_{\operatorname{SO}} \ar[d,"p"] \ar[r,"=",phantom] & \{ (e_1 , \cdots , e_n) \mid p(e_i) = p(e_j), e_i \text{ O.N.}  \} \subset E \times \cdots \times E \\
            & B 
        \end{tikzcd}
    \]

    spin structure on \(\xi \leftrightarrow \alpha \in H^1(P_{\operatorname{SO}};\mathbb{Z}_2)\) such that \(i^{\ast} \alpha \neq 0\).

    \(\leftrightarrow \alpha : \pi_1 P_{\operatorname{SO}} \to \mathbb{Z}_2\) such that \(\alpha \circ i \neq 0\). 

    This gives rise to the double cover \(\begin{tikzcd}
        P_{\text{spin}} \ar[d] \\ P_{\operatorname{SO}}
    \end{tikzcd}\) 

    Given the fibration, we have the Serre \(5\)-term exact sequence [with \(\mathbb{Z}_2\)-coefficients]

    \[
        \begin{tikzcd}
            H^1 B \ar[r] & H^1 P_{\operatorname{SO}} \ar[r] & H^1 \operatorname{SO} \ar[r,"\delta_3"] \ar[d,phantom,sloped,"="] & H^2 B \ar[r] & H^2 P_{\operatorname{SO}} \\
            & & \{ 0, g \} 
        \end{tikzcd}
    \]

    This is a consequence of the Serre Spetral Sequence.

    Claim: \(\delta_3(g) = \operatorname{w}_2(\xi)\).

    Proof: (i): \(\delta_3(g)\in H^1 B\) is a characteristic class for oriented vector bundle with metric [everything natural, we have a pullback].

    (ii): `universal case': \(\begin{tikzcd}
        \operatorname{SO}(n) \ar[r] & \operatorname{ESO}(n) \ar[d] \ar[r,phantom,"\simeq"] & \ast \\
        & \operatorname{BSO}(n) \ar[r,phantom,"="] & \widehat{G}_n  
    \end{tikzcd}\) 

    \[
        \begin{tikzcd}
            0 \ar[r] & H^1(\operatorname{SO}) \ar[r,tail,"\approx"] & H^2(\operatorname{BSO}(n)) \\
            & (0,g) & (0,\operatorname{w}_2)
        \end{tikzcd}
    \]

    END OF SPIN!

    Recall stability lemma:

    \[
        \begin{tikzcd}
            \pi_k \operatorname{O}(k+1) \ar[r, two heads] \ar[d,phantom,sloped,"="] & \pi_k \operatorname{O}(k+2) \ar[r,"\approx"] \ar[d,phantom,sloped,"="] & \pi_k \operatorname{O}(k+3) \ar[r,"\approx"] \ar[d,phantom,sloped,"="] & \, \\
            \pi_{k+1} \operatorname{BO}(k+1) \ar[r, two heads] & \pi_{k+1} \operatorname{BO}(k+2) \ar[r] & \pi_{k+1} \operatorname{BO}(k+3) \ar[r,"\approx"] & \,
        \end{tikzcd}
    \]

    For example,

    \[
        \begin{tikzcd}
            \pi_1 \operatorname{O}(2) \ar[r] & \pi_1 \operatorname{O}(3) \ar[r,"\approx"] & \pi_1 \operatorname{O}(4) \ar[r] & \, \\
            \mathbb{Z} \ar[r] & \mathbb{Z}_2 \ar[r,phantom,"="] & \mathbb{Z}_2
        \end{tikzcd}
    \]

    Corollary: \(\pi_2 \operatorname{O}(k) = 0\) for \(k \gg 0\).

    \begin{corollary}
        Let \(B\) be CW complex.

        \begin{enumerate}[label=\alph*)]
            \item \(\begin{tikzcd}
                \xi = \mathbb{R}^n \ar[r] & E \ar[d] \\ & B
            \end{tikzcd}\, n > \dim B\).
            
            \(\implies \exists\) nowhere zero secttion (\(\iff \xi = \alpha \oplus \epsilon\)).

            \item \(\begin{tikzcd}
                \xi , \eta = \mathbb{R}^n \ar[r] & E \ar[d] \\ & B
            \end{tikzcd}\; n > \dim B\).
            
            \(\xi \oplus \epsilon \cong \eta \oplus \epsilon\) [stability isomorphism] \(\implies \xi \cong \eta\) isomorphism.
        \end{enumerate} 
    \end{corollary}

    Now we can define stably orthonormal group:

    \[
        \operatorname{O} = \operatorname{colim}_{n \to \infty} O(n) (= \bigcup_{n} O(n) \text{ with topology})
    \]

    Then \(\pi_k \operatorname{O} = \pi_k \operatorname{O}(n)\) for \(n \geq k + 2\).

    Then we have Bott periodicity

    \[
        \pi_k \operatorname{O} = \begin{dcases}
            \mathbb{Z}_2, &\text{ if } k\equiv 0(8) ;\\
            \mathbb{Z}_2, &\text{ if }k\equiv 1(8)  ;\\
            0, &\text{ if } k\equiv 2(8);\\
            \mathbb{Z}, &\text{ if } k\equiv 3(8);\\
            0, &\text{ if } k\equiv 4(8);\\
            0, &\text{ if } k\equiv 5(8);\\
            0, &\text{ if } k\equiv 6(8);\\
            \mathbb{Z}, &\text{ if } k\equiv 7(8).
        \end{dcases}
    \]

    \[
        \pi_k U = \begin{dcases}
            0, &\text{ if } k\equiv 0(2) ;\\
            \mathbb{Z}, &\text{ if } k\equiv 1(2).
        \end{dcases}
    \]

    For \(k \leq 7\), the generators are all Hopf bundles over \(S^{k+1}\). There are \(4\) hopf bundles (reals, complex, quarternions, octonions) and they correspond to the non-zero \(\pi_k \operatorname{O}\).

    Canonical example: \(k = 1\).

    \[
        \begin{tikzcd}
            S^1 \ar[r] & {S^3 \; (z_1, z_2)} \ar[d] \\
            & {\mathbb{C} P^1\; [z_1:z_2]} \ar[r,phantom,"\cong"] & S^2
        \end{tikzcd}
    \]

    \[
        \begin{tikzcd}
            \mathbb{C} \ar[r] & E(\gamma^1) \ar[d] \\
            & \mathbb{C} P^1
        \end{tikzcd}
    \]

    \[
        \begin{tikzcd}
            S^3\ar[d] & {(z_1, z_2)} \\
            \mathbb{C} \cup  \infty  & z_1 / z_2
        \end{tikzcd}
    \]

    \begin{theorem}
        [Splitting Principle]

        We can have splitting principles for real bundles \(\xi  = \mathbb{R}^n \to E \to B\) or complex bundles \(\mathbb{C}^n \to E^{\prime} \to B^{\prime}\).

        Assume \(B, B^{\prime}\) are CW. Splitting principle says \(\exists\) maps \(S \xrightarrow{f} B , S^{\prime} \xrightarrow{f^{\prime}} B^{\prime}\) such that:

        \begin{enumerate}[label=\arabic*)]
            \item \(f^{\ast} E = L_1 \oplus \cdots \oplus L_n\) and \(f^{\prime \ast} E^{\prime} = L^{\prime}_1 \oplus \cdots \oplus L^{\prime}_n\), i.e. direct sum of line bundles.
            \item These maps are cohomology injections: \(f^{\ast} : H^{\ast} (B;\mathbb{F}_2) \rightarrowtail H^{\ast} (S;\mathbb{F}_2), f^{\prime \ast} : H^{\ast} (B^{\prime} ; \mathbb{Z}) \to H^{\ast} (S^{\prime} ; \mathbb{Z})\).
        \end{enumerate} 
    \end{theorem}


    idea: We can pretend every vector bundle is a sum of line bundle.

    For existence of SW (and chern) classes: 
    
    Instead of Steenrod squares, we can try to take \(f^{\ast} \operatorname{w}(E) = \operatorname{w}(L_1) \cdots \operatorname{w}(L_n)\).

    These are just line bundles so we can define them by orientations.

\end{document}