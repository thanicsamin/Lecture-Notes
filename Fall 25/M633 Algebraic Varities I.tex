\documentclass{article}
\usepackage{amsmath, amsthm, amssymb, amsfonts, mathtools,enumitem, stmaryrd,physics, cancel, tikz-cd, graphicx, float, booktabs, xurl}
\usetikzlibrary{arrows}
\usepackage{geometry}
    \geometry{
        a4paper,
        left = 40mm,
        top = 20mm,
        right = 40mm,
        bottom = 30mm
    }
\setlength{\parindent}{0pt}

\theoremstyle{definition}
\newtheorem{problem}{Problem}
\newtheorem{solution}{Solution}
\newtheorem*{example}{Example}
\newtheorem*{exercise}{Exercise}
\newtheorem*{definition}{Definition}
\newtheorem{theorem}{Theorem}
\newtheorem*{theorem*}{Theorem}
\newtheorem{proposition}[theorem]{Proposition}
\newtheorem*{proposition*}{Proposition}
\newtheorem{lemma}[theorem]{Lemma}
\newtheorem*{lemma*}{Lemma}
\newtheorem{corollary}[theorem]{Corollary}
\newtheorem*{corollary*}{Corollary}
\newtheorem*{remark}{Remark}

\title{M633 Algebraic Varities I}
\author{Thanic Nur Samin}
\date{}

\begin{document}
    \maketitle

    \section*{Monday, 8/25/2025}
    
    Textbook: The Rising Sea.
    
    Link: \url{https://math.stanford.edu/~vakil/216blog/FOAGjul2724public.pdf}

    It's basically an introduction to the scheme point of view for algebraic geometry.

    Grothendieck POV: if we get the definitions right, hard problems will become easier. We are trying to get the definition here.

    \section*{Motivation / A Pseudo History}

    Let \(X\) be a topological space (Compact, Hausdorff).

    \(C(X)=\) ring of real-valued continuous functions.

    Question: What are the maximal ideals of \(C(X)\)?

    Fact: TFAE: given a ring \(A\),

    \begin{enumerate}[label=\roman*)]
        \item \(I\) is maximal among proper ideals of \(A\)
        \item \(A / I\) is a field
        \item There exists a surjective homomorphism from \(A\) to a field \(F\), \(\phi : A \to F\) such that \(I = \ker \phi\)  
    \end{enumerate} 

    If \(x_0\in X\), we have the ring homomorphism \(\operatorname{eval}_{x_0}: f \mapsto f(x_0)\).

    This is an \(\mathbb{R}\)-linear map.

    This is also obviously surjective.

    Thus, \(\ker \operatorname{eval}_{x_0} = \left\{ f\in C(x) \mid f(x_0) = 0 \right\}\). This is a maximal ideal. In fact,

    \begin{theorem}
        All maximal ideals of \(C(X)\) are of the form \(\ker \operatorname{eval}_{x_0}\).
    \end{theorem}

    \begin{proof}
        Suppose not. Let \(\phi : C(X) \to F\) be a surjective homomorphism to a field, and let \((f_1, f_2, f_3, \cdots) = \ker \phi\).
        
        If \(\ker \phi \neq \ker \operatorname{eval}_{x_0} \implies \exists f_{x_0}\in \ker \phi\) such that \(f_{x_0}(x_0)\neq 0\). This is true for each point \(x_0\in X\).

        Therefore, \(f_{x_0}(x)\neq 0\) for all \(x\in U_{x_0}\) where \(U_{x_0}\) is an open neighborhood of \(x_0\).

        Since \(X\) is compact, there is \(x_1, \cdots , x_n \in X\) so that \(U_{x_j}\) cover \(x\).

        \(f_1, \cdots , f_n \in \ker \phi\) such that \(f_i(x) \neq 0 \forall x\in U_{x_i}\).

        Then \(f(x) \coloneqq \sum_{i} f_i^2(x) > 0\) for all \(x\in X\). Then \(\frac{1}{f}\in C(X) \implies 1\in \ker \phi\). Contradiction!  
    \end{proof}

    Given \(f\in C(X)\) define \(Z(f) = f ^{-1} (0)\). This is a closed subset of \(X\).
    
    We can do abuse of notation and say \(X = \operatorname{Max}(C(X))\).

    Then \(Z(f) = \left\{ \mathfrak{m} \in \operatorname{Max} (C(X)) \mid f\in \mathfrak{m} \right\} \) 

    Then, \(Z(f)^c\) open in \(X\)

    \(= \left\{ \mathfrak{m} \in \operatorname{Max} (C(X)) \mid f\notin \mathfrak{m} \right\} \) 

    We have successfully turned a topological space into a ring.

    If we have \(X \xrightarrow{cont} Y\) we have \(C(Y) \to C(X)\).

    Instead of arbitrary topological spaces, now we focus on \(\mathbb{C}^n\).

    Lets look at polynomials \(\mathbb{C}^n \to \mathbb{C}\).

    Ring of polynomial functions is \(\mathbb{C} [x_1, \cdots , x_n]\).

    \begin{theorem}
        [Weak Hilbert Nullstellensatz] Maximal ideals of this ring are exactly the kernels of evaluation maps at points \((a_1, \cdots , a_n) \in \mathbb{C}^n\).
    \end{theorem}

    Note that \(x_1 - a_1, \cdots , x_n - a_n\in \ker \operatorname{eval}_{(a_1, \cdots , a_n)}\). In fact, \(\ker \operatorname{eval}_{(a_1, \cdots , a_n)} = (x_1 - a_1, \cdots , x_n - a_n)\).

    \begin{proof}
        WLOG \(a_1 = \cdots = a_n = 0\). Then \(\ker \operatorname{eval}_{(0,\cdots ,0)}\) are exactly the polynomial with no constant term, which is exactly \((x_1, \cdots , x_n)\).
    \end{proof}

    Now we prove weak Nullstellensatz.

    \begin{proof}
        Let \(\mathfrak{m} \subset \mathbb{C} [x_1, \cdots , x_n]\) be a maximal ideal. Then \(F = \mathbb{C}[x_1, \cdots , x_n] / \mathfrak{m}\) is a field extension of \(\mathbb{C}\). So, \(F\) is transcendental. Choose \(x\in F \setminus \mathbb{C}\). Then \(x\) generates a subfield \(\mathbb{C} (x)\).

        Then, \(\dim _{\mathbb{C} \text{-v.s.}} \mathbb{C}(x)\) is uncountable. To prove this, note that \(\left\{ \frac{1}{x-c} \mid c\in \mathbb{C} \right\}\) are linearly independent.
        
        However, \(\dim_{\mathbb{C}\text{-v.s.}} \mathbb{C}[x_1, \cdots, x_n]\) is countable. 
    \end{proof}

    Given a system of polynomial equations:

    \(f_1 (x_1, \cdots , x_n) = 0\)
    
    \(\vdots\)
    
    \(f_m(x_1, \cdots , x_n) = 0\) 

    Find or describe the set of complex solutions.

    We want to find all \(\mathfrak{m} \in \operatorname{Max} (\mathbb{C} [x_1, \cdots , x_n])\) such that \(f_1, \cdots , f_m \in \mathfrak{m}\).

    Define \(I = (f_1, \cdots , f_m)\). Then, we want \(\{ \mathfrak{m} \mid I \subset \mathfrak{m} \}\).

    We have turned the problem of finding solutions to finding maximal ideal containing a certain ideal.

    From the theorem about order preserving bijection of ideal containing ideal and quotient,

    We want all maximal ideals \(\overline{\mathfrak{m}}\) in \(\mathbb{C} [x_1, \cdots , x_n] / I\).

    We want to do the most general thing. There is nothing special about polynomials!

    Let \(A\) be a commutative ring.  We think of \(\operatorname{Max}(A)\) as the associated space.

    \textbf{If somebody gives us a ring \(A\), we want to think of it as a ring of function on a space. \(\operatorname{Max}(A)\) is that space.}

    There is a problem with this idea: We would like to be able to go from \(\operatorname{Max}(A) \to \operatorname{Max}(B)\) whenever we have a ring homomorphism \(f: B \to A\).

    Suppose \(\mathfrak{m} \subset A\). We want t have \(f ^{-1} (\mathfrak{m})\). We want this to be maximal. It is not always maximal!
    
    Suppose we have a homomorphism \(\mathbb{C} [x] \hookrightarrow \mathbb{C} (x)\).

    There is only one maximal ideal on \(\mathbb{C}(x)\). It is \((0)\). Then \(f ^{-1} ((0)) = (0)\) but \((0)\) is not a maximal ideal in \(\mathbb{C}[x]\).

    The solution is to not use \(\operatorname{Max}(A)\), but rather \(\operatorname{Prime}(A)\).

    Let \(f: B \to A\) be a homomorphism and let \(P \subset A\) be a prime ideal.

    Claim: \(f^{-1} (P)\) is also prime.

    Proof: \(xy\in f ^{-1} (P) \implies f(xy)\in P \implies f(x)f(y)\in P \implies f(x) \in P \lor f(y)\in P \implies x\in f ^{-1} (P) \lor y\in f ^{-1} (P)\).

    This works! But how does this mess up the space? What additional points do we have?

    

\end{document}