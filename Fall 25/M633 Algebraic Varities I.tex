\documentclass{article}
\usepackage{amsmath, amsthm, amssymb, amsfonts, mathtools,enumitem, stmaryrd,physics, cancel, tikz-cd, graphicx, float, booktabs, xurl}
\usetikzlibrary{arrows}
\usepackage{geometry}
    \geometry{
        a4paper,
        left = 40mm,
        top = 20mm,
        right = 40mm,
        bottom = 30mm
    }
\setlength{\parindent}{0pt}

\theoremstyle{definition}
\newtheorem{problem}{Problem}
\newtheorem{solution}{Solution}
\newtheorem*{example}{Example}
\newtheorem*{exercise}{Exercise}
\newtheorem*{definition}{Definition}
\newtheorem{theorem}{Theorem}
\newtheorem*{theorem*}{Theorem}
\newtheorem{proposition}[theorem]{Proposition}
\newtheorem*{proposition*}{Proposition}
\newtheorem{lemma}[theorem]{Lemma}
\newtheorem*{lemma*}{Lemma}
\newtheorem{corollary}[theorem]{Corollary}
\newtheorem*{corollary*}{Corollary}
\newtheorem*{remark}{Remark}

\title{M633 Algebraic Varities I}
\author{Thanic Nur Samin}
\date{}

\begin{document}
    \maketitle

    \section*{Monday, 8/25/2025}
    
    Textbook: The Rising Sea.
    
    Link: \url{https://math.stanford.edu/~vakil/216blog/FOAGjul2724public.pdf}

    It's basically an introduction to the scheme point of view for algebraic geometry.

    Grothendieck POV: if we get the definitions right, hard problems will become easier. We are trying to get the definition here.

    \section*{Motivation / A Pseudo History}

    Let \(X\) be a topological space (Compact, Hausdorff).

    \(C(X)=\) ring of real-valued continuous functions.

    Question: What are the maximal ideals of \(C(X)\)?

    Fact: TFAE: given a ring \(A\),

    \begin{enumerate}[label=\roman*)]
        \item \(I\) is maximal among proper ideals of \(A\)
        \item \(A / I\) is a field
        \item There exists a surjective homomorphism from \(A\) to a field \(F\), \(\phi : A \to F\) such that \(I = \ker \phi\)  
    \end{enumerate} 

    If \(x_0\in X\), we have the ring homomorphism \(\operatorname{eval}_{x_0}: f \mapsto f(x_0)\).

    This is an \(\mathbb{R}\)-linear map.

    This is also obviously surjective.

    Thus, \(\ker \operatorname{eval}_{x_0} = \left\{ f\in C(x) \mid f(x_0) = 0 \right\}\). This is a maximal ideal. In fact,

    \begin{theorem}
        All maximal ideals of \(C(X)\) are of the form \(\ker \operatorname{eval}_{x_0}\).
    \end{theorem}

    \begin{proof}
        Suppose not. Let \(\phi : C(X) \to F\) be a surjective homomorphism to a field, and let \((f_1, f_2, f_3, \cdots) = \ker \phi\).
        
        If \(\ker \phi \neq \ker \operatorname{eval}_{x_0} \implies \exists f_{x_0}\in \ker \phi\) such that \(f_{x_0}(x_0)\neq 0\). This is true for each point \(x_0\in X\).

        Therefore, \(f_{x_0}(x)\neq 0\) for all \(x\in U_{x_0}\) where \(U_{x_0}\) is an open neighborhood of \(x_0\).

        Since \(X\) is compact, there is \(x_1, \cdots , x_n \in X\) so that \(U_{x_j}\) cover \(x\).

        \(f_1, \cdots , f_n \in \ker \phi\) such that \(f_i(x) \neq 0 \forall x\in U_{x_i}\).

        Then \(f(x) \coloneqq \sum_{i} f_i^2(x) > 0\) for all \(x\in X\). Then \(\frac{1}{f}\in C(X) \implies 1\in \ker \phi\). Contradiction!  
    \end{proof}

    Given \(f\in C(X)\) define \(Z(f) = f ^{-1} (0)\). This is a closed subset of \(X\).
    
    We can do abuse of notation and say \(X = \operatorname{Max}(C(X))\).

    Then \(Z(f) = \left\{ \mathfrak{m} \in \operatorname{Max} (C(X)) \mid f\in \mathfrak{m} \right\} \) 

    Then, \(Z(f)^c\) open in \(X\)

    \(= \left\{ \mathfrak{m} \in \operatorname{Max} (C(X)) \mid f\notin \mathfrak{m} \right\} \) 

    We have successfully turned a topological space into a ring.

    If we have \(X \xrightarrow{cont} Y\) we have \(C(Y) \to C(X)\).

    Instead of arbitrary topological spaces, now we focus on \(\mathbb{C}^n\).

    Lets look at polynomials \(\mathbb{C}^n \to \mathbb{C}\).

    Ring of polynomial functions is \(\mathbb{C} [x_1, \cdots , x_n]\).

    \begin{theorem}
        [Weak Hilbert Nullstellensatz] Maximal ideals of this ring are exactly the kernels of evaluation maps at points \((a_1, \cdots , a_n) \in \mathbb{C}^n\).
    \end{theorem}

    Note that \(x_1 - a_1, \cdots , x_n - a_n\in \ker \operatorname{eval}_{(a_1, \cdots , a_n)}\). In fact, \(\ker \operatorname{eval}_{(a_1, \cdots , a_n)} = (x_1 - a_1, \cdots , x_n - a_n)\).

    \begin{proof}
        WLOG \(a_1 = \cdots = a_n = 0\). Then \(\ker \operatorname{eval}_{(0,\cdots ,0)}\) are exactly the polynomial with no constant term, which is exactly \((x_1, \cdots , x_n)\).
    \end{proof}

    Now we prove weak Nullstellensatz.

    \begin{proof}
        Let \(\mathfrak{m} \subset \mathbb{C} [x_1, \cdots , x_n]\) be a maximal ideal. Then \(F = \mathbb{C}[x_1, \cdots , x_n] / \mathfrak{m}\) is a field extension of \(\mathbb{C}\). So, \(F\) is transcendental. Choose \(x\in F \setminus \mathbb{C}\). Then \(x\) generates a subfield \(\mathbb{C} (x)\).

        Then, \(\dim _{\mathbb{C} \text{-v.s.}} \mathbb{C}(x)\) is uncountable. To prove this, note that \(\left\{ \frac{1}{x-c} \mid c\in \mathbb{C} \right\}\) are linearly independent.
        
        However, \(\dim_{\mathbb{C}\text{-v.s.}} \mathbb{C}[x_1, \cdots, x_n]\) is countable. 
    \end{proof}

    Given a system of polynomial equations:

    \(f_1 (x_1, \cdots , x_n) = 0\)
    
    \(\vdots\)
    
    \(f_m(x_1, \cdots , x_n) = 0\) 

    Find or describe the set of complex solutions.

    We want to find all \(\mathfrak{m} \in \operatorname{Max} (\mathbb{C} [x_1, \cdots , x_n])\) such that \(f_1, \cdots , f_m \in \mathfrak{m}\).

    Define \(I = (f_1, \cdots , f_m)\). Then, we want \(\{ \mathfrak{m} \mid I \subset \mathfrak{m} \}\).

    We have turned the problem of finding solutions to finding maximal ideal containing a certain ideal.

    From the theorem about order preserving bijection of ideal containing ideal and quotient,

    We want all maximal ideals \(\overline{\mathfrak{m}}\) in \(\mathbb{C} [x_1, \cdots , x_n] / I\).

    We want to do the most general thing. There is nothing special about polynomials!

    Let \(A\) be a commutative ring.  We think of \(\operatorname{Max}(A)\) as the associated space.

    \textbf{If somebody gives us a ring \(A\), we want to think of it as a ring of function on a space. \(\operatorname{Max}(A)\) is that space.}

    There is a problem with this idea: We would like to be able to go from \(\operatorname{Max}(A) \to \operatorname{Max}(B)\) whenever we have a ring homomorphism \(f: B \to A\).

    Suppose \(\mathfrak{m} \subset A\). We want t have \(f ^{-1} (\mathfrak{m})\). We want this to be maximal. It is not always maximal!
    
    Suppose we have a homomorphism \(\mathbb{C} [x] \hookrightarrow \mathbb{C} (x)\).

    There is only one maximal ideal on \(\mathbb{C}(x)\). It is \((0)\). Then \(f ^{-1} ((0)) = (0)\) but \((0)\) is not a maximal ideal in \(\mathbb{C}[x]\).

    The solution is to not use \(\operatorname{Max}(A)\), but rather \(\operatorname{Prime}(A)\).

    Let \(f: B \to A\) be a homomorphism and let \(P \subset A\) be a prime ideal.

    Claim: \(f^{-1} (P)\) is also prime.

    Proof: \(xy\in f ^{-1} (P) \implies f(xy)\in P \implies f(x)f(y)\in P \implies f(x) \in P \lor f(y)\in P \implies x\in f ^{-1} (P) \lor y\in f ^{-1} (P)\).

    This works! But how does this mess up the space? What additional points do we have?

    \section*{Wednesday, 8/27/2025}
    
    Now we go back to the textbook.

    We start with some category theory. For this course, categories will be locally small. The objects might not be sets, but hom-sets will be sets.

    Let \(\mathcal{C}, D\mathcal{C}\) be categories. \(\mathcal{C} \xrightarrow{F} \mathcal{D}\) morphisms.

    Let \(F: \operatorname{ob} \mathcal{C} \to \operatorname{ob} \mathcal{D}\).

    Suppose \(X,Y\in \operatorname{ob} \mathcal{C}\).

    \(\phi: X \to Y\) means \(\phi \in \operatorname{Mor}_{\mathcal{C}}(X,Y)\).
    
    Then, \(F(\phi): F(X) \to F(Y), F(\phi)\in \operatorname{Mor}_{\mathcal{D}}(F(X),F(Y))\). 

    We have the following categories: Sets, Groups, Ab, Top, Rings, Comm, Field, \(R\)-mod, COmplexes of \(R\)-mod, Sheaves on \(X\), etc.

    \begin{definition}
        A functor is \textit{faithful} if \(\forall X,Y\in \operatorname{Ob} \mathcal{C}, \operatorname{Mor}_{\mathcal{C}} (X,Y) \to \operatorname{Mor}_{\mathcal{D}}(F(X),F(Y))\) is injective.

        It is \textit{fully faithful} if this map is a bijection.
    \end{definition}

    \(\operatorname{Top} ^{\ast} =\) category of pointed topological spaces. This contains pairs \((X,x)\), space with a point.
    
    \(\operatorname{Mor}_{\operatorname{Top}^{\ast}}\left( (X,x),(Y,y) \right) = \left\{ \text{cont. maps } f: X\to Y \text{ s.t. } f(x) = y \right\}\)

    This is useful: we can't find fundamental group without a base point.

    Then, \(\pi_1\) is a functor from \(\operatorname{Top}^{\ast}\) to Groups. Morphism \((X,x) \xrightarrow{f} (Y,y)\) gives us \(f_{\ast} : \pi_1(X,x) \to \pi_1(Y,y)\) which is a group homomorphism.

    We want to talk about natural transformation which is important for this course.

    \begin{definition}
        [Natural Transformation] Consider functors \(\mathfrak{f},\mathfrak{g}: \mathcal{C} \to \mathcal{D}\).

        A natural transformation \(T: \mathfrak{f} \to \mathfrak{g}\) assigns to each \(x \in \operatorname{ob} \mathcal{C}\) an element \(T(x) \in \operatorname{Mor}_{\mathcal{D}} (\mathfrak{f}(x), \mathfrak{g}(x))\) with compatibilty condition:

        Given \(x,y\in \operatorname{Ob} (\mathcal{C}), f\in \operatorname{Mor}_{\mathcal{C}}(x,y)\) such that the following diagram commutes:

        \[
            \begin{tikzcd}
                \mathfrak{f} (x) \ar[r,"T(x)"] \ar[d,"\mathfrak{f}(f)"'] & \mathfrak{g}(x) \ar[d,"\mathfrak{g}(f)"] \\ \mathfrak{f}(y) \ar[r,"T(y)"] & \mathfrak{g}(y) 
            \end{tikzcd}
        \]
    \end{definition}

    
    \begin{definition}
        If \(f\in \operatorname{Mor}_{\mathcal{C}}(x,y), g\in \operatorname{Mor}_{\mathcal{C}}(y,x)\) we say \(f\) and \(g\) are inverses iff \(f \circ g = \operatorname{id}_{y}, g \circ f = \operatorname{id}_{x}\).

        A morphism which has an inverse is called an isomorphism.
    \end{definition}
    
    Inverses are unique. If \(h\) is also an inverse of \(f\) then \(h \circ f \circ g = h \circ (f \circ g) = h \circ \operatorname{id}_{y} = h\) and \(h \circ f \circ  g= (h \circ f) \circ  g = \operatorname{id}_{x} \circ  g = g\).

    \begin{definition}
        Morphisms with inverses are isomorphisms.
    \end{definition}

    \begin{definition}
        A category in which every morphism has an inverse is called a groupoid.
    \end{definition}

    Lets talk about an example. Consider the cateory with \(1\) object \(\{ \ast \}\). Since our categories are locally small, the morphisms form a set. There is a composition law. This gives us:

    \textit{A category with one object is a monoid}.

    Of course, if we add the stipulation that every morphism must have an inverse,

    \textit{A groupoid with one object is a group}.

    We want a categorical analogue for injectivity and bijectivity. Consider the example of the same set with two topologies, one finer than the other. Then on the point level we can have a bijection, but one map is continuous and the inverse map is not.

    This gives us the concepts of monomorphism and epimorphism.

    Monomorphism loosely resembles injectivity.

    Epimorphism loosely resembles surjectivity.

    \begin{definition}
        \(f \in \operatorname{Mor}_{\mathcal{C}}(x,y)\) is a \textit{monomorphism} if \(\forall z\in \operatorname{Ob} \mathcal{C}\) and all \(g,h\in \operatorname{Mor}_{\mathcal{C}}(z,x)\) we have:
        
        \[
            f \circ  g = f \circ h \implies g = h
        \]

        \[
            \begin{tikzcd}
                z \ar[r, shift left = 2pt, "g"] \ar[r, shift right = 2pt, "h"'] & x \ar[r,"f"] & y
            \end{tikzcd}
        \]
    \end{definition}

    \begin{definition}
        \(f \in \operatorname{Mor}_{\mathcal{C}}(x,y)\) is epimorphic if \(\forall z\in \operatorname{Ob}\mathcal{C}, \forall g,h\in \operatorname{Mor}_{\mathcal{C}}(y,z)\),
        
        \[
            g \circ f = h \circ  f \implies g = h
        \]

        \[
            \begin{tikzcd}
                x \ar[r,"f"] & y \ar[r, shift left = 2pt, "g"] \ar[r, shift right = 2pt, "h"'] & z
            \end{tikzcd}
        \]
    \end{definition} 

    \begin{definition}
        [Natural Isomorphism] Given categories \(\mathcal{C}\) and \(\mathcal{D}\) and a functor \(\mathfrak{f}, \mathfrak{g}: \mathcal{C} \to \mathcal{D}\), a natural isomorphism is a natural transformation \(T\) from \(\mathfrak{f}\) to \(\mathfrak{g}\) such that for all \(x\in \operatorname{Ob} \mathcal{C}\),

        \[
            T(x) \in \operatorname{Mor}_{\mathcal{D}}(\mathfrak{f}(x), \mathfrak{g} (x))
        \]

        is an isomorphism.
    \end{definition}

    Nonexample of natural isomorphism: fix a field \(k\) and let \(C = \operatorname{Vect}_k\). Consider the double dual functor \(f: \mathcal{C} \to \mathcal{C}\) so that \(V \to (V^{\ast})^{\ast}\).

    [We take two duals since only one would mean this is a contravariant functor. We want the direction of the functors to be the same].

    Consider the identity functor \(\operatorname{id}_{\mathcal{C}}: V \to V\).
    
    We have a natural transformation \(\operatorname{id}_{\operatorname{Vect}_k} \to \mathfrak{f}\) by \(V \mapsto (V \to V^{\ast})\)
    
    Any \(v\in V\) defines a linear transformation \(T_v : V^{\ast} \to k\) given by \(T_v(v^{\ast}) = v^{\ast} (v)\). Then \(T_v \in (V^{\ast})^{\ast} = V^{\ast\ast}\).
    
    We have the following commutative diagram:

    \[
        \begin{tikzcd}
            V \ar[r,"T_v"] \ar[d,"A"] & V^{\ast\ast} \ar[d,"T(A)"] \\ W \ar[r,"T_w"] & W^{\ast\ast} 
        \end{tikzcd}
    \]

    If \(V\) is infinite dimensional, then \(\dim V^{\ast} > \dim V\). Then \(\dim V^{\ast\ast} > \dim V\). So this is only a natural transformation, not a natural isomorphism.

    Note that however in \(\operatorname{Vect}_k^{\text{fin}}\) the double dual is a natural isomorphism.

    Also see: equivalence of categories.

    \section*{Friday, 8/29/2025}
    
    We continue category theory today.

    \begin{definition}
        [Equivalence of Categories] If \(\mathcal{C}\) and \(\mathcal{D}\) are categories and \(F: \mathcal{C} \to \mathcal{D}\) and \(G: \mathcal{D} \to \mathcal{C}\) are functors such that \(F \circ G : \mathcal{D} \to \mathcal{D}\) and \(G \circ F: \mathcal{C} \to \mathcal{C}\) are naturally isomorphic to \(\operatorname{id}_{\mathcal{D}} \) and \(\operatorname{id}_{\mathcal{C}} \) respectively.
    \end{definition}

    For example, let \(\mathcal{C} =\) category with objects \(\varnothing , \{ 1 \} , \{ 1,2 \}, \{ 1,2,3 \}, \cdots\) and morphisms are functions.

    Let \(\mathcal{D}\) be the category of finite sets and morphisms are functions.

    We have an obvious functor: \(\mathcal{C} \to \mathcal{D}\) sends each \(\{ 1, 2, \cdots , n \} \) to itself.

    For \(\mathcal{D} \to \mathcal{C}\) we need to work a little bit harder, and we have to deal with axiom of choice and other stuff. To avoid these, we introduce the following easier definition:

    \begin{definition}
        If \(\mathcal{C}\) and \(\mathcal{D}\) are categories and \(F: \mathcal{C} \to \mathcal{D}\) and:

        \begin{enumerate}[label=\arabic*)]
            \item \(F\) is fully faithful
            \item \(F\) is essentially surjective. 
        \end{enumerate} 

        [Essentially surjective means every object is isomorphic to an object in the image. Every set with \(n\) elements is isomorphic to \(\{ 1,\cdots ,n \}\) for example.]

        Then \(\mathcal{C}\) and \(\mathcal{D}\) are equivalent.
    \end{definition}

    Given a category \(\mathcal{C}\) and \(A \in \operatorname{ob} \mathcal{C}\) we define functors:

    \(h_A: \mathcal{C} \to \operatorname{Sets}\) given by \(h_A(X) = \operatorname{Mor}_{\mathcal{C}} (A,X)\). This is contravariant.

    \(h^A: \mathcal{C} \to \operatorname{Sets}\) given by \(h^A(X) = \operatorname{Mor}_{\mathcal{C}} (X,A)\)

    Given \(A,B \in \operatorname{Ob} \mathcal{C}, \phi \in \operatorname{Mor}_{\mathcal{C}} (A,B)\), \(\phi\) deines a functor \(h_B(X) \to h_A(X)\) and \(h^A(X) \to h^B(X)\).

    \begin{definition}
        A contravariant functor \(F: \mathcal{C} \to \operatorname{Sets}\) is \textit{representable} if \(\exists A \in \operatorname{Ob} \mathcal{C}\) such that \(F = h_A\).
    \end{definition}

    \begin{theorem}
        [Yoneda Lemma] The set of natural transformations \(h_A \to h_B\) is naturally isomorphic to \(\operatorname{Mor}_{\mathcal{C}}(B,A)\).
    \end{theorem}

    \begin{proof}
        Let \(N\) be a natural transformation from \(h_A\) to \(h_B\). i.e. for \(X\in \operatorname{ob} \mathcal{C}\) we have:

        \[
            N(X) : \underset{= \operatorname{Mor}_{\mathcal{C}} (A,X)}{h_A(X)} \to \underset{=\operatorname{Mor}_{\mathcal{C}} (B,X)}{h_B(X)}  
        \]

        Let \(X = A\). Then, \(N(A): h_A(A) \to h_B(A) \implies N(A): \operatorname{Mor}_{\mathcal{C}} (A,A) \to \operatorname{Mor}_{\mathcal{C}} (B,A)\)
        
        Let \(N(A)(\operatorname{id}_{A})\eqqcolon \psi\).
        
        Composition by \(\psi\) gives a map \(h_A \to h_B\), i.e. for all \(Y\in \operatorname{Ob} \mathcal{C}\), composition by \(\psi\) gives \(h_A(Y) \to h_B(Y)\) which is the same as \(N(Y)\).

        Let \(f: X \to Y\). We have:

        \[
            \begin{tikzcd}
                \operatorname{Mor}_{\mathcal{C}} (A,X) \ar[r,"N(X)"] \ar[d,"f_{\ast}"] & \operatorname{Mor}(B,X) \ar[d,"f_{\ast}"] \\ \operatorname{Mor}(A,Y) \ar[r,"N(Y)"] & \operatorname{Mor}(B,Y)
            \end{tikzcd}
        \]

        Setting \(X = A\),

        \[
            \begin{tikzcd}
                \operatorname{Mor}_{\mathcal{C}} (A,A) \ar[r,"N(X)"] \ar[d,"f_{\ast}"] & \operatorname{Mor}(B,A) \ar[d,"f_{\ast}"] \\ \operatorname{Mor}(A,Y) \ar[r,"N(Y)"] & \operatorname{Mor}(B,Y)
            \end{tikzcd}
        \]

        taking \(\operatorname{id}_{A}\) and applying the commutativity,

        \[
            \begin{tikzcd}
                \operatorname{id}_{A} \ar[r,"N(X)"] \ar[d,"f_{\ast}"] & \psi \ar[d] \\ f \ar[r] & N(Y)(f) = f \circ \psi  
            \end{tikzcd}
        \]
    \end{proof}

    \subsection*{Universal Objects}

    \begin{definition}
        An object \(X\in \operatorname{ob} \mathcal{C}\) is an \textit{initial object} if \(\forall Y\in \operatorname{ob} \mathcal{C}, \vert \operatorname{Mor}_{\mathcal{C}} (X,Y) \vert = 1\).

        It is a final object if \(\forall Y\in \operatorname{ob} \mathcal{C}, \vert \operatorname{Mor}_{\mathcal{C}} (Y,X) \vert = 1\). 
    \end{definition}

    Up to \textit{unique isomorphism} an initial or final object in a category is unique if it exists.

    \begin{definition}
        Let \(L\) be a (commutative) ring and \(S\) a multiplicative system in \(A\), meaning \(1\in S, x,y\in S \implies xy\in S, 0\notin S\).

        The localization \(S ^{-1} A\) is the \textit{universal \(A\)-algebra in which every element of \(S\) is invertible.}

        \(\frac{a_1}{s_1} = \frac{a_2}{s_2}\) means \((a_1 s_2 - a_2 s_1) s_3 = 0\) for some \(s_3 \in S\).
    \end{definition}

    We need the construction to show that the localization exists. But it is easier to work with the universal property!

    \(S ^{-1} A\), assuming it exists, is universal among all \(A\)-algebras in which \(S\) is invertible.

    Consider all ring homomorphisms \(\left\{ \phi : A \to B \mid \phi(s) \text{ is a unit in \(B\) for all \(s\in S\)} \right\} \) 

    We can now define a category. Let this set be \(\operatorname{ob} \mathcal{C}\). Let the morphisms be as follows:

    \[
        \begin{tikzcd}
            & B \ar[dd]\\ A \ar[ru,"\phi"] \ar[rd,"\psi"] \\ & C
        \end{tikzcd}
    \]

    Existence of \(S ^{-1} A\) is expressed by the existence of an initial object in this category.

    We have the homomorphism \(A \to S ^{-1} A\) by \(a \mapsto \frac{a}{1}\).

    \[
        \begin{tikzcd}
            & S ^{-1} A \ar[dd, dotted, "!"]\\ A \ar[ru] \ar[rd,"s \mapsto b^{\ast}"] \\ & B
        \end{tikzcd}
    \]

    Another example: suppose \(A\) is a ring and \(M,N\) are \(A\)-modules. We can define tensor product \(M \otimes_A N\).

    The property we're interested in is: \(\operatorname{Hom}_A (M \otimes_A N, X) = A \text{-billinear}(M \times N, X)\). 

    Fix \(M,N\). Consider the functor \(X \mapsto \left\{ A \text{-billinear maps } M \times N \to X \right\}\).

    This functor is \textit{representable} by in the category of \(A\)-modules.

    \[
        \begin{tikzcd}
            & M \otimes_A N \ar[rd,dotted, "!"] \\ M \times N \ar[rr] \ar[ur] && X
        \end{tikzcd}
    \]

    Does (Sets) have an initial and final object? \(\varnothing\) is initial, any \(1\)-element set is final.

    What about the category of complex vector spaces?

    \(0\) is initial and final.

    A zero object is an object that is both initial and final.

    Category of infinite sets doesn't have an initial or final object.

    In the category of rings, \(\mathbb{Z} \to R\) always exists so it's initial. We don't take zero rings so there's no final object.

    Note that if we have a map of rings \(A \to B\), the map of schemes go in the opposite direction: \(\operatorname{Spec} B \to \operatorname{Spec} A\). So there should be a final object in the category of schemes.

    \section*{Wednesday, 9/3/2025}
    
    \subsection*{Products and Coproducts}

    Suppose we have a category \(\mathcal{C}\), index set \(I\) and for each \(\alpha \in I\) we have \(X_\alpha \in \operatorname{ob} \mathcal{C}\).

    We want to talk (in a categorical sense) about the product of all the \(X_\alpha\)'s. This should be analogous to the cartesian product, we should be able to extract the initial object.

    The product, thus, should be an object \(X\in \operatorname{ob} \mathcal{C}\) together with the maps \(\pi_\alpha \in \operatorname{Mor}_{\mathcal{C}} (X,X_\alpha)\) which is universal for such data.

    For example, in the case \(I = \{ 1,2 \}\) and \(X = X_1 \times X_2\), \(X\) is universal in the following sense:

    \[
        \begin{tikzcd}
            & Y \ar[ldd, bend right] \ar[rdd, bend left] \ar[d, dotted] \\ & X \ar[ld] \ar[rd] \\ X_1 & & X_2
        \end{tikzcd}
    \]

    For coproducts we just reverse the arrows.

    Category \(\mathcal{C}, \alpha \in I\) index set, \(X_\alpha \in \operatorname{ob} \mathcal{C}\).
    
    The coproduct \(\coprod_\alpha X_\alpha\) of the \(X_\alpha\)'s is an object \(X\in \operatorname{ob} \mathcal{C}\) together with maps \(i_\alpha \in \operatorname{Mor}_{\mathcal{C}} (X_\alpha , X)\) with is universal for such data. For \(I = \{ 1,2 \}\) and \(X = X_1 \coprod X_2\):

    \[
        \begin{tikzcd}
            & Y & \\ & X \ar[u, dotted] \\ X_1 \ar[uur, bend left] \ar[ur] & & X_2 \ar[uul, bend right] \ar[ul]
        \end{tikzcd}
    \]

    In the category of sets, product is the cartesian product, and coproduct is the disjoint union.

    In \(\operatorname{Ab}\), the product and coproduct of two objects are the same, the direct sum as long as the index set is finite.

    For infinite index set,

    \(\coprod_{i=1}^{\infty} = \bigoplus_{i=1}^{\infty} X_i = \left\{ (x_1, x_2, \cdots ) \mid x_i \in X_i, x_i = 0 \, \forall i \gg 0 \right\}\)
    
    Finite sums.

    \(\prod_{i=1}^{\infty} X_i = \left\{ (x_1, x_2, \cdots ) \mid x_i\in X_i \right\} \)
    
    Unrestricted.

    These are not the same! Infinite product of \(\mathbb{Z}\) is not free for example.

    We can write it like this:

    \(\operatorname{Ab} \times \operatorname{Ab} \xrightarrow{\coprod} \operatorname{Ab}\) 

    \(\operatorname{Ab} \times \operatorname{Ab} \xrightarrow{\prod} \operatorname{Ab}\)

    We have the following natural transformation:

    \((X_1, X_2) \mapsto \begin{matrix} (X_1 \oplus X_2 \to X_1 \times X_2) \\ (x_1, x_2) \mapsto (x_1, x_2) \end{matrix}\) 

    Something that is both a product and a coproduct is called a biproduct.

    \subsection*{Limits and Colimits}

    We can generalize the concepts of product to limit and coproduct to colimit.

    Limits/Colimits are the \textit{same thing} but instead of an index set \(I\) we use an \textit{index category} \(\mathcal{I}\).

    The data which determines the limit/colimit is a functor from \(\mathcal{I} \to \mathcal{C}\).

    For example: consider the following category of \(3\) elements (ignore the identity morphisms):

    \[
        \begin{tikzcd}
            1 \ar[d] \\ 0 & 2 \ar[l]
        \end{tikzcd}
    \]

    Consider functors from this category to a category \(\mathcal{C}\). We then have the following in \(\mathcal{C}\):

    \[
        \begin{tikzcd}
            X_1 \ar[d] \\ X_0 & X_2\ar[l]
        \end{tikzcd}
    \]

    The limit of such a diagram, if it exists consists of \(X\in \operatorname{ob} \mathcal{C}\) and maps \(\pi_i \in \operatorname{Mor}_{\mathcal{C}}(X,X_i)\) such that the diagram:

    \[
        \begin{tikzcd}
            X_1 \ar[d] & X \ar[l] \ar[ld] \ar[d] \\ X_0 & X_2\ar[l]
        \end{tikzcd}
    \]

    commutes with the universal property:

    \[
        \begin{tikzcd}
            & & Y \ar[dll] \ar[ddl] \ar[dl, dotted] \\ X_1 \ar[d, "f_1"] & X \ar[l] \ar[ld] \ar[d] \\ X_0 & X_2\ar[l, "f_2"]
        \end{tikzcd}
    \]

    This is specific case is called the fiber product.

    In (Sets) all limits and colimits exist.

    In the fiber product example, we can consider \(X = \coprod_{x_0\in X_0} f_1 ^{-1} (x_0) \times f_2 ^{-1} (x_0)\).

    We can look at the following category of natural number: \(\mathcal{I} \coloneqq \cdots \to 4 \to 3 \to 2 \to 1\) [we don't write arrows \(4 \to 1\) since it's a composition].

    Let \(\mathcal{C} = \operatorname{Ab}\). We can consider \(\mathcal{I} \to \operatorname{Ab}\) so that we have \(\cdots \to X_4 \to X_3 \to X_2 \to X_1\).
    
    Then taking limit gives us the projective limit \(\varprojlim_n X_n\).

    For example if \(X_i = \mathbb{Z} / p^i \mathbb{Z}_i\).

    We have \(\mathbb{Z} / p^{i+1} \mathbb{Z} \to \mathbb{Z} / p^i \mathbb{Z}\) by taking \(\mod p^i\). 

    Then \(\varprojlim_n \mathbb{Z} / p^n\mathbb{Z} = \mathbb{Z}_p = a_0 + a_1 p + a_2 p^2 + \cdots\).

    Note: the topology of \(\mathbb{Z}_p\) is important. Individual \(\mathbb{Z} /p^n \mathbb{Z}\) have discrete topology. They're finite and thus compact. The topology of \(\mathbb{Z}_p\) then comes from Tychonoff's theorem.

    \subsection*{Filtered Category}

    A \textit{filtered category} \(\mathcal{I}\) satisfies:

    \begin{enumerate}[label=\arabic*)]
        \item \(\mathcal{I}\) is non-empty.
        \item If \(x,y\in \operatorname{ob} \mathcal{I}\) there exists \(z\in \operatorname{ob} \mathcal{I}\) such that \(\operatorname{Mor}_{\mathcal{C}} (x,z)\neq \varnothing, \operatorname{Mor}_{\mathcal{C}} (y,z)\neq \varnothing\).
        \item If \(x,y\in \operatorname{ob} \mathcal{I}, f,g\in \operatorname{Mor}_{\mathcal{C}} (x,y)\) then \(\exists z\in \operatorname{ob} \mathcal{I}, h\in \operatorname{Mor}_{\mathcal{C}}(y,z)\) such that \(h \circ f = h \circ g\).   
    \end{enumerate} 

    Condition 2 implies given \(x,y\) we can always find \(z\) such that,

    \[
        \begin{tikzcd}
            x \ar[rd] \\ & z \\ y \ar[ru]
        \end{tikzcd}
    \]

    Condition 3 implies given

    \[
        \begin{tikzcd}
            x \ar[r, bend right] \ar[r, bend left] & y
        \end{tikzcd}
    \]

    We can find

    \[
        \begin{tikzcd}
            x \ar[r, bend right] \ar[r, bend left] & y \ar[r] & z
        \end{tikzcd}
    \]

    Advantage of having a filtered category: we can make colimits exist.

    \begin{theorem}
        The category of fields does not have general colimits but it does have \textit{filtered colimits}
    \end{theorem}

    Take a colimit in the category of sets and observe that it has a field structure.

    How do we add up two elements in different fields \(x\) and \(y\)? Take the field \(z\) and add there!

    \subsection*{Adjoint Functors}

    Suppose we have categories \(\mathcal{C}\) and \(\mathcal{D}\) and functors \(F: \mathcal{C} \to \mathcal{D}, G: \mathcal{D} \to \mathcal{C}\). TFAE:
  
    \begin{enumerate}[label=\arabic*)]
        \item \((F,G)\) is an adjoint pair
        \item \(F\) is the left-adjoint of \(G\)
        \item \(G\) is the right-adjoint of \(F\)
    \end{enumerate} 

    All these equate to saying:

    \begin{definition}
        There is a natural isomorphism between the following: \(\operatorname{Mor}_{\mathcal{D}}(F(X), Y)\) and \(\operatorname{Mor}_{\mathcal{C}}(X, G(Y))\). We denote this by \(N(X,Y)\). So \(\operatorname{Mor}_{\mathcal{D}}(F(X),Y) \xrightarrow{N(X,Y)} \operatorname{Mor}_{\mathcal{C}} (X,G(Y))\).
    \end{definition}

    The picture looks like the following: 
    Suppose we have \(X_1 \to X_2\) in \(\mathcal{C}\). 

    \[
        \begin{tikzcd}
            \operatorname{Mor}_{\mathcal{D}}(F(X_1),Y) \ar[r,"N(X_1\text{,}Y)"] & \operatorname{Mor}_{\mathcal{C}} (X_1, G(Y)) \\ \operatorname{Mor}_{\mathcal{D}}(F(X_2),Y) \ar[r,"N(X_2\text{,}Y)"] \ar[u,"F(f)^{\ast}"] & \operatorname{Mor}_{\mathcal{C}} (X_2, G(Y)) \ar[u,"f^{\ast}"] 
        \end{tikzcd}
    \]

\end{document}