\documentclass{article}
\usepackage{amsmath, amsthm, amssymb, amsfonts, mathrsfs, mathtools,enumitem, stmaryrd,physics, cancel, tikz-cd, graphicx, float, booktabs}
\usetikzlibrary{arrows}
\usepackage{geometry}
    \geometry{
        a4paper,
        left = 40mm,
        top = 20mm,
        right = 40mm,
        bottom = 30mm
    }
\setlength{\parindent}{0pt}

\theoremstyle{definition}
\newtheorem{problem}{Problem}
\newtheorem{solution}{Solution}
\newtheorem*{example}{Example}
\newtheorem*{exercise}{Exercise}
\newtheorem*{definition}{Definition}
\newtheorem{theorem}{Theorem}
\numberwithin{theorem}{subsection}
\newtheorem*{theorem*}{Theorem}
\newtheorem{proposition}[theorem]{Proposition}
\newtheorem*{proposition*}{Proposition}
\newtheorem{lemma}[theorem]{Lemma}
\newtheorem*{lemma*}{Lemma}
\newtheorem{corollary}[theorem]{Corollary}
\newtheorem*{corollary*}{Corollary}
\newtheorem*{remark}{Remark}

\title{M702 ANT}
\author{Thanic Nur Samin}
\date{\vspace{-5ex}}

\begin{document}
    \maketitle

    \section*{Tuesday, 1/14/2025}

    \begin{abstract}
        Chapter 1: Local Class Field Theory (LCFT).

        Chapter 2: \(p\)-divisible groups (eg LT formal groups) and associated Galois representations \(V\) and the \underline{Hodge-Tate Decomposition} of \(V \otimes_{\mathbb{Q}_p}\mathbb{C}_p\) and also the diagonal action of \(\mathscr{G}_K\).

        \underline{Tate}: \(p\)-divisible groups.

        Chapter 3: Sen theory, Fontaine's period rings \((\varphi, \Gamma)\)-modules.


    \end{abstract}
    
    \section{Local Class Field Theory (LCFT)}

    \subsection{Lubin Tate Theory}

    [N] Neukirch, Alg. NT
    
    [S] Serre, Local Class Field Theory (Cassels-Frohlich)

    [LT] Lubin, Tate Formal complex multiplication

    \(K =\)  non-archimedean local field (locally compact) \(\supset \mathcal{O} = \mathcal{O}_K =\) valuation ring \(\supset P_K =\) valuation ideal.

    Residue Field \(k = \mathcal{O} / P_K\), \(\operatorname{char} (k) = p > 0, q \coloneqq \vert k \vert = p^f\).

    Normalized Valuation \(v = v_K : K \twoheadrightarrow \mathbb{Z} \cup \{ \infty \}, \vert a \vert = q^{-v(a)}\).
    
    \(U_K = \mathcal{O}_K^\times\).

    \begin{definition}
        \(e(x) \in \mathcal{O} [[x]]\) (a formal power series) is called a Lubin-Tate (LT) series for the uniformizer \(\pi\) (fixed) if the following conditions are fulfilled:

        \begin{itemize}
            \item \(e(x) \equiv \pi x \mod\deg 2\).
            \item \(e(x) \equiv x^q \mod \pi\). 
        \end{itemize} 
    \end{definition}

    Set \(\mathscr{E}_\pi =\) set of LT series for the uniformizer \(\pi\).

    Recall: Let \(R\) be any \(\mathcal{O}\)-algebra (\(i: \mathcal{O}\to R\) ring homomorphism).

    A formal \(\mathcal{O}\)-module over \(R\) is a \(1\)-dimensional commutative formal group \(F(x,y) \in R[[x,y]]\) over \(R\) (some people call it a formal group law) together with a unital (sending \(1\) to \(1\)) ring homomorphism:

    \[
        [\cdot]_F : \mathcal{O} \to \operatorname{End}_R(F) = \{ f(x) \in R[[x]] \mid f(0)=0, f(F(x,y))=F(f(x),f(y)) \}
    \]

    such that \(\forall a\in \mathcal{O}: [a]_F (x) = i(a)x \mod\deg 2\).

    We have the following properties:

    \(F(x,y) = x + y + \text{higher order terms}\)

    Associativity: \(F(x,F(y,z))=F(F(x,y),z)\)

    Commutativity: \(F(x,y) = F(y,x)\).

    \(\implies\exists! \iota (x) \in R[[x]]: F(x,\iota(x)) = 0\). Also, \(\iota (x) = -x + \text{higher order terms}\).

    If \(R\) is a local \(\mathcal{O}\)-algebra with maximal ideal \(M\) \((i ^{-1} (M) = P_K, k = \mathcal{O} / P_K \to R / M)\) then a formal \(\mathcal{O}\)-module \(F\) over \(R\) is callled a LT \(\mathcal{O}\)-module over \(R\) if in addition it is a formal \(\mathcal{O}\)-module and for any uniformizer \(\pi\) of \(K\): \([\pi]_F (x) \equiv x^q \mod M\).

    \begin{remark}
        If \(F\) is a LT \(\mathcal{O}\)-module over \(\mathcal{O}\) [\(i: \mathcal{O} \xrightarrow{\operatorname{id}} \mathcal{O}\)] then \([\pi]_F (x) \in \mathscr{E}_{\pi}\) [meaning it is a Lubin Tate series] for any uniformizer \(\pi\). 
    \end{remark}

    \begin{example}
        \begin{enumerate}[label=\arabic*)]
            \item \(K = \mathbb{Q}_p, F = \widehat{\mathbb{G}}_m, \widehat{\mathbb{G}}_m(x,y) = x + y + xy = (1+x)(1+y)-1\).

            Then, \([\cdot]: \mathbb{Z}_p \to \operatorname{End}_{\mathbb{Z}_p}(\widehat{\mathbb{G}}_m), [a](x) = (1+x)^a - 1 \coloneqq \sum_{n=1}^{\infty} \binom{a}{n}x^n, \binom{a}{n}= \frac{a(a-1)\cdots (a-n+1)}{n!}\in \mathbb{Z}_p\) for any \(a\in \mathbb{Z}_p, n \geq 1\).

            \begin{exercise}
                \begin{enumerate}[label=\arabic*)]
                    \item \(\forall a\in \mathbb{Z}_p \forall n\geq 0, \binom{a}{n}\) as defined above is in \(\mathbb{Z}_p\).
                    \item If \(K\) is a proper extension of \(\mathbb{Q}_p\) then \(\binom{a}{n}\notin \mathcal{O}_K\) for infinitely many \(a\in \mathcal{O}_K\). 
                \end{enumerate} 
            \end{exercise}
            
            \item \(K = \mathbb{F}_q((t)), F = \widehat{\mathbb{G}}_a, \widehat{\mathbb{G}}_a(x,y) \equiv  x+y\). Set \([t](x) = tx + x^q\). Then,
            
            \[
                \left[ \underbrace{\sum_{\nu = 0}^{\infty} \alpha_\nu t^{\nu}}_{a}  \right](x) \coloneqq \sum_{\nu = 0}^{\infty} \alpha_\nu[t]^{\circ \nu}(x) = \sum_{n=1}^{\infty} a_n x^n \text{ where } a_1 = a
            \]

            gives \(F = \widehat{\mathbb{G}}_a\) the structure of a LT \(\mathcal{O}\)-module over \(\mathcal{O}\).

        \end{enumerate}
    \end{example}

    \begin{theorem}
        \begin{enumerate}[label=\roman*)]
            \item For all uniformizer \(\pi\) of \(K\) and any \(e \in \mathscr{E}_{\pi}\) there exists unique LT \(\mathcal{O}\)-module \(F_e\) over \(\mathcal{O}\) such that:
            
            \[
                [\pi]_{F_e}(x) = e(x)
            \]

            \item \(\forall e, e^{\prime} \in \mathscr{E}_{\pi}\) there is an isomorphism of formal \(\mathcal{O}\)-modules \(f: F_e \to F_{e^{\prime}}\) (\(f\in x \mathcal{O} [[x]], f(F_e(x,y))=F_{e^{\prime}}(f(x),f(y))\).
            
            \(\forall a\in \mathcal{O}: f([a]_{F_e}(x))=[a]_{F_{e^{\prime}}}(f(x))\).

            \(f^{\prime} (0)\in \mathcal{O}^\times\).
            
            \item Let \(K^{nr}\) be the maximal unramified extension of \(K\) (inside some fixed algebraic closure \(\overline{K}\)) and let \(K_{nr} \coloneqq \widehat{K^{nr}}\) be the completion of \(K^{nr}\) and let \(\mathcal{O}_{K_{nr}}\) be its valuation ring. Then for any two uniformizers \(\pi , \pi ^{\prime} \) of \(K\) and LT series \(e\in \mathscr{E}_{\pi}\) and \(e^{\prime} \in \mathscr{E}_{\pi^{\prime}}, \exists\) an isomorphism of formal \(\mathcal{O}\)-modules \(F_e \to F_{e^{\prime}}\) over \(\mathcal{O}_{K_{nr}}\).  
        \end{enumerate} 
    \end{theorem}

    \subsection*{Formal Complex Multiplication}

    Let \(\overline{K}\) be the fixed algebraic closure of \(K \supset \mathcal{O}_{\overline{K}} \supset P_{\overline{K}}\). Let \(\pi =\) the fixed uniformizer, \(e\in \mathscr{E}_{\pi}, F_e =\) LT \(\mathcal{O}\)-module over \(\mathcal{O}\).
    
    Set \(F[\pi^m] = \left\{ \alpha \in P_{\overline{K}} \mid \underbrace{[\pi^m]_{F_e}}_{e^{\circ m}(x)}(\alpha) = 0 \right\} \). This can be shown to be finite from Theorem 1.1.1.ii by setting \(e^{\prime}(x) = \pi x + x^q\). Then the isomorphism will provide a bijection to \(F_{e^{\prime}}[\pi^{m}, e^{\prime} \in \mathscr{E}_{\pi}]\). Then the zeros of the power series are the zeros of the iteration of the polynomial. Hence the set is finite.

    \(L_{\pi, m}\coloneqq K(F_e(\pi^m))\) called the field of \(\pi^m\)-torsion points of \(F_e\). It doesn't depend on \(e\), though it does depend on \(\pi\).

    \begin{example}
        if \(K=\mathbb{Q}_p\) and \(e(x) = (1+x)^p - 1\) then, \(L_{p,m} = \mathbb{Q}_p(\zeta - 1 \mid \zeta^{p^m} = 1) = \mathbb{Q}_{p}(\mu_{p^m})\).

        If we take \(e^{\prime}(x) = px + x^p\), the power series and the torsion points \(F_e[p^m]\) and \(F_{e^{\prime}}[p^m]\) are different but the fields \(\mathbb{Q}_{p}(F_{e[p^m]})\) and \(\mathbb{Q}_{p}(F_{e^{\prime}}[p^m])\)  has to be the same!
    \end{example}

    \begin{theorem}
        \begin{enumerate}[label=\roman*)]
            \item \(F_e[\pi^m]\) is a free \(\mathcal{O} / (\pi^m)\) module of rank \(1\) [note that \([\pi^m]\) annihilates \([a](x)\) since \([\pi^m]_{F_e}(\alpha) = 0\)].
            \item \(\forall m \geq 1\) the maps \(\mathcal{O} / (\pi^m) \to \operatorname{End}_{\mathcal{O}}(F_e[\pi^m]), a \mod \pi^m \mapsto [\alpha \mapsto [a](\alpha)]\).
            
            Also, \(\mathcal{O}^\times / (1+(\pi^m)) \to \operatorname{Aut}_{\mathcal{O}}(F_e[\pi^m])\), same formula are isomorphism (of finite groups).

            \item \(L_{\pi, m}\) does not depend on \(e\in \mathscr{E}_{\pi}\) but depends on \(\pi\). In particular, if \(e^{\prime} (x) = \pi x + x^q\) then \(L_{\pi,m}=K(F_{e^{\prime}}[\pi^m])\). 
            
            \item \(L_{\pi,m}\) is a finite purely ramified Galois extension (so it does not contain a proper unramified extension) of \(K\) of degree \((q-1)q^{m-1}\).
            
            The map \(G(L_{\pi,m} / K) \to \operatorname{Aut}_{\mathcal{O}}(F_e[\pi^m]) \overset{\text{ii, canonical}}{\cong} \mathcal{O}^\times / (1 + (\pi^m))\) given by \(\sigma \mapsto a \mod (1+(\pi^m))\).
            
            If \(\forall \alpha \in F_e[\pi^m]\colon \sigma (\alpha) = [a]_{F_e}(\alpha)\), is an isomorphism.

            \item If \(L_\pi = \bigcup_{m \geq 1} L_{\pi, m}\), then the maps in iv induce an isomorphism:
            
            \[
                G(L_\pi / K) = \lim_{\xleftarrow[m]{}} G(L_{\pi,m} / K) \xrightarrow[]{\cong} \lim_{\leftarrow} \mathcal{O}^\times / (1 + (\pi^m)) \cong \mathcal{O}^\times
            \]
        \end{enumerate} 
    \end{theorem}

    \section*{Thursday, 1/16/2025}
    
    Recall: we fixed an algebraic closure \(\overline{K}\). Residue field of \(\overline{K} = \overline{k} = \) algebraic closure of \(k = \mathbb{F}_q\).
    
    \begin{theorem}
        If \(L / K\) is abelian, \(L_{\pi} \subset L\), and \(L / L_\pi\) is purely rammified, then \(L_\pi = L\).
    \end{theorem}
    
    \begin{proof}
        Proof uses the Hasse-Arf theorem, which says that the jumps (or breaks) of the upper ramification filtration (\(G(L / K)^t, t \geq -1\)) are integers.
    \end{proof}

    \begin{remark}
        \(G(L_\pi / K)^m = \operatorname{Gal}(L_\pi / L_{\pi,m}), m \geq 0\).
        
        \(L_{\pi, 0} \coloneqq K\).
    \end{remark}

    Let \(K^{ab} \subset \overline{K}\) be the maximal abelian subextension.

    \begin{theorem}
        For any uniformizer \(\pi\) one has \(K^{ab} = K^{nr}\).
        
        \(K^{nr} =\) maximal unramified extension \(= K(\mu_n \mid p \nmid n)\). 
    \end{theorem}

    \begin{proof}
        Set \(L_\pi^{nr} \coloneqq K^{nr}.L_\pi \subset K^{ab}\). This gives us an exact sequence:

        \begin{center}
            \begin{tikzcd}
                1 \ar[r] & G(K^{ab} / L_\pi^{nr}) \ar[r] & G(K^{ab} / L_\pi) \ar[r, two heads] & G(L_\pi^{nr} / L_\pi) \ar[r] & 1
            \end{tikzcd}
        \end{center}
        
        \begin{center}
            \begin{tikzcd}
                G(L_\pi^{nr} / L_\pi) \ar[r,"\cong"] & G(\overline{k}\mid k) = \langle\varphi\rangle^{\text{top}}
            \end{tikzcd}
        \end{center}

        Where \(\varphi(\overline{\alpha}) = \overline{\alpha}^q, \overline{\alpha} \in \overline{k}\).
        
        \[
            \langle \varphi \rangle ^{\text{top}} \coloneqq \lim_{\leftarrow} \varphi^\mathbb{Z} / \varphi^{n\mathbb{Z}} \cong \lim_{\leftarrow} \mathbb{Z} / n \mathbb{Z} \eqqcolon \widehat{\mathbb{Z}}  
        \]

        Choose \(\widetilde{\varphi} \in G(K^{ab} / L_\pi)\) such that \(\eval{\widetilde{\varphi}}_{K^{nr}} = \varphi\)
        
        \(L_\pi \subset L \coloneqq (K^{ab})^{\overline{\langle \widetilde{\varphi} \rangle } }\).
        
        \(\overline{\langle \widetilde{\varphi} \rangle }\) is the closed subgroup of \(G(K^{ab} / L_{\pi})\) generated by \(\widetilde{\varphi}\). 

    \end{proof}

    \section*{Tuesday, 1/21/2025}
    
    \underline{Recall}: \(K =\) local nonarch.. field, \(\pi =\) uniformizer, \(e \in \mathscr{E}_\pi\) a LT seriess for \(\pi\), \(F_e\) a LT formal \(\mathcal{O}\)-module, \(L_\pi = \bigcup K(F_e[\pi^m]) \subset K^{ab}\) with topological isomorphism \(\operatorname{Gal}(L_\pi / K) \xrightarrow[\iota_\pi]{\cong} U_K = \mathcal{O}_K^\times\).

    1.1.6 \(\implies\) 

    \begin{center}
        \begin{tikzcd}
            K^\times \ar[r, equal] & U_K \times \pi^\mathbb{Z} \ar[r] & G(K^{ab} / K) \\ & (a,\pi^n) \ar[r,mapsto] & \underbrace{\iota_\pi^{-1}(a)}_{\text{acts trivially on }K^{nr}}\widetilde{\varphi}^n
        \end{tikzcd}
    \end{center}

    \(\widetilde{\varphi} =\) Frobenius element of \(G(K^{ab} / K)\).

    The map:

    \begin{center}
        \begin{tikzcd}
            U_K \ar[r, "\cong"', "\text{can}"] & G(L_\pi / K) & G(L_\pi K^{nr} / K^{nr}) \ar[l, "\cong", "\text{can}"'] \ar[r,hook] & G(K^{ab} / K)
        \end{tikzcd}
    \end{center}

    is canoncial. Here, \(K^{ab} = L_\pi K^{nr}\).

    \begin{definition}
        The Weil group \(W_K\) is defined by:

        \[
            W_K = \left\{ \sigma \in G(\overline{K} / K) \mid \eval{\sigma}_{K^{nr}} \in \varphi^\mathbb{Z}_{K^{nr}} \right\} 
        \]
    \end{definition}

    Here \(\varphi_{K^{nr}}=\) \underline{the} arithmetic Frobenius of \(K^{nr}\).

    We equip \(W_K\) with the coarsest topology which makes the \underline{inertia subgroup}:
    
    \[
        I_K = \left\{ \sigma \in G(\overline{K} / K) \mid \eval{\sigma}_{K^{nr}} = \operatorname{id}_{K^{nr}} \right\}
    \]

    an open subgroup, and \(I_K\) is equipped witth its profinite topology. Then,

    \[
        W_K = \sqcup_{n\in\mathbb{Z}} I_L \widetilde{\varphi}^\mathbb{Z}
    \]

    (disjoint union of open cosets) with \(\widetilde{\varphi}\) as in 1.1.6. 

    \begin{proposition}
        The abelianization \(W_K^{ab} = W_K / \overline{[W_K, W_K]}\) is isomorphic to:

        \[
            \left\{ \sigma \in G (K^{ab} \mid K) \mid \eval{\sigma}_{K^{nr}} \in \varphi^\mathbb{Z}  \right\} 
        \]

        The image of the homomorphism:

        \[
            K^\times \to G(K^{ab} / K)
        \]

        of 1..1.6 is \(W_K^{ab}\).

        \(U_K \supset 1 + (p^m)\) is open.
    \end{proposition}

    \begin{definition}
        Let \(\Gamma\) be a topological group and \(\rho: \Gamma \to \operatorname{Aut} (V)\) be a representation of \(\Gamma\) as an \(E\)-vector space (\(E\) = any field). \(\rho\) is called \underline{smooth} if \(\forall v\in V\) we have:

        \[
            \operatorname{Stab}_{\rho}(v) = \left\{ \gamma \in \Gamma \mid \rho (\gamma)(v) = v \right\} 
        \]

        is open.
    \end{definition}

    \begin{proposition}
        [\(\ell\)-adic local Langlands correspondence for \(GL_1\)]

        Let \(\ell \neq p\) be a prime. Then the isomorphism \(K^\times \to W^{ab}_{K}\) from 1.1.7 induces a bijection:
        
        \[
            \begin{Bmatrix}
                \text{continuous homomorphisms} \\ W_K \to GL_1(\overline{\mathbb{Q}_\ell}) = \overline{\mathbb{Q}_{\ell}}^\times 
            \end{Bmatrix} \to \begin{Bmatrix}
                \text{smooth irreducible} \\ \text{rep's of } GL_1(K) = K^\times \\ \text{on } \overline{\mathbb{Q}_{\ell}}\text{-vector space}
            \end{Bmatrix} / \cong
        \]

        \[
            \chi \mapsto \left[ K^\times \xrightarrow{\cong} W_K^{ab} \overset{\chi}{\dashrightarrow} \overline{\mathbb{Q}_{\ell}}^\times \right] 
        \]

        \begin{center}
            \begin{tikzcd}
                \chi \ar[r, mapsto] & \text{[} K^\times \ar[r,"\cong"] & W_K^{ab} \ar[r,"\chi",dash] & \overline{\mathbb{Q}}_{\ell}^\times \text{]} \\
                
                & & W_K \ar[u] \ar[ur,"\chi"]
            \end{tikzcd}
        \end{center}

    \end{proposition}

    \begin{proof}
        Main point: a smooth irreducible representation of \(K^\times\) on a \(\overline{\mathbb{Q}}_{\ell}^\times\) vector space is \(1\)-dimensional.
    \end{proof}

    \begin{remark}
        Proposition 1.1.8 is also true when \(\overline{\mathbb{Q}}_{\ell}\) is replaced by \(\mathbb{C}\) and with the appropriate modifications, when \(\overline{\mathbb{Q}}_{\ell}\) is replaced by \(\overline{Q}_p\).
    \end{remark}

    \subsection{1-dim formal groups: the functional equation lemma}

    Cf. Hazewinkel, Formal groups and Applications = [H1 Formal]

    Here we let:

    \begin{itemize}
        \item \(K\) = any commutative ring
        \item \(A \subset K\) subring
        \item \(p\) prime
        \item \(q\) power of \(p\)
        \item \(\sigma: K \to K\) ring homomorphism
        \item \(I \subset A\) ideal
        \item \(s_1, s_2, s_3, \cdots \in K\) 
    \end{itemize} 

    We assume:

    \begin{itemize}
        \item \(\sigma (A) \subset A\)
        \item \(\forall a\in A: \sigma (a) \cong a^q \mod I\) 
        \item \(p\in I\) so \(A / I\) is an \(\mathbb{F}_p\)-algebra
        \item \(\forall i \geq 1: s_i I \subset A\) 
        \item \(\forall b\in K \forall r \geq 0: b I^r \subset A \implies \sigma (b) I^r \subset A\). 
    \end{itemize} 

    \begin{lemma}

        Let \(g(x) = \sum_{i=1}^{\infty} b_i x^i \in x A[[x]]\).

        By HW1, \(\exists! f_g(x) = \sum_{i=1}^{\infty} d_i x^i \in x K[[x]]\) so that,

        \[
            f(x) = g(x) + \sum_{i=1}^{\infty} s_i (\sigma_{\ast} ^ i f)(x^{q^i}) \tag*{(1.2.1)}
        \]

        where \(\sigma_{\ast}^i f_g\) is power series obtained from \(f_g\) obtained by applying \(\sigma^i\) to all coefficients.
    \end{lemma}

    Indeed, \(d_n = \begin{dcases}
        b_n, &\text{ if } q\nmid n ;\\
        b_n + s_1 \sigma(d_{n / q}) + \cdots +s_r \sigma(d_{n / q^r}), &\text{ if } n = q^r m, q \nmid m.
    \end{dcases}\) 

    \begin{lemma}[The functional equation lemma (FEL)]

        Let the data be as above. Let \(g(x) = \sum_{i=1}^{\infty} b_i x^i\) and \(\overline{g}(x) = \sum_{i=1}^{\infty} \overline{b_i} x^i\) be in \(x A[[x]]\) and assume \(b_1 \in A^\times\). Then, \(f_g(x) = b_1 x +\) higher order terms \(\implies f_g\) has inverse \(f_g ^{-1}\) w.r.t.\ composition. Then,

        \begin{enumerate}[label=\roman*)]
            \item \(F_g(x,y) \coloneqq f_g ^{-1} (f_g(x) + f_g(y))\) is a formal group over \(A\).
            \item \(f_g ^{-1} (f_{\overline{g}}(x)) \in x A[[x]]\).
            \item Given \(h(x) = \sum_{i=1}^{\infty} c_n x^n \in x A[[x]], \exists \widehat{h} (x) = \sum_{n=1}^{\infty} \widehat{c_n} x^n\) s.t. \(f_g(h(x))=f_{\widehat{h}}(x)\).
            \item If \(\alpha(x) \in x A[[x]], \beta(x) \in K[[X]]\), then \(\forall r \geq 0: \alpha(x) \equiv \beta(x) \mod I^r A[[x]] \iff f_g(\alpha(x)) \equiv f_g(\beta (x)) \mod I^r A[[x]]\)   
        
        \end{enumerate} 

    \end{lemma}

    \begin{lemma}
        [HW1] Write \(f_g(x) = \sum_{i=1}^{\infty} d_i x^i\) and write \(n = q^r m, q \nmid m\). Then \(d_n I^r \subset A\).  
    \end{lemma}

    \begin{lemma}
        Let \(G(x,y) \in A[[x,y]]\) and \(n = q^r m\), and \(\ell > 0\). Then,
        
        \[
            G(x,y)^{q^\ell n} \cong \left( (\sigma_{\ast}^{\ell} G)(x^{q^{\ell}, y^{q^{\ell}}}) \right)^n \mod I^{r+1} 
        \]

        \((\sigma(a) \equiv a^q \mod I)\) 
    \end{lemma}

    \begin{proof}
        [Proof of (i) of FEL] Note that \(f_g ^{-1} (x) = b_1 ^{-1} x + h.o.t\). Then,
        \[
            F_g(x,y) = b_1 ^{-1} (b_1 x + b_1 y + h.o.t) = x + y + h.o.t \tag*{(1)}
        \]
        
        and associativity follows from the definition.

        Write \(F(x,y)=F_1(x,y)+F_2(x,y)+F_3(x,y) + \cdots\) with \(F_d(x,y) \in K[x,y]\) homogeneous of degree \(d\).

        We want to show, \(\forall d\geq 1, F_d(x,y) \in A[x,y]\).

        We prove this by induction. Case \(d=1\) already done.

        Assume \(d \geq 2\) and the statement is true for \(F_1, \cdots , F_{d-1}\).

        \underline{Note}:
        \[
            \forall r \geq 2: (F_1(x,y) + \cdots +F_{d-1} (x,y))^r \equiv F(x,y)^r \mod \deg d+1\tag*{(2)}
        \] 

        (2) and 1.2.4 together imply that \(\forall i \geq 1, n = q^r m, q \nmid m\) (\(n=1, r=0\) are ok).

        \[
            F(x,y)^{q^i n} \cong \left( (\sigma_{\ast}^i F) (x^{q^i},y^{q^i}) \right) ^ n \mod \deg d+1, I^{n+1} \tag*{(3)}
        \]

        By definition,

        \[
            f(F(x,y)) = f(x) + f(y) \tag*{(4)}
        \]

        (4) \(\implies\) (5):

        \[
            (\sigma_{\ast} f)((\sigma_{\ast} ^i F)(x,y)) = (\sigma_{\ast}^i f)(x) + (\sigma_{\ast}^i f)(y) \tag*{(5)}
        \]

        (1.1.2) = (6)

        \[
            f(x) = g(x) + \sum_{i=1}^{\infty} s_i (\sigma_{\ast}^i f)(x^{q^i}) \tag*{(6)}
        \]

        Substitute \(F(x,y)\) for \(x\) in (6). We get (7):

        \[
            f(F(x,y)) = g(F(x,y)) + \sum_{i=1}^{\infty} s_i \sum_{n=1}^{\infty} \sigma^i(d_n) F(x,y)^{q^i n} \tag*{(7)}
        \]

        Then we use the 12.4 congruence and our knowledge about the integrality of \(s_i\). Eventually it turns out that \(F_d(x,y) \equiv 0 \mod A[[x,y]]\). Thus \(F_d\) has coefficients in \(A\). 
        
    \section*{Thursday, 1/23/2025}
    
        Write \(n = q^r m, q \nmid m\).


        \(F(x,y)^{q^i n}\) in (7) satisfies (3).

        1.2.3 \(d_n I^r \subset A \implies \sigma(d_n)I^r \subset A\). Iterating, \(\sigma^i(d_n)I^r \subset A\).
        
        Also, \(s_i I \subset A\). Multiplying both sides,

        \[
            s_i \sigma^i(d_n) I^{r+1} \subset A
        \]

        Multiply (3) by \(s_i \sigma^i(d_n)\),

        \[
            s_i \sigma^i(d_n) F(x,y)^{q^i n} \equiv s_i \sigma^i(d_n)((\sigma_{\ast}^i F)(x^{q^i},y^{q^i}))^n \mod A,\deg d+1 \tag*{(8)}
        \]

        (7) and (8) together imply that,

        \[
            \cancel{f(F(x,y))} \equiv g(F(x,y)) + \sum_{i=1}^{\infty} s_i (\sigma_{\ast} ^i f)((\sigma_{\ast}^i F)(x^{q^i},y^{q^{i} })) \mod A, \deg d+1
        \]

        \[
            \overset{(5)}{\equiv} g(F(x,y)) + \sum_{i=1}^{\infty} s_i ((\sigma_{\ast}^i f)(x^{q^i})+(\sigma_{\ast}^i f)(y^{q^i})) 
        \]

        \[
            \overset{(6)}{=} g(F(x,y)) + \cancel{f(x)} - g(x) + \cancel{f(y)} - g(y)
        \]

        Upshot:
        
        \[
            g(F(x,y)) \equiv g(x) + g(y) \overset{\text{by assoc. on \(g\)}}{\equiv} 0 \mod A, \deg d+1 \tag*{(9)} 
        \]

        \[
            \implies 0 \overset{\mod A, \deg d+1}{\equiv}  g(F(x,y)) = b_1 F(x,y) + b_2 F(x,y)^2 + \cdots
        \]

        \[
            b_2 F^2 + \cdots \equiv b_2(\underbrace{F_1 + \cdots +F_{d-1}}_{\in A[x,y] \text{ by ind. hyp.} })^2 + b_3 (F_1 + \cdots + F_{d-1})^3 + \cdots \mod \deg d+1
        \]

        \[
            \implies 0\equiv b_1(\underbrace{F_1 + \cdots + F_{d-1}}_{\in A[x,y]} + F_d) \equiv b_1 F_d(x,y)
        \]

        Since \(b_1 \in A^\times\) we have, \(F_d(x,y) \in A[x,y]\).

        Statement ii (\(f_g ^{-1} (f_{\overline{g}}(x)) \in A[[x]]\) ) is proved in the same way.

        Satetement iii: \(\forall h(x) = \sum_{n=1}^{\infty} c_n x^n \in x A[[x]]\), suppose \(\exists \widehat{h}(x) = \sum_{n=1}^{\infty} \widehat{c_n}x^n \in A[[x]]\) such that \(f_g(h(x)) = f_{\widehat{h}}(x)\) which is defined by the Functional Equation of same type (i.e. all the other data are the same).
        
        Set \(\widehat{f}(x) = f(h(x))\) by assoc. \(h(x) \in xA[[x]]\).

        Recall:

        \[
            f(x) = g(x) + \sum_{i=1}^{\infty} s_i(\sigma_{\ast} ^i f)(x^{q^i})
        \]

        Then,

        \[
            \widehat{f} (x) - \sum_{i=1}^{\infty} s_i (\sigma_{\ast}^i \widehat{f})(x^{q^i}) = f(h(x)) - \sum_{i = 1}^{\infty} s_i (\sigma_{\ast}^i f)((\sigma_{\ast}^i h)(x^{q^i}))
        \]

        When \(n = q^r m, q \nmid m\),

        \[
            = f(h(x)) - \sum_{i=1}^{\infty} s_i \sum_{n=1}^{\infty} \sigma^i (d_n) \underbrace{\left( (\sigma_{\ast}^i h)(x^{q^i}) \right)^n }_{\underset{1.2.4}{\equiv} h(x,y)^{q^{i} n} \mod I^{r+1}} 
        \]

        Use 1.2.3 and \(s_i I \subset A\) to deduce that,

        \[
            \underset{\mod A}{\equiv} f(h(x)) - \sum_{i=1}^{\infty} s_i (\sigma_{\ast}^i f)(h(x)) \equiv g(h(x)) \equiv 0 \mod A 
        \]

        Set \(\widehat{h}(x) \coloneqq \widehat{f}(x) - \sum_{i=1}^{\infty} s_i (\sigma_{\ast}^i \widehat{f})(x^{q^i}) \in x A[[x]]\).
        
        Construction \(\implies \widehat{f}(x) = f_{\widehat{h}}(x)\) [unique solution to the functional equation].

    \end{proof}

    For statement iv: [H, Formal, ch 1, sec. 2.4.]

    So, we can write many formal group laws of the form \(F(x,y) = f ^{-1} (f(x)+f(y))\). where \(f\) is invertible. \(f\) is logarithm for this formal group law.

    Applications:

    \begin{enumerate}[label=\arabic*)]
        \item If \(K / \mathbb{Q}_p\) is a finite extension, \(\exists\) polynomials \(p_1 (x), p_2(x), \cdots \in K[x]\) such that:
        
        \[
            [a]_{F_e}(x) = \sum_{n=1}^{\infty} p_n(a) x^n
        \]

        with \(\forall n \geq 1 \forall a\in \mathcal{O}_K, p_n(a) \in \mathcal{O}_K\).

        Where \(F_e\) is a LT \(\mathcal{O}_K\)-module.

        eg when \(K = \mathbb{Q}_p, e(x) = (1+x)^p - 1 \implies p_n(x) = \binom{x}{n}\).
        
        \(p_n(a) \in \mathbb{Z}_p\) if \(a\in \mathbb{Z}_p\) but if \(K \neq \mathbb{Q}_p\) then \(\exists a\in \mathcal{O}_K\) such that \(\binom{a}{n}\notin \mathcal{O}_K\).

        \item Formal groups over \(\mathbb{F}_p\) or \(\overline{\mathbb{F}}_p\).
        
        Fix \(n \geq 1\). Set \(A = \mathbb{Z}, p\) a prime, \(I = p\mathbb{Z}, K = \mathbb{Q}, \sigma = \operatorname{id}_{}, q = p\).

        Define \(s_i^{(n)} = \begin{dcases}
            0, &\text{ if } i \neq n ;\\
            \frac{1}{p}, &\text{ if } i = n.
        \end{dcases}\) 

        Let \(g(x) = x, f_n(x) \in \mathbb{Z}\left[ \frac{1}{p} \right] [[x]]\) be the unique power series satisfying the functional equation:

        \[
            f_n(x) = x + p ^{-1} f_n(x^{p^n})
        \]

        Then,

        \[
            f_n(x) = x + \frac{x^{p^n}}{p} + \frac{x^{p^{2n}}}{p^2} + \cdots = \sum_{i=1}^{\infty} \frac{x^{p^{ni}}}{i} \tag*{$(\ast)$}
        \]

        FEL \(\implies F_n(x,y) = f_n ^{-1} (f_n(x)+f_n(y))\in \mathbb{Z} [[x]]\) by FEL.

        Exercise: if \(\ell\) is a prime \(\neq p\) then \(F_n(x,y)\mod \ell\) is isomorphic to \(\widehat{\mathbb{G}}_{a,\mathbb{F}_\ell}\).
        
    \end{enumerate} 

    Set \(\overline{F}_n(x,y) = F_n(x,y) \mod p \mathbb{Z} \in \mathbb{F}_p[x,y]\) a formal group over \(\mathbb{F}_p\).
    
    \begin{proposition}
        \begin{enumerate}[label=\roman*)]
            \item \([p]_{F_n} \equiv x^{p^n} \mod p\).
            \item If \(n\neq m \in \mathbb{Z}_{>0}\), then for any field \(k\) of characteristic \(p\), we have:
            
            \[
                \operatorname{Hom}_{\text{formal grp } / K}  \left( \overline{F}_n \otimes k, \overline{F}_m \otimes k \right) = \{ 0 \} 
            \]

            In particular, \(\overline{F}_n\) and \(\overline{F}_m\) are not isomorphic over any \(k\).
        \end{enumerate} 
    \end{proposition}

    \begin{proof}
        \begin{enumerate}[label=\roman*)]
            \item Set \(\alpha(x) = [p]_{F_n}(x) \in \mathbb{Z} [[x]]\) and \(\beta(x) = x^{q^n}\).
            
            Recall that \([p]_{F_n}(x) = f_n ^{-1} (p f_n (x))\).

            \[
                f_n(\alpha(x)) = p \cdot f_n(x), f_n(\beta(x)) = f_n(x^{p^n}).
            \]

            \((\ast) \implies f_n([p](x)) - f_n(x^{p^n}) = px \equiv 0\mod p\).
            
            FEL iv \(\implies \alpha(x) \equiv \beta(x) \mod p\).

            \item Let \(h(x) \in x k[[x]]\) be a non-zero homomorphism \(\overline{F}_n \otimes k \to \overline{F}_m \otimes k\).
            
            Let \(h(x) = ux^t + h.o.t, u\in k^\times , t \geq 1\). Then,

            \[
                \implies h \left( [p]_{\overline{F}_n}(x) \right) = [p]_{\overline{F}_m}(h(x)) 
            \]

            \[
                \implies u x^{p^n t} + h.o.t = u p^m x^{p^m t} + h.o.t.
            \]

            \[
                \implies  p^{n} t = p^m t \implies p^n = p^m
            \]

            Which is a contradiction.

        \end{enumerate} 
    \end{proof}

    \begin{remark}
        \begin{enumerate}[label=\arabic*)]
            \item One can show [H, Formal, 18.5.1] that a \(1\)-dimensional (commutative) formal group over a separably closed field \(k\) of \(\operatorname{char} p\) is isomorphic to exactly one of \(\overline{F}_n \otimes k\) for a unique \(n \geq 1\) or \(\widehat{\mathbb{G}}_{a,k}\).
            
            We define the \underline{height} of \(F\) to be:
            
            \[
                ht(F) \coloneqq \begin{dcases}
                    n, &\text{ if } F \cong \overline{F}_n \otimes k ;\\
                    \infty, &\text{ if } F \cong \widehat{\mathbb{G}}_{a,k}.
                \end{dcases}
            \]

            \item Let \(K = \mathbb{Q}_p (\zeta_{p^n - 1})\) unramified extension of degree \(n\) over \(\mathbb{Q}_p\). Let \(q = p^n, e(x) = f_n ^{-1} (p f_n(x))\) which we know is a Lubin-Tate series for the uniformizer \(\pi = p\) of \(K\).
            
            Clearly, \(e(x) = px + h.o.t, e(x) \equiv x^{p^n} = x^q \mod p\). These are exactly the conditions for LT series.

            \(\implies F_n = f_n ^{-1} (f_n(x) + f_n(y)) = F_e\) is the LT \(\mathcal{O}_K\)-module for \(e(x)\) by LT theory.

            One can show the canonical map:

            \begin{center}
                \begin{tikzcd}
                    \mathcal{O}_K \ar[r] & \operatorname{End}(F_n \otimes_{\mathbb{Z}} \mathcal{O}_K) \ar[r] \ar[d,"="] & \operatorname{End}(\overline{F}_n \otimes_{\mathbb{F}_p}\mathbb{F}_q) \\ & \operatorname{End}(F_e, \mathcal{O}_K)
                \end{tikzcd}
            \end{center}

            is injective but not surjective.

            \[
                \phi(x) = x^p \, \text{ is an endomorphism of } \overline{F}_n
            \]

            \(\implies \operatorname{End}(\overline{F}_n \otimes \mathbb{F}_q) = \mathcal{O}_K[\phi]\) where \([a]_{\overline{F}_n} \circ \phi = \phi \circ [\phi(a)]_{\overline{F}_n}\)
            
            and \(\mathcal{O}_K[\phi]\otimes_{\mathbb{Z}_p} \mathbb{Q}_p\) is a divisional algebra over \(\mathbb{Q}_p\).

            \[
                \operatorname{End}(\overline{F}_n \otimes_{\mathbb{F}_q}\mathbb{F}_q) \otimes_{\mathbb{Q}_p} \mathbb{Q}_p \eqqcolon D_n  
            \]

            We have: \(\dim_{\mathbb{Q}_p}(D_n) = n^2\) 
            
            Furthermore, \(\text{center}(D_n) = \mathbb{Q}_p\). \(D_n\) also contains \(K\) but it is not in the center.  
        \end{enumerate} 
    \end{remark}

    \subsection{LCFT following Hazewinkel}

    [H] = `Local Class Field Theory is easy'

    In this section, a \underline{local field} is, by convention, a field \(K\) which is complete for a discrete non-trivial non-archimedean absolute value \(\vert \cdot \vert\). i.e., \(\vert K^\times \vert\) is a non-trivial discrete subgroup of \(R_{> 0}\).

    Examples:

    \begin{enumerate}[label=\arabic*)]
        \item \(\mathbb{Q}_p, \mathbb{F}_{q}((t))\)
        \item \((\mathbb{Q}_p)_{nr} = \widehat{\mathbb{Q}_p^{nr}}, \overline{\mathbb{F}_q}((t)) = (\underbrace{\mathbb{F}_q((t)) \otimes_{\mathbb{F}_q} \overline{\mathbb{F}_q}}_{\mathbb{F}_q((t))^{nr}})^\wedge\) [the \(t\)-adic completion]
        \item \(\mathbb{C}((t))\) or \(k((t))\) [they are complete w.r.t.\ a \(t\)-adic absolute value].
    \end{enumerate} 

    Outline:

    \begin{enumerate}[label=\arabic*)]
        \item Assume \(K\) has algebraically closed residue field \(k\), and \(L / K\) is finite abelian extension. [Note that abelian here automatically means Galois]. Set \(U(K) = \mathcal{O}_K^\times\), units of the valuation ring, and \(V(L / K) = \langle \sigma (u) u ^{-1} \mid \sigma \in G(L / K), u \in U(L) \rangle\) [subgroup generated by these elements].
        
        Fix a uniformizer \(\pi_L\) of \(L\) and define:

        \[
            i: G(L / K) \to 
        \]
        
    \end{enumerate} 

\end{document}